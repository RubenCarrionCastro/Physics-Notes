\documentclass[12pt, oneside]{Thesis}
\usepackage[spanish,mexico]{babel}
\usepackage[sectionbib, square, comma, numbers, sort&compress]{natbib} 
\hypersetup{urlcolor=black, colorlinks=true} 
\usepackage{amsmath} 
\usepackage{mathrsfs}
\usepackage{indentfirst}
\usepackage{enumitem}
\usepackage{url}
\usepackage{braket} 
\usepackage{mathtools}
\usepackage{bibunits} 
\usepackage{float}
\usepackage{blindtext}
% \usepackage{blindtext}
\usepackage{multicol}
\usepackage{hyperref}
\usepackage[colorinlistoftodos, prependcaption, textsize=tiny]{todonotes}
%\setlength{\parindent}{0.5cm}
\newcommand{\tr}{\mathrm{tr}}
\newcommand{\diag}{\mathrm{diag}}
\renewcommand{\qedsymbol}{\rule{0.5em}{0.5em}}
\usepackage[spanish]{cleveref}
\usepackage[spanish]{babelbib}
%\selectbiblanguage{spanish}
\usepackage{siunitx}
\usepackage{booktabs}
\usepackage{arydshln}
\usepackage{bbm}
\usepackage[many]{tcolorbox}
\usepackage[toc]{appendix}
\usepackage{framed}
\usepackage{calrsfs}
\usepackage[mathscr]{euscript}
\usepackage{tensor}
\usepackage{autonum}
\usepackage[singlelinecheck=false]{caption}
% Sangría para captions (2em a modo de ejemplo)

\usepackage{cancel}
\newtheorem{observacion}[teorema]{Observaci\'on}
\newtheorem{hipotesis}[teorema]{Hip\'otesis}
\usepackage{xpatch}
\usepackage{xcolor}
\newcommand{\bl}{\color{blue}}
% \newcommand{ }{\color{red}}
\newcommand{\cc}{\color{cyan}}
\newcommand{\cm}{\color{magenta}}
\xpatchcmd{\proof}{\itshape}{\normalfont\proofnamefont}{}{}
% \renewcommand{\refname}{Bibliografía de Figuras}
% \renewcommand{\bibname}{Nuevo Título de Bibliografía}
% \usepackage[usenames,dvipsnames]{color}
%Este pequeño bloque permite que el número de la ecuación permanezca del tamaño del resto del texto
%si se decide hacer small el tamaño de la ecuación.
%%%%%%%%%%%%%%%%%
\makeatletter
\def\maketag@@@#1{\hbox{\m@th\normalfont\normalsize#1}}
\makeatother
%%%%%%%%%%%%%%%%
\newtheorem{propiedad}{Propiedades}[section]
\newtheorem{axioma}{Axioma}[section]
\newtheorem{thm}{Teorema}[section]
\newtheorem{theorem}{Teorema}[section]
\newtheorem{proposition}[thm]{Proposición} 
\newtheorem{lemma}[thm]{Lema}
\newtheorem{corollary}[thm]{Corolario} 
\newtheorem{conv}[thm]{Convención}
\newtheorem{defi}[thm]{Definición}
\newtheorem{definition}[theorem]{Definición}
\newtheorem{notation}[thm]{Notación} 
\newtheorem{exe}[thm]{Ejemplo}
\newtheorem{conjecture}[thm]{Conjetura} 
\newtheorem{prob}[thm]{Problema}
\newtheorem{remark}[thm]{Observación}
\newtheorem{example}[thm]{Ejemplo}
\newtheorem{note}[thm]{Nota}
\newtheorem{Trucaso}[thm]{Trucaso}
\newtheorem{dem}[thm]{Demostración}

\newcommand{\brackets}[1]{\left[#1\right]}
\newcommand{\curlybraces}[1]{\left\{#1\right\}}
\newcommand{\qedh}{\hfill\hspace{5mm}\qedsymbol}
\newcommand{\scalar}[2]{\langle #1, #2 \rangle}
\newcommand{\ptensor}[2]{#1 \otimes #2}
\newcommand{\pcart}[2]{#1 \times #2}
\newcommand{\funct}[3]{#1:\hspace{1mm} #2\to #3}
\newcommand{\abs}[1]{|#1|}

\newtcolorbox[auto counter, number within=section]{mytheorem}[2][]{
  enhanced,
  breakable,
  title=Teorema~\thetcbcounter: #2,
  #1,
}
\newtcolorbox[auto counter, number within=section]{propositionbox}[2][]{
  enhanced,
  breakable,
  title=Proposition~\thetcbcounter: #2,
  #1,
}

\newtcolorbox[auto counter, number within=section]{corollarybox}[2][]{
  enhanced,
  breakable,
  title=Corollary~\thetcbcounter: #2,
  #1,
}

\newtcolorbox[auto counter, number within=section]{remarkbox}[2][]{
  enhanced,
  breakable,
  title=Remark~\thetcbcounter: #2,
  #1,
}

\newtcolorbox[auto counter, number within=section]{notebox}[2][]{
  enhanced,
  breakable,
  title=Note~\thetcbcounter: #2,
  #1,
}


% \newenvironment{Figura}
%   {\par\medskip\noindent\minipage{\linewidth}}
%   {\endminipage\par\medskip}
\newenvironment{Figura}[1][\linewidth] % ancho por defecto
  {\par\medskip\centering
   \begin{minipage}{#1}
   \captionsetup{justification=justified}}
  {\end{minipage}\par\medskip}


\newcommand{\proofnamefont}{\bfseries}
\setlength {\marginparwidth}{2cm}
\begin{document}


	\frontmatter 
	\setstretch{1.5} 
	\fancyhead{} 
	\rhead{\thepage} 
	\lhead{} 
	\pagestyle{fancy} 
	\newcommand{\HRule}{\rule{\linewidth}{0.5mm}} 
	\hypersetup{pdftitle={\titulo}}
	\hypersetup{pdfauthor=\autor}
%---------------------------------------------------------------------------
	
  \begin{minipage}[t]{1.5cm}
    \vspace{30pt}
      \begin{picture}(50,100)
        %\put(-113,55){\includegraphics[scale=0.15]{Pictures/UGR-Logo.png}
        %\put(-40,-550){\rule[2pt]{1pt}{21cm}}
        %\put(-37,-550){\rule[2pt]{2mm}{21cm}}
        %\put(-29,-550){\rule[2pt]{1pt}{21cm}}
        %\put(-18,-550){\rule[2pt]{1pt}{21cm}}
        %\put(-15,-550){\rule[2pt]{2mm}{21cm}}
        %\put(-7,-550){\rule[2pt]{1pt}{21cm}}
        %\put(-63,53){\includegraphics[scale=0.25]{Pictures/Logo-FS.png}}
        
        \put(-67,60)
        {\includegraphics[scale=0.25]{Logotipo_I_Facultad_de_Ciencias_Fondo_blanco_negativo.png}}
        \put(-40,-550){\rule[2pt]{1pt}{21cm}}
        \put(-37,-550){\rule[2pt]{2mm}{21cm}}
        \put(-29,-550){\rule[2pt]{1pt}{21cm}}
        \put(-18,-550){\rule[2pt]{1pt}{21cm}}
        \put(-15,-550){\rule[2pt]{2mm}{21cm}}
        \put(-7,-550){\rule[2pt]{1pt}{21cm}}
      \end{picture}
  \end{minipage}
  \begin{minipage}[t]{14cm}
  	\vspace{40pt}
   	\begin{picture}(400,100)
   		\put(-25,178){\rule[2pt]{16cm}{1pt}}
   		\put(-25,170){\rule[2pt]{16cm}{2mm}}
   		\put(-25,167){\rule[2pt]{16cm}{1pt}}
   		\put(-25,130){
        
   		%\put(70,100){
        \begin{tabular}[t]{c}
          {\huge\textbf{ UNIVERSIDAD DE CÓRDOBA}}\\
          %{\huge\textbf{UNIVERSIDAD DE CÓRDOBA}}\\
          %{\huge\textbf{FISICA SOCIETY}}\\
             \\
           {\LARGE\textbf{\sc Facultad de Ciencias}}\\
              \\
           {\Large\textbf{Grado de F\'isica}}
        \end{tabular}}
    \end{picture}

  \begin{center}
  	{
    	\Large\bfseries 
     	Apuntes de
    	\par
    }%
    \vskip 3.5em%
    {
    	\begin{tabular}[t]{c}%
    		{\huge \textbf{FÍSICA DEL ESTADO SÓLIDO}}
     	\end{tabular} \par
    }
    \vskip 3em%
\fbox{\parbox{8.1cm}{\Large$\left[-\frac{\hbar^2}{2m}\nabla^2+V(\vec{r})\right]\Psi(\vec{r})=E\Psi(\vec{r})$\strut}}
      \vskip 1.5em %
      {
      	\begin{tabular}[t]{c}%
      		{\Large Autor:}
      	\end{tabular} 
        \par
      }
      \vskip 1.5em
      {
      	\begin{tabular}[t]{c}%
      		{\Large Rubén Carrión Castro}
      	\end{tabular}
      	\par
      }%
      \vskip 1.5em%
	  \vskip 20pt
	  {
		  \Large
			\begin{tabular}{ll}
				Profesores de la asignatura: & Dr. Pedro Rodríguez García   \\
			& Dr. Alberto Jiménez Solano\end{tabular}
		}
   \end{center}
   \vfill
   \begin{center}
     \begin{tabular}{c}
      {
          \large Cursado en la Universidad de Córdoba
      }
          \quad \hfill \quad {\large \today}
     \end{tabular}
   \end{center} \par%
\end{minipage}
{ 

%----------------------------------------------------------------------------------------
%	QUOTATION PAGE
%----------------------------------------------------------------------------------------
\pagestyle{empty}
\null\vfill 
\vfill\vfill\vfill\vfill\vfill\vfill\null 
\cleardoublepage 
	%\setstretch{1.3} % Return the line spacing back to 1.3

\pagestyle{empty} % Page style needs to be empty for this page

% \begin{flushright}
% \textit{Para mis compañeros de Córdoba que no tienen Relatividad General \ldots}
% \end{flushright} % Dedication text

\addtocontents{toc}{\vspace{2em}} % Add a gap in the Contents, for aesthetics

	% \input{Agradecimientos}
	%\listoftodos
	\pagestyle{fancy} % The page style headers have been "empty" all this time, now use the "fancy" headers as defined before to bring them back

\lhead{\emph{\'{I}ndice general}} % Set the left side page header to "Contents"
\tableofcontents % Write out the Table of Contents
	\mainmatter 
	\pagestyle{fancy} 
    
    %% INTRODUCCIÓN
%\chapter{Introducci\'on} % Main chapter title
\chapter*{Prefacio}
\addcontentsline{toc}{chapter}{Prefacio} 
\label{Introduccion} 
\lhead{\emph{Prefacio}} 
\noindent\textit{Todos somos muy ignorantes. Lo que ocurre es que no todos ignoramos las mismas cosas.} (A. Einstein)
\\ \\
Estos apuntes están hechos para acercar a estudiantes como yo a la Relatividad General, empezaremos con unas nociones de tensores, que hice en la Universidad de Córdoba con el doctor Jónatan Herrera. Seguiremos con la Teoría Especial de la Relatividad, luego un repaso de Geometría Diferencial y finalmente introduciremos la Relatividad General.
\lhead{\emph{Capítulo 1: Estructura Cristalina y Red Recíproca}}
\chapter{Estructura Cristalina y Red Recíproca}

\section{Introducción}

Las estructuras cristalinas son las estructuras de la mayoría de los sólidos en la naturaleza y dependiendo de ésta, las propiedades eléctricas, ópticas, térmicas y mecánicas del sólido pueden variar, pues depende de su estructura interna. Por tanto, conocer la estructura del sólido nos permitirá predecir comportamientos físicos, nos ayudará a interpretar fenómenos de difracción y lograremos diseñar materiales con propiedades específicas.\\ \\
Los sólidos cristalinos deben ser simétricos, y por ende, deben tener una estructura interna ordenada y periódica. Esta periodicidad se describe mediante una red tridimensional. Esta red está formada por una distribución periódica de nodos, y a su vez, estos nodos representan las intersecciones de la red. Las redes son una estructura matemática, no es física, pero este modelo nos será de gran utilidad para describir a los sólidos. Cada punto de la red se asocia a una base, que será un conjunto de átomos o moléculas. La base nos permite modelar toda la red. Cada estructura mínima periódica de la red se define como celda unidad, o en otras palabras, es la unión mínima cerrada de nodos.\\ \\
Definimos un cristal como el conjunto de una red y una base de átomos en relación a la celda unidad. Decimos que la red de Bravais es la 'estructura vacía' y la base la 'rellena'. Estos conceptos los detallaremos y estudiaremos en profundidad a lo largo de este capítulo.
 \section{Estructura Cristalina}
\subsection{Red de Bravais y vectores primitivos}
\subsubsection{Red de Bravais}
Vamos a comenzar dando una definición más formal de la red de Bravais.
\begin{definition}{Red de Bravais}
	Una red de Bravais es un conjunto infinito de puntos discretos con una disposición que se ve exactamente igual desde cualquiera de sus puntos.
\end{definition}
En otras palabras, una red de Bravais (tridimensional) consiste en todos los puntos con vectores de posición $\vec{R}$ de la forma,
\begin{equation}
	\vec{R}=n_1\vec{a}_1+n_2\vec{a}_2+n_3\vec{a}_3
\end{equation}
donde $\vec{a}_1$, $\vec{a}_2$ y $\vec{a}_3$ son tres vectores no coplanarios, y $n_1$, $n_2$ y $n_3$ abarcan todos los valores enteros. Es decir, cada punto de la red se obtiene combinando vectores primitivos $\vec{a}_i$ con coeficientes enteros $n_i$, para $i=1,2,3$. Es decir, el vector $\vec{R}$ nos define todo el espacio de la red.\\ \\
Por el estudio de los poliedros simétricos en matemáticas, sabremos que existirán un total de 5 redes de Bravais para sólidos bidimensionales y 14 redes de Bravais para los tridimensionales.
\subsubsection{Vectores primitivos}
Las propiedades más importantes de los vectores primitivos son las siguientes:
\begin{enumerate}
	\item $\vec{a}_1$, $\vec{a}_2$ y $\vec{a}_3$ son vectores no coplanarios que generan toda la red.
	\item No son únicos, es decir, existen infinitas elecciones posibles de vectores primitivos. Dependiendo de cuál tomemos los cálculos resultarán más o menos sencillos.
	\item La celda unitaria generada por estos vectores se denomina celda primitiva. Dentro de esta celda primitiva puede haber más de un átomo, con la posibilidad de átomos en los nodos de esta celda.
\end{enumerate}
\begin{Figura}
	\centering
	\includegraphics[width=.8\textwidth]{Imagenes/Capitulo1/VectoresPrimitivos.png}
	\captionof{figure}{Ejemplos de redes de Bravais: (a) 2D (oblicua) y (b) 3D (cúbica simple)}
	\label{fig1-1}
\end{Figura}
\begin{Figura}
	\centering
	\includegraphics[width=.8\textwidth]{Imagenes/Capitulo1/VectoresPrimiticos2.png}
	\captionof{figure}{Diferentes elecciones de pares de vectores primitivos para una red de Bravais 2D. Para mayor claridad, se muestra con distintos orígenes.}
	\label{fig1-2}
\end{Figura}
\newpage
\begin{multicols}{2}

\begin{Figura}
	\centering
	\includegraphics[width=.8\textwidth]{Imagenes/Capitulo1/VectoresPrimitivos3.png}
	\captionof{figure}{Red tipo panal. Se observa que desde distintos puntos de la red no se observa lo mismo.}
	\label{fig1-3}
\end{Figura}

\begin{note}
	La red tipo panal no es una red de Bravais, pues no conserva la misma orientación al trasladarse desde todos los puntos de la red, pues tiene simetría espejo (al trasladarla no se superpone idénticamente); pero puede describirse como una red de Bravais con una base doble.
\end{note}

\begin{Figura}
	\centering
	\includegraphics[width=.8\textwidth]{Imagenes/Capitulo1/VectoresPrimitivos4.png}
	\captionof{figure}{Red y base doble para formar un cristal con estructura tipo panal.}
	\label{fig1-4}
\end{Figura}

\end{multicols}
\subsection{Redes cúbica simple, centrada en el cuerpo y centrada en las caras}
\begin{multicols}{2}
La red cúbica simple (SC) es la red más simple que podamos formar, pues consta de un cubo cuyos vértices corresponden a los nodos de la red. Como los nodos se encuentran en los vértices del cubo, cada nodo tendrá seis veccinos más cercanos. Esta red es muy poco común en la naturaleza y solo el polonio está formado por esta red en condiciones normales. 
\begin{Figura}
	\centering
	\includegraphics[width=.8\textwidth]{Imagenes/Capitulo1/SC.png}
	\captionof{figure}{Estructura SC.}
	\label{fig1-5}
\end{Figura}
\end{multicols}
La red cúbica centrada en el cuerpo (BCC) corresponde también a un cubo con nodos en los vértices y nodos en el centro del cubo, por lo que cada nodo tiene 8 vecinos más cercanos. Es una red más compacta que la SC y es mucho más común, pues elementos como el hierro, el cromo, el molibdeno, etc. la poseen. Podremos modelarla como una SC más un nodo en el centro del cubo.
\begin{Figura}
	\centering
	\includegraphics[width=.8\textwidth]{Imagenes/Capitulo1/BCC.png}
	\captionof{figure}{(a) Estructura BCC. (b) Puntos de una red BCC, donde se ilustra que el entorno es el mismo para todos ellos.}
	\label{fig1-6}
\end{Figura}
La red cúbica centrada en las caras (FCC) corresponde también a un cubo con nodos en los vértices y nodos en el centro de las caras del cubo, por lo que cada nodo tendrá 12 vecinos más cercanos. Es todavía más común que las anteriores y se presentan en elementos más comunes como el aluminio, el cobre, el oro y la plata. Podremos modelarla como una SC más nodos en el centro de todas las caras del cubo.
\begin{Figura}
	\centering
	\includegraphics[width=.8\textwidth]{Imagenes/Capitulo1/FCC.png}
	\captionof{figure}{(a) Estructura FCC. (b) Puntos de una red FCC, donde se ilustra que el entorno es el mismo para todos ellos.}
	\label{fig1-7}
\end{Figura}
\begin{note}
	Si tomamos una red periódica bidimensional, cuya celda unidad sea un cuadrado con átomos de cloro en los vértices y átomos de sodio en el centro, podremos tomar una base del cristal tipo SC, tal que $B\equiv\curlybraces{Cl\to(0,0,0),\hspace{2mm}Na\to\left(\frac{1}{2},\frac{1}{2},0\right)}$. 
\end{note}
\newpage
\subsubsection{Número de coordinación}
\begin{multicols}{2}
El número de coordinación será el número de vecinos más cercanos a un nodo o átomo que tomemos de referencia. Dependerá de la estructura de la red y la disposición de los átomos. Así, para la SC el número de coordinación será 6, para la BCC, 8, y para la FCC, 12.
\begin{Figura}
	\centering
	\includegraphics[width=.8\textwidth]{Imagenes/Capitulo1/VecinosCercanos.png}
	\captionof{figure}{Grupos de vecinos de una red SC.}
	\label{fig1-8}
\end{Figura}
\end{multicols}
\subsubsection{Factor de empaquetamiento}
El factor de empaquetamiento (APF) se define como la proporción del volumen de la celda unitaria ocupada por los átomos. Se calcula como:
\begin{equation}
	APF=\frac{\text{Volumen de los átomos de la celda}}{\text{Volumen de la celda}}
\end{equation}
\begin{Figura}
	\centering
	\includegraphics[width=.8\textwidth]{Imagenes/Capitulo1/FactorEmpaquetamiento.png}
	\captionof{figure}{Cálculo del factor de empaquetamiento para la estructura SC.}
	\label{fig1-9}
\end{Figura}
\subsection{Celdas primitiva, convencional y Wigner-Seitz}
\subsubsection{Celda primitiva}
La celda primitiva es una región del espacio que se repite por traslación que contiene toda la información de la red. Corresponderá con la celda unitaria más pequeña posible, conteniendo un único punto de red. No reflejan necesariamente toda la simetría de la red. Su volumen se calcula como,
\begin{equation}
	V=|\vec{a}_1\cdot(\vec{a}_2\times\vec{a}_3)|
\end{equation}
\begin{Figura}
	\centering
	\includegraphics[width=.8\textwidth]{Imagenes/Capitulo1/CeldasPrimitivas.png}
	\captionof{figure}{Diferentes elecciones de celdas primitivas para la misma red de Bravais 2D.}
	\label{fig1-10}
\end{Figura}
Como hemos mencionado antes que la elección de los vectores primitivos, y por tanto, de la celda primitiva, no es unívoca. Entonces todas las celdas primitivas de una misma red de Bravais necesariamente poseerán el mismo volumen. Dependiendo del tipo de red, los vectores primitivos serán unos u otros. Veamos la elección más óptima de vectores primitivos para la SC, BCC y FCC:
\begin{itemize}
	\item Vectores primitivos para la red SC:
		\begin{equation}
			\vec{a}_1=a(1,0,0);\hspace{5mm}\vec{a}_2=a(0,1,0);\hspace{5mm}\vec{a}_3=a(0,0,1)
		\end{equation}
	\item Vectores primitivos para la red BCC:
		\begin{equation}
			\vec{a}_1=\frac{a}{2}\left(1,1,-1\right);\hspace{5mm}\vec{a}_2=\frac{a}{2}(-1,1,1);\hspace{5mm}\vec{a}_3=\frac{a}{2}(1,-1,1)
		\end{equation}
	\item Vectores primitivos para la red FCC:
		\begin{equation}
			\vec{a}_1=\frac{a}{2}(1,1,0);\hspace{5mm}\vec{a}_2=\frac{a}{2}(0,1,1);\hspace{5mm}\vec{a}_3=\frac{a}{2}(1,0,1)
		\end{equation}
\end{itemize}
\begin{multicols}{2}
\begin{Figura}
	\centering
	\includegraphics[width=.8\textwidth]{Imagenes/Capitulo1/VecPrimBCC.png}
	\captionof{figure}{Diagrama de los vectores primitivos para la red BCC.}
	\label{fig1-11}
\end{Figura}
\begin{Figura}
	\centering
	\includegraphics[width=.8\textwidth]{Imagenes/Capitulo1/VecPrimFCC.png}
	\captionof{figure}{Diagrama de los vectores primitivos para la red FCC.}
	\label{fig1-12}
\end{Figura}
\end{multicols}
\begin{Figura}
	\centering
	\includegraphics[width=.8\textwidth]{Imagenes/Capitulo1/VecPrimBCC2.png}
	\captionof{figure}{(a) Tres vectores primitivos para la red de Bravais BCC. Por ejemplo, el punto $P$ se define como $\vec{P}=-\vec{a}_1+2\vec{a}_3$. (b) Selección de vectores primitivos más simétricos. $P$ está descrito por $\vec{P}=\vec{a}_1+2\vec{a}_2+\vec{a}_3$.}
	\label{fig1-13}
\end{Figura}
\begin{note}
	Siempre que definamos un cristal, debemos definir la red primitiva y la base de la red.
\end{note}
\subsubsection{Celda convencional}
La celda convencional es más grande que la primitiva y se elige de modo que refleje mejor la simetría del cristal. Es mucho más intuitiva para representar las redes cúbicas, pues simplemente la red será un cubo, incluyendo los nodos correspondientes a la SC, BCC y FCC, respectivamente.
\begin{Figura}
	\centering
	\includegraphics[width=.8\textwidth]{Imagenes/Capitulo1/CeldaConvencionalSC.png}
	\captionof{figure}{Celda convencional para la red SC (izquierda), BCC (centro) y FCC (derecha). La longitud del lado del cubo, $a$, se denomina \textit{parámetro de red}.}
	\label{fig1-14}
\end{Figura}
\subsubsection{Celda de Wigner-Seitz}
La celda de Wigner-Seitz es una estructura del espacio que contiene todos los puntos más cercanos a un nodo de red dado que a cualquier otro. Se construye trazando planos perpendiculares a los segmentos que unen un punto de red con sus vecinos y siempre corresponde a una celda primitiva.
\begin{Figura}
	\centering
	\includegraphics[width=.8\textwidth]{Imagenes/Capitulo1/CeldaWeignerSeitz.png}
	\captionof{figure}{Celda de Wigner-Seitz para distintas redes de Bravais: (a) 2D, (b) BCC y (c) FCC.}
	\label{fig1-15}
\end{Figura}
Para formar esta celda en el plano bidimensional, trazamos rectas desde un nodo de referencia al resto de nodos de toda la red. En estas rectas tomamos las bisectrices, las extendemos al infinito y la celda de Wigner-Seitz será aquella región del espacio cerrada por el mínimo número de bisectrices.
\subsection{Estructuras especiales y notables}
\subsubsection{Estructura hexagonal compacta (HCP)}
\begin{Figura}
	\centering
	\includegraphics[width=.8\textwidth]{Imagenes/Capitulo1/HCP.png}
	\captionof{figure}{(a) Red de Bravais hexagonal. (b) Estructura HCP}
	\label{fig1-16}
\end{Figura}
La red de Bravais hexagonal consta de capas triangulares apiladas con separación $c$. Mientras que, la estructura HCP consta de dos subredes hexagonales interpenetradas, desplazadas una distancia $c/2$ a lo largo del eje $z$ y lateralmente, de modo que una capa queda sobre los centros de los triángulos de la otra.
\begin{Figura}
	\centering
	\includegraphics[width=.8\textwidth]{Imagenes/Capitulo1/HCP2.png}
	\captionof{figure}{(a) Esferas apiladas tipo ABAB, estructura HCP. (b) Esferas apiladas tipo ABCABC, estructura FCC.}
	\label{fig1-17}
\end{Figura}
En la imagen anterior tenemos una vista superior de las tres primeras capas en un apilamiento de esferas. La primera capa forma una red triangular plana. Las esferas de la segunda se sitúan sobre intersticios alternos de la primera. Ahora tendremos dos opciones para apilar la tercera capa. Si las esferas de la tercera capa se colocan directamente sobre las de la primera, la estructura es HCP (secuencia ABAB...). Si, en cambio, la tercera capa se coloca sobre los intersticios no ocupados de la primera, la estructura es FCC (secuencia ABCABC..., con la diagonal del cuerpo del cubo en dirección vertical). Esta última apilación es la forma más compacta y eficiente de apilar esferas que se conoce.
\subsubsection{Estructura tipo diamante}
\begin{multicols}{2}
\begin{Figura}
	\centering
	\includegraphics[width=.8\textwidth]{Imagenes/Capitulo1/Diamante.png}
	\captionof{figure}{Estructura diamante. Los cuatro vértices más cercanos de cada átomo forman los vértices de un tetraedro regular.}
	\label{fig1-18}
\end{Figura}
La estructura tipo diamante consiste de dos redes FCC interpenetradas, desplazadas una cuarta parte de su longitud, a lo largo de la diagonal del cubo. Vemos que cada átomo tiene cuatro vecinos cercanos, formando un tetraedro. Puede describirse como una red FCC con base una base doble de átomos,
\begin{equation}
	\mathscr{B}=\curlybraces{\left(0,0,0\right),\hspace{2mm}\left(\frac{1}{4},\frac{1}{4},\frac{1}{4}\right)}
\end{equation}
Algunos átomos reales que poseen esta estructura son el Silicio, el Germanio y el Carbono.
\end{multicols}
\subsubsection{Estructura tipo cloruro de sodio}
\begin{multicols}{2}
\begin{Figura}
	\centering
	\includegraphics[width=.8\textwidth]{Imagenes/Capitulo1/NACL.png}
	\captionof{figure}{Estructura tipo NaCl. Los iones A y B se alternan en una red FCC. Cada ion está rodeado por seis iones opuestos.}
	\label{fig1-19}
\end{Figura}
La estructura tipo NaCl consta de cantidades iguales de iones A y B que ocupan puntos alternos de una red FCC. Cada ion tiene seis iones del otro tipo como vecinos más cercanos, formando una coordinación octaédrica. Podemos describirla como una red de Bravais con una base:
\begin{equation}
	\mathscr{B}=\curlybraces{(0,0,0)_A,\left(\frac{1}{2},\frac{1}{2},\frac{1}{2}0\right)_B}
\end{equation}
 Algunas moléculas reales que poseen esta estructura son el NaCl, el KCl, el LiF, el FeO y el MgO.
\end{multicols}
\subsubsection{Estructura tipo cloruro de cesio}
\begin{multicols}{2}
\begin{Figura}
	\centering
	\includegraphics[width=.8\textwidth]{Imagenes/Capitulo1/CECL.png}
	\captionof{figure}{Estructura tipo CsCl. Los iones A y B se disponen en una red SC con una base. Cada ion tiene ocho vecinos del tipo opuesto.}
	\label{fig1-20}
\end{Figura}
La estructura tipo CsCl consta de cantidades iguales de iones A y B que ocupan posiciones alternas en un cubo. Cada ion tiene ocho vecinos del tipo opuesto formando una coordinación cúbica. Podemos describirla como una red cúbica simple con base:
\begin{equation}
\mathscr{B}=\curlybraces{\left(0,0,0\right)_A,\left(\frac{1}{2},\frac{1}{2},\frac{1}{2}\right)_B}
\end{equation}
Algunos ejemplos son el CsCl, el CsBr y el TiCl.
\end{multicols}
\subsubsection{Estructura tipo blenda de zinc}
\begin{multicols}{2}
\begin{Figura}
	\centering
	\includegraphics[width=.8\textwidth]{Imagenes/Capitulo1/ZINC.png}
	\captionof{figure}{Estructura tipo ZnS. Iones A y B organizados en una red tipo diamante, donde cada átomo está rodeado por cuatro del tipo opuesto en coordinación tetraédrica.}
	\label{fig1-21}
\end{Figura}
La red tipo ZnS consta de iones A y B dispuestos en una estructura análoga a la del diamante. Cada ion tiene coordinación tetraédrica con cuatro iones del tipo opuesto. Podremos describirla como una red FCC con una base doble de dos átomos de diferente tipo:
\begin{equation}
	\mathscr{B}=\curlybraces{(0,0,0)_A,\left(\frac{1}{4},\frac{1}{4},\frac{1}{4}\right)_B}
\end{equation}
\end{multicols}
\section{Red recíproca y primera zona de Brillouin}
\subsection{Red recíproca}
Sabemos que una red de Bravais está definida por un conjunto de vectores de posición $\vec{R}$ y sabemos que la forma matemática para captar patrones, es la transformada de Fourier\footnote{La transformada de Fourier descompone cualquier señal en ondas planas, por tanto, si hay patrón, en Fourier aparecen picos en frecuencias muy concretas en el espacio recíproco.}. Si se considera una onda plana, $e^{i\vec{k}\cdot\vec{r}}$ sobre la red, en general no es periódica con la red. Solo para ciertos valores específicos de $\vec{k}$ la onda es compatible con la periodicidad de la red. El conjunto de estos vectores $\vec{K}$, forma una nueva red en el espacio de ondas, denominada \textbf{red recíproca} (o vectores de la red recíproca). Por tanto, una onda plana será periódica respecto a la red de Bravais si y solo si $\vec{K}$ pertenece a la red recíproca, esta condición se matematiza como,
\begin{equation}
	e^{i\vec{K}\cdot(\vec{r}+\vec{R})}=e^{i\vec{K}\cdot\vec{r}}\Leftrightarrow e^{i\vec{K}\cdot\vec{R}}=1\Rightarrow\vec{K}\cdot\vec{R}=2\pi m,\hspace{5mm}m\in\mathbb{Z}
\end{equation}
\begin{note}
La red recíproca también es una red de Bravais.
\end{note}
\subsubsection{Vectores de la red recíproca}
Podemos construir los vectores de la red recíproca usando los vectores de la red directa, $\vec{a}_1$, $\vec{a}_2$ y $\vec{a}_3$, así, la red recíproca está definida por:
\begin{equation}
	\vec{b}_1=2\pi\frac{\vec{a}_2\times\vec{a}_3}{\vec{a}_1\cdot(\vec{a}_2\times\vec{a}_3)};\hspace{5mm}\vec{b}_2=2\pi\frac{\vec{a}_3\times\vec{a}_1}{\vec{a}_1\cdot(\vec{a}_2\times\vec{a}_3)};\hspace{5mm}\vec{b}_3=2\pi\frac{\vec{a_1}\times\vec{a}_2}{\vec{a}_1\cdot(\vec{a}_2\times\vec{a}_3)}
\end{equation}
Vemos que el denominador corresponde con el volumen del tetraedro formado por los vectores primitivos del espacio real, es decir,
\begin{equation}
		\text{Volumen}=\vec{a}_1\cdot(\vec{a}_2\times\vec{a}_3)
\end{equation}
Además, vemos que el numerador tiene dimensión de $[\text{Longitud}]^2$, pues corresponde con una superficie. Por tanto, los vectores primitivos de la red recíproca, $\vec{b}_i$, tendrán dimensión de $[\text{Longitud}]^{-1}$.\\ \\
Sabemos que el espacio real está determinado por los vectores primitivos de la forma,
\begin{equation}
	\vec{R}=n_1\vec{a}_1+n_2\vec{a}_2+n_3\vec{a}_3
\end{equation}
Por analogía, el espacio recíproco estará determinado por los $\vec{b}_i$ de la forma,
\begin{equation}
	\vec{K}=m_1\vec{b}_1+m_2\vec{b}_2+m_3\vec{b}_3
\end{equation}
Luego, suponiendo que $m_i\in\mathbb{Z}$, pues $n_i\in\mathbb{Z}$, podemos aplicar la condición anterior,
\begin{equation}
		\vec{K}\cdot\vec{R}=\left(m_1\vec{b}_1+m_2\vec{b}_2+m_3\vec{b}_3\right)\cdot\left(n_1\vec{a}_1+n_2\vec{a}_2+n_3\vec{a}_3\right)=2\pi m 
\end{equation}
Es evidente que para que esta condición se cumpla, debemos imponer una relación de dualidad,
\begin{equation}
	\vec{b}_i\cdot\vec{a}_j=2\pi\delta_{ij}
\end{equation}
donde $m\equiv m_in_j\in\mathbb{Z}$.
\subsubsection{Propiedades de la red recíproca}
\begin{enumerate}
	\item El volumen de la celda primitiva del espacio recíproco, $V_r$, está relacionado con el de la celda primitiva de la red directa, $V$, mediante:
\begin{equation}
	V_r=\frac{(2\pi)^3}{V}
\end{equation}
\begin{proof}
El volumen de la red recíproca lo podemos definir análogo al del espacio directo, i.e
\begin{equation}
V_r=\vec{b}_1\cdot(\vec{b}_2\times\vec{b}_3)
\end{equation}
Luego,
\[\begin{array}{rl}
\vec{b}_2\times\vec{b}_3&=\left(\frac{2\pi}{V}\right)^2\left(\vec{a}_3\times\vec{a}_1\right)\times\left(\vec{a}_1\times\vec{a}_2\right)=\\ \\
&=\left(\frac{2\pi}{V}\right)^2\curlybraces{\vec{a}_1\left[\left(\vec{a}_3\times\vec{a}_1\right)\cdot\vec{a}_2\right]-\vec{a_2}\left[\left(\vec{a}_3\times\vec{a}_1\right)\cdot\vec{a}_1\right]}=\\ \\
&= \left(\frac{2\pi}{V}\right)^2\curlybraces{\vec{a}_1\left[\vec{a}_3\cdot\left(\vec{a}_1\times\vec{a}_2\right)\right]-\vec{a_2}\left[\vec{a}_3\cdot\cancelto{0}{\left(\vec{a}_1\times\vec{a}_1\right)}\right]}\\ \\
&=\left(\frac{2\pi}{V}\right)^2\vec{a}_1\left[\vec{a}_3\cdot\left(\vec{a}_1\times\vec{a}_2\right)\right]=\left(\frac{2\pi}{V}\right)^2\vec{a}_1V=\frac{(2\pi)^2}{V}\vec{a}_1
\end{array}\]
Pues como el volumen es constante, mientras hagamos los productos de forma cíclica tendremos que
\begin{equation}
	V=\vec{a}_1\cdot(\vec{a}_2\times\vec{a}_3)=\vec{a}_2\cdot(\vec{a}_3\times\vec{a}_1)=\vec{a}_3\cdot(\vec{a}_1\times\vec{a}_2)
\end{equation}
Por tanto,
\[
\vec{b}_1\cdot(\vec{b}_2\times\vec{b_3})=\frac{(2\pi)^3}{V^2}\left(\vec{a}_2\times\vec{a}_3\right)\cdot\vec{a}_1=\frac{(2\pi)^3}{V^2}V=\frac{(2\pi)^3}{V}=V_r
\]
\end{proof}
\item La red recíproca de una red de Bravais también es una red de Bravais.
\item La red recíproca de una red recíproca coincide con la red directa original.
\item Ejemplos:
\[\begin{array}{rcl}
	\text{\textit{Espacio directo}} & & \text{\textit{Espacio recíproco}}\\ \\
	\text{SC de lado }a & \rightarrow & \text{SC de lado }2\pi/a\\ \\
	\text{BCC de lado }a & \rightarrow & \text{FCC de lado }4\pi/a\\ \\
	\text{FCC de lado }a & \rightarrow & \text{BCC de lado }4\pi/a
\end{array}\]
Vamos a desarrollar estos ejemplos:
\begin{example}
Tomamos una red SC de lado $a$, tal que sus vectores primitivos en el espacio directo son,
\begin{equation}
\vec{a}_1=a(1,0,0);\hspace{5mm}\vec{a}_2=a(0,1,0);\hspace{5mm}\vec{a}_3=a(0,0,1)
\end{equation}
Luego, el volumen directo será $V=a^3$. Así, los vectores primitivos del espacio recíproco son,
\begin{equation}
\vec{b}_1=2\pi\frac{\vec{a}_2\times\vec{a}_3}{V}=\frac{2\pi}{a^3}\begin{vmatrix}
\hat{i} & \hat{j} & \hat{k}\\
0 & a & 0\\
0 & 0 & a 
\end{vmatrix}=\frac{2\pi}{a^3}a^2\hat{i}=\frac{2\pi}{a}(1,0,0)
\end{equation}
Análogamente,
\begin{equation}
	\vec{b}_2=\frac{2\pi}{a}(0,1,0);\hspace{5mm}\vec{b}_3=\frac{2\pi}{a}(0,0,1)
\end{equation}
Luego, es claro que la red en el espacio recíproco coincide con una SC de lado $a^*=2\pi/a$.
\end{example}
\begin{example}
Tomamos una red BCC de lado $a$, tal que sus vectores primitivos en el espacio directo son,
\begin{equation}
\vec{a}_1=\frac{a}{2}(1,1,-1);\hspace{5mm}\vec{a}_2=\frac{a}{2}(-1,1,1);\hspace{5mm}\vec{a}_3=\frac{a}{2}(1,-1,1)
\end{equation}
Luego, el volumen directo será $V=a^3/2$.  Así, los vectores primitivos del espacio recíproco son,
\begin{equation}
\vec{b}_1=2\pi\frac{\vec{a}_2\times\vec{a}_3}{V}=\frac{4\pi}{a^3}\frac{a^2}{4}\begin{vmatrix}
\hat{i} & \hat{j} & \hat{k}\\
-1 & 1 & 1\\
1 & -1 & 1
\end{vmatrix}=\frac{\pi}{a}(2,2,0)=\frac{2\pi}{a}(1,1,0)
\end{equation}
Análogamente,
\begin{equation}
\vec{b}_2=\frac{2\pi}{a}(0,1,1);\hspace{5mm}\vec{b}_3=\frac{2\pi}{a}(1,0,1)
\end{equation}
Vemos que estos vectores corresponden con una FCC en el espacio recíproco de lado $a^*=4\pi/a$. Debemos tener en cuenta que tanto para la FCC, como para la BCC, la constante que acompaña a los vectores primitivos es $a^*/2$, por tanto, el lado de la FCC es la constante que acompaña los vectores primitivos multiplicada por 2.
\end{example}
\begin{example}
Tomamos una red FCC de lado $a$, tal que sus vectores primitivos en el espacio directo son,
\begin{equation}
\vec{a}_1=\frac{a}{2}(1,1,0);\hspace{5mm}\vec{a}_2=\frac{a}{2}(0,1,1);\hspace{5mm}\vec{a}_3=\frac{a}{2}(1,0,1)
\end{equation}
Luego, el volumen directo será $a^3/4$. Así, los vectores primitivos del espacio recíproco son,
\begin{equation}
\vec{b}_1=2\pi\frac{\vec{a}_2\times\vec{a}_3}{V}=\frac{8\pi}{a^3}\frac{a^2}{4}\begin{vmatrix}
\hat{i} & \hat{j} & \hat{k}\\
0 & 1 & 1\\
1 & 0 & 1
\end{vmatrix}=\frac{2\pi}{a}(1,1,-1)
\end{equation}
Análogamente,
\begin{equation}
	\vec{b}_2=\frac{2\pi}{a}(-1,1,1);\hspace{5mm}\vec{b}_3=\frac{2\pi}{a}(1,-1,1)
\end{equation}
Siguiendo el ejemplo anterior, vemos que en el espacio recíproco obtenemos una BCC de lado $a^*=4\pi/a$.
\end{example}
\end{enumerate}
La red recíproca es tan importante porque permite describir las estructuras cristalinas en el espacio de momentos, siendo la herramienta fundamental para analizar fenómenos periódicos a escala atómica. Siendo fundamental para conectar medidas experimentales con la geometría de la red cristalina.
\subsubsection{Primera zona de Brillouin}
La primera zona de Brillouin corresponde a la celda de Weigner-Seitz en el espacio recíproco. Contiene todos los puntos del espacio recíproco que están más cerca del origen que de cualquier otro punto de la red recíproca. Esta zona representa la región fundamental para describir fenómenos periódicos en el espacio de momentos; permite determinar la estructura de bandas electrónicas, define las condiciones de difracción de Bragg y facilita el análisis de superficies de Fermi. Por simetría, bastará analizar las propiedades físicas dentro de esta zona para obtener todas las características del cristal.
\section{Planos cristalinos e índices de Miller}
\subsection{Planos cristalinos}
Los planos cristalinos se definen como un conjunto de átomos ubicados en planos paralelos dentro del cristal y permiten describir la simetría interna, ayudando a clasificar direcciones preferentes. Por tanto, su orientación influye en propiedades como la resistencia mecánica y modos de deslizamiento; los fenómenos de fractura y propagación de grietas; y la intensidad y el ángulo en la difracción de rayos X. Estos planos se caracterizan por la distancia interplanar y su orientación relativa a la celda unitaria.\\ \\
Para identificar estos planos de manera sistemática se emplean los \textit{índices de Miller}, un sistema de notación que utiliza las intersecciones de los planos con los ejes cristalográficos y que veremos a continuación.
\subsection{Índices de Miller}
El procedimiento para obtener los índices de Miller de un plano concreto de la red es el siguiente:
\begin{enumerate}
	\item Determinar las intersecciones del plano tomado con los ejes cristalográficos $x$, $y$, $z$ en términos de los vectores primitivos.
	\item Tomar los inversos de las intersecciones.
	\item Multiplicar por un factor común para obtener enteros sin fracciones.
\end{enumerate}
Veamos un ejemplo sencillo.
	\begin{Figura}
	\centering
	\includegraphics[width=.5\textwidth]{Imagenes/Capitulo1/Miller.png}
	\captionof{figure}{Ejemplo de planos cristalinos en una SC. El plano $(1  0  0)$ corta el eje $x$ en 1 y el paralelo a los ejes $y$ y $z$.}
	\label{fig1-22}
\end{Figura}
Tomaremos como ejemplo el plano paralelo a $YZ$ que corta con el eje $x$ en una SC. Primero vemos que los vectores unitarios de la SC encerrados en este plano son solo el vector $\vec{a}_1=(1,0,0)$, y los demás vectores no cortan este plano, por lo que los tomaremos como infinitos. Obteniendo $(1 \infty \infty )$. Ahora tomamos los inversos del plano obtenido, tal que $(1~ 1/\infty~  1/\infty)=(1  0  0) $. Obteniendo el plano $(1  0  0)$, que al tener números enteros no deberemos multiplicar por ningún factor de escala. 

\begin{tcolorbox}[title=Nomenclatura]
\begin{itemize}
	\item $\mathbf{(hkl):}$ representa un plano específico.
	\item $\mathbf{\left\lbrace h k l\right\rbrace:}$ representa una familia de planos equivalentes por simetría.
	\item $\mathbf{[u  v  w]:}$ representa una dirección cristalográfica.
\end{itemize}
\end{tcolorbox}
\subsection{Familias de planos cristalinos}
En celdas de alta simetría (como la SC), los planos con propiedades equivalentes por simetría se agrupan en familias. Por ejemplo, la familia de planos $\left\lbrace 100\right\rbrace$ incluye los planos $(100)$, $(010)$, $(001)$, $(\bar{1}00)$, $(0\bar{1}0)$ y $(00\bar{1})$, los vemos en la siguiente figura.
	\begin{Figura}
	\centering
	\includegraphics[width=.6\textwidth]{Imagenes/Capitulo1/Miller2.png}
	\captionof{figure}{Familia $\left\lbrace 100\right\rbrace$ en una red cúbica simple.}
	\label{fig1-23}
\end{Figura}
\begin{note}
	Los planos que sean miembros de una misma familia tienen igual espaciado interplanar y propiedades físicas idénticas.
\end{note}
\begin{remark}
	La densidad superficial de nudos se define como el número de nudos por unidad de área dentro del plano que estemos tratando, tal que
	\begin{equation}
		\rho_{(hkl)}=\frac{n}{s}
	\end{equation}
	donde $n$ es el número de nudos dentro del plano $(hkl)$ y $s$ es la superficie de este plano.\\ \\
Es importante remarcar que los nudos que se consideran dentro del plano pueden ser fraccionarios, debido a que los nudos de las esquinas tienen $1/4$ dentro del plano y en la $n$ se vé también reflejado. Normalmente al tener 4 esquinas con un nudo en cada una, tendremos que $n\propto4\cdot1/4=1$, pero habrá otras ocasiones, en redes más complejas, donde el número no sea entero.
\end{remark}
\subsection{Espaciado interplanar y red recíproca}
En la red recíproca, cualquier vector se escribe como
\begin{equation}
\vec{K}=m_1\vec{b}_1+m_2\vec{b}_2+m_3\vec{b}_3
\end{equation}
A cada plano $(hkl)$ se le asocia un vector recíproco,
\begin{equation}
\vec{G}_{hkl}=h\vec{b}_1+k\vec{b}_2+l\vec{b}_3
\end{equation}
que contiene la información de la orientación del plano y del espaciado interplanar. Estos vectores son perpendiculares a los plano $(hkl)$. \\ \\
El espaciado interplanar $d_{hkl}$ es la distancia entre planos consecutivos de la familia $\left\lbrace hkl\right\rbrace$ y se relaciona con el módulo del vector recíproco mediante,
\begin{equation}
d_{hkl}=\frac{2\pi}{|\vec{G}_{hkl}|}
\end{equation}
\begin{theorem}
	Para cualquier familia de planos separados por una distancia $d$, existe un vector de la red recíproca perpendicular a dichos planos, cuya magnitud mínima es $2\pi/d$. Recíprocamente, para cada vector de la red recíproca $\vec{K}$ existe una familia de planos normal a él, separados por $d=2\pi/|\vec{K}|$.
\end{theorem}
\subsubsection{Espaciado interplanar en redes cúbicas}
En una red cúbica simple, los vectores de la base de la celda convencional son ortogonales, de módulo $a$ (suponiendo que el lado del cubo es $a$),  tal que
\begin{equation}
\vec{a}_1=a\hat{x},\hspace{5mm}\vec{a}_2=a\hat{y},\hspace{5mm}\vec{a}_3=a\hat{z}
\end{equation}
Usando la definición de red recíproca, se obtienen los vectores recíprocos,
\begin{equation}
\vec{b}_1=\frac{2\pi}{a}\hat{x},\hspace{5mm}\vec{b}_2=\frac{2\pi}{a}\hat{y},\hspace{5mm}\vec{b}_3=\frac{2\pi}{a}\hat{z}
\end{equation}
La combinación lineal con los índices de Miller resulta,
\begin{equation}
\vec{G}_{hkl}=h\vec{b}_1+k\vec{b}_2+l\vec{b}_3=\frac{2\pi}{a}(h\hat{x}+k\hat{y}+l\hat{z})
\end{equation}
cuyo módulo es,
\begin{equation}
|\vec{G}_{hkl}|=\frac{2\pi}{a}\sqrt{h^2+k^2+l^2}
\end{equation}
recordando la relación $d_{hkl}=2\pi/|\vec{G}_{hkl}|$, resulta
\begin{equation}
d_{hkl}=\frac{a}{\sqrt{h^2+k^2+l^2}}\equiv\frac{\text{Parámetro de red}}{\left(\text{suma cuadrada de los índices de Miller}\right)^{1/2}}
\end{equation}
La celda convencional de todas las redes cúbicas (SC, BCC, FCC) es un cubo de lado $a$. Geométricamente, la distancia entre planos $(hkl)$ en ese cubo es siempre el resultado anterior.\\ \\
Lo que distingue a SC, BCC y FCC no es la fórmula de $d_{hkl}$, sino las \textbf{condiciones de reflexión} que impone el \textbf{factor de estructura}.
\section{Conclusiones}
\begin{itemize}
	\item La \textbf{estructura cristalina} se describe mediante la combinación de \textbf{Red de Bravais} y \textbf{base}, determinando la simetría y propiedades del material.
	\item Las distintas celdas (\textbf{primitiva}, \textbf{convencional}, \textbf{Weigner-Seitz}) permiten representar y analizar la red con distintos niveles de simetría y conveniencia.
	\item La \textbf{red recíproca} y la \textbf{primera zona de Brillouin} son herramientas fundamentales para describir fenómenos en el \textbf{espacio de momentos}, como la \textbf{difracción} y la \textbf{estructura electrónica}.
	\item Los \textbf{índices de Miller} proporcionan un sistema para identificar y clasificar \textbf{planos cristalinos}, conectando la geometría cristalina con propiedades \textbf{mecánicas} y \textbf{ópticas}.
	\item Comprender estas relaciones es clave para \textbf{interpretar experimentos}, \textbf{diseñar materiales} y \textbf{predecir su comportamiento}.
\end{itemize}

\lhead{\emph{Capítulo 2: Difracción de Rayos X}}
\chapter{Difracción de Rayos X}

\section{Formulación de Bragg y von Laue}
\subsection{Formulación de Bragg}
En la formulación de Bragg, los cristales se modelan como familias de planos separados por una distancia $d$. Cada plano tiene reflexión especular, teniendo interferencia constructiva entre planos sucesivos.
\begin{Figura}
	\centering
	\includegraphics[width=.6\textwidth]{Imagenes/Capitulo2/Bragg}
	\captionof{figure}{Reflexión de Bragg para planos separados por $d$.}
	\label{fig1-24}
\end{Figura}
Usando este modelo, obtenemos la conocida \textbf{condición de Bragg}, que viene dada por
\begin{equation}
m\lambda=2d\sin\theta
\end{equation}
donde no usamos índices de refracción, porque al usar rayos X como haz solo ven vacío, al tener longitudes de onda tan energéticas. Además, para un haz no monocromático, aparecerán múltiples órdenes $n$ y múltiples familias de planos $(hkl)$.\\ \\
Al tener varias familias de planos, estas producen reflexiones distintas, porque aunque la dirección del rayo incidente sea la misma, la dirección y longitud de onda del rayo reflejado serán diferentes, debido a la condición de Bragg con $d$ reemplazada por $d'$.
\begin{Figura}
	\centering
	\includegraphics[width=.6\textwidth]{Imagenes/Capitulo2/Bragg2}
	\captionof{figure}{Distintas familias de planos producen reflexiones distintas.}
	\label{fig1-25}
\end{Figura}
En esta figura podemos observar que el haz azul solo detecta los planos horizontales, mientras que el haz naranja solo detecta los planos oblicuos.
\subsection{Formulación de von Laue}
En la formulación de von Laue, modelamos el cristal como una red de Bravais con dispersores en posiciones $\vec{R}$, de tal forma que cada dispersor pueda dispersar radiación en todas direcciones. Los picos de difracción ocurrirán cuando las ondas dispersadas interfieran constructivamente.\\ \\
Luego, para dos dispersores separados por $\vec{d}$, la diferencia de camino óptico viene dada por
\begin{equation}
	d\cos\theta+d\cos\theta'=\vec{d}\cdot(\hat{n}-\hat{n}')=m\lambda\Longrightarrow\vec{d}\cdot(\vec{k}-\vec{k}')=2\pi m
\end{equation}
\begin{Figura}
	\centering
	\includegraphics[width=.6\textwidth]{Imagenes/Capitulo2/Laue}
	\captionof{figure}{Diferencia de camino óptico para los rayos dispersados desde dos puntos separados por $\vec{d}$.}
	\label{fig1-26}
\end{Figura}
Para todos los puntos de la red de Bravais, $\vec{R}$, tendremos que
\begin{equation}
	\vec{R}\cdot(\vec{k}-\vec{k}')=2\pi m\Longrightarrow e^{i(\vec{k}-\vec{k}')\cdot\vec{R}}=1
\end{equation}
Por tanto, se producirá interferencia constructiva únicamente si $\vec{K}=\vec{k}-\vec{k}'$ pertenece a la red recíproca.
\subsubsection{Interpretación geométrica}
\begin{Figura}
	\centering
	\includegraphics[width=.6\textwidth]{Imagenes/Capitulo2/Geom}
	\captionof{figure}{Condición de Laue.}
	\label{fig1-27}
\end{Figura}
La condición de Laue nos dice que si $\vec{k}-\vec{k}'$ corresponde a $\vec{K}$ y ambos tienen la misma magnitud, la punta de $\vec{k}$ está equidistante del origen $O$ y de la punta de $\vec{K}$. Por lo tanto, se encuentra en el plano que bisecta perpendicularmente la línea que une el origen con $\vec{K}$.\\ \\
Luego, sabiendo que si $\vec{k}'-\vec{k}$ pertenece a la red recíproca, entonces $\vec{K}$ también, la condición elástica $\left(|\vec{k}|=|\vec{k}'|\right)$ se reescribe como,
\begin{equation}
	|\vec{k}'|=|\vec{k}-\vec{K}|
\end{equation}
elevando al cuadrado y simplificando, obtenemos
\begin{equation}
	|\vec{k}'|^2=|\vec{k}|^2=|\vec{k}-\vec{K}|^2=|\vec{k}|^2+|\vec{K}|^2-2\vec{k}\cdot\vec{K}\Rightarrow\vec{k}\cdot\vec{K}=\frac{|\vec{K}|^2}{2}\Rightarrow\vec{k}\cdot\hat{u}_k=\frac{|\vec{K}|^2}{2}
\end{equation}
Acabamos de demostrar la consecuencia directa de la condición de Laue que nos implicaba que la punta de $\vec{k}$ está en un plano que bisecta perpendicularmente el segmento que une el origen con $\vec{K}$. Estos son los llamados planos de Bragg en el espacio recíproco.
\subsubsection{Ecuaciones de von Laue}
Las ecuaciones de von Laue establecen la condición para la interferencia constructiva en la difracción de rayos X, que es
\begin{equation}
		\vec{R}\cdot\vec{K}=2\pi m
\end{equation}
donde $\vec{R}$ es un vector de la red directa, $\vec{K}=\vec{k}'-\vec{k}$ es un vector de la red recíproca y $m\in\mathbb{Z}$.\\ \\
Para difracción elástica, donde 
\begin{equation}
	|\vec{k}'|=|\vec{k}|=\frac{2\pi}{\lambda}
\end{equation}
donde hemos usado la condición de red recíproca.  Proyectando sobre $\vec{R}$, tenemos que
\begin{equation}
	|\vec{R}|\left(\cos\phi'-\cos\phi\right)=\lambda m
\end{equation}
donde $\phi$ y $\phi'$ son los ángulos entre $\vec{R}$ y $\vec{k}$, $\vec{k}'$ repectivamente.\\ \\
Sabiendo que $[uvw]$ son las direcciones cristalográficas, el vector de la red directa $\vec{R}$ lo podemos escribir como,
\begin{equation}
	\vec{R}=u\vec{a}_1+v\vec{a}_2+w\vec{a}_3
\end{equation}
tomamos como vector recíproco a
\begin{equation}
	\vec{K}\equiv\vec{G}_{hkl}=h\vec{b}_1+k\vec{b}_2+l\vec{b}_3
\end{equation}
Luego, si tomamos la dirección $\hat{a}_1$, tendremos la proyección que $\vec{R}=\vec{a}_1$, luego $(u,v,w)=(1,0,0)$. Por tanto, la condición queda
\begin{equation}
	\vec{a}_1\cdot\vec{K}=2\pi m_1=2\pi h 
\end{equation}
Sabiendo que estamos en la difracción elástica, es decir, $|\vec{k}'|=|\vec{k}|=2\pi/\lambda$. Entonces,
\begin{equation}
	\vec{a}_1\cdot(\vec{k}'-\vec{k})=a_1\frac{2\pi}{\lambda}\cos\alpha'-a_1\frac{2\pi}{\lambda}\cos\alpha=a_1\frac{2\pi}{\lambda}(\cos\alpha'-\cos\alpha)=2\pi h\Rightarrow a_1(\cos\alpha'-\cos\alpha)=\lambda h
\end{equation}
donde $\alpha'$ y $\alpha$ son los ángulos entre $\vec{a}_1$ y $\vec{k}'$ y $\vec{k}$ respectivamente.\\ \\
Análogamente para las direcciones $\hat{a}_2$ y $\hat{a}_3$, donde $(u,v,w)=(0,1,0)$ y $(u,v,w)=(0,0,1)$, respectivamente, tenemos
\begin{equation}
	\begin{array}{rcl}
		\hat{a}_1 & \to & a_1(\cos\alpha'-\cos\alpha)=\lambda h\\
		\hat{a}_2 & \to & a_2(\cos\beta'-\cos\beta)=\lambda k\\
		\hat{a}_3 & \to & a_3(\cos\gamma'-\cos\gamma)=\lambda l
	\end{array}
\end{equation}
donde $h$, $k$ y $l$ son los índices de Miller asociados al plano responsable de la reflexión. Esto se debe a que el vector $\vec{G}$ es un caso partircular del vector $\vec{K}$ que nos da la información sobre los planos cristalográficos asociados a los índices de Miller.
\subsubsection{Equivalencia entre Bragg y von Laue}
Recordamos que la condición de von Laue establece que, con difracción elástica $(|\vec{k}|=|\vec{k}'|)$, establece que
\begin{equation}
\vec{K}=\vec{k}'-\vec{k}
\end{equation}
es un vector de la red recíproca. Esto implica que $\vec{k}$ y $\vec{k}'$ forman el mismo ángulo $\theta$ con el plano perpendicular a $\vec{K}$, debido a que $\vec{K}$ bisecta el ángulo entre $\vec{k}$ y $\vec{k}'$.
\begin{Figura}
	\centering
	\includegraphics[width=.6\textwidth]{Imagenes/Capitulo2/Equiv}
	\captionof{figure}{Plano que contiene $\vec{k}$, $\vec{k}'$ y $\vec{K}$ cumpliendo von Laue. La línea discontinua es el plano perpendicular a $\vec{K}$.}
	\label{fig1-28}
\end{Figura}
En el espacio directo, este plano corresponde a una familia de planos cristalinos $(hkl)$ responsable de la reflexión. Así, la dispersión descrita por von Laue puede interpretarse como una reflexión de Bragg con ángulo de Bragg $\theta$.
\begin{dem}
Vamos a demostrar la equivalencia entre la formulación de Bragg y von Laue. Para ello, vamos a partir de la condición de von Laue para difracción elástica y vamos a obtener la condición de Bragg.\\ \\
Tomando planos cristalográficos con índices de Miller $(hkl)$, el vector recíproco más corto perpendicular a estos planos es
\[\vec{K}_0=h\vec{b}_1+k\vec{b}_2+l\vec{b}_3\]
que corresponde con el vector $\vec{G}_{hkl}$ visto en el tema anterior. Sabemos que
\[|\vec{G}_{hkl}|=\frac{2\pi}{d}\equiv|\vec{K}_0|\]
donde $d$ es la separación entre los planos perpendiculares a $\vec{K}_0$. \\ \\
Al ser este vector el más corto, tendremos que
\[\vec{K}=n\vec{K}_0\]
Por otro lado, como tenemos difracción elástica, tendremos que
\[|\vec{k}'|=|\vec{k}|=\frac{2\pi}{\lambda}\]
Luego, el módulo de $\vec{K}$ será
\[|\vec{K}|^2=|\vec{k}'-\vec{k}|^2=|\vec{k}'|^2+|\vec{k}|^2-2\vec{k}'\cdot\vec{k}=2|\vec{k}|^2-2|\vec{k}|^2\cos(2\theta)=2|\vec{k}|^2(1-\cos(2\theta))\]
Usando las propiedades del ángulo doble tenemos que,
\[1-\cos(2\theta)=2\sin^2\theta\]
Así queda,
\[|\vec{K}|=2k^22\sin^2\theta=4k^2\sin^2\theta\Rightarrow|\vec{K}|=2k\sin\theta\overset{\left(k=2\pi/\lambda\right)}{=}\frac{4\pi}{\lambda}\sin\theta\]
Por otro lado, sabemos que 
\[|\vec{K}|=n|\vec{K}_0|=n\frac{2\pi}{d}\]
Por tanto,
\[n\frac{2\pi}{d}=\frac{4\pi}{\lambda}\sin\theta\Rightarrow n\lambda=2d\sin\theta\]
que es exactamente la condición de Bragg. \qedh
\end{dem}

Como tenemos finitos planos de Bragg, la probabilidad de colisionar con uno de ellos es prácticamente nula, por lo que tendremos que diseñar 		 experimentos para que esta probabilidad aumente. Estas son las denominadas geometrías experimentales que trataremos ahora.

\section{Geometrías experimentales}

Sabemos que un pico de difracción ocurre solo si la punta del vector $\vec{k}$ coincide con un plano de Bragg en el espacio-$k$. Estos planos son discretos y no cubren todo el espacio recíproco, por tanto, con $\lambda$ y dirección fijas, será poco probable que $\vec{k}$ cumpla la condición de Bragg. Para solucionar este problema, necesitamos aumentar la probabilidad de encontrar difracción, para ello podemos hacer:
\begin{itemize}
    \item \textbf{Variar el módulo }de $\vec{k}$, es decir, cambiar la longitud de onda $\lambda$ del haz incidente.
    \item \textbf{Cambiar la dirección }de $\vec{k}$, es decir, rotar el cristal respecto al haz.
\end{itemize}
\subsection{Esfera de Ewald}
La esfera de Ewald es una construcción geométrica en el espacio recíproco que representa las posibles direcciones de difracción para un haz incidente dado. La esfera es de radio $|\vec{k}|=2\pi/\lambda$, cuyo centro es la punta del vector de onda incidente $\vec{k}$, por lo que tomamos el origen en un nodo fijo de la superficie de la esfera. Por tanto, un punto de la red recíproca que esté sobre la esfera corresponde a una dirección de difracción que cumple la condición de Bragg.
\begin{Figura}
    \centering
    \includegraphics[width=0.8\textwidth]{Imagenes/Capitulo2/Ewald.png}
    \captionof{figure}{Esfera de Ewald: condiciones geométricas para la difracción en el espacio recíproco.}
    \label{ewald}
\end{Figura}
Necesitamos que $\vec{k}'-\vec{k}=\vec{K}$ sea un vector de la red recíproca, es decir, un vector de traslación de nodos. Por tanto, necesitamos que por lo menos un punto coincida con la superficie de la esfera.

\subsection{Método de von Laue (haz monocromático)}

\begin{Figura}
    \centering
    \includegraphics[width=0.8\textwidth]{Imagenes/Capitulo2/LauePoli.png}
    \captionof{figure}{Método de von Laue: región accesible de la red recíproca entre las esferas de Ewald para $\lambda_1$ y $\lambda_2$, donde se producen picos de Bragg.}
    \label{laue}
\end{Figura}

El método de von Laue consiste en dejar fijo el cristal, tomando dos esferas de Ewuald con un mismo nodo de origen para un $\vec{k}_1$ y un $\vec{k}_2$, por lo que si tomamos una superposición de esferas de Ewald con $\vec{k}\in[\vec{k}_1,\vec{k}_2]$, tendremos un rango de longitudes de onda $\lambda\in[\lambda_1,\lambda_2]$. Por tanto, la región accesible, es decir, la región donde tengamos puntos, son picos de Bragg, será el volumen comprendido entre ambas esferas de Ewald, que está encerrado entre los radios $\frac{2\pi}{\lambda_2}$ y $\frac{2\pi}{\lambda_1}$.\\ \\
Este método resulta útil para orientar monocristales con estructura conocida, pues orientamos el cristal de tal forma que un nodo coincida con el origen de las esferas de Ewald. Este método permite registrar múltiples reflexiones en una sola exposición. Pero está limitado a cristales de buena calidad y a rangos amplios de $\lambda$.\\ \\
En el mundo real, los cristales no tienen todos la misma orientación, sino que están construidos por 'ladrillos' del cristal con distintas orientaciones.
\subsection{Método de rotación del cristal (haz monocromático)}
\begin{Figura}
    \centering
    \includegraphics[width=0.8\textwidth]{Imagenes/Capitulo2/LauePoli2.png}
    \captionof{figure}{Método de rotación: al girar el cristal, los puntos de la red recíproca trazan círculos que producen reflexiones de Bragg al intersectar la esfera de Ewald.}
    \label{laue2}
\end{Figura}
Tomamos la $\lambda$ fija y rotamos el cristal, por lo que la red recíproca describe \textit{círculos}. Se denominan Bragg cuando estos círculos cortan la esfera de Ewald. Es útil para determinar parámetros de red y orientar monocristales, permitiendo explorar un gran número de reflexiones con una sola longitud de onda.

\subsection{Método de polvo (Debye-Scherrer)}
\begin{Figura}
    \centering
    \includegraphics[width=0.8\textwidth]{Imagenes/Capitulo2/LauePoli3.png}
    \captionof{figure}{(a) Esfera de Ewald (radio $|\vec{k}|$) centrada en la punta de $\vec{k}$ y esfera de radio $|\vec{K}|$ centrada en $O$. Su intersección es un círculo que genera un cono de rayos dispersados opuesto a $\vec{k}$. (b) Corte plano que contiene $\vec{k}$, donde el triángulo isósceles resultante cumple $|\vec{K}|=2|\vec{k}|\sin(\phi/2)$, siendo $\phi$ el ángulo de dispersión.}
\end{Figura}
La esfera centrada en $O$ representa todo el cristal, mientras que la esfera centrada en la punta de $\vec{k}$ representa una esfera de Ewald que vamos rotando. En la intersección de ambas esferas siempre habrá difracción si tenemos un nodo dentro.\\ \\
Tenemos orientaciones cristalinas aleatorias (policristal o polvo). Cada $\vec{K}$ genera una esfera centrada en $O$. La intersección con la esfera de Ewald son anillos/conos de difracción. La geometría es $|\vec{K}|=2|\vec{k}|\sin(\phi/2)$. \\ \\
Este método nos permite identificar fases cristalinas y calcular parámetros de red. La ventaja es que no requiere orientar ni rotar la muestra.

\section{Factor de estructura geométrico}

Cuando una celda unidad contiene más de un átomo idéntico (base), la intensidad de los picos de difracción depende de su disposición interna. La amplitud total de difracción será proporcional a la suma de las contribuciones de cada átomo de la celda:
\begin{equation}
    A(\vec{K})\propto\sum_{j=1}^Ne^{i\vec{K}\cdot\vec{r}_j}
\end{equation}
con $\vec{r}_j$ ($j=1,2,\dots,N$) las posiciones relativas de los $N$ átomos de la base y $\vec{K}=\vec{k}'-\vec{k}$ un vector de la red recíproca.\\ \\
Se define el \textbf{factor de estructura geométrico}:
\begin{equation}
    S(\vec{K})=\sum_{j=1}^Ne^{i\vec{K}\cdot\vec{r}_j}
\end{equation}
La intensidad en un pico de Bragg será por tanto proporcional a $|S(\vec{K}|^2$.
\subsection{Interpretación física}
$S(\vec{K})$ describe la interferencia constructiva o destructiva de las ondas dispersadas por los átomos de la base. Si $S(\vec{K})=0$ para cierto vector del espacio recíproco $\vec{K}$, no habrá difracción en esa dirección, lo que se conoce como \textit{extinción sistemática}.\\ \\
El factor de estructura es solo una de las contribuciones a la dependencia de la intensidad con $\vec{K}$, junto con factores como el factor de forma atómico o las dependencias geométricas y angulares. No obstante, cuando $S(\vec{K})=0$, la cancelación es total y no se observa difracción, independientemente de las demás contribuciones.\\ \\
Para las redes con cierta simetría, como las BBC o FCC, presentan extinciones sistemáticas que se deben a la interferencia destructiva entre los átomos de la base.
\subsubsection*{Ejemplo: Red Cúbica Simple (SC)}
\begin{multicols}{2}
\begin{Figura}
    \centering
    \includegraphics[width=\textwidth]{Imagenes/Capitulo2/Sc.png}
    \label{cap2-SC}
    \captionof{figure}{Red Cúbica Simple.}
\end{Figura}
Una SC de parámetro de red $a$ se describe con una base de un único átomo, situado en el origen $\vec{r}_1=(0,0,0)$. Al ser una SC, el vector recíproco es
\begin{equation}
    \vec{K}=\frac{2\pi}{a}(h\hat{u}_x+k\hat{u}_y+l\hat{u}_z)
\end{equation}
con $h$, $h$ y $l$ los índices de Miller. El factor de estructura será,
\begin{equation}
    S(\vec{K})=\sum_je^{i\vec{K}\cdot\vec{r}_j}=e^{i\vec{K}\cdot\vec{r}_1}=1
\end{equation}
    Por tanto, en una SC \textit{no hay extinciones sistemáticas}, pues aparecen todas las reflexiones $(hkl)$.
\end{multicols}

\subsubsection*{Ejemplo: Red Cúbica Centrada en el Cuerpo (BCC)}
\begin{Figura}
\centering
\includegraphics[width=0.4\textwidth]{Imagenes/Capitulo2/BCC.png}
\label{cap2-BCC}
\captionof{figure}{BCC.}
\end{Figura}
Una BCC de parámetro de red $a$, se puede entender como una SC con base doble, con dos orígenes, $\vec{r}_1=(0,0,0)$ y $\vec{r}_2=a\left(\frac{1}{2},\frac{1}{2},\frac{1}{2}\right)$. Al tratarse de una SC, un vector recíproco será $\vec{K}=\frac{2\pi}{a}(h\hat{u}_x+k\hat{u}_y+l\hat{u}_z)$ con $h$, $k$ y $l$ los índices de Miller.\\ \\
Por tanto, el factor de estructura será,
\begin{equation}
\begin{array}{rl}
S(\vec{K})&=\sum\limits_{j=1}^Ne^{i\vec{K}\cdot\vec{r}_j}=e^{i\vec{K}\cdot\vec{r}_1}+e^{i\vec{K}\cdot\vec{r}_2}=1+e^{i\pi(h+k+l)}\\ \\
&=1+\cos(\pi(h+k+l))+i\sin(\pi(h+k+l))=\\ \\
&=1+(-1)^{h+k+l}=\left\lbrace\begin{array}{ll}
2 & \text{si }h+k+l=2m\\
0 & \text{si }h+k+l=2m+1 
\end{array}\right.\hspace{0.5cm}m\in\mathbb{Z}
\end{array}
\end{equation}
Por tanto, en una BCC solo aparecen reflexiones con $h,k,l$ cuando $h+k+l$ es par.
\subsubsection*{Ejemplo: Red Cúbica Centrada en las Caras (FCC)}
\begin{Figura}
    \centering
    \includegraphics[width=0.5\textwidth]{Imagenes/Capitulo2/FCC.png}
    \captionof{figure}{FCC.}
\end{Figura}
Una FCC de parámetro de red $a$ se puede considerar como una SC con una base de cuatro átomos,
\begin{equation}
    \vec{r}_1=(0,0,0);\hspace{2mm}\vec{r}_2=a\left(0,\frac{1}{2},\frac{1}{2}\right);\hspace{2mm}\vec{r}_3=a\left(\frac{1}{2},0,\frac{1}{2}\right);\hspace{2mm}\vec{r}_4=a\left(\frac{1}{2},\frac{1}{2},0\right)
\end{equation}
Al ser una SC, un vector recíproco es
\begin{equation}
    \vec{K}=\frac{2\pi}{a}(h\hat{u}_x+k\hat{u}_y+l\hat{u}_z)
\end{equation}
con $h$, $k$ y $l$ los índices de Miller. Por tanto, el factor de estructura será
\begin{equation}
    \begin{array}{rl}
        S(\vec{K}) &=\sum\limits_{j=1}^Ne^{i\vec{K}\cdot\vec{r}_j}=e^{i\vec{K}\cdot\vec{r}_1}+e^{i\vec{K}\cdot\vec{r}_2}+e^{i\vec{K}\cdot\vec{r}_3}+e^{i\vec{K}\cdot\vec{r}_4}=  \\
         & =1+e^{i\pi(k+l)}+e^{i\pi(h+l)}+e^{i\pi(h+k)}=\\
         &=1+(-1)^{k+l}+(-1)^{h+l}+(-1)^{h+k}
         \end{array}
\end{equation}
donde
\begin{Figura}
    \centering
    \includegraphics[width=0.8\textwidth]{Imagenes/Capitulo2/Tabla1.png}
    \captionof{table}{Valores de $S(\vec{K})$ para diferentes combinaciones de índices de Miller en una red FCC.}
\end{Figura}
Por tanto,
\begin{equation}
    S(\vec{k})=\left\lbrace\begin{array}{ll}
        4 & \text{si }h,k,l\text{ todos pares o todos impares} \\
        0 & \text{en cualquier otro caso}
    \end{array}\right.
\end{equation}
Por tanto, en una FCC sólo aparecen reflexiones con $h,k,l$ todos pares o todos impares.
\\ \\
Como resumen tenemos,
\begin{Figura}
    \centering
    \includegraphics[width=0.8\textwidth]{Imagenes/Capitulo2/Tabla2.png}
    \captionof{table}{Extinciones sistemáticas en redes cúbicas simples, centradas en el cuerpo y en las caras.}
\end{Figura}

\section{Factor de forma atómico}
Sabemos que cada átomo dispersa los rayos X como consecuencia de la interacción del haz incidente con la nube electrónica. La amplitud dispersada por un átomo depende de la distribución de densidad electrónica y del ángulo de dispersión.\\ \\
Si un átomo se modelara como un punto, su dispersión sería igual en todas las direcciones. Sin embargo, la distribución electrónica extendida provoca interferencias internas que modifican la amplitud. La onda dispersada por un átomo es la suma coherente de las ondas emitidas por cada elemento de volumen $d^3r$ de su nube electrónica. Cada elemento situado en $\vec{r}$ contiene una densidad electrónica $\rho_e(\vec{r})$ y dispersa una onda proporcional a $\rho_e(\vec{r})d^3r$, con una fase determinada por la diferencia de camino óptico entre la onda incidente y la dispersada,
\begin{equation}
    \phi(\vec{r})=\vec{K}\cdot\vec{r}
\end{equation}
donde $\vec{K}=\vec{k}'-\vec{k}$ es el vector de dispersión.\\ \\
La amplitud total dispersada por el átomo en la dirección $\vec{K}$ será entonces la suma coherente de todas las contribuciones,
\begin{equation}
    f(\vec{K})\propto\int\rho_e(\vec{r})e^{i\vec{K}\cdot\vec{r}}d^3r
\end{equation}
Esta expresión es, matemáticamente, la transformada de Fourier de la densidad electrónica $\rho_e(\vec{r})$, por lo que se define el \textbf{factor de forma atómico} como,
\begin{equation}
    f(\vec{K})=\int\rho_e(\vec{r})e^{i\vec{K}\cdot\vec{r}}d^3r
\end{equation}
donde $\rho_e(\vec{r})$ es la densidad electrónica del átomo y $\vec{K}$ el vector de dispersión.
\subsection{Interpretación física}
El factor de forma atómico representa la amplitud relativa con la que un átomo dispersa los rayos X en una dirección dada. \\ \\
En $\vec{K}=0$, el término $e^{i\vec{K}\cdot\vec{r}}=1$ para todos los puntos $\vec{r}$. Esto quiere decir que todas las contribuciones de la nube electrónica están en fase, produciendo interferencia constructiva total. Además, la amplitud total es entonces la suma coherente de todas las cargas electrónicas, tal que
\begin{equation}
    f(0)=\int\rho_e(\vec{r})d^3r=Z
\end{equation}
donde $Z$ es el número total de electrones del átomo. Físicamente, para $\theta=0$ el átomo se comporta como un \textbf{dispersor puntual} con amplitud máxima $f(0)=Z$.\\ \\
A medida que aumenta el módulo $|\vec{K}|$ (es decir, al aumentar el ángulo de dispersión), las interferencias entre electrones a distintas posiciones, reducen el valor de $f(\vec{K})$. Por tanto, $f(\vec{K})$ decrece con el ángulo de dispersión.
\subsection{Expresión general y aproximaciones}
Para átomos esféricamente simétricos, $f(\vec{K})$ solo depende del módulo $K=|\vec{K}|$, y se puede escribir como,
\begin{equation}
    f(\vec{K})=\int\rho_e(\vec{r})e^{i\vec{K}\cdot\vec{r}}d^3r\Rightarrow f(K)=4\pi\int_0^{\infty}\rho_e(r)\frac{\sin(Kr)}{Kr}r^2dr\label{radial}
\end{equation}
Reescribiendo en coordenadas esféricas $(r,\theta,\varphi)$, donde
\begin{equation}
    d^3r=r^2\sin\theta drd\theta d\varphi\hspace{4mm}y\hspace{4mm}\vec{K}\cdot\vec{r}=Kr\cos\theta
\end{equation}
Sustituyendo en la definición del factor de forma,
\begin{equation}
    f(K)=\int_0^{\infty}\rho_e(r)r^2dr\int_0^{2\pi}\int_0^{\pi}e^{iKr\cos\theta}\sin\theta d\theta d\varphi
\end{equation}
Resolviendo las integrales angulares,
\begin{equation}
    \int_0^{2\pi}d\varphi=2\pi\hspace{4mm}y\hspace{4mm}\int_0^{\pi}e^{iKr\cos\theta}\sin\theta d\theta=\frac{2\sin(Kr)}{Kr}
\end{equation}
En resumen, el factor de forma atómico depende del vector de dispersión.\\ \\
Si la densidad electrónica es \textbf{esféricamente simétrica}, $\rho_e(\vec{r})=\rho_e(r)$, la dispersión es igual en todas las direcciones, y por tanto $f(\vec{K})\equiv f(K)$. En este caso, la integral se reduce a la expresión radial (\ref{radial}).\\ \\
En la práctica, los valores de $f(K)$ se obtienen experimentalmente o mediante cálculos para cada elemento y tipo de radiación. Una aproximación común es expresar $f(K)$ como suma de términos gaussianos,
\begin{equation}
    f(K)=\sum_{i=1}^4a_ie^{-b_i\left(\frac{K}{4\pi}\right)^2}+c
\end{equation}
donde los coeficiente $a_i$, $b_i$ y $c$ son ajustados empíricamente.
\subsection{Contribución a la intensidad de difracción}
La intensidad de un pico de difracción está dada, de forma simplificada, por
\begin{equation}
    I(\vec{K})\propto|F(\vec{K}|^2=|S(\vec{K}\cdot f(\vec{K}|^2
\end{equation}
donde $S(\vec{K})$ es el factor de estructura geométrico, que representa la interferencia entre átomos de la celda, y $f(\vec{K})$ es el factor de forma atómico, que representa la interferencia interna del átomo. Como ambas magnitudes se relacionan con interferencias, para ángulos grandes tendremos que incluso cuando $S(\vec{K})\neq0$, $f(\vec{K})$ puede reducir significativamente la intensidad.\\ \\
En resumen, $f(\vec{K})$ cuantifica la dispersión individual de cada átomo, tiene un máximo cuando $\vec{K}=0$ y decrece con el ángulo de difracción. Dependerá del número atómico $Z$, de la densidad electrónica y de la radiación utilizada. Es un factor multiplicativo en la intensidad total de difracción; siendo fundamental para interpretar correctamente los difractogramas y calcular intensidades absolutas.

\section{Conclusiones}

\begin{itemize}
    \item La difracción de rayos X es la técnica fundamental para \textbf{determinar estructuras cristalinas}.
    \item Las formulaciones de \textbf{Bragg} y \textbf{von Laue} son \textbf{equivalentes}.
    \item La \textbf{esfera de Ewald} es una herramienta simple para entender la relación entre el espacio recíproco y la difracción.
    \item El \textbf{factor de estructura} explica qué reflexiones se extinguen en función de la simetría de red.
    \item El \textbf{factor de forma atómico} incorpora la \textbf{distribución electrónica}, modulando la intensidad de las reflexiones.
    \item En conjunto, estos conceptos permiten interpretar difractogramas y conectar \textbf{simetría cristalina} con los \textbf{patrones experimentales}.
\end{itemize}

\lhead{\emph{Capítulo 3: Fuerzas Interatómicas y tipos de Enlace}}
\chapter{Fuerzas Interatómicas y Enlaces en las Redes Cristalinas}

\section{Introducción}

Hasta ahora, hemos estudiado los sólidos como un sistema constituido por partículas discretas que forman una estructura tridimensional, periódica, perfecta y hemos prestado la atención principal a las regularidades de la estructura y la simetría de las redes cristalinas. Al hacer esto, no hemos dicho nada de las fuerzas que mantienen juntas las partículas cerca de sus posiciones de equilibrio. Las fuerzas que sujetan las partículas en el cristal son de la misma naturaleza que las interatómicas que condicionan la formación de las moléculas complejas. Estas fuerzas, como ahora ya se ha establecido exactamente, son, en lo fundamental, las fuerzas de atracción electrostática entre las partículas con cargas de signos contrarios y las fuerzas de repulsión entre las partículas con cargas de igual signo.\\ \\
Las valoraciones de los potenciales de interacción entre las partículas en el cristal muestran que las fuerzas magnéticas son muy pequeñas y que las fuerzas de gravitación son despreciables. Por tanto, el carácter de las fuerzas de interacción entre los átomos está determinado en primer lugar por la estructura de las capas electrónicas de los átomos que interaccionan. Luego, los electrones de valencia van a jugar un papel fundamental.\\ \\
Nos interesará entender el enlace entre un par de átomos y la interacción entre todas las parejas de átomos para formar el cristal.\\ \\
El carácter de las fuerzas interatómicas se toma a veces como base para la clasificación de los sólidos. De acuerdo con esta clasificación (según el enlace) todos los sólidos se dividen en cuatro tipos:
\begin{itemize}
    \item \textbf{Cristales de gases inertes (cristales moleculares)}: En este tipo de cristales las fuerzas interatómicas que prevalecen son las Fuerzas de Van der Waals. Un ejemplo es el cristal de argón.
    \item \textbf{Cristales iónicos}: En este tipo de cristales las fuerzas que prevalecen es la interacción eléctrica entre iones. Por ejemplo el NaCl.
    \item \textbf{Cristales covalentes}: En este tipo de cristales las fuerzas que prevalecen son los enlaces covalentes entre los átomos vecinos. Por ejemplo el diamante o el germanio.
    \item \textbf{Cristales metálicos}: En este tipo de cristales, los electrones de valencia de cada átomo son compartidos por todos los iones del cristal, siendo este mar de electrones el 'pegamento' de estos cristales. Por ejemplo el Na metálico.
\end{itemize}

Advertimos que no existe un procedimiento unívoco de clasificación de los sólidos. Así, todos los sólidos pueden clasificarse por las propiedades de simetría de sus estructuras cristalinas y por sus propiedades eléctricas.
\section{Energía de enlace}
El problema de las fuerzas interatómicas en los sólidos es análogo al problema de las fuerzas intermoleculares, por lo que reducimos el problema a generalizar la respuesta obtenida para las moléculas. Así, empezaremos por estudiar las fuerzas que mantienen unidos los átomos en una molécula diatómica.
\begin{multicols}{2}
Sean $A$ y $B$ dos átomos. Si estos átomos están lejos uno del otro se comportarán como libres. La energía del sistema formado por los dos átomos aislados es igual a la suma de las energías de dichos átomos, la cual puede tomarse como nula. Los átomos no interaccionan entre sí mientras la distancia $r$ entre ellos es grande en comparación con $(r_a+r_b)$, siendo $r_a$ y $r_b$ los radios de los átomos $A$ y $B$, respectivamente.


\begin{Figura}
    \centering
    \includegraphics[width=\textwidth]{Imagenes/Capitulo3/Potencial.png}
    \captionof{figure}{Dependencia de la energía potencial total de interacción de dos átomos (curva continua) respecto de la distancia entre ellos.}
    \label{cap3-potencial}
\end{Figura}
\end{multicols}
Si al disminuir la distancia entre los átomos disminuye la energía del sistema, en comparación con la energía total de los átomos aislados, entre los átomos surge una fuerza de atracción, a la cual corresponde la disminución de la energía potencial del sistema $U(r)$. A cierta distancia $r=r_0$ la energía $U(r)$ alcanza un valor mínimo, al cual corresponde la fuerza
\begin{equation}
    F=-\left(\frac{\partial U}{\partial r}\right)_{r=r_0}=0
\end{equation}
Si los átomos continúan acercándose entre sí, empiezan a actuar fuerzas de repulsión, representando la energía potencial total de la interacción en forma de suma de dos términos, uno de los cuales (negativo) corresponde a la energía de las fuerzas de atracción, y el otro (positivo), a la energía de las fuerzas de repulsión, tal que
\begin{equation}
    U(r)=U_a(r)+U_r(r)
\end{equation}
En la Figura \ref{cap3-potencial}, se representan esquemáticamente las curvas de estos potenciales y la curva sumaria, correspondiente a la energía del potencial total de interacción. Cuando $r=r_0$, lo que corresponde al mínimo de energía del sistema, las fuerzas de atracción equilibran a las de repulsión $(F_a-F_r=0)$ y se forma la molécula $AB$ de la configuración más estable, en la cuál los núcleos de los átomos vibran con la frecuencia propia $\omega_0$. Advertimos que, cerca de la posición de equilibrio, la forma de la curva $U=U(r)$ se aproxima a la de una parábola, como puede verse desarrollando $U(r)$ en serie de Taylor en el entorno $r=r_0$,
\begin{equation}
    U(r)=-U_0(r_0)+\cancelto{0}{\left(\frac{\partial U}{\partial r}\right)_{r=r_0}}(r-r_0)+\frac{1}{2}\left(\frac{\partial^2 U}{\partial r^2}\right)_{r=r_0}(r-r_0)^2+\frac{1}{6}\left(\frac{\partial^3U}{\partial r^3}\right)_{r=r_0}(r-r_0)^3+\dots
\end{equation}
donde cancelamos la primera derivada porque estamos en un mínimo.\\ \\
La energía de cohesión (o de enlace) es la energía necesaria para separar las partes constituyentes del cristal a distancia infinita unas de otras, entendiendo como 'partes constituyentes' a las moléculas o átomos en los cristales moleculares, átomos en cristales covalentes y metálicos, iones en cristales iónico.
\section{Cristales de gases inertes (cristales moleculares)}
\textit{A los cristales moleculares pertenecen los sólidos cuyas redes cristalinas tienen ocupados sus nudos por moléculas iguales con los enlaces saturados o por átomos de gases inertes}.\\ \\
\textit{Una de las particularidades características de los cristales moleculares es que las partículas (átomos, moléculas) se mantienen unidas en el cristal por fuerzas de Van der Waals muy débiles}. La energía de enlace en estos cristales es muy pequeña e igual a $0.02-0.15$ eV. Estas energías de enlace tan pequeñas condicionan temperaturas de fusión muy bajas para estos cristales.\\ \\
\textit{La estructura que forman estos cristales es una FCC, por lo que el empaquetamiento es denso}.\\ \\
\textit{La existencia de las fuerzas de Van der Waals la refleja el hecho de que un átomo isótropo neutro (o molécula neutra) puede polarizarse bajo la influencia de un campo eléctrico, incluso dos átomos isótropos neutros inducen uno en otro pequeños momentos dipolares eléctricos}. El origen de las fuerzas de Van der Waals puede explicarse partiendo de los siguientes razonamientos. En los átomos de los gases inertes, los electrones exteriores forman agrupaciones estables, muy resistentes, de ocho electrones en los estados $s^2p^6$, lo que hace que sobre el movimiento de los electrones influya poco la presencia de los átomos vecinos. En promedio, la distribución de la carga en el átomo aislado tiene simetría esférica, la carga positiva del núcleo es igual a la carga negativa de los electrones que lo rodean, el átomo es eléctricamente neutro y los centros de las cargas se encuentran en el centro del núcleo. Cuando acercamos los átomos, la nube electrónica de uno de ellos puede desplazarse, de manera que los centros de las cargas positivas y negativas no coincidan, surgiendo un momento dipolar atómico $\vec{p}_1$, que provocará un dipolo inducido en el otro átomo, $\vec{p}_2$. De este modo, a medida que se aproximan dos átomos entre sí, su configuración estable se hace equivalente a dos dipolos eléctricos.\\ \\
Como la atracción de las cargas de signos opuestos, más próximas entre sí, aumenta, al acercarse aquellas con más fuerza que la repulsión de las cargas del mismo signo, que están más lejos, el resultado será la atracción mutua de los átomos. Vamos a hacer el cálculo del potencial, modelando los átomos como osciladores acoplados.
\begin{Figura}
    \centering
    \includegraphics[width=0.8\textwidth]{Imagenes/Capitulo3/oscilador.png}
    \captionof{figure}{Esquema del modelo de la molécula diatómica como osciladores.}
\end{Figura}
El Hamiltoniano sin interacción será
\begin{equation}
    H_0=\frac{p_1^2}{2m}+\frac{1}{2}kx_1^2+\frac{p_2^2}{2m}+\frac{1}{2}kx_2^2
\end{equation}
con $k=m\omega_0^2$. El Hamiltoniano de interacción será,
\begin{equation}
    H'=\frac{e^2}{4\pi\epsilon_0}\left[\frac{1}{R}+\frac{1}{R+x_1-x_2}-\frac{1}{R+x_1}-\frac{1}{R-x_2}\right]
\end{equation}
Como $|x_1|,|x_2|\ll R$ se puede simplificar $h'$ usando el desarrollo, $(1+d)^{-1}\approx1-d+d^2$. Se tiene,
\begin{equation}
    H'\approx\frac{e^2}{4\pi\epsilon_0R}\left\lbrace1+\left[1-\frac{x_1-x_2}{R}+\left(\frac{x_1-x_2}{R}\right)^2\right]-\left[1-\frac{x_1}{R}+\left(\frac{x_1}{R}\right)^2\right]-\left[1+\frac{x_2}{R}+\left(\frac{x_2}{R}\right)^2\right]\right\rbrace
\end{equation}
Luego, quedándonos a primer orden,
\begin{equation}
    H'\approx-\frac{1}{4\pi\epsilon_0}\left(\frac{2e^2x_1x_2}{R^3}\right)
\end{equation}
En el Hamiltoniano total, $H=H_0+H'$, pasamos a modos normales con el cambio de variable,
\begin{equation}
    x_1=\frac{1}{\sqrt{2}}(x_S+x_A);\hspace{4mm}x_2=\frac{1}{\sqrt{2}}(x_S-x_A)
\end{equation}
pues un modo normal involucra el movimiento coordinado de todos los átomos del cristal. A $x_S$ y $x_A$ corresponden los momentos $p_S$ y $p_A$ que satisfacen,
\begin{equation}
    p_1=\frac{1}{\sqrt{2}}(p_S+p_A);\hspace{4mm}p_2=\frac{1}{\sqrt{2}}(p_S-p_A)
\end{equation}
Luego, el Hamiltoniano total queda,
\begin{equation}
    H=\left[\frac{p_S^2}{2m}+\frac{m}{2}\left(\omega_0^2-\frac{1}{4\pi\epsilon_0}\frac{2e^2}{mR^3}\right)x_S^2\right]+\left[\frac{p_A^2}{2m}+\frac{m}{2}\left(\omega_0^2+\frac{1}{4\pi\epsilon_0}\frac{2e^2}{mR^3}\right)x_A^2\right]
\end{equation}
que representa dos osciladores independientes con frecuencias,
\begin{equation}
    \omega_{S,A}=\omega_0\left[1\mp\frac{1}{4\pi\epsilon_0}\frac{2e^2}{m\omega_0^2R^3}  \right]^{1/2}\approx\omega_0\left[1\mp\frac{a}{2}-\frac{a^2}{8}\right]
\end{equation}
con $a=\frac{1}{4\pi\epsilon_0}\frac{2e^2}{m\omega_0^2R^3}$ y usando que $(1\mp d)^{1/2}\approx 1\mp\frac{d}{2}-\frac{d^2}{8}+\mathscr{O}(d^3)$.
Sabemos que la energía de un oscilador de frecuencia $\omega$ es
\begin{equation}
    E_n=\hbar\omega\left(n+\frac{1}{2}\right)
\end{equation}
con $n=0,1,2,\dots$ La energía del estado fundamental de los dos osciladores independientes (descrito por $H_0$) es
\begin{equation}
    U_0=2\frac{\hbar\omega_0}{2}=\hbar\omega_0
\end{equation}
La energía del estado fundamental de los dos osciladores acoplados (descrito por $H$) es
\begin{equation}
    U_{acopl}=\frac{\hbar}{2}(\omega_S+\omega_A)=\hbar\omega_0\left(1-\frac{a^2}{8}\right)
\end{equation}
donde hemos usado las definiciones de $\omega_S$ y $\omega_A$. Por tanto,
\begin{equation}
    U(R)=U_{acopl}-U_0=-\frac{\hbar\omega_0a^2}{8}=-\frac{\hbar\omega_0}{8}\frac{1}{(4\pi\epsilon_0)^2}\left(\frac{2e^2}{m\omega_0^2R^3}\right)^2\propto\frac{-1}{R^6}
\end{equation}
que representa el potencial atractivo y podemos agrupar las constantes, tal que
\begin{equation}
    U(R)=-\frac{A}{R^6}
\end{equation}
Este potencial corresponde a las fuerzas de Van der Waals o de interacción dipolo - dipolo inducido.
\begin{note}
    El potencial de Van der Waals aparece directamente al realizar el cálculo cuántico. No ha sido necesario suponer que una fluctuación provoca un dipolo en una molécula. En el tratamiento cuántico, esa fluctuación está asociada a la energía del punto cero del oscilador.
\end{note}

Por tanto, las fuerzas atractivas entre átomos en los cristales de gases inertes son fuerzas de Van der Waals, también llamadas de interacción dipolo - dipolo inducido.\\ \\
\textit{Si la distancia entre los átomos sigue disminuyendo, sus capas electrónicas empiezan a superponerse y entre los átomos surgen fuerzas de repulsión considerables. La repulsión, en el caso de los gases inertes, aparece principalmente como resultado de la acción del Principio de exclusión de Pauli}. Al superponerse las capas electrónicas, los electrones del primer átomo tienden a ocupar parcialmente los estados del segundo. Como los átomos de los gases inertes poseen capas atómicas estables, en las cuales todos los estados energéticos están ya ocupados, al superponerse las capas, los electrones deben pasar a estados cuánticos libres con energía más alta, ya que, de acuerdo con el principio de Pauli, los electrones no pueden ocupar una misma región del espacio sin que aumente su energía cinética. El aumento de la energía cinética hace que aumente la energía total del sistema de los dos átomos que interaccionan y, por tanto, que aparezcan \textbf{fuerzas de repulsión}.\\ \\
Para que el potencial total tenga un mínimo, es necesario que a distancias pequeñas, el potencial de las fuerzas de repulsión sea mayor que el de las fuerzas de atracción. El potencial de las fuerzas de repulsión suele representarse en forma de ley de potencias,
\begin{equation}
    U_r=\frac{B}{R^{12}}
\end{equation}
Ahora la energía potencial total de la interacción entre dos átomos, situados a la distancia $R$ uno del otro, se puede escribir como,
\begin{equation}
    U(R)=-\frac{A}{R^{6}}+\frac{B}{R^{12}}
\end{equation}
donde $A,B\in\mathbb{R}^+\backslash\{0\}$. Para describir la interacción de los átomos eléctricamente neutros y de las moléculas eléctricamente neutras y no polares, se usa el \textbf{potencial de Lennard-Jones},
\begin{equation}
    U=4\epsilon\left[\left(\frac{\sigma}{R}\right)^{12}-\left(\frac{\sigma}{R}\right)^6\right]
\end{equation}
Vemos que este potencial depende de dos parámetros,
\begin{equation}
    \epsilon=\frac{A^2}{4B};\hspace{4mm}\sigma=\left(\frac{B}{A}\right)^{1/6}
\end{equation}
El parámetro $\sigma$ corresponde a la distancia interatómica con la cual la energía potencial total es nula, y el parámetro $\epsilon=-U_{min}$ tiene la dimensión de energía y es igual al mínimo de la energía potencial para $R_0/\sigma=2^{1/6}\approx1.12$. Estos parámetros se obtienen en la posición de equilibrio, usando que $\left.\frac{dU}{dR}\right|_{R_0}=0$.
\begin{proof}
    Vamos a obtener estos parámetros de equilibrio.
    \begin{align}
        0=\left.\frac{dU}{dR}\right|_{R_0}=4\epsilon\left[\sigma^{12}\frac{-12}{R_0^{13}}-\sigma^6\frac{-6}{R_0^7}\right]=\cancel{\frac{24\epsilon\sigma^6}{R_0^7}}\left[1-\frac{2\sigma^6}{R_0^6}\right]\Rightarrow1-\frac{2\sigma^6}{R_0^6}=0\Rightarrow\frac{2\sigma^6}{R_0^6}=1\Rightarrow\frac{R_0}{\sigma}=2^{1/6}
    \end{align}
    Luego, sustituyendo en $U(R_0)\equiv U_{min}$,
    \begin{equation}
        U_{min}=4\epsilon\left[\left(\frac{1}{2^{1/6}}\right)^{12}-\left(\frac{1}{2^{1/6}}\right)^6\right]=4\epsilon\left[\frac{1}{4}-\frac{1}{2}\right]=-\epsilon
    \end{equation}
\end{proof}

Este desarrollo lo hemos hecho para una molécula diatómica, por lo que debemos extenderlo para todo el cristal. Por lo que, si $U(r)$ es la energía de interacción de dos átomos, la energía de interacción de los $N$ átomos del cristal es
\begin{equation}
    U_C=\frac{1}{2}\sum_{i=1}^N\sum_{\overset{j=1}{j\neq i}}U(r_{ij})=\frac{N}{2}\sum_{\overset{j=1}{j\neq i}}^NU(r_{ij})
\end{equation}
donde en la última suma, $i$ es un átomo cualquiera de referencia. En ella, se han despreciado los efectos de borde, por lo que la suma da el mismo valor para cualquier $i$.\\ \\
Por tanto, debemos expresar $r_{ij}$ en unidades de distancia entre los vecinos más próximos,
\begin{equation}
    r_{ij}=R~p_{ij}
\end{equation}
Luego, la energía de interacción de todos los átomos del cristal es
\begin{equation}
    U_C(R)=\frac{N}{2}4\epsilon\left[\sum_{\overset{j=1}{j\neq i}}^N\left(\frac{\sigma}{p_{ij}R}\right)^{12}-\sum_{\overset{j=1}{j\neq i}}^N\left(\frac{\sigma}{p_{ij}R}\right)^6\right]
\end{equation}
donde $R$ es la distancia entre vecinos más próximos, $p_{ij}R$ es la distancia entre los átomos $i$ y $j$, el subíndice $i$ indica un átomo cualquiera de referencia, y se suma sobre todos los $N$ átomos del cristal excepto del $i$. Reordenando,
\begin{equation}
    U_C(R)=2N\epsilon\left[\left(\frac{\sigma}{R}\right)^{12}A_{12}-\left(\frac{\sigma}{R}\right)^6A_6\right]\hspace{4mm}\text{con}\hspace{4mm}
    A_{12}=\sum_{\overset{j=1}{j\neq i}}^N\left(\frac{1}{p_{ij}}\right)^{12};\hspace{4mm}A_6=\sum_{\overset{j=1}{j\neq i}}^N\left(\frac{1}{p_{ij}}\right)^6
\end{equation}
denominadas \textbf{sumas estructurales}. Luego, hemos cargado la dependencia espacial en $p_{ij}$, con las sumas estructurales, que son sumas geométricas. Vamos a calcular estas sumas para distintas redes, sabiendo que, para redes cúbicas, el parámetro de red es $a$, luego tendremos que $p_{ij}=\frac{r_{j}}{r_1}$, donde $r_1$ es la distancia de los vecinos más próximos.
\begin{note}\label{cap3-nota1}
    Para encontrar el número de vecinos por capa, podemos calcularlos usando las permutaciones, dado la posición de un átomo de la capa. Veamos todos los casos:
    \begin{enumerate}
    \item \label{cap-3-1}Si tenemos un átomo de la capa de vecinos en la posición $(\pm a,0,0)$, tenemos 3 permutaciones, y una componente no nula, luego $2^1=2$, por tanto tendremos 
    \begin{equation}
        3\times2=6\text{ vecinos}
    \end{equation}
   \item \label{cap-3-2} Si tenemos un átomo de la capa de vecinos en la posición $(\pm a,\pm a,0)$, tendremos las mismas permutaciones de antes, que son 3, y ahora tenemos 2 componentes no nulas, luego $2^2=4$, por tanto tendremos
    \begin{equation}
        3\times4=12\text{ vecinos}
    \end{equation}
    \item \label{cap-3-3} Si tenemos un átomo de la capa de vecinos en la posición $(\pm a,\pm a,\pm b)$, tendremos las mismas permutaciones anteriores, que son 3, y ahora tenemos 3 componentes no nulas, luego $2^3=8$, por tanto tendremos
    \begin{equation}
        3\times8=24\text{ vecinos}
    \end{equation}
    \item \label{cap-3-4} Si tenemos un átomo de la capa de vecinos en la posición $(\pm a,\pm b,0)$, tenemos 3 valores distintos, por tanto las permutaciones serán $3!=6$, y como tenemos dos posiciones no nulas, $2^2=4$, por tanto tendremos,
    \begin{equation}
        6\times4=24\text{ vecinos}
    \end{equation}
    \item \label{cap-3-5} Si tenemos un átomo de la capa de vecinos en la posición $(\pm a,\pm a,\pm a)$, al ser todos iguales la permutación es 1, pero tenemos 3 posiciones no nulas, luego $2^3=8$, por tanto tendremos,
    \begin{equation}
        1\times8=8\text{ vecinos}
    \end{equation}
    \item \label{cap-3-6} Si tenemos un átomo de la capa de vecinos en la posición $(\pm a,\pm b,\pm c)$, como tenemos tres elementos distintos, las permutaciones serán $3!=6$, además, como cada coordenada puede tener signo $\pm$, tendremos $2^3=8$, luego, el número de vecinos en esta capa es
    \begin{equation}
        6\times8=48 \text{ vecinos}
    \end{equation}
    \end{enumerate}
\end{note}
Sabemos que los valores más precisos calculados de estas sumas estructurales son los siguientes,

\begin{table}[H]
\centering
\begin{tabular}{@{}c:c:c:c@{}}
\toprule
\textbf{} & \textbf{FCC}                  & \textbf{SC}    & \textbf{BCC}   \\ \midrule
$A_{12}$           & $12.13188$ & $6.20215$ & $9.11418$                     \\
$A_6$           & $14.45392$ & $8.40192$ & $12.25367$                  \\ \bottomrule
\end{tabular}
\caption{Valores de $A_{12}$ y $A_6$ más precisos obtenidos de la referencia \cite{Schwerdtfeger_2021}.}
\label{cap3-tabla1}
\end{table}

\begin{itemize}
    \item \textbf{Para una Red FCC}:\\ \\
    Recordando la estructura FCC,
    \begin{Figura}
        \centering
        \includegraphics[width=\textwidth]{Imagenes/Capitulo3/FCC.png}
        \captionof{figure}{Red FCC donde se marcan los vecinos cercanos según más proximidad al átomo de referencia.}
        \label{cap3-FCC}
    \end{Figura}
    \begin{Figura}
        \centering
        \includegraphics[width=\textwidth]{Imagenes/Capitulo3/FCC-2.jpeg}
        \captionof{figure}{Red FCC donde se marcan las coordenadas de los vecinos cercanos al origen.}
        \label{cap3-FCC-2}
    \end{Figura}
Tomamos como átomo de referencia una esquina (átomo rojo), situado en $(0,0,0)$.
\\
- Para los primeros vecinos más próximos, tomamos uno de estos vecinos, vemos que está situado en el punto $a\left(0,\frac{1}{2},\frac{1}{2}\right)$ que tendrán una distancia real $r_1=a/\sqrt{2}$, pues sabiendo que el lado vale $a$, tomando el módulo tenemos,
\begin{equation}
    r_1^2=\frac{a^2}{4}+\frac{a^2}{4}=\frac{a^2}{2}\Rightarrow r_1=\frac{a}{\sqrt{2}}
\end{equation}
Luego, normalizando con $r_1$, tenemos que $p_{ij}^{(1)}=\frac{r_1}{r_1}=1$, con 12 átomos (ver nota \ref{cap3-nota1}, apartado \ref{cap-3-2}).\\ 
- Para los segundos vecinos más cercanos, tomando uno de estos vecinos en el punto $a(0,0,1)$, tenemos que $r_2=a$, pues son los rosas, y vemos que tenemos 6 átomos (ver nota \ref{cap3-nota1}, apartado \ref{cap-3-1}), por tanto, $p_{ij}^{(2)}=\frac{r_2}{r_1}=\frac{a}{a/\sqrt{2}}=\sqrt{2}$.\\
- Para los terceros vecinos más cercanos, tomando uno de estos vecinos en el punto $a\left(1,\frac{1}{2},\frac{1}{2}\right)$, luego la distancia real será $r_3=a\sqrt{3/2}$, pues tomando el módulo,
\begin{equation}
    r_3^2=a^2+\frac{a^2}{4}+\frac{a^2}{4}=a^2+\frac{a^2}{2}=3\frac{a^2}{2}\Rightarrow r_3=a\sqrt{\frac{3}{2}}
\end{equation}
Luego, normalizando con $r_1$ tenemos que $p_{ij}^{(3)}=\frac{r_3}{r_1}=\frac{a\sqrt{3/2}}{a/\sqrt{2}}=\sqrt{3}$, siendo 24 átomos (ver nota \ref{cap3-nota1}, apartado \ref{cap-3-3}).\\
- Para los cuartos vecinos más cercanos, tomando un vecino en el punto $a(0,1,1)$ tenemos que $r_4=a\sqrt{2}$, pues tomando el módulo tenemos,
\begin{equation}
    r_4^2=a^2+a^2=2a^2\Rightarrow r_4=a\sqrt{2}
\end{equation}
Luego, $p_{ij}^{(4)}=\frac{r_4}{r_1}=\frac{a\sqrt{2}}{a/\sqrt{2}}=2$, siendo 12 vecinos (ver nota \ref{cap3-nota1}, apartado \ref{cap-3-2}).\\
- Para los vecinos quintos más cercanos, tenemos 24 átomos (ver nota \ref{cap3-nota1}, apartado \ref{cap-3-4}), que tomando uno de ellos en el punto $a\left(\frac{1}{2},\frac{3}{2},0\right)$, la distancia real será $r_5=a\sqrt{5/2}$, pues tomando el módulo,
\begin{equation}
    r_5^2=\frac{a^2}{4}+\frac{9a^2}{4}=\frac{5a^2}{2}\Rightarrow r_5=a\sqrt{\frac{5}{2}}
\end{equation}
Luego, normalizando con $r_1$ tenemos que $p_{ij}^{(5)}=\frac{r_5}{r_1}=\frac{a\sqrt{5/2}}{a/\sqrt{2}}=\sqrt{5}$.
\\
- Para los vecinos sextos más cercanos, tenemos 8 átomos (ver nota \ref{cap3-nota1}, apartado \ref{cap-3-5}), que tomando uno de ellos en el punto $a(1,1,1)$, la distancia real es $r_6=a\sqrt{3}$, pues tomando el módulo,
    \begin{equation}
        r_6^2=a^2+a^2+a^2=3a^2\Rightarrow r_6=a\sqrt{3}
    \end{equation}
    Luego, normalizando con $r_1$ tenemos que $p_{ij}^{(6)}=\frac{r_6}{r_1}=\frac{a\sqrt{3}}{a/\sqrt{2}}=\sqrt{6}$.
- Para los vecinos séptimos más cercanos, tenemos 48 átomos (ver nota \ref{cap3-nota1}, apartado \ref{cap-3-6}), que tomando uno de ellos en el punto $a(1,\frac{1}{2},\frac{3}{2})$, la distancia real es $r_7=a\sqrt{7/2}$, pues tomando el módulo,
    \begin{equation}
        r_7^2=a^2+a^2\frac{1}{4}+a^2\frac{9}{4}=\frac{7}{2}a^2\Rightarrow r_7=a\sqrt{\frac{7}{2}}
    \end{equation}
    Luego, normalizando con $r_1$ tenemos que $p_{ij}^{(7)}=\frac{r_7}{r_1}=\frac{a\sqrt{7/2}}{a/\sqrt{2}}=\sqrt{7}$.\\
    En resumen,
    % Please add the following required packages to your document preamble:
% 
\begin{table}[H]
\centering
\begin{tabular}{@{}c:c:c:c:c@{}}
\toprule
\textbf{Capa} & \textbf{Vector ejemplo}                  & \textbf{Distancia}    & $\mathbf{p=r/r_1}$  & \textbf{Nº átomos} \\ \midrule
$1$           & $\left(0,\frac{1}{2},\frac{1}{2}\right)$ & $a\frac{1}{\sqrt{2}}$  & $1$        & $12$               \\
$2$           & $\left(0,0,1\right)$                     & $a$                   & $\sqrt{2}$ & $6$                \\
$3$           & $\left(0,\frac{1}{2},1\right)$           & $a\sqrt{\frac{3}{2}}$ & $\sqrt{3}$ & $24$               \\
$4$           & $\left(0,1,1\right)$                     & $a\sqrt{2}$           & $2$        & $12$               \\
$5$           & $\left(\frac{1}{2},\frac{1}{2},1\right)$ & $a\sqrt{\frac{5}{2}}$ & $\sqrt{5}$ & $24$               \\
$6$           & $\left(1,1,1\right)$                     & $a\sqrt{3}$           & $\sqrt{6}$ & $8$                \\
$7$           & $\left(1,\frac{1}{2},\frac{3}{2}\right)$ & $a\sqrt{\frac{7}{2}}$ & $\sqrt{7}$ & $48$               \\ \bottomrule
\end{tabular}
\caption{Tabla resumen de los valores obtenidos sobre los vecinos próximos de una red FCC.}
\end{table}


    
Luego, sustituyendo en las expresiones de $A_{12}$ y $A_6$, tenemos
\begin{equation}\hspace*{-.5cm}
\begin{array}{rl}
    A_{12}&=\sum\limits_{i=1}^{12}\left(\frac{1}{1}\right)^{12}+\sum\limits_{i=1}^6\left(\frac{1}{\sqrt{2}}\right)^{12}+\sum\limits_{i=1}^{24}\left(\frac{1}{\sqrt{3}}\right)^{12}+\sum\limits_{i=1}^{12}\left(\frac{1}{2}\right)^{12}+\sum\limits_{i=1}^{24}\left(\frac{1}{\sqrt{5}}\right)^{12}+\sum\limits_{i=1}^{8}\left(\frac{1}{\sqrt{6}}\right)^{12}+\sum\limits_{i=1}^{48}\left(\frac{1}{\sqrt{7}}\right)^{12}=\\ \\
    &=12+\frac{6}{2^{6}}+\frac{24}{3^6}+\frac{12}{2^{12}}+\frac{24}{5^6}+\frac{8}{6^6}+\frac{48}{7^6}=12.1317
\end{array}
\end{equation}
    Para $A_6$ tenemos,
    \begin{equation}
    \begin{array}{rl}
        A_6&=\sum\limits_{i=1}^{12}\left(\frac{1}{1}\right)^6+\sum\limits_{i=1}^6\left(\frac{1}{\sqrt{2}}\right)^6+\sum\limits_{i=1}^{24}\left(\frac{1}{\sqrt{3}}\right)^6+\sum\limits_{i=1}^{12}\left(\frac{1}{2}\right)^6+\sum\limits_{i=1}^{24}\left(\frac{1}{\sqrt{5}}\right)^{6}+\sum\limits_{i=1}^{8}\left(\frac{1}{\sqrt{6}}\right)^{6}+\sum\limits_{i=1}^{48}\left(\frac{1}{\sqrt{7}}\right)^{6}=\\ \\
        &=12+\frac{6}{2^3}+\frac{24}{3^3}+\frac{12}{2^6}+\frac{24}{5^3}+\frac{8}{6^3}+\frac{48}{7^3}=14.1954
    \end{array}
    \end{equation}
    Sabemos que el valor real debe ser $A_6\approx 14.45$, y la discrepancia se debe a la truncación de $A_6$ en el séptimo vecino, pues por definición hay que usar todos. Para $A_{12}$ sí podemos truncar aquí, pues las contribuciones superiores son muy ínfimas.

\item \textbf{Para una Red SC}:\\ \\
    Recordando la estructura SC,
    \begin{Figura}
        \centering
        \includegraphics[width=0.8\textwidth]{Imagenes/Capitulo3/SC.jpeg}
        \captionof{figure}{Red SC donde se marcan los vecinos cercanos según más proximidad al átomo de referencia.}
        \label{cap3-SC}
    \end{Figura}
    \begin{Figura}
        \centering
        \includegraphics[width=0.8\textwidth]{Imagenes/Capitulo3/SC-2.jpeg}
        \captionof{figure}{Red SC donde se marcan las coordenadas de los vecinos cercanos al origen.}
        \label{cap3-SC-2}
    \end{Figura}
Tomamos como átomo de referencia una esquina (átomo rojo), situado en $(0,0,0)$.
\\
- Para los primeros vecinos más próximos, tomamos uno de estos vecinos situado en el punto $a\left(0,0,1\right)$ que tendrán una distancia real $r_1=a$, tomando el módulo tenemos,
\begin{equation}
    r_1^2=a^2\Rightarrow r_1=a
\end{equation}
Luego, normalizando con $r_1$, tenemos que $p_{ij}^{(1)}=\frac{r_1}{r_1}=1$, con 6 átomos (ver nota \ref{cap3-nota1}, apartado \ref{cap-3-1}).\\ 
- Para los segundos vecinos más cercanos, tomando uno de estos vecinos en el punto $a(1,1,0)$, tenemos que $r_2=a\sqrt{2}$, pues tomando el módulo,
\begin{equation}
    r_2^2=a^2+a^2=2a^2\Rightarrow r_2=a\sqrt{2}
\end{equation}
Vemos que tenemos 12 átomos (ver nota \ref{cap3-nota1}, apartado \ref{cap-3-2}), por tanto, $p_{ij}^{(2)}=\frac{r_2}{r_1}=\frac{a\sqrt{2}}{a}=\sqrt{2}$.\\
- Para los terceros vecinos más cercanos, tomando uno de estos vecinos en el punto $a\left(1,1,1\right)$, luego la distancia real será $r_3=a\sqrt{3}$, pues tomando el módulo,
\begin{equation}
    r_3^2=a^2+a^2+a^2=3a^2\Rightarrow r_3=a\sqrt{3}
\end{equation}
Luego, normalizando con $r_1$ tenemos que $p_{ij}^{(3)}=\frac{r_3}{r_1}=\frac{a\sqrt{3}}{a}=\sqrt{3}$, siendo 8 átomos (ver nota \ref{cap3-nota1}, apartado \ref{cap-3-5}).\\
En resumen,
\begin{table}[H]
\centering
\begin{tabular}{@{}c:c:c:c:c@{}}
\toprule
\textbf{Capa} & \textbf{Vector ejemplo}                  & \textbf{Distancia}    & $\mathbf{p=r/r_1}$  & \textbf{Nº átomos} \\ \midrule
$1$           & $\left(1,0,0\right)$ &                           $a$                     & $1$                    & $6$               \\
$2$           & $\left(1,1,0\right)$                     & $a\sqrt{2}$                   & $\sqrt{2}$            & $12$                \\
$3$           & $\left(1,1,1\right)$                     & $a\sqrt{3}$ &                     $\sqrt{3}$           & $8$               \\ \bottomrule
\end{tabular}
\caption{Tabla resumen de los valores obtenidos sobre los vecinos próximos de una red SC.}
\end{table}
Luego, sustituyendo en las expresiones de $A_{12}$ y $A_6$, tenemos
\begin{equation}
    A_{12}=\sum_{i=1}^{6}\left(\frac{1}{1}\right)^{12}+\sum_{i=1}^{12}\left(\frac{1}{\sqrt{2}}\right)^{12}+\sum_{i=1}^{8}\left(\frac{1}{\sqrt{3}}\right)^{12}=6+\frac{12}{2^6}+\frac{8}{3^6}=6.19847
\end{equation}
Para el $A_6$ tenemos,

\begin{equation}
    A_{6}=\sum_{i=1}^{6}\left(\frac{1}{1}\right)^{6}+\sum_{i=1}^{12}\left(\frac{1}{\sqrt{2}}\right)^{6}+\sum_{i=1}^{8}\left(\frac{1}{\sqrt{3}}\right)^{6}=6+\frac{12}{2^3}+\frac{8}{3^3}=7.7963
\end{equation}

Comparando con los valores reales $A_{12}\approx6.20$ y $A_6\approx8.40$, vemos que de nuevo $A_{12}$ sí se aproxima bien con este truncamiento, mientras que para $A_6$ habrá que usar más vecinos.

\item \textbf{Para una Red BCC}:\\ \\
    Recordando la estructura BCC,
    \begin{Figura}
        \centering
        \includegraphics[width=0.8\textwidth]{Imagenes/Capitulo3/BCC.jpeg}
        \captionof{figure}{Red BCC donde se marcan los vecinos cercanos según más proximidad al átomo de referencia.}
        \label{cap3-SC}
    \end{Figura}
    \begin{Figura}
        \centering
        \includegraphics[width=0.8\textwidth]{Imagenes/Capitulo3/BCC-2.jpeg}
        \captionof{figure}{Red BCC donde se marcan las coordenadas de los vecinos cercanos al origen.}
        \label{cap3-SC-2}
    \end{Figura}
Tomamos como átomo de referencia una esquina (átomo rojo), situado en $(0,0,0)$.
\\
- Para los primeros vecinos más próximos, tomamos uno de estos vecinos situado en el punto $a\left(\frac{1}{2},\frac{1}{2},\frac{1}{2}\right)$ que tendrán una distancia real $r_1=\frac{a}{2}\sqrt{3}$, tomando el módulo tenemos,
\begin{equation}
    r_1^2=\frac{a^2}{4}+\frac{a^2}{4}+\frac{a^2}{4}=3\frac{a^2}{4}\Rightarrow r_1=\frac{a\sqrt{3}}{2}
\end{equation}
Luego, normalizando con $r_1$, tenemos que $p_{ij}^{(1)}=\frac{r_1}{r_1}=1$, con 8 átomos (ver nota \ref{cap3-nota1}, apartado \ref{cap-3-5}).\\ 
- Para los segundos vecinos más cercanos, tomando uno de estos vecinos en el punto $a(1,0,0)$, tenemos que $r_2=a$. Vemos que tenemos 6 átomos (ver nota \ref{cap3-nota1}, apartado \ref{cap-3-1}). Normalizando con $r_1$ tenemos, $p_{ij}^{(2)}=\frac{r_2}{r_1}=\frac{a}{a\sqrt{3}/2}=\frac{2}{\sqrt{3}}$.\\
- Para los terceros vecinos más cercanos, tomando uno de estos vecinos en el punto $a(1,1,0)$, tenemos que $r_3=a\sqrt{2}$, pues tomando el módulo,
\begin{equation}
    r_3^2=a^2+a^2=2a^2\Rightarrow r_3=a\sqrt{2}
\end{equation}
Vemos que tenemos 12 átomos (ver nota \ref{cap3-nota1}, apartado \ref{cap-3-2}), por tanto, $p_{ij}^{(3)}=\frac{r_3}{r_1}=\frac{a\sqrt{2}}{a\sqrt{3}/2}=\frac{2\sqrt{2}}{\sqrt{3}}$.\\
- Para los cuartos vecinos más cercanos, tomando uno de estos vecinos en el punto $a\left(1,1,1\right)$, luego la distancia real será $r_4=a\sqrt{3}$, pues tomando el módulo,
\begin{equation}
    r_4^2=a^2+a^2+a^2=3a^2\Rightarrow r_4=a\sqrt{3}
\end{equation}
Luego, normalizando con $r_1$ tenemos que $p_{ij}^{(4)}=\frac{r_4}{r_1}=\frac{a\sqrt{3}}{a\sqrt{3}/2}=2$, siendo 8 átomos (ver nota \ref{cap3-nota1}, apartado \ref{cap-3-5}).\\
En resumen,
\begin{table}[H]
\centering
\begin{tabular}{@{}c:c:c:c:c@{}}
\toprule
\textbf{Capa} & \textbf{Vector ejemplo}                  & \textbf{Distancia}    & $\mathbf{p=r/r_1}$  & \textbf{Nº átomos} \\ \midrule
$1$           & $\left(\frac{1}{2},\frac{1}{2},\frac{1}{2}\right)$                     & $a\sqrt{3}/2$ &                     $1$           & $8$               \\ 
$2$           & $\left(1,0,0\right)$ &                           $a$                     & $2/\sqrt{3}$                    & $6$               \\
$3$           & $\left(1,1,0\right)$                     & $a\sqrt{2}$                   & $2\sqrt{2}/\sqrt{3}$            & $12$                \\
$4$           & $\left(1,1,1\right)$                     & $a\sqrt{3}$ &                     $2$           & $8$               \\ \bottomrule
\end{tabular}
\caption{Tabla resumen de los valores obtenidos sobre los vecinos próximos de una red BCC.}
\end{table}
Luego, sustituyendo en las expresiones de $A_{12}$ y $A_6$, tenemos
\begin{equation}\hspace*{-2.8cm}
    A_{12}=\sum_{i=1}^{8}\left(\frac{1}{1}\right)^{12}+\sum_{i=1}^{6}\left(\frac{1}{2/\sqrt{3}}\right)^{12}+\sum_{i=1}^{12}\left(\frac{1}{2\sqrt{2}/\sqrt{3}}\right)^{12}+\sum_{i=1}^{8}\left(\frac{1}{2}\right)^{12}=8+\frac{6}{(2/\sqrt{3})^{12}}+\frac{12}{(2\sqrt{2}/\sqrt{3})^{12}}+\frac{8}{2^{12}}=9.1032
\end{equation}
Para el $A_6$ tenemos,
\begin{equation}
    \hspace*{-1.6cm}
    A_{6}=\sum_{i=1}^{8}\left(\frac{1}{1}\right)^{6}+\sum_{i=1}^{6}\left(\frac{1}{2/\sqrt{3}}\right)^{6}+\sum_{i=1}^{12}\left(\frac{1}{2\sqrt{2}/\sqrt{3}}\right)^{6}+\sum_{i=1}^{8}\left(\frac{1}{2}\right)^{6}=12+\frac{6}{(2/\sqrt{3})^{6}}+\frac{12}{(2\sqrt{2}/\sqrt{3})^{6}}+\frac{8}{2^{6}}=11.2891
\end{equation}
Comparando con los valores reales $A_{12}\approx9.11$ y $A_6\approx12.25$, vemos que de nuevo, $A_{12}$ se aproxima bien con este truncamiento, mientras que $A_6$ no, por lo que habrá que usar más vecinos.
\end{itemize}

Ahora, sustituyendo los valores de $A_{12}$ y $A_6$ en la energía obtenidos para las distintas redes, podemos obtener los \textbf{parámetros de equilibrio del cristal}. Vamos a hacerlo para todas las redes con los valores de la table \ref{cap3-tabla1}:
\begin{itemize}
    \item \textbf{Red FCC}: Sustituyendo obtenemos que la energía es,
    \begin{equation}
        U_C(R)=2N\epsilon\left[12.12\left(\frac{\sigma}{R}\right)^{12}-14.45\left(\frac{\sigma}{R}\right)^6\right]
    \end{equation}
    Aplicando la condición de equilibrio en $R_0$, obtenemos que
    \begin{equation}
        0=\left.\frac{dU_C(R)}{dR}\right|_{R_0}=2N\epsilon\left[12.13\sigma^{12}\frac{-12}{R_0^{13}}-14.45\sigma^6\frac{-6}{R_0^7}\right]=\frac{2N\epsilon6\sigma^6}{R_0^7}\left[14.45-12.13\sigma^6\frac{2}{R_0^6}\right]
    \end{equation}
    Luego,
    \begin{equation}
        14.45-12.13\sigma^6\frac{2}{R_0^6}=0\Rightarrow14.45=24.26\left(\frac{\sigma}{R_0}\right)^6\Rightarrow\left(\frac{R_0}{\sigma}\right)^6=1.67889\Rightarrow\frac{R_0}{\sigma}=1.09
    \end{equation}
    Por tanto, la distancia $R_0$ entre vecinos más cercanos en el equilibrio para una red FCC es $R_0=1.09\sigma$. Este $R_0$ es menor que la distancia de equilibrio para un par de átomos aislados, que recordamos que es $R_0=1.12\sigma$.\\
    Ahora, sustituyendo este valor de $R_0$ en la ecuación de la energía, obtendremos la energía potencial en el equilibrio, tal que
    \begin{equation}
        U_C(R_0)=2N\epsilon\left[12.13\frac{1}{1.09^{12}}-14.45\frac{1}{1.09^6}\right]=-8.6N\epsilon
    \end{equation}
    Entonces, la energía potencial en el equilibrio por átomo es, $U_C(R_0)/N=-8.6\epsilon$. Por tanto, la energía de enlace por átomo es $8.6\epsilon$.
    \item \textbf{Red SC}: Sustituyendo obtenemos que la energía es,
    \begin{equation}
        U_C(R)=2N\epsilon\left[6.20\left(\frac{\sigma}{R}\right)^{12}-8.40\left(\frac{\sigma}{R}\right)^6\right]
    \end{equation}
    Aplicando la condición de equilibrio en $R_0$, obtenemos que
    \begin{equation}
        0=\left.\frac{dU_C(R)}{dR}\right|_{R_0}=2N\epsilon\left[6.2\sigma^{12}\frac{-12}{R_0^{13}}-8.4\sigma^6\frac{-6}{R_0^7}\right]=\frac{2N\epsilon6\sigma^6}{R_0^7}\left[8.4-6.2\sigma^6\frac{2}{R_0^6}\right]
    \end{equation}
    Luego,
    \begin{equation}
        8.4-6.2\sigma^6\frac{2}{R_0^6}=0\Rightarrow8.4=12.4\left(\frac{\sigma}{R_0}\right)^6\Rightarrow\left(\frac{R_0}{\sigma}\right)^6=1.4762\Rightarrow\frac{R_0}{\sigma}=1.067
    \end{equation}
    Por tanto, la distancia $R_0$ entre vecinos más cercanos en el equilibrio para una red SC es $R_0=1.067\sigma$. Este $R_0$ es menor que la distancia de equilibrio para un par de átomos aislados, que recordamos que es $R_0=1.12\sigma$ y menor que la distancia de equilibrio de una red FCC.\\
    Ahora, sustituyendo este valor de $R_0$ en la ecuación de la energía, obtendremos la energía potencial en el equilibrio, tal que
    \begin{equation}
        U_C(R_0)=2N\epsilon\left[6.2\frac{1}{1.067^{12}}-8.4\frac{1}{1.067^6}\right]=-5.7N\epsilon
    \end{equation}
    Entonces, la energía potencial en el equilibrio por átomo es, $U_C(R_0)/N=-5.7\epsilon$. Por tanto, la energía de enlace por átomo es $5.7\epsilon$.
    \item \textbf{Red BCC}: Sustituyendo obtenemos que la energía es,
    \begin{equation}
        U_C(R)=2N\epsilon\left[9.11\left(\frac{\sigma}{R}\right)^{12}-12.25\left(\frac{\sigma}{R}\right)^6\right]
    \end{equation}
    Aplicando la condición de equilibrio en $R_0$, obtenemos que
    \begin{equation}
        0=\left.\frac{dU_C(R)}{dR}\right|_{R_0}=2N\epsilon\left[9.11\sigma^{12}\frac{-12}{R_0^{13}}-12.25\sigma^6\frac{-6}{R_0^7}\right]=\frac{2N\epsilon6\sigma^6}{R_0^7}\left[12.25-9.11\sigma^6\frac{2}{R_0^6}\right]
    \end{equation}
    Luego,
    \begin{equation}
        12.25-9.11\sigma^6\frac{2}{R_0^6}=0\Rightarrow12.25=18.22\left(\frac{\sigma}{R_0}\right)^6\Rightarrow\left(\frac{R_0}{\sigma}\right)^6=1.49\Rightarrow\frac{R_0}{\sigma}=1.068
    \end{equation}
    Por tanto, la distancia $R_0$ entre vecinos más cercanos en el equilibrio para una red BCC es $R_0=1.068\sigma$. Este $R_0$ es menor que la distancia de equilibrio para un par de átomos aislados, que recordamos que es $R_0=1.12\sigma$.\\
    Ahora, sustituyendo este valor de $R_0$ en la ecuación de la energía, obtendremos la energía potencial en el equilibrio, tal que
    \begin{equation}
        U_C(R_0)=2N\epsilon\left[9.11\frac{1}{1.068^{12}}-12.25\frac{1}{1.068^6}\right]=-8.24N\epsilon
    \end{equation}
    Entonces, la energía potencial en el equilibrio por átomo es, $U_C(R_0)/N=-8.24\epsilon$. Por tanto, la energía de enlace por átomo es $8.24\epsilon$.
\end{itemize}
\section{Cristales iónicos}
\subsection{Introducción}
Los \textit{cristales iónicos son compuestos en los cuales prevalece el carácter del enlace químico, en cuya base se encuentra la interacción electrostática entre los iones con carga}. Son representantes típicos de los cristales iónicos los haluros de los metales alcalinos, por ejemplo, del tipo de estructura NaCl y CsCl.\\ \\
Al formarse los cristales del tipo NaCl, los átomos de los alógenos, que tienen gran afinidad por el electrón, capturan los electrones de valencia de los metales alcalinos, que tienen bajos potenciales de ionización, y con esto se forman iones positivos y negativos cuyas capas electrónicas se asemejan a las capas $s^2p^6$ de simetría esférica, llenas, de los gases inertes más próximos. Como resultado de la atracción coulombiana de los aniones y los cationes se produce la superposición de los seis orbitales $p$ exteriores y se forma la red del tipo NaCl cuya simetría y número de coordinación, igual a 6, corresponde a los seis enlaces de valencia de cada átomo con sus vecinos. En resumen,
\begin{tcolorbox}[title=Cristales iónicos]
En los cristales iónicos la atracción está condicionada principalmente por la interacción coulombiana entre los iones cargados. Además de la atracción entre los iones con cargas de signos contrarios, existe en ellos la repulsión debida, por una parte, a la repulsión de las cargas de igual signo y, por otra, a la acción del principio de exclusión de Pauli, ya que cada ion tiene configuraciones electrónicas estables, como las de los gases inertes, con las capas llenas.\\ \\
En el modelo simple del cristal iónico puede admitirse que los electrones son esferas rígidas, impenetrables y cargadas, aunque en realidad, bajo la acción de los campos eléctricos de los iones vecinos, la forma esférica simétrica de los iones se altera algo a causa de la polarización.
\end{tcolorbox}
\subsection{Energía potencial de interacción entre iones}
Como en el caso de los cristales moleculares, para calcular la energía de enlace de los cristales iónicos, vamos a partir de las representaciones clásicas ordinarias, considerando que los iones se encuentran en nudos de la red cristalina, su energía cinética es insignificante, y las fuerzas que actúan entre ellos son centrales. \\ \\
Tendremos dos términos en el potencial, uno debido a la fuerza repulsiva debida al principio de exclusión de Pauli (es de corto alcance y muy intenso), que modelamos por
\begin{equation}
    \lambda e^{-r_{ij}/\rho}\hspace{4mm}\text{o bien}\hspace{4mm}\frac{b}{r_{ij}^n}
\end{equation}
donde $r_{ij}$ es la distancia entre los iones $i$ y $j$; $n$, $\lambda$, $\rho$ y $b$ son parámetros (determinados experimentalmente). Tomaremos la representación exponencial.\\ \\
Por otro lado, tendremos un término debido a la fuerza electrostática entre dos iones,
\begin{equation}
    \pm\frac{1}{4\pi\epsilon_0}\frac{q^2}{r_{ij}}
\end{equation}
donde el signo $+$ es para iones de igual signo (se repelen), el signo $-$, para iones de distinto signo (se atraen) y $q$ es el valor absoluto de la carga de los iones. Entonces, la energía potencial total entre dos iones queda,
\begin{equation}
    U_{ij}=\lambda e^{-r_{ij}/\rho}\pm\frac{1}{4\pi\epsilon_0}\frac{q^2}{r_{ij}}
\end{equation}
donde denotamos $U_{ij}\equiv U(r_{ij})$.\\ \\
Aplicando el mismo razonamiento que usamos para los cristales moleculares, podemos escribir $r_{ij}=p_{ij}R$, donde $R$ es la distancia entre vecinos más cercanos. Por tanto, teniendo en cuenta que \textit{el principio de exclusión de Pauli es importante solo entre los iones más cercanos y que los iones más cercanos tienen signo opuesto}, tendremos que
\begin{equation}
    U_{ij}(R)=\left\lbrace\begin{matrix}
        \lambda e^{-R/\rho}-\frac{q^2}{4\pi\epsilon_0R} & \text{si }i\text{ y }j\text{ son vecinos más cercanos}\\ \\
        \pm\frac{1}{4\pi\epsilon_0}\frac{q^2}{p_{ij}R} & \text{si }i\text{ y }j\text{ no son vecinos más cercanos}
    \end{matrix}\right.
\end{equation}
Este desarrollo lo hemos razonado para dos iones, luego, tomando un cristal con $2N$ iones, la energía potencial de todo el cristal es,
\begin{equation}
    U_C(R)=\frac{1}{2}\sum_{i=1}^{2N}\sum_{\overset{j=1}{j\neq i}}^{2N}U_{ij}(R)=N\sum_{\overset{j=1}{j\neq i}}^{2N}U_{ij}(R)
\end{equation}
Por tanto,
\begin{equation}
    U_C(R)=N\left[Z\lambda e^{-R/\rho}-\frac{q^2}{4\pi\epsilon_0R}\underbrace{\sum_{\overset{j=1}{j\neq i}}^{2N}\frac{\pm1}{p_{ij}}}{\alpha}\right]=N\left[Z\lambda e^{-R/\rho}-\frac{q^2}{4\pi\epsilon_0R}\alpha\right]
\end{equation}
donde $Z$ es el número de vecinos más cercanos y $\alpha$ es la denominada \textbf{constante de Madelung}, siendo una suma geométrica, en la que el signo $+$ es para iones con cargas de signo diferente, y el $-$ para iones con cargas de igual signo, pues hemos sacado un signo menos factor común.\\ \\
Esta constante de Madelung puede obtenerse igual que las sumas estructurales anteriores. Vamos a calcular la del NaCl, recordando que la estructura de tipo NaCl es una red FCC con iones alternos. Luego, 
\begin{Figura}
    \centering
    \includegraphics[width=0.4\linewidth]{Imagenes//Capitulo3/NaCl.png}
    \captionof{figure}{Esquema para el cálculo de la constante de Madelung para la estructura NaCl.}
    \label{cap3-NaCl}
\end{Figura}
En este caso,
\begin{equation}
    \frac{\alpha}{R}=\frac{1}{R}\sum_{j\neq i}\frac{1}{p_{ij}}=\frac{1}{R}\left(\frac{6}{\sqrt{1}}-\frac{12}{\sqrt{2}}+\frac{8}{\sqrt{3}}-\frac{6}{\sqrt{4}}+\frac{24}{\sqrt{5}}-\dots\right)=\frac{1.748}{R}
\end{equation}
Los números $+6$, $-12$, $+8$, $-6$, $+24$, etc., aparecen porque a la distancia $R\sqrt{1}$ el ion $Cl^-$ está rodeado por 6 iones $Na^+$; a la distancia $R\sqrt{2}$, por 12 iones $Cl^-$, etc. Además, al tener 6 vecinos más cercanos, $Z=6$. Luego, la distancia de equilibrio será,
\begin{equation}
    0=\left.\frac{dU}{dR}\right|_{R_0}\Rightarrow Z\lambda e^{-R_0/\rho}=\frac{q^2\alpha\rho}{4\pi \epsilon_0 R_0^2}
\end{equation}
Luego, despejando $R_0$ e introduciendo los datos del NaCl obtenemos que $R_0\sim10\rho$.
\begin{note}
    $R_0\sim10\rho\gg\rho$ es consistente con considerar interacción repulsiva por el principio de exclusión de Pauli solo entre vecinos más cercanos.
\end{note}
La energía del cristal en equilibrio es $U_C(R_0)$, tal que
\begin{equation}
    U_C(R_0)=N\left[\frac{q^2\alpha\rho}{4\pi\epsilon_0R_0^2}-\frac{q^2\alpha}{4\pi\epsilon_0R_0}\right]=\underbrace{-\frac{N\alpha q^2}{4\pi\epsilon_0R_0}}_{E_M}\left[1-\frac{\rho}{R_0}\right]
\end{equation}
donde $E_M$ se denomina \textbf{energía de Madelung}, que permanece inalterable al cambiar de modelo repulsivo, pues el modelo atractivo siempre es electrostático. Esta energía da el aporte electrostático a la energía de enlace, siendo principal pues $\rho/R_0\sim0.1$. La energía de enlace es $-U_C(R_0)>0$.
\\ \\
Vamos a obtener la energía de Madelung usando el modelo repulsivo polinómico, $b/r_{ij}^n$, usando que
\begin{equation}
    U_{ij}=\frac{b}{r_{ij}^n}+\frac{1}{4\pi\epsilon_0}\frac{q^2}{r_{ij}}
\end{equation}
sumamos las interacciones por todo el cristal, suponiendo $2N$ iones,
\begin{equation}
    U_C=\frac{\cancel{2}N}{\cancel{2}}\left[\underbrace{\sum_{j=1}^N\frac{1}{p_{ij}^n}}_{A_n}\left(\frac{b}{R^n}\right)-\frac{q^2}{4\pi\epsilon_0R}\underbrace{\sum_{j=1}^N\frac{\pm1}{p_{ij}}}_{\alpha}\right]=N\left[A_n\left(\frac{b}{R^n}\right)-\frac{q^2\alpha}{4\pi\epsilon_0R}\right]
\end{equation}
Aplicando la condición de equilibrio,
\begin{equation}
    \left.\frac{dU(R)}{dR}\right|_{R_0}=0\Rightarrow N\left[-A_n\left(\frac{nb}{R_0^{n+1}}\right)+\frac{q^2\alpha}{4\pi\epsilon_0R_0^2}\right]=0
\end{equation}
Despejamos $R_0$,
\begin{equation}
    A_n\frac{nb}{R_0^{n+1}}=\frac{q^2\alpha}{4\pi\epsilon_0R_0^2}\Rightarrow\frac{R_0^{n+1}}{R_0^2}=R_0^{n-1}=\frac{nbA_n4\pi\epsilon_0}{q^2\alpha}\Rightarrow R_0=\sqrt[n-1]{\frac{4\pi\epsilon_0nbA_n}{q^2\alpha}}
\end{equation}
Luego, sustituyendo en el potencial,
\begin{equation}
    U_C(R_0)=-\frac{N\alpha q^2}{4\pi\epsilon_0R_0}\left[1-\frac{1}{n}\right]
\end{equation}
donde vemos que la energía de Madelung es la misma que en el caso anterior. Además, asumiendo que $R_0\sim10\rho$, usando el mismo modelo para el mismo cristal, tenemos que
\begin{equation}
    \cancel{-\frac{N\alpha q^2}{4\pi\epsilon_0R_0}}\left[1-\frac{1}{n}\right]=\cancel{-\frac{N\alpha q^2}{4\pi\epsilon_0R_0}}\left[1-\frac{\rho}{R_0}\right]\Rightarrow\frac{1}{n}=\frac{\rho}{R_0}\sim\frac{\cancel{\rho}}{10\cancel{\rho}}\Rightarrow n=10
\end{equation}
Ahora vamos a proceder a calcular el módulo de compresibilidad ($B$), que se relaciona con el parámetro de red $a$ de los cristales iónicos. Sabiendo que tendremos un mínimo $a_0$ del parámetro de red en el mínimo de la energía $E_0$, por lo que, haciendo un desarrollo en serie de la energía tenemos,
\begin{equation}
    E(a)=E_0+(a-a_0)\cancelto{0}{\left.\frac{\partial E}{\partial a}\right|_{a_0}}+\frac{1}{2}(a-a_0)^2\left.\frac{\partial^2E}{\partial a^2}\right|_{a_0}
\end{equation}
por lo que podremos darle un comportamiento parabólico en torno al mínimo de energía. La idea calve es partir de un valor experimental del módulo de compresibilidad, que nos permita obtener un valor de los parámetros de ajuste, $n$.\\ \\
Por definición, el módulo de compresibilidad, $\kappa$, es
\begin{equation}
    \frac{1}{\kappa}=-V\left(\frac{dP}{dV}\right)
\end{equation}
Luego, suponiendo $T=cte$, por el Primer Principio de la Termodinámica, tenemos que 
\begin{equation}
    dU=dQ-pdV
\end{equation}
Luego,
\begin{equation}
    \frac{1}{\kappa}=V\left(\frac{\partial^2U}{\partial V^2}\right)
\end{equation}
Hacemos el cambio de variable $U=U(R)$, tal que
\begin{equation}
    \frac{\partial U}{\partial V}=\frac{\partial U}{\partial R}\frac{\partial R}{\partial V}=\underbrace{\frac{\partial}{\partial V}\left(\frac{\partial U}{\partial R}\right)\frac{\partial R}{\partial V}}_{\alpha}+\underbrace{\frac{\partial U}{\partial R}\frac{\partial}{\partial V}\left(\frac{\partial R}{\partial V}\right)}_{\tilde{\beta}}
\end{equation}
donde $\frac{\partial}{\partial V}=\frac{\partial}{\partial R}\left(\frac{\partial R}{\partial V}\right)$, por tanto
\begin{equation}
    \alpha=\frac{\partial }{\partial R}\left(\frac{\partial U}{\partial R}\right)\frac{\partial R}{\partial V}\frac{\partial R}{\partial V}=\frac{\partial^2U}{\partial R^2}\left(\frac{\partial R}{\partial   V}\right)^2
\end{equation}
\begin{equation}
    \tilde{\beta}=\left(\frac{\partial U}{\partial R}\right)\frac{\partial }{\partial R}\left(\frac{\partial R}{\partial V}\right)\frac{\partial R}{\partial V}=\frac{\partial}{\partial R}\left(\frac{\partial R}{\partial V}\right)\frac{\partial R}{\partial V}\cancelto{0}{\left(\frac{\partial U}{\partial R}\right)}=0
\end{equation}
donde anulamos la derivada de $U$ respecto a $R$ porque en el punto de equilibrio es un mínimo. Luego, sustituyendo tenemos que
\begin{equation}
    \frac{1}{\kappa}=V\left(\frac{\partial^2U}{\partial R^2}\right)_{R_0}\left(\frac{\partial R}{\partial V}\right)^2_{R_0}
\end{equation}
donde $\left(\frac{\partial^2U}{\partial R^2}\right)_{R_0}$ dependerá del modelo de interacción y $\left(\frac{\partial R}{\partial V}\right)^2_{R_0}$ dependerá de la geometría del cristal.\\ \\
Recordando que para el NaCl tenemos que $R_0\sim10\sigma$ con $n\sim10$, tenemos que
\begin{equation}
    \frac{\partial^2U_C(R)}{\partial R^2}=\frac{\partial^2}{\partial 2}\left[NA_n\frac{b}{R^n}-N\frac{e^2\alpha}{4\pi\epsilon_0R}\right]=N\left[\underbrace{A_nn(n+1)\frac{b}{R^{n+2}}}_{\gamma}-\frac{2\alpha e^2}{4\pi\epsilon_0R^3}\right]
\end{equation}
Aplicando la condición de equilibrio, vemos que
\begin{equation}
    \frac{nbA_n}{R_0^{n+1}}=\frac{e^2\alpha}{4\pi\epsilon_0 R_0^2}
\end{equation}
multiplicando por $\frac{n+1}{R_0}$ a ambos lados, vemos que
\begin{equation}
    \frac{n(n+1)A_nb}{R_0^{n+2}}=\gamma|_{R_0}=\frac{e^2\alpha(n+1)}{4\pi\epsilon_0 R_0^3}
\end{equation}
Por tanto, en $R=R_0$, tendremos que
\begin{equation}
\left.\frac{\partial^2U_C}{\partial R^2}\right|_{R_0}=N\left[\frac{e^2\alpha(n+1)}{4\pi\epsilon_0R_0^3}-\frac{2e^2\alpha}{4\pi\epsilon_0R_0^3}\right]=N\frac{e^2\alpha(n-1)}{4\pi\epsilon_0R_0^3}>0
\end{equation}
que vemos que es positivo porque es un mínimo. Ahora calculamos la otra derivada,
\begin{equation}
    \left(\frac{\partial R}{\partial V}\right)^2=\frac{1}{\left(\frac{\partial V}{\partial R}\right)^2}
\end{equation}
Para calcular el volumen usamos que $v=\vec{a}_1\cdot(\vec{a}_2\times\vec{a}_3)$, que es el volumen de la celda primitiva y los vectores $\vec{a}_i$ son los vectores de la red. Después lo multiplicaremos por el número de celdas que tenemos en el cristal. Suponiendo que estamos calculando el cristal de NaCl, sabemos que su estructura es tipo FCC, luego, recordando los vectores de la estructura FCC, tendremos
\begin{equation}
    \vec{a}_1=\frac{a}{2}(1,1,0);\hspace{5mm}\vec{a}_2=\frac{a}{2}(0,1,1);\hspace{5mm}\vec{a}_3=\frac{a}{2}(1,0,1)
\end{equation}
Luego,
\begin{equation}
    v=\frac{a^2}{4}\vec{a}_1\cdot\begin{vmatrix}
        \hat{i} & \hat{j} & \hat{k}\\
             0 &        1 &     1\\
             1 &        0 &     1
    \end{vmatrix}=\frac{a^3}{8}\begin{pmatrix}1 &1 &0\end{pmatrix}\cdot\begin{pmatrix}
        1\\
        1\\
        -1
    \end{pmatrix}=\frac{a^3}{4}
\end{equation}
Recordando que el parámetro de red es todo el lado del cubo, el radio será la mitad de este, luego $a=2R$. Por tanto, el volumen será
\begin{equation}
    v=\frac{8R^3}{4}=2R^3
\end{equation}
Suponiendo que tenemos $N$ pares de iones en el cristal, el volumen del cristal NaCl\footnote{La generalización del volumen de los cristales que hemos dado en clase es usando que $V=2N\beta R^3$, donde $\beta$ es un parámetro que depende del cristal y tomamos como $\beta=1$ para este caso. En la referencia \cite{alma991010074599707712} aparece que el volumen general es $V=N\gamma R^3$, donde $\gamma$ es un factor que depende del cristal, siendo $\gamma=2$ para el NaCl, $\gamma=1.54$ para el CsCl, $\gamma=3$ para el ZnS, etc.} será,
\begin{equation}
    V=2NR^3
\end{equation}
Luego,
\begin{equation}
    \frac{\partial V}{\partial R}=6NR^2\Rightarrow\left(\frac{\partial V}{\partial R}\right)^2=36N^2R^4\Rightarrow\left(\frac{\partial R}{\partial V}\right)^2_{R_0}=\frac{1}{36N^2R_0^4}
\end{equation}
Por tanto,
\begin{equation}
    \frac{1}{\kappa}=2NR_0^3\frac{Ne^2\alpha}{4\pi\epsilon_0 R_0^3}\frac{(n-1)}{36 N^2R_0^4}=\frac{e^2\alpha(n-1)}{4\pi\epsilon_0}\frac{1}{18 R_0^4}
\end{equation}
Despejando $n$\footnote{Usando la generalización del volumen, tenemos que $n=1+\frac{4\pi\epsilon_09\gamma R_0^4}{Z_1Z_2e^2\alpha\kappa}$, donde $Z_1$ y $Z_2$ son el número de iones por átomo, para el NaCl, $Z_1=Z_2=1$, pues tenemos solo un ion de Na y otro de Cl.} vemos que
\begin{equation}
    n=1+\frac{18 R_0^44\pi\epsilon_0}{\alpha e^2\kappa}
\end{equation}
Usando los valores experimentales del NaCl, $\kappa=3.3\cdot10^{-11}$ m$^2$/N, $\alpha=1.75$, $R_0=2.81$ \AA, $e=1.6\cdot10^{-19}$ C y $\epsilon_0=1/(4\pi9\cdot10^9)$, tenemos que $n=1+8,44=9.44$. 
Luego, la energía potencial es $U_C(R_0)=12.9\cdot10^{-19}$ J ($\sim8$ eV), cosa que concuerda con el valor experimental ($\sim7.9$ eV).

\lhead{\emph{Capítulo 4: Vibraciones de la red}}
\chapter{Vibraciones de la red. Propiedades térmicas.}
En el sólido, los átomos a toda temperatura, incluso a $0$ K, realizan sin cesar vibraciones (oscilaciones) alrededor de su posición de equilibrio media. Cuando las amplitudes de las oscilaciones son pequeñas, estas pueden considerarse \textbf{armónicas}. Al elevar la temperatura, la excitación de las vibraciones de uno de los átomos se transmite a los átomos más próximos, los cuales, a su vez, la comunican a sus vecinos y así sucesivamente. Este proceso es semejante al de propagación de las ondas sonoras en un sólido, por lo que podremos interpretar las vibraciones como un conjunto de ondas elásticas de distinta longitud de onda que interaccionan y se propagan por todo el volumen del cristal. Como el sólido tiene dimensiones limitadas, a una temperatura dada se establece un estado estacionario de vibraciones, como resultado de la superposición de las ondas estacionarias.
\section{Cristal unidimensional monoatómico.}
Como modelo unidimensional de sólido, vamos a considerar una cadena de $N$ átomos iguales, de masa $M$ y distancia interatómica $a$, que puedan desplazarse a lo largo de una recta.
\begin{Figura}
    \centering
    \includegraphics[width=0.75\textwidth]{Imagenes//Capitulo4/Cadena.png}
    \captionof{figure}{Cadena lineal de átomos iguales.}
    \label{Cap4-cadena}
\end{Figura}
Cada átomo posee en este sistema un grado de libertad, en tanto que el sistema en conjunto, $N$ grados de libertad.\\
Con este modelo vamos a estudiar las propiedades de los modos normales de vibración. En principio, lo analizaremos utilizando como hipótesis la interacción, mediante la Ley de Hooke, solo a vecinos más próximos. Es decir, sea $u_n(x,t)$ el desplazamiento en cierto instante $t$ del $n-$ésimo átomo respecto de sus posición de equilibrio en el punto de coordenadas $x_n=na$, si los desplazamientos de los átomos de las posiciones de equilibrio son pequeños en comparación con la distancia $a$, las fuerzas de interacción entre los átomos se pueden considerar cuasiestáticas; de acuerdo con la Ley de Hooke, estas fuerzas serán proporcionales a los desplazamientos, tal que
\begin{equation}
    F_{(n+1)\to n}=\alpha(u_{n+1}-n_u)
\end{equation}
donde $u_i$ representa el desplazamiento del equilibrio del átomo $i$. Los átomos estarán unidos como por muellecitos elásticos, cada uno de los cuales se caracteriza por la constante de elasticidad $C$, y el desplazamiento $u_n$ describe las vibraciones del átomo en torno a la posición de equilibrio.
\begin{Figura}
    \centering
    \includegraphics[width=0.8\textwidth]{Imagenes//Capitulo4/muelle.png}
    \captionof{figure}{Ilustración de la cadena monoatómica con interacciones a primeros vecinos. El desplazamiento de cada átomo respecto a a su posición de equilibrio lo llamamos $u_n$
 donde $n$ es la celda de la red de Bravais donde se encuentra el átomo.}
    \label{cap4-muelle}
\end{Figura}
Hallemos la ecuación del movimiento del $n-$ésimo átomo. Vamos a suponer que solo actúan fuerzas de corta acción, es decir, el átomo solo interacciona con sus vecinos más cercanos, el $(n-1)-$ésimo y el $(n+1)-$ésimo, ya que la acción que ejercen sobre él el resto de átomos es despreciable. Luego,
\begin{equation}
    F_n=\alpha(u_{n+1}-u_n)-\alpha(u_n-u_{n-1})=\alpha(u_{n+1}+u_{n-1}-2u_n)
\end{equation}
donde $\alpha$ es una constante de fuerza relacionada con la constante de elasticidad por la expresión $C=\alpha a$. Aplicando la Segunda Ley de Newton, tenemos que
\begin{equation}
    M\frac{d^2u_n}{dt^2}=\alpha(u_{n+1}+u_{n-1}-2u_n)
\end{equation}
siendo una ecuación diferencial de segundo orden. Sabemos que la solución a este tipo de ecuaciones son del tipo de soluciones de onda plana, tal que
\begin{equation}
    u(x_n,t)=u_n(q)=u_0e^{i(qx_n-\omega(q) t)}
\end{equation}
donde $u_0$ representa el desplazamiento del átomo con $n=0$ en el instante $t=0$, $q=2\pi/\lambda$ es el número de onda y $\omega(q)$ la frecuencia angular del modo dado.\\ \\
Sustituyendo la solución en la Segunda Ley de Newton, teniendo en cuenta que $x_{n+1}=x_n+a$ y $x_{n-1}=x_n-a$, tenemos que
\begin{equation}
    -M\omega^2 \cancel{e^{iqx_n}}=\alpha \cancel{e^{iqx_n}}\left[e^{iqa}+e^{-iqa}-2\right]
\end{equation}
Aplicando que $\cos\alpha=\frac{e^{i\alpha}+e^{-i\alpha}}{2}$, tenemos que
\begin{equation}
    M\omega^2=-\alpha\left[2\cos(qa)-2\right]=2\alpha\left[1-\cos(qa)\right]
\end{equation}
Usando que $\sin(2\alpha)=\frac{1-\cos\alpha}{2}$, obtenemos
\begin{equation}
    M\omega^2=4\alpha\sin^2\left(\frac{qa}{2}\right)
\end{equation}
Despejando $\omega$ obtenemos la \textbf{relación de dispersión},
\begin{equation}
    \omega^2=\frac{4\alpha}{M}\sin^2\left(\frac{qa}{2}\right)\Rightarrow\omega(q)=\omega_m\left|\sin\left(\frac{qa}{2}\right)\right|
\end{equation}
donde usamos que las frecuencias solo pueden ser positivas con $\omega_m=\sqrt{4\alpha/M}$.\\ \\
El resultado de la relación de dispersión nos indica que los parámetros de dependencia temporal, $\omega$, y espacial, $q$, están ligados en nuestro modelo mediante una función periódica, lo que nos permite restringir nuestro estudio a una zona donde se dé esa periodicidad,
\begin{Figura}
    \centering
    \includegraphics[width=0.6\linewidth]{PZB.png}
    \captionof{figure}{Curva de dispersión para una cadena lineal monoatómica.}
    \label{cap4-PZB}
\end{Figura}
Observando la Figura \ref{cap4-PZB}, vemos que el intervalo entre $-\pi/a$ y $\pi/a$ será periódico, por lo que esta zona será la Primera Zona de Brillouin. En este resultado podemos destacar:
\begin{enumerate}
    \item \textit{La solución $\omega(q)$ es periódica con periodo $\frac{2\pi}{a}$, que corresponde al tamaño de la Primera Zona de Brillouin (PZB) para este modelo.}
    \begin{proof}
        Para demostrar esto, vamos a imponer que 
        \begin{equation}
            q'=q+\frac{2\pi}{a}
        \end{equation}
        donde $q'$ está en la PZB. Luego, sustituyendo en la relación de dispersión tendremos que
        \begin{equation}
            \omega(q')=\omega\left(q+\frac{2\pi}{a}\right)=\omega_m\left|\sin\left(\frac{qa}{2}+\frac{2\pi a}{2 a}\right)\right|=\omega_m\left|\sin\left(\frac{qa}{2}+\pi\right)\right|=\omega_m\left|\sin\left(\frac{qa}{2}\right)\right|=\omega(q)
        \end{equation}
        pues
        \begin{equation}
            \sin\left(\frac{q'a}{2}\right)=\sin\left(\frac{qa}{2}+\pi\right)=\sin\left(\frac{qa}{2}\right)\cos(\pi)+\cos\left(\frac{qa}{2}\right)\sin(\pi)=-\sin\left(\frac{qa}{2}\right)
        \end{equation}
        Luego,
        \begin{equation}
            \left|\sin\left(\frac{q'a}{2}\right)\right|=\left|-\sin\left(\frac{qa}{2}\right)\right|=\left|\sin\left(\frac{qa}{2}\right)\right|
        \end{equation}
        Entonces, vemos que el intervalo $[-\pi/a,\pi/a]$ será el menor intervalo donde se contiene $\omega(q)$, pues si tenemos la relación de dispersión fuera de este, por periodicidad, podremos volver a este intervalo, que corresponderá con la PZB. Además, el tamaño de la PZB será $2\pi/a$, pues $\pi/a+\pi/a=2\pi/a$.
    \end{proof}
    \item \textit{En el límite $q\to0$, podremos considerar el sistema continuo y la velocidad de fase y de grupo de las ondas será igual a la velocidad del sonido en el material.}
    \begin{proof}
    En los valores $q\to0$, con la aproximación de $\sin\alpha\approx\alpha$, tenemos que 
    \begin{equation}
        \omega(q)\approx\omega_m\frac{a}{2}|q|
    \end{equation}
    Como recordamos, para una onda plana la velocidad de fase viene dada por la relación,
    \begin{equation}
        v_f=\frac{\omega}{q}
    \end{equation}
    en nuestro caso, al ser $q\to0$, implica que $\lambda\to\infty$ y $\lambda\gg a$, por lo que el sistema se puede considerar continuo y $v_f=v_{sonido}\equiv v_s$ con
    \begin{equation}
        v_s=\omega_ma/2
    \end{equation}
    Para la velocidad de grupo, tendremos que
    \begin{equation}
        v_g=\frac{\partial\omega}{\partial q}=\omega_m\frac{a}{2}=v_s
    \end{equation}
    para la aproximación de $q\to0$.\\
    Sin usar esta aproximación, las velocidades de fase y de grupo no serán exactamente igual a la del sonido, sino que serán,
    \begin{equation}
        v_s=\frac{\omega}{|q|}=\frac{\omega_m}{|q|}\left|\sin\left(\frac{qa}{2}\right)\right|\times\frac{a/2}{a/2}=\omega_m\frac{a}{2}\left|\frac{\sin\left(qa/2\right)}{qa/2}\right|=v_s\left|\frac{\sin\left(qa/2\right)}{qa/2}\right|
    \end{equation}
    y
    \begin{equation}
        v_g=\frac{\partial \omega}{\partial q}=\omega_m\frac{a}{2}\left|\cos\left(\frac{qa}{2}\right)\right|=v_s\left|\cos\left(\frac{qa}{2}\right)\right|
    \end{equation}
    \end{proof}
    \item \textit{Todos los modos de oscilación están contenidos en la Primera Zona de Brillouin.} 
    \begin{proof}
    Para demostrarlo partimos de que cualquier número de onda $q'$ se puede expresar como,
    \begin{equation}
        q'=q+\frac{2\pi}{a}m
    \end{equation}
    donde $q'$ está en la PZB y $m\in\mathbb{Z}$. El modo de oscilación asociado a $q'$ será,
    \begin{equation}
        u_n(q')=u_n\left(q+\frac{2\pi}{a}m\right)=u_0\exp\left\lbrace\underbrace{i\left(q+\frac{2\pi}{a}m\right)(na)}_{(I)}-\underbrace{i\omega\left(q+\frac{2\pi}{a}m\right)t}_{(II)}\right\rbrace
    \end{equation}
    En el argumento exponencial $(I)$, tenemos que
    \begin{equation}
        i\left[qna+\frac{2\pi}{a}mna\right]=i\left[qna+2\pi mn\right]\Rightarrow e^{iqna}\cancelto{1}{e^{2\pi mn}}=e^{iqna}
    \end{equation}
    donde la segunda exponencial es 1 porque $n,m\in\mathbb{Z}$, luego, aplicando Euler, tenemos que $e^{i2\pi nm}=\cancelto{1}{\cos(2\pi nm)}+i\cancelto{0}{\sin(2\pi nm)}=1$.\\
    En el argumento exponencial $(II)$, tenemos que
    \begin{equation}
        \omega\left(q+\frac{2\pi}{a}m\right)=\omega(q)
    \end{equation}
debido a la periodicidad de la relación de dispersión, pues el periodo $\frac{2\pi}{a}$ corresponde al tamaño de la PZB.\\
Por lo tanto,
\begin{equation}
    u_n\left(q+\frac{2\pi}{a}m\right)=u_0e^{i\left[qna-\omega(q)t\right]}=u_n(q)
\end{equation}
y como queremos demostrar, cualquier modo de oscilación fuera de la PZB está representada por un modo dentro de la PZB, por lo que la PZB es suficiente para describir el modelo propuesto.
\end{proof}
\item \textit{La forma de los valores de $q$ son discretos.}
\begin{proof}
Para determinar la forma del número de onda $q$ vamos a utilizar una ecuación más, que es la condición cúbica de frontera (Born-Von Kármán). Es decir,
\begin{equation}
    u(x_n\pm L,t)=u(x_n,t)
\end{equation}
lo que implica que
\begin{equation}
    u_0=e^{i\left[qx_n\pm L-\omega t\right]}=u_0e^{i\left[qx_n-\omega t\right]}\Rightarrow e^{\pm iqL}=1\Rightarrow qL=2\pi m\Rightarrow q_m=\frac{2\pi}{L}m
\end{equation}
siendo $m\in\mathbb{Z}$, por lo tanto, los valores de $q$ son discretos; debido a que $L$ es grande, los valores discretos se podrán asemejar a un cuasi-continuo.
\end{proof}
\end{enumerate}
\begin{remark}
    A cada $q_m$ corresponde un modo de oscilación con longitud de onda $\lambda_m=2\pi/q_m$ y frecuencia $\omega(q_m)$.
\end{remark}
\begin{remark}
Describiremos los modos de oscilación sólo con vectores en
la primera zona de Brillouin (PZB). En vez de usar un $q’$ que
esté fuera de esa zona, le restaremos o sumaremos al $q’$ las
veces que sea necesaria el valor $2\pi/a$, de forma que el resultado sea un $q$ dentro de la PZB. Tendremos una única rama, llamada rama acústica, cuya pendiente (en la zona $q\sim0$, $\lambda\gg a$) corresponde con la velocidad del sonido.
\end{remark}
Terminamos calculando una magnitud, $\rho_q$, denominada \textbf{densidad de estados} en el espacio recíproco. Sabemos que el número total de modos es igual al número de grados de libertad del modelo, que son $N$, y calculamos mediante la analogía de la densidad normal, sabiendo que
\begin{equation}
    \rho=\frac{M}{V}
\end{equation}
donde $M$ es la masa y $V$ el volumen. Para nuestra 'densidad', vamos a identificar la masa con el número de grados de libertad y el volumen con la longitud de la PZB, pues es unidimensional, así tendremos que
\begin{equation}
    \rho_q=\frac{N}{L_{PZB}}=\frac{L/a}{2\pi/a}=\frac{L}{2\pi}
\end{equation}
que será la densidad de estados en el espacio recíproco unidimensional, discreto y restringido a la PZB. Además, como el número de modos en la PZB coincide con el número de grados de libertad del sistema, tendremos que
\begin{equation}
    \rho_q\frac{2\pi}{a}=\frac{L}{2\pi}\frac{2\pi}{a}=\frac{L}{a}=\frac{Na}{a}=N
\end{equation}
\begin{remark}
$~~$
    \begin{itemize}
        \item Un modo de oscilación es un movimiento colectivo de los átomos del cristal (los involucra a todos) dado por $u(x_n,t)=u_n(q)=u_0e^{i(qx_n-\omega t}$ con $q\equiv q_m$.
        \item Cualquier oscilación admisible de los átomos del cristal puede expresarse como superposición de modos de oscilación. Esto no es más que el desarrollo de Fourier de una oscilación de los átomos del cristal,
        \begin{equation}
            u(x_n,t)=\frac{1}{\sqrt{N}}\sum_{q_m\in PZB}\xi(q_m)e^{i\left(q_mx_n-\omega(q_m)t\right)}\approx\frac{L}{2\pi}\int_{-\pi/a}^{\pi/a}\xi(q)e^{i\left(qx_n-\omega(q)t\right)}dq
        \end{equation}
        donde $\sqrt{N}$ es un factor de normalización y la segunda igualdad se basas en que los valores $q_m$ forman un cuasi-continuo.
    \end{itemize}
\end{remark}
Ahora vamos a proceder a calcular\textbf{ la relación de dispersión} $\mathbf{\omega(q)}$ \textbf{general}, es decir, considerando la interacción hasta los vecinos $n-$ésimos. Para ello, sabemos que para cada uno de los átomos de la cadena lineal, la segunda ley de Newton la podemos escribir, de acuerdo a la ley de Hooke, como
\begin{equation}
    M\omega^2u_n=-\sum_{l=\pm1}^{\pm N}C_l(u_{n+l}-u_n)
\end{equation}
donde hemos tomado un átomo de referencia en $n$, $l$ es un índice tanto positivo como negativo y $C_l$ es la constante de fuerza que cumple que $C_l>C_{l+1}$. Además, la solución anterior es,
\begin{equation}
    u_n=u_0e^{i(qna-\omega t)}
\end{equation}
donde usamos que $x_n=na$. Luego, esta solución la podremos extender a nuestro sistema sumando los términos por distancia al átomo de referencia, tal que
\begin{equation}
\hspace*{-0.8cm}
    \sum_{l=\pm1}^{\pm N}(u_{n+l}-u_n)=\sum_{l=\pm 1}^{\pm N}\left[u_0 e^{i(q(n+l)a-\omega t)}-u_0e^{i(qna-\omega t)}\right]=-u_0e^{e^{iqna-\omega t}}\sum_{l=\pm1}^{\pm N}\left[1-e^{iqla}\right]=-u_n\sum_{l=\pm1}^{\pm N}\left[1-e^{iqla}\right]
\end{equation}
Entonces, sustituyendo en la segunda Ley de Newton tenemos,
\begin{equation}
    \cancel{u_n}M\omega^2=\cancel{u_n}\sum_{l=\pm1}^{\pm N}C_l\left[1-e^{iqla}\right]
\end{equation}
podemos considerar simétrico el valor de la constante de fuerza, es decir, $C_l=C_{-l}$, y por tanto, quitamos los valores negativos del sumatorio, tal que
\begin{equation}
    \sum_{l=\pm1}^{\pm N}C_l\left[1-e^{iqla}\right]=\sum_{l>0}^N\left[C_l\left(1-e^{iqla}\right)+C_{-l}\left(1-e^{-iqla}\right)\right]=\sum_{l>0}^NC_l\left[\left(1-e^{iqla}\right)+\left(1-e^{-iqla}\right)\right]
\end{equation}
donde aparece un nuevo término debido a que tenemos que tener en cuenta los valores de $-l$. Por tanto,
\begin{equation}
    M\omega^2=\sum_{l>0}^NC_l\left[\left(1-e^{iqla}\right)+\left(1-e^{-iqla}\right)\right]=2\sum_{l>0}^N\left[1-\cos(qla)\right]=4\sum_{l>0}^NC_l\sin^2\left(\frac{qla}{2}\right)
\end{equation}
donde hemos aplicado que $\cos\alpha=\frac{e^{i\alpha}+e^{-i\alpha}}{2}$ y que $\sin^2\alpha=\frac{1-\cos(2\alpha)}{2}$. Luego, despejando $\omega$ tendremos que
\begin{equation}
    \omega^2=\frac{4}{M}\sum_{l>0}^NC_l\sin^2\left(\frac{qla}{2}\right)\Rightarrow \omega(q)=\sqrt{\sum_{l>0}^N\omega_{m,l}\sin^2\left(\frac{qla}{2}\right)}
\end{equation}
donde $\omega_{m,l}=\frac{4C_l}{M}$. Vemos que para $l=1$, tenemos que $C_l=\alpha$, por lo que $\omega_{m,1}=\omega_m$ y obtenemos la relación de dispersión a vecinos más próximos.
\section{Cristal unidimensional diatómico. Vibraciones de una cadena lineal diatómica}
En el apartado anterior hemos determinado los modos normales de las vibraciones de una red de Bravais unidimensional monoatómico. Ahora vamos a estudiar las vibraciones de los átomos de una red unidimensional con base, cuando a la celda unidad, con parámetro $2a$, corresponden dos átomos. Supongamos que a lo largo de la recta se encuentran $N$ celdas unidad. Este sistema posee $2N$ átomos, luego tendremos $2N$ grados de libertad.
\begin{Figura}
    \centering
    \includegraphics[width=0.7\textwidth]{Imagenes//Capitulo4/cap4-diatomica.png}
    \captionof{figure}{Cadena lineal diatómica con átomos alternos diferentes de masa $M_1$ y $M_2$.}
    \label{cap4-diatomica}
\end{Figura}
Consideramos una cadena con átomos de masa $M_1$ en las posiciones $x_{2n+1}=(2n+1)a$ y átomos de masa $M_2$ en las posiciones $x_{2n}=2na$. Suponiendo que la longitud de la cadena es $L$, y como ya hemos dicho, tendremos $N$ celdas unidad de longitud $2a$, entonces la longitud será $L=2Na$. Sean $u_{2n}$ el desplazamiento de un átomo de masa $M_2$ a lo largo de la dirección $x$, en cierto instante $t$, respecto de su posición de equilibrio, y $u_{2n+1}$, el desplazamiento de un átomo de masa $M_1$, respecto la suya. Reconsiderando que los desplazamientos son pequeños en comparación con $a$, entonces volvemos a tener que las fuerzas de interacción son cuasielásticas, que serán proporcionales a los desplazamientos, tal que
\begin{equation}
    F_{2n}=\alpha(u_{2n+1}-u_{2n})-\alpha(u_{2n}-u_{2n-1})=\alpha(u_{2n+1}+u_{2n-1}-2u_{2n})
\end{equation}
\begin{equation}
    F_{2n+1}=\alpha(u_{2n+1}-u_{2n+1})-\alpha(u_{2n+1}-u_{2n})=\alpha(u_{2n+1}+u_{2n}-2u_{2n+1})
\end{equation}
siendo $\alpha$ la constante de fuerza. Reusando la Segunda Ley de Newton, establecemos las ecuaciones de movimiento. Debemos tener en cuenta que debido a la diferencia de masa, las ecuaciones de movimiento de los átomos de masa $M_1$ son diferentes a las de los átomos $M_2$, luego
\begin{equation}
    M_1\frac{d^2u_{2n+1}}{dt^2}=\alpha(u_{2n+2}+u_{2n}-2u_{2n+1})
\end{equation}
\begin{equation}
    M_2\frac{d^2u_{2n}}{dt^2}=\alpha(u_{2n+1}+u_{2n-1}-2u_{2n})
\end{equation}
Vemos que, dando valores a $n$, tenemos un sistema de ecuaciones diferenciales acopladas para los desplazamientos de los $2N$ átomos. Teniendo en cuenta que las vibraciones de los átomos de masas diferentes pueden efectuarse con amplitudes $u_{01}$ y $u_{02}$ diferentes, las soluciones a estas ecuaciones las buscaremos en la forma de ondas progresivas del tipo,
\begin{equation}
    u_{2n+1}(x,t)\equiv u_{2n+1}=u_{01}e^{i(qx_{2n+1}-\omega t)}
\end{equation}
\begin{equation}
    u_{2n}(x,t)\equiv u_{2n}=u_{02}e^{i(qx_{2n}-\omega t)}
\end{equation}
Debemos notar que un modo normal corresponde a una forma de oscilación de todos los átomos, y por lo tanto, está dado por el conjunto de las soluciones anteriores (no por una de ellas solamente) para todos los átomos.\\ \\
Sustituyendo estas soluciones en las ecuaciones diferenciales obtenemos,
\begin{equation}
    M_1u_{01}(-\omega^2)e^{iqx_{2n+1}}\cancel{e^{-i\omega t}}=\cancel{e^{-i\omega t}}\alpha\left(u_{02}e^{iqx_{2n+2}}+u_{02}e^{iqx_{2n}}-2u_{01}e^{iqx_{2n+1}}\right)
\end{equation}
Multiplicando esta ecuación por $e^{-iqx_{2n+1}}$ y usando que $x_{2n+2}=(2n+2)a$, $x_{2n}=2na$ y $x_{2n+1}=(2n+1)a$, obtenemos que
\begin{equation}
    -u_{01}\omega^2M_1=\alpha\left[u_{02}e^{iqa}+u_{02}e^{-iqa}-2u_{01}\right]
\end{equation}
Luego,
\begin{equation}
    \left(2\alpha-\omega^2M_1\right)u_{01}-\left(2\alpha\cos(qa)\right)u_{02}=0\label{cap4-u01}
\end{equation}
De forma análoga, obtenemos
\begin{equation}
    -\left(2\alpha\cos(qa)\right)u_{01}+\left(2\alpha-\omega^2M_2\right)u_{02}=0\label{cap4-u02}
\end{equation}
Vemos que las ecuaciones (\ref{cap4-u01}) y (\ref{cap4-u02}) forman un sistema de dos ecuaciones para $u_{01}$ y $u_{02}$. Para que tenga solución no trivial, es necesario que el determinante del sistema sea nulo, es decir
\begin{equation}
    \begin{vmatrix}
        \left(2\alpha-\omega^2M_1\right) & -2\alpha\cos(qa)\\
        -2\alpha\cos(qa) & \left(2\alpha-\omega^2M_2\right)
    \end{vmatrix}=\left(2\alpha-\omega^2M_1\right)\left(2\alpha-\omega^2M_2\right)-4\alpha^2\cos^2(qa)=0
\end{equation}
Desarrollamos,
\begin{equation}
\begin{array}{l}
    4\alpha-2\alpha\omega^2M_2-2\alpha\omega^2M_1+\omega^4M_1M_2-4\alpha^2\cos^2(qa)=\\
    =\omega^4M_1M_2-2\alpha\left(M_1+M_2\right)\omega^2+4\alpha^2\left(1-\cos^2(qa)\right)=\\
    =\omega^4M_1M_2-
    2\alpha\left(M_1+M_2\right)\omega^2+4\alpha^2\sin^2(qa)=0
    \label{cap4-om4}
\end{array}
\end{equation}
Hemos obtenido una ecuación de segundo grado (haciendo el cambio de variable $\xi=\omega^2$), para $\omega^2$, luego
\begin{equation}
    \omega^2=\frac{2\alpha(M_1+M_2)}\pm\sqrt{4\alpha^2(M_1+M_2)^2-16\alpha^2M_1M_2\sin^2(qa)}{2M_1M_2}    
\end{equation}
Luego, obtenemos la \textbf{relación de dispersión} siguiente,
\begin{equation}
    \omega^2=\alpha\left[\frac{1}{M_1}+\frac{1}{M_2}\right]\pm\alpha\sqrt{\frac{\left(M_1+M_2\right)^2}{M_1^2M_2^2}-\frac{4\sin^2(qa)}{M_1M_2}}
\end{equation}
o bien,
\begin{equation}
    \omega(q)=\sqrt{\alpha\left[\frac{1}{M_1}+\frac{1}{M_2}\right]\pm\alpha\sqrt{\frac{\left(M_1+M_2\right)^2}{M_1^2M_2^2}-\frac{4\sin^2(qa)}{M_1M_2}}}
\end{equation}
donde tomamos la raíz positiva porque $\omega\geq0$ siempre. De acuerdo con esta relación de dispersión, al graficar $\omega$ frente a $q$, tendremos dos ramas, la rama superior se denomina \textbf{rama óptica} y la inferior se denomina \textbf{rama acústica}.
\begin{Figura}
    \centering
    \includegraphics[width=0.75\textwidth]{Imagenes//Capitulo4/ramas.png}
    \captionof{figure}{Curvas de dispersión para la cadena lineal diatómica reducida a la PZB.}
    \label{cap4-ramas}
\end{Figura}
Las propiedades de la relación de dispersión son las siguientes:
\begin{enumerate}
    \item $\omega(q)$ es periódica en $q$ con periodo $\pi/a$.
    \begin{proof}
        Como en $\omega(q)$ la única dependencia con $q$ es $\sin^2(qa)$, vemos que tomando $q'=q\pm\pi/a$, tenemos que
        \begin{equation}
            \sin^2(q'a)=\sin^2[(q\pm\pi/a)a]=\sin^2(qa\pm\pi)=[\sin(qa)\cos(\pm\pi)]^2=\sin^2(qa)
        \end{equation}
    \end{proof}
    Debemos notar que como el lado de la celda unidad es $2a$, el lado de la celda unidad en la red recíproca es $2\pi/(2a)=\pi/a$. Luego, como los vectores de la red recíproca son $G_m=(\pi/a)m$ con $m\in\mathbb{Z}$, vemos que $\omega(q)=\omega(q+G)$. Además, la PZB corresponde al intervalo $\left[-\frac{\pi}{2a},\frac{\pi}{2a}\right]$ en el espacio $q$.
    \item $\omega(q)$ es simétrica, es decir, $\omega(q)=\omega(-q)$.
    \begin{proof}
        Usando la misma dependencia anterior,
        \begin{equation}
            \sin^2(-qa)=[\sin(-qa)]^2=[-\sin(qa)]^2=[\sin(qa)]^2=\sin^2(qa)
        \end{equation}
    \end{proof}
    \item La relación de dispersión presenta dos ramas, una para el signo $+$ y otra para el signo $-$.
\end{enumerate}
\subsection*{Rama acústica}
La rama acústica es la rama con signo $-$. Debido a que es simétrica y periódica, bastará con estudiar su forma en el intervalo $[0,\pi/2a]$. Supondremos que $M_2>M_1$ y estudiamos el comportamiento de $\omega(q)$ en los extremos del intervalo.
\begin{itemize}
    \item En $q=0$ se tiene que $\omega(q=0)=0$, cosa que ocurre en las ondas elásticas (como el sonido), y por esto se denomina \textbf{rama acústica}.
    \item En $q=\pi/2a$, se tiene
    \begin{equation}
        \omega^2(q=\pi/2a)=\alpha\left(\frac{1}{M_1}+\frac{1}{M_2}\right)-\alpha\sqrt{\left(\frac{1}{M_1}-\frac{1}{M_2}\right)^2}=2\frac{\alpha}{M_2}
    \end{equation}
    Luego,
    \begin{equation}
        \omega(q=\pi/2a)=\sqrt{\frac{2\alpha}{M_2}}
    \end{equation}
\end{itemize}

\subsection*{Rama óptica}
La rama óptica será la rama con el signo $+$ que estudiaremos también en los extremos del intervalo:
\begin{itemize}
    \item En $q=0$ se tiene,
    \begin{equation}
        \omega^2(q=0)=2\alpha\left[\frac{1}{M_1}+\frac{1}{M_2}\right]\Rightarrow\omega(0)=\sqrt{2\alpha\left[\frac{1}{M_1}+\frac{1}{M_2}\right]}
    \end{equation}
    \item En $q=\pi/2a$ se tiene,
    \begin{equation}
        \omega^2(q=\pi/2a)=\alpha\left[\frac{1}{M_1}+\frac{1}{M_2}\right]+\alpha\left[\frac{1}{M_1}-\frac{1}{M_2}\right]=\frac{2\alpha}{M_1}\Rightarrow\omega(q=\pi/2a)=\sqrt{\frac{2\alpha}{M_1}}
    \end{equation}
\end{itemize}
Observando la Figura \ref{cap4-ramas}, vemos que aparece una frecuencia máxima y además una banda de frecuencia prohibidas, que son las frecuencias
\begin{equation}
    \omega>\sqrt{2\alpha\left[\frac{1}{M_1}+\frac{1}{M_2}\right]}\hspace{4mm}\text{y}\hspace{4mm}\sqrt{2\alpha/M_2}<\omega\sqrt{2\alpha/M_1}
\end{equation}
Estas frecuencias prohibidas no se propagan en el cristal.\\ \\
Para valores de $q\approx0$, la solución de modos normales tiene la forma siguiente,
\begin{equation}
    u_{2n+1}=u_{01}e^{-i\omega t};\hspace{5mm}u_{2n}=u_{02}e^{-i\omega t}
\end{equation}
Vemos que en la rama acústica, para $q\approx0$ tenemos $\omega\approx0$, luego, de la ecuación del movimiento
\begin{equation}
    (2\alpha-\omega^2M_1)u_{01}-(2\alpha\cos(qa))u_{02}=0
\end{equation}
se tiene
\begin{equation}
    2\alpha u_{01}-2\alpha u_{02}=0\Rightarrow u_{01}=u_{02}
\end{equation}
Por lo que los átomos de la celda oscilan en fase.\\
Para la rama óptica, vemos que para $q\approx0$ tenemos que
\begin{equation}
    \omega\approx\sqrt{\frac{2\alpha}{M_1+M_2}}
\end{equation}
que sustituyendo este valor en la ecuación del movimiento, queda
\begin{equation}
    \left[2\alpha-2\alpha M_1\left(\frac{1}{M_1}+\frac{1}{M_2}\right)\right]u_{01}-2\alpha u_{02}=0\Rightarrow u_{02}=-\frac{M_1}{M_2}u_{01}
\end{equation}
Por lo que los átomos de una celda oscilan en oposición de fase.\\
Si en lugar de átomos, tenemos iones cargados, el movimiento de los iones en el modo óptico produce un momento dipolar oscilante en la celda. Esto se produce a la frecuencia característica de esta rama, que es $\omega\sim10^{13}$ s$^{-1}$, por lo que hay fuerte absorción y reflexión de la luz infrarroja. En la rama acústica, los iones se mueven aproximadamente en fase y los centros de carga positiva y negativa coinciden en todo momento.\\ \\
Además, para valores pequeños, $qa\ll1$, en la expresión (\ref{cap4-om4}) desarrollamos el $\sin^2(qa)$ en serie de Maclaurin ($\sin^(qa)\approx q^2a^2$) a primer orden. Luego,
\begin{equation}
    \omega^4M_1M_2-2\alpha(M_1+M_2)\omega^2+4\alpha^2q^2a^2=0
\end{equation}
Resolviendo tenemos,
\begin{equation}
    \omega^2=\alpha\left[\frac{1}{M_1}+\frac{1}{M_2}\right]\pm\alpha\sqrt{\frac{(M_1+M_2)^2}{M_1^2M_2^2}-\frac{4q^2a^2}{M_1M_1}}
\end{equation}
Usando las propiedades de las raíces de la ecuación cuadrática $x^2+px+q=0$, que son
\begin{equation}
    x_1+x_2=-p;\hspace{5mm}x_1x_2=q
\end{equation}
Luego, tenemos el sistema,
\begin{equation}
    \left\lbrace\begin{matrix}
        \omega_1+\omega_2=2\alpha(M_1+M_2)\\
        \omega_1\omega_2=4\alpha^2q^2a^2
    \end{matrix}\right.
\end{equation}
que resolviendo queda,
\begin{equation}
    \omega_1=\sqrt{2\alpha\left(\frac{1}{M_1}+\frac{1}{M_2}\right)}
\end{equation}
\begin{equation}
    \omega_2=qa\sqrt{\frac{2\alpha}{M_1+M_2}}
\end{equation}
donde $\omega_1$ representa la \textbf{rama óptica} y $\omega_2$, la \textbf{rama acústica}. Luego, análogamente al caso monoatómico, la velocidad del sonido será
\begin{equation}
    v_s=a\sqrt{\frac{2\alpha}{M_1+M_2}}
\end{equation}
pues es la pendiente de la rama acústica. Además, vemos que para valores pequeños de $q$, la velocidad de fase y de grupo de la onda coinciden con la velocidad del sonido,
\begin{equation}
    v_s=v_f=v_g
\end{equation}
Por otro lado, si $M_1=M_2$, tenemos que la velocidad del sonido es
\begin{equation}
    v_s=\sqrt{\frac{\alpha}{M}}=\sqrt{\frac{C}{\rho}}
\end{equation}
que es la misma expresión que obtuvimos en la cadena monoatómica con $\rho=M/a$.
\begin{tcolorbox}
    \begin{itemize}
        \item En la \textbf{rama acústica} el cristal se mueve como un todo hacia delante y atrás, \textbf{ondas elásticas}.
        \item En la \textbf{rama óptica}, la frecuencia
        \begin{equation}
            \omega=\left(\frac{2\alpha}{M_1}\right)^{1/2}\Rightarrow\omega\left(\frac{2\cdot5\times10^2}{10^{-23}}\right)^{1/2}\approx3\times10^{13}\text{ s}^{-1}
        \end{equation}
        que corresponde con el \textbf{infrarrojo} y puede producir un pico de absorción en iones cargados NaCl, entonces, tendremos un modelo de \textbf{oscilador armónico forzado} en presencia de un campo eléctrico externo.
    \end{itemize}
\end{tcolorbox}
\subsection*{Importancia de la PZB}
Demostremos que $q$ y $q+m\pi/a$ con $m\in\mathbb{Z}$, dan el mismo modo de oscilación.
\begin{proof}
    El modo correspondiente a $q$ está dado por
    \begin{equation}
        u_{2n+1}=u_{01}e^{i(qx_{2n+1}-\omega(q)t)};\hspace{5mm}u_{2n}=u_{02}e^{i(qx_{2n}-\omega(q)t)}
    \end{equation}
    donde $u_{01}$ y $u_{02}$ se relacionan por,
    \begin{equation}
        (2\alpha-\omega^2M_1)u_{01}-(2\alpha\cos(qa))u_{02}=0\Rightarrow u_{01}=\frac{2\alpha\cos(qa)}{2\alpha-\omega^2M_1}u_{02}
    \end{equation}
    Es decir, para un número de onda $q$, el modo está dado por,
    \begin{equation}
        u_{2n+1}=\frac{2\alpha\cos(qa)}{2\alpha-\omega^2M_1}u_{02}e^{i(qx_{2n+1}-\omega(q)t)};\hspace{5mm}u_{2n}=u_{02}e^{i(qx_{2n}-\omega(q)t)}
    \end{equation}
    Luego, el modo para $q'=q+m\pi/a$ se obtiene sustituyendo $q$ por $q'$, recordando que $\omega(q+m\pi/a)=\omega(q)$, $x_{2n}=2na$ y $x_{2n+1}=(2n+1)a$, tenemos
    \begin{equation}
        u_{2n}(q+m\pi/a)=u_{02}e^{i\left[(q+m\pi/a)2na-\omega(q+m\pi/a)t\right]}=u_{02}e^{i\left[q2na-\omega(q)t\right]}e^{i2mn\pi}=u_{02}e^{i\left[qx_{2n}-\omega(q) t\right]}=u_{2n}(q)
    \end{equation}
    donde usamos que
    \begin{equation}
        e^{i2mn\pi}=\cos(2mn\pi)+i\sin(2mn\pi)\overset{mn=s}{=}\cancelto{1}{\cos(2s\pi)}+i\cancelto{0}{\sin(2s\pi)}=1
    \end{equation}
    pues al ser $m,n\in\mathbb{Z}$, entonces $s\in\mathbb{Z}$ y $2s$ es par. Además,
    \begin{equation}
        u_{2n+1}(q+m\pi/a)=u_{01}e^{i\left[(q+m\pi/a)(2n+1)a-\omega(q+m\pi/a)t\right]}=u_{01}e^{i\left[q(2n+1)a-\omega(q+m\pi/a)t\right]}e^{im(2n+1)\pi}
    \end{equation}
Usando que,
\begin{equation}
    e^{(2n+1)\pi}=\cancelto{-1}{\cos(2n+1)\pi}+i\cancelto{0}{\sin(2n+1)\pi}
\end{equation}
tenemos
    \begin{equation}
        u_{2n+1}(q+m\pi/a)=\frac{2\alpha\cos(qa+m\pi)}{2\alpha-\omega^2(q)M_1}e^{i\left[q(2n+1)a-\omega(q+m\pi/a)t\right]}(-1)^m
    \end{equation}
Luego, usando que
\begin{equation}
    \cos(qa+m\pi)=\cos(qa)\cancelto{(-1)^m}{\cos(m\pi)}-\sin(qa)\cancelto{0}{\sin(m\pi)}=\cos(qa)(-1)^m
\end{equation}
Por tanto,
\begin{equation}
    u_{2n+1}(q+m\pi/a)=\frac{2\alpha\cos(qa)\cancel{(-1)^m}}{2\alpha-\omega^2(q)M_1}e^{i\left[q(2n+1)a-\omega(q)t\right]}\cancel{(-1)^m}=\frac{2\alpha\cos(qa)}{2\alpha\omega^2M_1}e^{i\left[qx_{2n+1}-\omega t\right]}=u_{2n+1}(q)
\end{equation}
\end{proof}
Por tanto, cualquier modo de vibración con vector de onda $q'$ fuera del intervalo $\left[\frac{-\pi}{2a},\frac{\pi}{2a}\right]$ se puede describir por un modo con vector de onda $q$ dentro de dicho intervalo, es decir, dentro de la PZB. Para ello, basta tomar $q'=q\pm n\pi$/a siendo $n$ el número natural necesario para que $q$ esté contenido en la PZB. Los modos con el $q'$ y el $q$ indicados son idénticos, es decir, \textbf{son el mismo modo}.
\begin{tcolorbox}
   \textbf{ Considerando solo los vectores $q$ dentro de la Primera Zona de Brillouin podemos describir todos los modos de oscilación.}
\end{tcolorbox}
Imponiendo las condiciones cíclicas de frontera,
\begin{equation}
    u_n(x\pm L,t)=u_n(x,t)
\end{equation}
podemos obtener los valores permitidos de $q$, sabiendo que $n\in\mathbb{N}$. Usando que $x_n=na$ obtenemos,
\begin{equation}
    u_n(x\pm L,t)\propto e^{i\left[q(na+L)\right]}\hspace{4mm}y\hspace{4mm}u_n(x,t)\propto e^{i[qna]}
\end{equation}
Igualando,
\begin{equation}
    e^{iq(na+L)}=\cancel{e^{iqna}}e^{\pm iqL}=\cancel{e^{iqna}}\Rightarrow e^{\pm iqL}=1\Rightarrow \cos(qL)\pm i\sin(qL)=1
\end{equation}
Como tenemos un número complejo igualado a un número real, la parte real del número complejo será igual al número real y la parte imaginaria será cero, obteniendo así el sistema,
\begin{equation}
    \left\lbrace\begin{matrix}
        \cos(qL)&=&1&=&\cos(2m\pi)\\
        \sin(qL)&=&0&=&\sin(n\pi)
    \end{matrix}\right.
\end{equation}
Nos quedamos con la primera ecuación porque es más restrictiva. Así, igualando los argumentos de los cosenos obtenemos,
\begin{equation}
    qL=2m\pi\Rightarrow q_m=\frac{2\pi}{L}m
\end{equation}
siendo estos $q_m$ los valores permitidos de $q$.\\ \\
La densidad de puntos en el espacio recíproco la calculamos análogamente al apartado anterior, tal que
\begin{equation}
    \rho_q=\frac{1}{2\pi/L}=\frac{L}{2\pi}
\end{equation}
El número de valores de $q$ en la PZB es,
\begin{equation}
    \rho_q\frac{\pi}{a}=\frac{L}{2\pi}\frac{\pi}{a}=\frac{L}{2a}=\frac{2Na}{2a}=N
\end{equation}
Pero como para cada valor de $q$ tenemos dos valores de $\omega$, uno por la rama óptica y otro por la rama acústica, el número total de modos de la PZB es $2N$, que coincide con el número de átomos, pues tenemos $N$ celdas unitarias y cada celda tiene dos átomos. Como estamos en una dimensión, cada átomo tiene solo un grado de libertad, y por tanto, tendremos que $2N$ son los grados de libertad del cristal, es decir, el número total de modos de la PZB coincide con el número de grados de libertad del cristal.

\section{Cristales tridimensionales}
En los cristales tridimensionales, que son los reales, los que ocurre es una generalización a tres dimensiones de los encontrado para los cristales lineales estudiados. Diremos lo que ocurre sin demostrarlo; si quisiéramos demostrarlo, nos guiaríamos por lo hecho en los cristales unidimensionales. El método es el mismo, pero la matemática sería más complicada debido a encontrarnos en tres dimensiones espaciales. Vamos a estudiar dos casos, los cristales tridimensionales con una base monoatómica y con una base poliatómica.
\subsection{Cristales tridimensionales monoatómicos}
Estos cristales son el análogo tridimensional del cristal unidimensional monoatómico. \\ \\
Supongamos que la red tridimensional está formada por átomos iguales de masa $M$ y que en el volumen $V$ del cristal hay $N$ celdas primitivas elementales de Bravais. Como cada átomo tiene en la red tres grados de libertad, el cristal en conjunto se caracteriza por $3N$ grados de libertad. La posición de equilibrio del $n-$ésimo átomo vendrá dada por el vector $\vec{r}_n$, por lo que el desplazamiento, alrededor de su posición de equilibrio, será también un vector tridimensional $\vec{u}_n$. Entonces, tendremos un sistema de ecuaciones diferenciales acopladas tridimensional para la ecuación de movimiento, cuya solución será también del tipo de onda, tal que
\begin{equation}
    \vec{u}_n=\vec{u}_0e^{i\left(\vec{q}\cdot\vec{r}_n-\omega t\right)}
    \label{cap4-sol3d-1}
\end{equation}
donde $\vec{q}$ es un vector del espacio recíproco que representa el \textbf{vector de onda}. Al sustituir la solución en el sistema, obtendremos tres ecuaciones homogéneas escalares para las componentes $u_{0x}$, $u_{0y}$ y $u_{0z}$ del vector $\vec{u}$. Estas ecuaciones son homogéneas porque no hay fuerza externa, ya que todas las fuerzas están originadas por la interacción entre los átomos, y dependen de los desplazamientos de los mismos. Sabemos que para que el sistema de ecuaciones tenga solución distinta de la trivial, tenemos que imponer que el determinante del sistema sea nulo, siendo un determinante de rango 3, pues tenemos tres ecuaciones con tres incógnitas que son las componentes de $\vec{u}$, y depende de $\omega^2$, ya que al derivar respecto al tiempo dos veces la solución $\vec{u}_n$ aparece $\omega^2$.\\ \\
De la condición de determinante nulo obtenemos entonces una ecuación de sexto grado para $\omega$, que podremos reescribir como una ecuación de tercer grado para $\omega^2$, obteniendo tres ramas de dispersión. Es decir, para cada valor del vector $\vec{q}$ obtenemos tres soluciones para $\omega^2$, y por lo tanto, tres valores de $\omega$, pues $\omega\geq0$ y en $\sqrt{\omega^2}$ tomamos solo la raíz positiva.
\begin{Figura}
    \centering
    \includegraphics[width=0.5\textwidth]{Imagenes/Capitulo4/dispersionmono3d.png}
    \captionof{figure}{Curvas de dispersión para la red tridimensional primitiva de Bravais.}
    \label{cap4-disp3d-1}
\end{Figura}
De esta forma, \textit{para cada valor del vector de onda }$\vec{q}$\textit{ hay tres modos de vibraciones que determinan tres ramas de relaciones de dispersión}.\\ \\
Uno de los tres modos, $L$, corresponde a la onda longitudinal, y los otros dos, $T_1$ y $T_2$, a las ondas transversales. Observando la Figura \ref{cap4-disp3d-1}, vemos que 
\begin{itemize}
    \item Las tres ramas cumplen que en $\vec{q}=0$, $\omega=0$, y por lo tanto, son ramas acústicas.
    \item Puede haber degeneración (coincidencia de alguna de las soluciones de la ecuación de tercer grado) y por lo tanto, algunas ramas de la relación de dispersión coinciden.
    \item En diferentes direcciones en el cristal, las ondas no se propagarán de la misma manera. Esto se entiende porque la fuerza sobre un átomo tiene componentes de diferente valor a lo largo de diferentes direcciones. Como la dirección de propagación de la onda la da el vector $\vec{q}$, en diferentes direcciones la relación de dispersión será diferente. Por lo tanto, al dar la gráfica de la relación de dispersión, debe aclararse en qué dirección se está tomando el vector $\vec{q}$.
    \item Se cumple que $\omega(\vec{q})=\omega(-\vec{q})$, por la simetría de inversión que siempre tiene la red.
    \item La relación de dispersión es periódica, $\omega(\vec{G}+\vec{q})=\omega(\vec{q})$, donde $\vec{G}$ es un vector del espacio recíproco.
\end{itemize}
Para hallar el intervalo de variación y determinar el número de valores permitidos de $\vec{q}$, volvemos a utilizar la condición de periodicidad de Born-Kármán, para lo cual, supondremos que el cristal tiene la forma de paralelepípedo regular de aristas $N_1\vec{a}_1$, $N_2\vec{a}_2$ y $N_3\vec{a}_3$, donde los vectores $\vec{a}_1$, $\vec{a}_2$ y $\vec{a}_3$, son los vectores primitivos de la red. El número total de celdas será $N=N_1\cdot N_2\cdot N_3$.\\ \\
De acuerdo con la condición de periodicidad para cada desplazamiento, escribimos
\begin{equation}
    \vec{u}(\vec{r}_j,t)=\vec{u}(\vec{r}_j\pm N_i\vec{a}_i,t)
\end{equation}
donde $i=1,2,3$ y $\vec{r}_j$ será el vector que localiza al átomo $j-$ésimo del cristal. Aplicando la solución (\ref{cap4-sol3d-1}), tenemos
\begin{equation}
    \vec{u}_0e^{i(\vec{q}\cdot\vec{r}_j-\omega t)}=\vec{u}_0e^{i\left[\vec{q}\cdot(\vec{r}_j\pm N_i\vec{a}_i)-\omega t\right]}=\vec{u}_0e^{i(\vec{q}\cdot\vec{r}_j-\omega t)}e^{\pm iN_i\vec{q}\cdot\vec{a}_i}
\end{equation}
Simplificando tendremos,
\begin{equation}
    e^{\pm iN_i\vec{q}\cdot\vec{a}_i}=1
\end{equation}
que será la ecuación que deben satisfacer los valores permitidos de $\vec{q}$, resolviendo tendremos,
\begin{equation}
    \cos(N_i\vec{q}\cdot\vec{a}_i)\pm i\sin(N_i\vec{q}\cdot\vec{a}_i)=1
\end{equation}
Luego, tendremos dos ecuaciones,
\begin{equation}
    \left\lbrace\begin{matrix}
        \cos(N_i\vec{q}\cdot\vec{a}_i)=1&\Rightarrow&N_i\vec{q}\cdot\vec{a}_i=2m_i\pi\\
        \sin(N_i\vec{q}\cdot\vec{a}_i)=0&\Rightarrow&N_i\vec{q}\cdot\vec{a}_i=m_i\pi
    \end{matrix}\right.
\end{equation}
donde $m_i\in\mathbb{Z}$ y dependerá de $i=1,2,3$. Luego, nos quedamos con la primera ecuación, pues es más restrictiva, tal que
\begin{equation}
    \vec{q}\cdot\vec{a}_i=\frac{2m_i\pi}{N_i}
\end{equation}
Usando los vectores primitivos de la red recíproca, podremos reescribir $\vec{q}$ como,
\begin{equation}
    \vec{q}=q_1\vec{b}_1+q_2\vec{b}_2+q_3\vec{b}_3
\end{equation}
Luego, aplicando la condición de ortogonalidad, $\vec{b}_i\cdot\vec{a}_j=2\pi\delta_{ij}$, tendremos que
\begin{equation}
    \vec{q}\cdot\vec{a}_i=\left(q_1\vec{b}_1+q_2\vec{b}_2+q_3\vec{b}_3\right)\cdot\vec{a}_i=2\pi q_i
\end{equation}
Por tanto, 
\begin{equation}
    q_i=\frac{m_i}{N_i}
\end{equation}
con $m_i\in\mathbb{Z}$. Entonces tenemos que
\begin{equation}
    \vec{q}=\frac{m_1}{N_1}\vec{b}_1+\frac{m_2}{N_2}\vec{b}_2+\frac{m_3}{N_3}\vec{b}_3
\end{equation}
Debemos notar que los $\vec{q}$ son vectores del espacio recíproco, pero no de la red recíproca.\\ \\
Por otro lado, en el espacio $q$, un valor permitido ocupa el volumen
\begin{equation}
    \frac{\vec{b}_1}{N_1}\cdot\left(\frac{\vec{b}_2}{N_2}\times\frac{\vec{b}_3}{N_3}\right)
\end{equation}
luego, la densidad de valores permitidos es
\begin{equation}
    \rho_q=\frac{N_1N_2N_3}{\vec{b}_1\cdot\left(\vec{b}_2\times\vec{b}_3\right)}\equiv\frac{N}{V_c^*}
\end{equation}
donde $V_c^*=\vec{b}_1\cdot(\vec{b}_2\times\vec{b}_3)$ es el volumen de una celda primitiva de la red recíproca. Esta densidad puede también escribirse como,
\begin{equation}
    \rho_q=\frac{NV_c}{V_c^*V_c}=\frac{V}{(2\pi)^3}
\end{equation}
donde $V$ es el volumen del cristal y $V_c=\vec{a}_1\cdot(\vec{a}_2\times\vec{a}_3)$ es el volumen de una celda primitiva de la red en el espacio directo.\\ \\
Por otra parte, como vemos en la Figura \ref{cap4-disp3d-1}, hemos representado las ramas en el intervalo $[-\pi/a,+\pi/a]$, que es el intervalo donde se encuentra la Primera Zona de Brillouin, para poder demostrar esto, debemos tener la forma de la relación de dispersión, pero no nos hará falta, pues análogamente al caso unidimensional monoatómico, tendremos que $\omega(\vec{q}')=\omega\left(\vec{q}+\frac{2\pi}{a}\right)$, donde $\vec{q}'$ es un vector de onda contenido en la PZB y $2\pi/a$ es el tamaño de la PZB.\\ \\
Ahora vamos a ver que $\vec{q}$ y $\vec{q}+\vec{G}$ describen el mismo modo de oscilación, para ello, escribimos el vector posición $\vec{r}_j$ en función de los vectores primitivos del espacio directo, tal que
\begin{equation}
    \vec{r}_j=m_1\vec{a}_1+m_2\vec{a}_2+m_3\vec{a}_3
\end{equation}
con $m_1,m_2,m_3\in\mathbb{Z}$, coincidiendo las posiciones de equilibrio de los átomos con los puntos de la red. Entonces, vemos que
\begin{equation}
    \vec{u}_0e^{i(\vec{q}+\vec{G})\cdot\vec{r}_j}=\vec{u}_0e^{i(\vec{q}\cdot\vec{r}_j+2\pi m)}=\vec{u}_0e^{i\vec{q}\cdot\vec{r}_j}
\end{equation}
pues $e^{2\pi mi}=i$ y además,
\begin{equation}
    \vec{G}\cdot\vec{r}_j=\left(n_1\vec{b}_1+n_2\vec{b}_2+n_3\vec{b}_3\right)\cdot\left(m_1\vec{a}_1+m_2\vec{a}_2+m_3\vec{a}_3\right)=2\pi\left(m_1n_1+m_2n_2+m_3n_3\right)=2\pi m
\end{equation}
donde $n_i,m_i\in\mathbb{Z}$, luego, hemos demostrado que las oscilaciones para $\vec{q}$ y $\vec{q}+\vec{G}$ coinciden, por lo que describen el mismo modo de oscilación.\\ \\
Por otro lado, vamos a demostrar que el número de modos en la PZB es igual al número de grados de libertad del cristal. Para ello, usando la ecuación de $\rho_q$, vemos que
\begin{equation}
    \rho_q=\frac{N}{V_c^*}\Rightarrow N=\rho_qV_c^*
\end{equation}
pero como para cada valor de $\vec{q}$ tenemos tres valores de $\omega$, uno por cada rama acústica, el número total de modos de la PZB es $3N$. Además, por lo discutido en la introducción de cristales tridimensionales, sabemos que el cristal tiene $3N$ grados de libertad, pues tenemos $N$ átomos en la red y cada átomo tiene 3 grados de libertad. Por tanto, vemos que el número de grados de libertad coincide con el número de modos de oscilación dentro de la PZB.\\ \\
Acabamos diciendo que, al igual que en los cristales unidimensionales, bastará trabajar solo con los vectores $\vec{q}$ en la PZB, pues cualquier modo de oscilación puede ser descrito por estos vectores y todos los modos se describen por ellos.
\subsection{Cristales tridimensionales poliatómicos}
Estos cristales son el análogo tridimensional del cristal unidimiensional diatómico, pero generalizado para el número de átomos diferentes que tengamos en el cristal.\\ \\
Supongamos un cristal tridimensional con una base de $r$ átomos distintos por celda unidad, luego, al describir las ecuaciones del movimiento para cada átomo de la celda, tendremos $r$ ecuaciones vectoriales, equivalentes a $3r$ ecuaciones escalares.\\ \\
Al proponer soluciones de onda plana,
\begin{equation}
    \vec{u}_{n,j}=\vec{u}_{0,j}e^{i(\vec{q}\cdot\vec{r}_{n,j}-\omega t)}
\end{equation}
con $j=1,2,3,\dots,r$, obtenemos un sistema de ecuaciones homogéneas para los $\vec{u}_{0,j}$. Para que exista solución distinta de la trivial, el determinante del sistema debe ser nulo. Este determinante es de orden $3r$, por lo que da una ecuación de grado $3r$ en $\omega^2$. Así, tendremos $3r$ soluciones ($\omega\geq0$ siempre), teniendo así $3r$ ramas en la relación de dispersión. Luego, tres de estas ramas son acústicas, en ellas $\omega(q=0)=0$, y $3r-3$ ramas ópticas. De nuevo, se cumple que:
\begin{enumerate}
    \item $\omega(\vec{q}+\vec{G})=\omega(\vec{q})$
    \item Los modos de oscilación descritos por $\vec{q}$ y $\vec{q}+\vec{G}$ son el mismo modo.
    \item Las condiciones cíclicas de frontera llevan a que
    \begin{equation}
        \vec{q}=\frac{m_1}{N_1}\vec{b}_1+\frac{m_2}{N_2}\vec{b}_2+\frac{m_3}{N_3}\vec{b}_3
    \end{equation}
    y la densidad de valores del vector de onda en el espacio $q$ está dada por
    \begin{equation}
        \rho_q=\frac{NV_c}{V_c^*V_c}=\frac{V}{(2\pi)^3}
    \end{equation}
    \item El número de modos en la PZB coincide con el número de grados de libertad de los átomos del cristal. Ahora, el número de modos en la PZB es 
    \begin{equation}
        3r\rho_qV_c^*=3rN
    \end{equation}
    pues para cada valor de $\vec{q}$ tenemos $3r$ ramas de $\omega$. El número de grados de libertad es $3rN$, pues en cada una de las $N$ celdas tendremos $r$ átomos con tres grados de libertad.
    \item Podremos trabajar solo con los $\vec{q}$ en la PZB.
    \item $\omega(\vec{q})=\omega(-\vec{q})$
\end{enumerate}
Representando las ramas acústicas y ópticas para un determinado tipo de átomo del cristal tenemos,
\begin{Figura}
    \centering
    \includegraphics[width=0.5\linewidth]{Imagenes//Capitulo4/cap4-disp3d-2.png}
    \captionof{figure}{Curvas de dispersión para una red tridimensional con base poliatómica para el tipo de átomo $r=0$. La rama $L$ corresponde con el modo longitudinal acústico, las ramas $T_1$ y $T_2$, con los modos transversales acústicos, la rama $L0$, con el modo longitudinal óptico y las ramas $T_1$ y $T_2$, con los modos transversales ópticos.}
\end{Figura}
Cabe recalcar que las ramas acústicas son las mismas, independientemente del tipo de átomo, las ramas ópticas son las que varían.
\section{Propiedades térmicas de los cristales}
Para estudiar las propiedades térmicas haremos uso de la estadística de Bose-Einstein, sabiendo que
\begin{itemize}
    \item la energía media de un modo $\omega$ es
    \begin{equation}
        \bar{\epsilon}=\frac{\hbar\omega}{e^{\hbar\omega/kT}-1}
    \end{equation}
    \item el número medio de fotones con energía $\hbar\omega$ es,
    \begin{equation}
        \bar{n}(\omega)=\frac{1}{e^{\hbar\omega/kT}-1}
    \end{equation}
    \item la energía total de los fotones es,
    \begin{equation}
        U=\int_0^{\omega_m}\bar{n}(\omega)\hbar\omega G(\omega)d\omega
    \end{equation}
    donde $G(\omega)d\omega$ representa la \textbf{densidad de modos}.
\end{itemize}
Empezamos calculando la densidad de estados, que definimos como 
\begin{definition}
    La densidad de estados $G(\omega)$ se define de forma que $G(\omega)d\omega$ dé el número de modos de oscilación en el intervalo $[\omega,\omega+d\omega]$.
\end{definition}
 Es decir, $G(\omega)$ es el número de modos de oscilación por unidad de frecuencia y puede entenderse como una densidad de probabilidad. 
 Para la densidad de estados en ondas elásticas en 3D, sabemos que la relación de dispersión será la correspondiente a grandes longitudes de onda $(\lambda\gg a)$, tal que
 \begin{equation}
     \omega=v_sq
 \end{equation}
donde $q=|\vec{q}|$, siendo $\vec{q}$ un vector tridimensional. Vamos a estudiar un cristal cúbico de lado $L$.\\ \\
Al ser independientes entre sí, la densidad de modos en 3D se puede factorizar de acuerdo al resultado en 1D, obtenido en los apartados anteriores,
\begin{equation}
    \rho_{1D}=\frac{L}{2\pi}
\end{equation}
obteniendo que
\begin{equation}
    \rho_{3D}=\rho_{1D,x}\cdot\rho_{1D,y}\cdot\rho_{1D,z}=\frac{L^3}{(2\pi)^3}=\frac{V}{(2\pi)^3}
\end{equation}
siendo el mismo resultado que para cristales tridimensionales. En el espacio $q$ en 3D, cada punto representado corresponde a un vector de onda permitido, luego, el número de puntos en un casquete esférico de radio $q$ y espesor $dq$ es,
\begin{equation}
    \rho_q4\pi q^2dq=\frac{V}{2\pi^2}\left(\frac{\omega}{v_x}\right)^2\frac{d\omega}{v_s}
\end{equation}
Sabemos que en 3D para cada $\vec{q}$ hay una onda longitudinal y dos transversales (las tres ramas), luego el número de modos accesibles será,
\begin{equation}
    G(\omega)d\omega=3\rho_{3D}dV_q=3\frac{V}{(2\pi)^3}4\pi q^2dq
\end{equation}
donde $dV_q$ es el diferencial en esféricas. 
Vemos que
\begin{itemize}
    \item $G(\omega)\propto\omega^2$
    \item $G(\omega)\propto V$
\end{itemize}
siendo la densidad de estados para ondas elásticas.
\begin{note}
    En ocasiones se toma $V=1$, es decir, se toma la densidad de estados por unidad de volumen.
\end{note}
Para la densidad de estados general en cristales tridimensionales se cumple que,
\begin{itemize}
    \item La densidad de estados es la suma de la densidad de estados de cada una de las ramas,
    \begin{equation}
        G(\omega)=\sum_jG_j(\omega)
    \end{equation}
    \item La forma de $G(\omega)$ es complicada y depende del material, pero podemos afirmar que
    \begin{enumerate}
        \item $G(\omega)\propto\omega^2$ cuando $\omega\to0$. Pues si $\omega\to0$ estamos en las ramas acústicas y en el límite de grandes longitudes de onda, $q\to0$.
        \item Para $\omega>\omega_m$, $G(\omega)=0$. Pues para valores mayores que la frecuencia máxima no hay modos de oscilación.
    \end{enumerate}
\end{itemize}
Vamos a trabajar en la aproximación de Debye para no preocuparnos de lo anterior,
\begin{tcolorbox}[title=Aproximación de Debye]
     En la aproximación de Debye se supone que:
     \begin{itemize}
         \item $\omega=v_sq$ es válida para cualquier valor de $\omega$.
         \item Se introduce una frecuencia de corte, la frecuencia de Debye, $\omega_D$. Su valor se halla exigiendo que el número total de modos de oscilación sea igual al número de grados de libertad del sistema. En un cristal 3D con $N$ átomos,
         \begin{equation}
             \int_0^{\omega_D}G(\omega)d\omega=3N
         \end{equation}
     \end{itemize}
 \end{tcolorbox}
Entonces, tenemos que $\omega=v_sq$ y $d\omega=v_sdq$, por tanto
\begin{equation}
    G(\omega)d\omega=3\frac{V}{(2\pi)^3}4\pi\frac{\omega^2}{v_s^3}d\omega=\frac{3V\omega^2}{2\pi^2v_s^3}d\omega
\end{equation}
Luego,
\begin{equation}
    G(\omega)=\frac{3V\omega^2}{2\pi^2v_s^3}
\end{equation}

Como la aproximación de Debye implica que existe una relación lineal durante todo el rango de valores de $q$ ($\omega=v_sq$) hasta le valor máximo de $\omega$ ($\omega_D$), el número total de modos tiene que ser igual a $N$ para cada rama, y como tenemos tres ramas, tendremos
\begin{equation}
    3N=\int_0^{\omega_D}G(\omega)d\omega=\int_0^{\omega_D}\frac{3V\omega^2}{2\pi^2v_s^3}d\omega=\frac{3V}{2\pi^2v_s^3}\frac{1}{3}\omega_D^3=\frac{V}{2\pi^2v_s^3}\omega_D^3
\end{equation}
Por tanto,
\begin{equation}
    \omega_D^3=\frac{6\pi^2N}{V}v_s^3
\end{equation}
Vamos a resolver ahora la energía considerando las tres ramas acústicas,
\begin{equation}
    U=\int_0^{\omega_D}\frac{\hbar\omega}{e^{\beta\hbar\omega}-1}\frac{3V\omega^2}{2\pi^2v_s^3}d\omega=\int_0^{\omega_D}\frac{3V\hbar}{2\pi^2v_s^3}\frac{\omega^3}{e^{\beta\hbar\omega}-1}d\omega
\end{equation}
donde $\beta=1/kT$. Hacemos el cambio de variable siguiente,
\begin{equation}
    x=\beta\hbar\omega=\frac{\hbar\omega}{kT}\Rightarrow x_D=\frac{\hbar\omega_D}{kT}\equiv\frac{\theta_D}{T}
\end{equation}
donde $\theta_D$ es la \textbf{temperatura de Debye}. Aplicando la definición de la temperatura de Debye a la condición de normalización obtenemos,
\begin{equation}
    \omega_D=\left(\frac{6\pi^2N}{V}\right)^{1/3}v_s\Rightarrow\theta_D=\frac{\hbar}{k}\left(\frac{6\pi^2N}{V}\right)^{1/3}v_s\Rightarrow v_s^{-3}=\left(\frac{\hbar}{k}\right)^3\frac{6\pi^2N}{V}\theta_D^{-3}
\end{equation}
Sustituyendo,
\begin{equation}
    U=\frac{3V\hbar}{2\pi^2\theta_D^3}\left(\frac{\hbar}{k}\right)^3\frac{6\pi^2N}{V}\left(\frac{kT}{\hbar}\right)^3\left(\frac{kT}{\hbar}\right)\int_0^{x_D}\frac{x^3}{e^x-1}dx=9NkT\left(\frac{T}{\theta_D}\right)^3\int_0^{x_D}\frac{x^3}{e^x-1}dx
\end{equation}
Ahora calculamos el valor de $C_v$, sabiendo que
\begin{equation}
    C_v=\left(\frac{\partial U}{\partial T}\right)_V
\end{equation}
Luego,
\begin{equation}
    C_v=\frac{\partial U}{\partial\beta}\frac{\partial\beta}{\partial T}=\frac{\partial}{\partial\beta}\left(\frac{3V\hbar}{2\pi^2v_s^2}\int_0^{\omega_D}\frac{\omega^3}{e^{\beta\hbar\omega}-1}d\omega\right)\frac{-1}{kT^2}=\frac{3V}{2\pi^2v_s^3}\frac{\hbar^2}{kT^2}\int_0^{\omega_D}\frac{\omega^4e^{\beta\hbar\omega}}{\left(e^{\beta\hbar\omega}-1\right)^2}d\omega
\end{equation}
Haciendo el cambio de variable anterior obtenemos,
\begin{equation}
    C_v=\frac{3V\hbar}{2\pi^2\theta_D^3}\left(\frac{\hbar}{k}\right)^3\frac{6\pi^2N}{V}\left(\frac{kT}{\hbar}\right)^4\left(\frac{kT}{\hbar}\right)\frac{\hbar}{kT^2}\int_0^{x_D}\frac{x^4e^x}{\left(e^x-1\right)^2}dx=9Nk\left(\frac{T}{\theta_D}\right)^3\int_0^{x_D}\frac{x^4e^x}{\left(e^x-1\right)^2}dx
\end{equation}
Para comprobar la validez del resultado, vamos a estudiar el comportamiento de $C_v$ en los límites de temperaturas altas y bajas:
\begin{itemize}
    \item Si $T\uparrow\uparrow$, el término $\beta\hbar\omega$ es pequeño, por lo que si desarrollamos la exponencial, $e^x\sim1+x$, e introducimos el resultado en $U$, obtenemos
    \begin{equation}
        U\sim9NkT\left(\frac{T}{\theta_D}\right)^3\int_0^{x_D}\frac{x^3}{(1+x-1)}dx=9NkT\left(\frac{T}{\theta_D}\right)^3\frac{x_D^3}{3}=3NkT
    \end{equation}
    Por tanto, derivando
    \begin{equation}
        C_v=3Nk
    \end{equation}
    que equivale al Principio de Equipartición y es independiente de la temperatura. Se denomina \textbf{Ley de Dulong y Petit}.
    \item Si $T\downarrow\downarrow$, entonces $x_D\uparrow\uparrow$, por lo que el límite de la integral se puede llevar a $+\infty$, con lo que nos queda una integral tabulada, $I=\pi^4/15$, y nos sale la energía como,
    \begin{equation}
        U=9NkT\left(\frac{T}{\theta_D}\right)^3\frac{\pi^4}{15}
    \end{equation}
    Por tanto, derivando
    \begin{equation}
        C_v=\frac{12}{5}Nk\pi^4\left(\frac{T}{\theta_D}\right)^3
    \end{equation}
    siendo $C_v\propto T^3$.
\end{itemize}
\subsection{Interacción en el gas de Fonones}
Un modo de oscilación de frecuencia $\omega$ tiene niveles de energía dados por $(n+1/2)\hbar\omega$. La energía del nivel $n$ (asociada a las oscilaciones de los átomos del cristal) podemos describirla diciendo que en el cristal tenemos, además de la energía del punto cero $\hbar\omega/2$, $n$ 'partículas', cada una con energía $\hbar\omega$. Estas 'partículas' se denominan \textbf{fonones}. Esto es análogo a la descripción de una onda electromagnética en base a fotones.\\ \\
A los fonones se le asocia la cantidad de movimiento dada por la relación de De Broglie,
\begin{equation}
    p=h/\lambda=h/(2\pi/q)=\hbar q
\end{equation}
El fonón interactúa con otras partículas dentro del cristal como si tuviera una cantidad de movimiento $\vec{p}=\hbar\vec{q}$, que se llama \textbf{cuasi-momento del fonón}. Sin embargo, la presencia de fonones no altera la cantidad de movimiento real del cristal y el fonón no puede propagarse en el vacío.\\ \\
La interacción de fonones se da por dos tipos de colisión, tipo $N$ y tipo $U$. En la colisión se va a producir una \textit{conservación del cuasi-momento},
\begin{equation}
    \hbar\vec{K}=\hbar\vec{K}'+\hbar\vec{q}+\hbar\vec{G}
\end{equation}
donde para el tipo $N$, $\vec{G}=0$, y para el tipo $U$, $\vec{G}\neq0$.
\begin{enumerate}
    \item Para el proceso tipo $N$ (proceso normal), la suma está dentro de la Primera Zona de Brillouin. En este caso, no se afecta el momento cristalino total llevado por los fonones.
    \item Para el proceso tipo $U$ (proceso umklapp o 'dar una voltereta'), la suma sí da un vector fuera de la PZB, por lo que el momento cristalino del fonón varía drásticamente su dirección respecto al de los fonones originales. En este caso, $U(T)\to k_t(T)$, es decir, este proceso produce una resistividad térmica.
\end{enumerate}

Sabemos que un material no conduce siempre de forma ideal, ya que solo se produce cuando no hay pérdida del momento. Por tanto, para el tipo $N$ hay conductividad $\infty$ y para $U$, la conductividad $k_T$ es finita, es decir, que existe una resistencia térmica.\\ \\
Vamos a analizar el término $k_T(T)$ que, recordando la teoría cinética de los gases (Apéndice \ref{cinetica}), vendrá dado por
\begin{equation}
    k_T(T)=\frac{1}{3}C_v(T)\bar{v}(T)l(T)
\end{equation}
donde
\begin{enumerate}
    \item $C_v(T)$ es constante para $T\uparrow\uparrow$ y es $\propto T^3$ cuando $T\downarrow\downarrow$. Estos resultados obedecen al Modelo de Debye, donde imponemos que $\omega=v_sq$.
    \item Vemos que $\bra{v}(T)$ es constante, independientemente de la temperatura.
    \item El recorrido libre medio $l(T)$ viene dado por 
    \begin{equation}
        l(T)=\bar{v}\tau
    \end{equation}
    Si hubiera dependencia con $T$, debería estar en $\tau$, que es el tiempo medio entre colisiones. Este dependerá del número de partículas que, a su vez, dependerá de la temperatura según la estadística de Bose-Einstein de la forma,
    \begin{equation}
        \bar{n}(T)=\frac{1}{e^{\hbar\omega/kT}-1}
    \end{equation}
    donde $\tau\propto1/\bar{n}$, ya que a más partículas, menos tendremos menos tiempo entre colisiones.\\
    Cuando $T\uparrow\uparrow$, vemos que 
    \begin{equation}
        \bar{n}(T)\sim\frac{kT}{\hbar\omega}
    \end{equation}
    ya que $e^x\sim1+x$ si $x\downarrow\downarrow$. Cuando $T\downarrow\downarrow$, $\bar{n}$ será muy pequeño, de forma que $\tau$ será muy alto, pues
    \begin{equation}
        \lim_{T\to0}\bar{n}(T)=\lim_{T\to0}\frac{1}{e^{\hbar\omega/kT}-1}\to\frac{1}{e^{\hbar\omega/0}-1}\to\frac{1}{e^{\infty}-1}\to\frac{1}{\infty}\to0
    \end{equation}
    Por tanto, el recorrido libre medio será igual al volumen de la muestra y habrá muy pocas colisiones.
\end{enumerate}
Por tanto,
\begin{itemize}
    \item Para $T\uparrow\uparrow$, tendremos que $C_v=cte$, $\bar{v}=cte$ y $l\propto1/T$, quedando que $k_T\propto1/T$.
    \item Para $T\downarrow\downarrow$, tendemos que $C_v\propto T^3$, $\bar{v}=cte$ y $l=cte$, quedando que $k_T\propto T^3$
\end{itemize}
Esto nos indica que va a haber un máximo entre las temperaturas altas y las temperaturas bajas. Para estas temperaturas 'medio bajas' nos sale que $k_T\propto T^3e^{\theta_D/T}$, ya que
\begin{equation}
    \bar{n}(T<<1)\approx\frac{1}{e^{\hbar\omega_D/kT}}=e^{-\frac{\hbar\omega_D}{kT}}=e^{-\theta_D/T}\Rightarrow \tau\propto e^{\theta_D/T}\Rightarrow l(T)\propto e^{\theta_D/T}
\end{equation}
y seguimos en el régimen donde $C_v\propto T^3$.\\
En resumen, tenemos el siguiente comportamiento,
\begin{Figura}
    \centering
    \includegraphics[width=0.5\linewidth]{Imagenes//Capitulo4/K.png}
    \captionof{figure}{Curva de $k_T$ con la temperatura.}
    \label{cap4-K}
\end{Figura}
\lhead{\emph{Capítulo 5: Bandas de energía}}
\chapter{Bandas de energía y modelo semiclásico}
\section{Teorema de Bloch}
\begin{theorem}
    Las funciones de onda que son soluciones de la ecuación de Schrödinger con potencial periódico, cuyo periodo es igual al de la red, representa ondas planas moduladas por cierta función con la periodicidad de red, es decir,
    \begin{equation}
        \Psi_k(\vec{r})=u_k(\vec{r})e^{i\vec{k}\cdot\vec{r}}
    \end{equation}
    denominada \textbf{función de Bloch}, satisface que
    \begin{equation}
        \left[-\frac{\hbar^2\nabla^2}{2m}+V(\vec{r})\right]\Psi_k(\vec{r})=E\Psi_k(\vec{r})
    \end{equation}
    con la condición de periodicidad,
    \begin{equation}
        V(\vec{r})=V(\vec{r}+\vec{R})
    \end{equation}
    siendo $\vec{R}$ un vector de la red, tal que
    \begin{equation}
        \vec{R}=n_1\vec{a}_1+n_2\vec{a}_2+n_3\vec{a}_3
    \end{equation}
\end{theorem}
Para demostrarlo, debemos definir primero el operador de traslación siguiente,
\begin{equation}
    \mathscr{T}_Rf(\vec{r})=f(\vec{r}+\vec{R})
\end{equation}
cuyas propiedades son,
\begin{enumerate}
    \item \label{i}Conmuta con el operador Hamiltoniano, $\mathscr{H}$, pues
    \begin{equation}
        \mathscr{T}_R\left[\mathscr{H}(\vec{r})\phi(\vec{r})\right]=\mathscr{H}(\vec{r}+\vec{R})\phi(\vec{r}+\vec{R})=\mathscr{H}(\vec{r})\mathscr{T}_R\phi(\vec{r})
    \end{equation}
    donde usamos la periodicidad del Hamiltoniano.
    \item\label{ii} Los operadores de traslación conmutan entre sí, tal que
    \begin{equation}
        \mathscr{T}_R[\mathscr{T}_{R'}\phi(\vec{r})]=\mathscr{T}_R\phi(\vec{r}+\vec{R}')=\underbrace{\phi(\vec{r}+\vec{R}'+\vec{R})}_{\mathscr{T}_{R'+R}\phi(\vec{r})}=\underbrace{\phi(\vec{r}+\vec{R}+\vec{R}')}_{\mathscr{T}_{R+R'}\phi(\vec{r})}=\mathscr{T}_R\phi(\vec{r}+\vec{R}')=\mathscr{T}_R[\mathscr{T}_{R'}\phi(\vec{r})]
    \end{equation}
    \item\label{iii} A partir de la propiedad (\ref{i}), podemos concluir que existen autofunciones comunes a $\mathscr{H}$ y los $\mathscr{T}_R$, y los denotamos por $\Psi$, tal que
    \begin{equation}
        \left.\begin{array}{r}
        \mathscr{H}\Psi=E\Psi  \\
        \mathscr{T}_R\Psi=C(R)\Psi
        \end{array}\right\rbrace
    \end{equation}
    siendo $C(R)$ el autovalor de $\mathscr{T}_R$, cuyas propiedades son
    \begin{enumerate}
        \item \label{A}$C(R+R')=C(R)C(R')$
        \begin{proof}
        $$\mathscr{T}_{R'}\mathscr{T}_R\Psi=\mathscr{T}_{R'}C(R)\Psi=C(R)\mathscr{T}_{R'}\Psi=C(R)C(R')\Psi$$
        Por la propiedad (\ref{ii}),
        $$\mathscr{T}_R\mathscr{T}_{R'}\Psi=\mathscr{T}_{R+R'}\Psi=C(R+R')\Psi$$
        Por tanto,
        $$C(R)C(R')=C(R+R')$$
        \end{proof}
        \item \label{B}$C(\vec{a}_i)=e^{i2\pi x_i}$, donde $\vec{a}_i$ es un vector primitivo de la red y $x_i\in\mathbb{R}$.
        \begin{proof}
            Debido a la propiedad (\ref{A}), podemos elegir como forma para los $C(R)$ esta exponencial, pues
            \begin{equation}
                C(R)C(R')=e^{\alpha R}e^{\alpha R'}=e^{\alpha(R+R')}=C(R+R')
            \end{equation}
            las constantes saldrán de la normalización.
        \end{proof}
        \item Por la aplicación masiva de la propiedad (\ref{A}), cualquier vector de la red, $\vec{R}=n_1\vec{a}_1+n_2\vec{a}_2+n_3\vec{a}_3$, un autovalor asociado al operador traslación $\mathscr{T}_R$ será,
        \begin{equation}
            C(R)=C(n_1\vec{a}_1)C(n_2\vec{a}_2)C(n_3\vec{a}_3)=[C(\vec{a}_1)]^{n_1}[C(\vec{a}_2)]^{n_2}[C(\vec{a}_3)]^{n_3}
        \end{equation}
        Aplicando la propiedad (\ref{B}), tendremos que
        \begin{equation}
            C(R)=e^{i2\pi\left(n_1x_1+n_2x_2+n_3x_3\right)}
        \end{equation}
        y finalmente, este resultado se puede tomar como el producto de un vector de la red real, $\vec{R}$, de componentes $\vec{R}=n_1\vec{a}_1+n_2\vec{a}_2+n_3\vec{a}_3$, por un vector de la red recíproca, $\vec{k}=x_1\vec{b}_1+x_2\vec{b}_2+x_3\vec{b}_3$, con $\vec{a}_i\cdot\vec{b}_j=2\pi\delta_{ij}$, por lo que nos queda,
        \begin{equation}
            C(\vec{R})=e^{i\vec{R}\cdot\vec{k}}
        \end{equation}
    \end{enumerate}
\end{enumerate}
En resumen, vistas las propiedades anteriores, ahora podemos escribir que
\begin{equation}
    \mathscr{T}_R\Psi(\vec{r})=\Psi(\vec{r}+\vec{R})=C(\vec{R})\Psi(\vec{r})=e^{i\vec{k}\cdot\vec{R}}\Psi(\vec{r})
\end{equation}
de tal forma que la aplicación del operador de traslación es igual a la aparición de una fase con forma de onda plana.\\ \\
Nos falta por demostrar que las $u_k(\vec{r})$ son periódicas. Para ello, elegimos 
\begin{equation}
    u_k(\vec{r})=e^{-i\vec{k}\cdot\vec{r}}\Psi(\vec{r})
\end{equation}
Por lo tanto,
\begin{equation}
    u_k(\vec{r}+\vec{R})=e^{-i\vec{k}\cdot(\vec{r}+\vec{R})}\Psi(\vec{r}+\vec{R})=e^{-i\vec{k}\cdot\vec{r}}e^{-i\vec{k}\cdot\vec{R}}\Psi(\vec{r}+\vec{R})
\end{equation}
Dado que, $\Psi(\vec{r}+\vec{R})=\mathscr{T}_R\Psi(\vec{r})$, que podemos escribir como $\Psi(\vec{r}+\vec{R})=\exp[i\vec{k}\cdot\vec{R}]\Psi(\vec{r})$, sustituyendo queda,
\begin{equation}
    u_k(\vec{r}+\vec{R})=e^{-i\vec{k}\cdot\vec{r}}e^{-i\vec{k}\cdot\vec{R}}e^{i\vec{k}\cdot\vec{R}}\Psi(\vec{r})=e^{-i\vec{k}\cdot\vec{r}}\Psi(\vec{r})=u_k(\vec{r})
\end{equation}
luego es periódica. Despejando $\Psi(\vec{r})$, nos queda que
\begin{equation}
    \Psi(\vec{r})=e^{i\vec{k}\cdot\vec{r}}u_k(\vec{r})
\end{equation}
obteniendo las \textbf{funciones de Bloch}.\\ \\
Así, en efecto, la función de onda del electrón del cristal representa una onda progresiva $e^{i\vec{k}\cdot\vec{r}}$ modulada por la función periódica $u_k(\vec{r})$, que tiene el periodo de la red y que depende del vector de onda $\vec{k}$. Del vector de onda $\vec{k}$ depende también la energía del electrón. La forma concreta de esta dependencia $E(\vec{k})$ puede hallarse resolviendo la ecuación de Schrödinger. Encontrar la dependencia $E(\vec{k})$, denominada \textbf{estructura de bandas de energía}, es uno de los problemas más importantes de la Física del Estado Sólido.
\section{Método Tight Binding. Modelo TBM}
Asumimos que el potencial que une el electrón al átomo es fuerte, por tanto, es importante para explicar bandas cercanas al núcleo.\\ \\
Vamos a suponer orbitales localizados. El objetivo es pasar de un nivel individual de energía a la formación de una banda debido a la interacción con los otros átomos del sólido; el método para deslocalizar las funciones de onda, es utilizar las funciones de Bloch, que como vimos anteriormente, tienen carácter colectivo y no individual.\\ \\
La función de Bloch elegida es,
\begin{equation}
    \Psi_k(x)=\frac{1}{N^{1/2}}\sum_{j=1}^Ne^{ikX_j}\phi_{\nu}(x-X_j)
\end{equation}
donde $X_j$ es la posición de los átomo en la red y $\phi_{\nu}(x-X_j)$ representan el orbital atómico, que es grande para $x=X_j$, pero decrece rápido.\\ \\
Imponemos poco solapamiento (overlap). Como conclusión previa, demostraremos que la función elegida cumple con la definición de las funciones de Bloch. Para ello, multiplicamos por los factores $e^{ikx}$ y $e^{-ikx}$, quedando
\begin{equation}
    \Psi_k(x)=\frac{1}{N^{1/2}}e^{ikx}\sum_{j=1}^Ne^{-ik(x-X_j)}\phi_{\nu}(x-X_j)
\end{equation}
donde vemos que el término $e^{ikx}$ representa una onda plana en 1D, y el término del sumatorio es un factor periódico de la red. Por tanto, la función elegida describe la función de onda de un electrón propagándose por la red, siendo una función de Bloch.\\ \\
Por otra parte, la función elegida también cumple que, cerca de una posición atómica $x=X_j$, tenemos
\begin{equation}
    \Psi_k(x)\approx e^{ikX_j}\phi_{\nu}(x-X_j)
\end{equation}
y se comporta como un orbital atómico, que es el criterio requerido por la Teoría TBM.\\ \\
Una vez determinada la función de onda elegida, podemos empezar los cálculos de la energía y la aparición de bandas,
\begin{equation}
    E(k)=\Braket{\Psi_k|H|\Psi_k}
\end{equation}
donde utilizamos el Hamiltoniano,
\begin{equation}
   H=-\frac{\hbar^2}{2m}\frac{d^2}{dx^2}+V(x) 
\end{equation}
donde 
\begin{equation}
    V(x)=v(x)+V'(x)
\end{equation}
\begin{Figura}
    \centering
    \includegraphics[width=0.5\linewidth]{Imagenes//Capitulo5/cap5-TBM.png}
    \captionof{figure}{La división del potencial del cristal en (a) un potencial atómico y (b) el resto del potencial del cristal.}
    \label{cap5-TBM}
\end{Figura}
Sustituyendo en la expresión de la energía,
\begin{equation}
    E(k)=\frac{1}{N}\sum_{jj'}e^{ik(X_j-X_{j'})}\Braket{\phi_{\nu}(x-X_{j'})|H|\phi_{\nu}(x-X_j)}
\end{equation}
Aunque aparece una doble sumatoria sobre $j$ y $j'$, debido a las condiciones del problema, una vez fijada la $j'$, la suma sobre $j$ es igual para todas las distintas $j'$ y por lo tanto, tendremos $N$ términos iguales. Así, podremos quitar uno de los índices del sumatorio por simetría.\\ \\
Por lo tanto, la expresión para la energía queda,
\begin{equation}
    E(k)=\sum_{j=-N/2}^{(N/2)-1}e^{iX_j}\Braket{\phi_{\nu}(x)|H|\phi_{\nu}(x-X_j)}
\end{equation}
donde hemos considerado que $X_{j'}=0$ (sin perder generalidad). A partir de este punto empezamos a simplificar la ecuación anterior. Para ello, vemos que el término correspondiente al electrón completamente localizado en $j=0$ es
\begin{equation}
    E(k)=\Braket{\phi_{\nu}(x)|H|\phi_{\nu}(x)}
\end{equation}
Por otra parte, vemos que el término siguiente,
\begin{equation}
    \sum_{j\neq0}e^{ikX_j}\Braket{\phi_{\nu}(x)|H|\phi_{\nu}(x-X_j)}
\end{equation}
incluye los efectos correspondientes al resto de los átomos.\\ \\
La parte correspondiente al término para $j=0$ la vamos a dividir de acuerdo a la Figura \ref{cap5-TBM}, es decir, $V(x)=v(x)+V'(x)$, y obtenemos dos ecuaciones,
\begin{equation}
    \Braket{\phi_{\nu}(x)|-\frac{\hbar^2}{2m}\frac{d^2}{dx^2}+v(x)|\phi_{\nu}(x)}=E_v
\end{equation}
Este término es igual a la energía $E_v$, que representa la \textbf{energía atómica}, ya que es el operador para un átomo libre. El otro término será,
\begin{equation}
    \Braket{\phi_{\nu}(x)|V'(x)|\phi_{\nu}(x)}=-\int\phi_{\nu}^*(x)V'(x)\phi_{\nu}^*(x)dx\equiv\beta
\end{equation}
donde $\beta$ es pequeño, ya que $\phi_{\nu}(x)$ es apreciable en el origen (recordar $j=0$), y $V'(x)$ es pequeño en el origen (ver Figura \ref{cap5-TBM}).\\ \\
El segundo término también lo vamos a dividir en dos partes. Para ello, vamos a considerar interacción únicamente con primeros vecinos, ya que las $\phi_{\nu}$ están fuertemente localizadas, quedándonos con el caso $X_1=a$, tal que
\begin{equation}
    E(k,X_1=a)=\Braket{\phi_{\nu}(x)|H|\phi_{\nu}(x-a)}
\end{equation}
obteniendo dos términos, que son
\begin{equation}
    \left.\begin{matrix}
        \Braket{\phi_{\nu}(x)|-\frac{\hbar^2}{2m}\frac{d^2}{dx^2}+v(x-a)|\phi_{\nu}(x-a)}\approx0\\
        \Braket{\phi_{\nu}(x)|V'(x-a)\phi_{\nu}(x-a)}
    \end{matrix}\right\rbrace
\end{equation}
donde el primer término es aproximadamente nulo, debido a que las $\phi_{\nu}$ no se solapan (overlap) y lo despreciamos al ser $E_v\Braket{\phi_{\nu}(x)|\phi_{\nu}(x-a)}$.\\ \\
El segundo término será 
\begin{equation}
    \Braket{\phi_{\nu}(x)|V'(x-a)\phi_{\nu}(x-a)}=-\int\phi_{\nu}^*(x)V'(x-a)\phi_{\nu}(x-a)dx\equiv\gamma
\end{equation}
Si consideramos que $V'(x-a)$ es apreciable cerca del origen, se puede evaluar la integral con el valor $\gamma$, denominada \textbf{integral de solapamiento}.\\ \\
Al ser el problema simétrico, para $X_2=-a$ obtenemos los mismos resultados, tal que
\begin{equation}
    E(k)=E_v-\beta-\gamma\sum_{j=\pm1}e^{ikX_j}=E_v-\beta-2\gamma\cos(ka)=E_0+4\gamma\sin^2\left(\frac{ka}{2}\right)
\end{equation}
donde $E_0=E_v-\beta-2\gamma$.
\subsection*{Conclusiones}
\begin{enumerate}
    \item La energía del electrón individual, $E_v$, se ha transformado en una banda de energía $E=E(k)$ al utilizar las funciones de Bloch, permitiendo deslocalización.
    \item La representación de la banda es,

    \item La altura de la banda es $4\gamma$, proporcional a la integral de desplazamiento, luego, a mayor solapamiento, más interacción y banda más alta.
    \item $E_0<E_v$.
    \item Aparece el concepto de masa efectiva; cerca del fondo de la banda ($k\sim0$),
    \begin{equation}
        E(k\downarrow\downarrow)=E_0+\gamma a^2k^2
    \end{equation}
    donde usamos que $\sin(x)\sim x$. Luego, igualando,
    \begin{equation}
        \gamma a^2k^2=\frac{\hbar^2}{2m^*}k^2\Rightarrow m^*=\frac{\hbar^2}{2a^2}\frac{1}{\gamma}
    \end{equation}
    por lo que $m^*\propto 1/\gamma$. Como $m^*$ es proporcional a $1/\gamma$, entonces la masa disminuye al aumentar el solapamiento $(\gamma)$, y la probabilidad de efecto túnel (tunneling) aumenta. Al contrario, al bajar $\gamma$, el electrón estará confinado y $m^*\uparrow\uparrow$.
\end{enumerate}

\section{Modelo semiclásico}

Comenzamos considerando al electrón asociado a un paquete de funciones de Bloch. Para verlo como analogía, empezamos con el caso del electrón libre, puesto que lo que vamos a estudiar es la dinámica del movimiento del electrón en presencia de campos aplicados.\\ \\
Empezamos con la velocidad para el caso del electrón libre, que es
\begin{equation}
    \vec{v}=\frac{\vec{p}}{m_0}
\end{equation}
y como $\vec{p}=\hbar\vec{k}$, tendremos que
\begin{equation}
    \vec{v}=\frac{\hbar\vec{k}}{m_0}
\end{equation}
por lo que $\vec{v}$ es proporcional y paralelo a $\vec{k}$.\\ \\
Si el electrón se describe asociado al paquete, sabemos que la velocidad de grupo es,
\begin{equation}
    \vec{v}_g=\vec{\nabla}_k\omega(\vec{k})
\end{equation}
y como $\omega=E/\hbar$, tenemos
\begin{equation}
    \vec{v}_g=\frac{1}{\hbar}\vec{\nabla}_kE(\vec{k})
\end{equation}
Para el caso del electrón libre tenemos que $\vec{v}\parallel\vec{k}$, pero esto no sucederá para el electrón de Bloch.\\ \\
Ahora podemos ver el efecto de las fuerzas externas, que serán debidas a los campos aplicados sobre el movimiento del electrón (como problemas de inducción eléctrica).\\ \\
Si suponemos un campo eléctrico aplicado, $\vec{\mathscr{E}}$, la fuerza experimentada por el electrón será.
\begin{equation}
    \vec{F}=-q_e\vec{\mathscr{E}}
\end{equation}
que es la fuerza de Lorentz, de forma que la potencia absorbida por el electrón es,
\begin{equation}
    \mathscr{P}=\frac{dE(\vec{k})}{dt}=-q_e\vec{\mathscr{E}}\cdot\vec{v}
\end{equation}
Aplicamos el cambio de variable siguiente,
\begin{equation}
    \frac{dE(k)}{dt}=\vec{\nabla}_kE(\vec{k})\frac{d\vec{k}}{dt}
\end{equation}
Luego, nos queda
\begin{equation}
    -q_e\vec{\mathscr{E}}\cdot\vec{v}=-\frac{1}{\hbar}q_e\vec{\mathscr{E}}\cdot\vec{\nabla}_kE(\vec{k})\Rightarrow -\frac{q_e}{\hbar}\vec{\mathscr{E}}\cdot\cancel{\vec{\nabla}_kE(\vec{k})}=\cancel{\vec{\nabla}_kE(\vec{k})}\frac{d\vec{k}}{dt}\Rightarrow-\frac{q_e}{\hbar}\vec{\mathscr{E}}=\frac{d\vec{k}}{dt}
\end{equation}
Tenemos entonces que,
\begin{equation}
    \vec{F}_e=-q_e\mathscr{E}=\hbar\frac{d\vec{k}}{dt}
\end{equation}
Por analogía a la Segunda Ley de Newton ($\vec{F}=\frac{d\vec{p}}{dt}$), al producto $\hbar\vec{k}$ se le denomina \textbf{momento cristalino del electrón} (o cuasi-momento). Luego,
\begin{equation}
    \vec{p}=\hbar\vec{k}
\end{equation}
Ahora vamos a pasar a estudiar la aceleración, que nos lleva al concepto de masa efectiva (lo vemos en 1D),
\begin{equation}
    a=\frac{dv}{dt}=\frac{dv}{dk}\frac{dk}{dt}=\frac{1}{\hbar}\frac{d^2E(k)}{dk^2}\frac{dk}{dt}\Rightarrow a=\frac{1}{\hbar}\frac{d^2E(k)}{dk^2}F_e
\end{equation}
obteniendo la forma que queríamos. De esta forma, con la Segunda Ley de Newton, obtenemos que
\begin{equation}
    m^*=\frac{F}{a}=\frac{\hbar^2}{\left(\frac{d^2E(k)}{dk^2}\right)}
\end{equation}
que es la definición de la \textbf{masa efectiva}. \\ \\
En 3 dimensiones tenemos,
    \begin{equation}
        a_i=\frac{dv_i}{dt}=\frac{d}{dt}\curlybraces{\frac{1}{\hbar}\frac{\partial E}{\partial k_i}}=\frac{1}{\hbar}\sum_{j=1}^3\frac{\partial^2E}{\partial k_i\partial k_j}\frac{F_j}{\hbar}=\sum_{j=1}^3\left(\frac{1}{\hbar^2}\frac{\partial^2E}{\partial k_i\partial k_j}\right)F_j
    \end{equation}
    De forma que el tensor de masa efectiva será
    \begin{equation}
        \frac{1}{m_{ij}}=\frac{1}{\hbar^2}\frac{\partial^2E}{\partial k_i\partial k_j}\Rightarrow[m^*]_{ij}=\frac{\hbar^2}{\left(\frac{\partial^2E(k)}{\partial k_i\partial k_j}\right)}
    \end{equation}
Sus propiedades son,
\begin{enumerate}
    \item El tensor $m^*$ es simétrico, por lo que se puede escribir solo con componentes de la diagonal principal.
    \item En 1D es un escalar.
    \item El concepto de $m^*$ nos permite tratar al electrón de Bloch como un electrón libre con este valor de masa.
    \item Para la aproximación en 1D y con $E(k)$ parabólico, podemos escribir que
    \begin{equation}
        E(k)=\frac{\hbar^2k^2}{2m^*}
    \end{equation}
    y la $m^*$ caracteriza la desviación con respecto al electrón libre (fondo del banda).
    \item En 3D sí se cumple la relación anterior,
    \begin{equation}
        [m^*]^{-1}=\frac{1}{m^*}\begin{pmatrix}
            1 & 0 & 0\\
            0 & 1 & 0\\
            0 & 0 & 1
        \end{pmatrix}
    \end{equation}
    En el caso de que $E(k)=\left(\alpha_1k_x^2+\alpha_2k_y^2+\alpha_3k_z^2\right)$ tenemos tres componentes para la masa efectiva, que serán
    \begin{equation}
        m_{xx}^*=\frac{\hbar^2}{2\alpha_1};\hspace{5mm}m_{yy}^*=\frac{\hbar^2}{2\alpha_2};\hspace{5mm}m_{zz}^*=\frac{\hbar^2}{2\alpha_3}
    \end{equation}
    Que será el caso para semiconductores, y a diferencia del electrón libre, las direcciones de fuerza y aceleración no tienen por qué coincidir. Hablamos de elipsoide de masa efectiva.
\end{enumerate}
\subsection{Conducción eléctrica}
Como aplicación al concepto de masa efectiva para el electrón, vamos a estudiar el proceso de conducción eléctrica como reacción de un campo eléctrico aplicado. El flujo para un electrón vendrá dado por,
\begin{equation}
    \vec{j}_e=-\frac{q_e}{V}\vec{v}_e(\vec{k})
\end{equation}
puesto que el flujo de electrones lo interpretamos como el número de electrones que atraviesa una unidad de volumen a una determinada velocidad, y hemos supuesto un electrón.\\ \\
En primer lugar, podemos ver que en el equilibrio, es decir, sin un campo eléctrico aplicado, no existe una velocidad neta, y por tanto, no hay conducción, es decir,
\begin{equation}
    \vec{v}(-\vec{k})=\frac{1}{\hbar}\left.\frac{\partial E(\vec{k})}{\partial \vec{k}}\right|_{\vec{k}=-\vec{k}}=\frac{1}{\hbar}\frac{\partial E(-\vec{k})}{\partial(-\vec{k})}=\frac{1}{\hbar}\frac{\partial E(\vec{k})}{\partial(-\vec{k})}=-\frac{1}{\hbar}\frac{\partial E(\vec{k})}{\partial\vec{k}}=\vec{v}(\vec{k})
\end{equation}
donde usamos que $E(\vec{k})=E(-\vec{k})$, pues la energía debe ser simétrica. Hemos obtenido que $\vec{v}(-\vec{k})=-v(\vec{k})$, entonces las componentes de la velocidad total se anulan en todas partes, pues
\begin{equation}
    \vec{v}_{total}(\vec{k})=\vec{v}(\vec{k})+\vec{v}(-\vec{k})=\vec{v}(\vec{k})-\vec{v}(\vec{k})=0
\end{equation}
Cuando aplicamos un campo eléctrico externo sobre el cristal, se establecerá una dirección que favorezca al movimiento de los electrones, por lo que sí tendremos una velocidad neta. Tendremos que el flujo diferencial sea,
\begin{equation}
    d\vec{j}_e=-\frac{q_e}{V}\frac{d\vec{v}_e(\vec{k})}{dt}dt
\end{equation}
Sabemos que la fuerza de Lorentz será la fuerza que se somete a los electrones para que se produzca su movimiento y además, sabemos que la segunda ley de Newton nos dice que la fuerza es igual a la masa del cuerpo por su aceleración, entonces tendremos que
\begin{equation}
    \vec{F}_e=m^*(\vec{k})\vec{a}_e=-q_e\vec{\mathscr{E}}\Rightarrow\vec{a}_e=\frac{d\vec{v}_e}{dt}=\frac{-q_e}{m^*(\vec{k})}\vec{\mathscr{E}}
\end{equation}
Luego,
\begin{equation}
    d\vec{j}_e=-\frac{q_e}{V}\frac{-q_e}{m^*(\vec{k})}\vec{\mathscr{E}}dt=\frac{q_e^2}{Vm^*(\vec{k})}\vec{\mathscr{E}}dt\Rightarrow\frac{d\vec{j}_e}{dt}=\frac{q_e^2}{Vm^*(\vec{k})}\vec{\mathscr{E}}
\end{equation}
Ahora, asumiendo un modelo probabilístico de Drude, donde $\tau_e$ sea el tiempo medio entre colisiones, el valor medio del flujo de electrones puede estudiarse como,
\begin{equation}
    \frac{\vec{j}_e}{dt}=<\vec{j}_e>=\frac{q_e^2\tau_e}{Vm^*(\vec{k})}\vec{\mathscr{E}}
\end{equation}
Este estudio lo hemos hecho para un único electrón del cristal, entonces asumiendo que tenemos $N$ electrones en el cristal, entonces tendremos
\begin{equation}
    <\vec{j}_e>=\frac{N}{V}\frac{q_e^2\tau_e}{m_e^*(\vec{k})}\vec{\mathscr{E}}=n_e\frac{q_e^2\tau_e}{m_e^*}
\end{equation}
donde $n_e$ representa la densidad electrónica, pues será el número de electrones por unidad de volumen. Hemos impuesto la misma masa efectiva para todos los electrones.\\ \\
De la expresión conocida para el flujo de electrones,
\begin{equation}
    \vec{j}_e=\sigma_e\vec{\mathscr{E}}
\end{equation}
donde $\sigma_e$ representa la conductividad electrónica, que despejamos igualando esta expresión con la que hemos obtenido, tal que
\begin{equation}
    \sigma_e=\frac{q_e^2n_e\tau_e}{m_e^*}
\end{equation}
\subsection{Huecos}
Un hueco es un estado electrónico no ocupado en una banda, es decir, los huecos serán los puntos de la banda donde no tengamos electrones localizados. La masa efectiva de un hueco electrónico será la masa efectiva del electrón que estuviera en este estado no ocupado, con el signo cambiado, tal que
\begin{equation}
   \frac{1}{m_h^*(\vec{k})}=-\frac{1}{m_e^*(\vec{k})}
\end{equation}
De forma análoga al electrón, obtenemos un flujo de huecos y una conductividad, tal que
\begin{equation}
    <\vec{j}_h>=\frac{n_hq_e^2\tau_h}{m_h^*}\vec{\mathscr{E}};\hspace{5mm}\sigma_h=\frac{q_e^2n_h\tau_h}{m_h^*}
\end{equation}
cabe recalcar que este flujo irá en sentido contrario al flujo de los electrones, pues, suponiendo un mismo campo eléctrico para ambos, la masa efectiva de los huecos es la masa efectiva de los electrones con signo opuesto.
\subsection{Densidad de estados}
De forma análoga al caso de la densidad de estados para las vibraciones de la red, visto anteriormente, vamos a calcular la densidad de estados electrónica, que será una magnitud de gran importancia en fenómenos de transporte.
\begin{definition}
    La densidad de estados electrónica será el número de estados electrónicos por unidad de volumen en el intervalo $[E,E+dE]$, es decir,
    \begin{equation}
        D(E)dE=\rho_kdV_k
    \end{equation}
    Para el caso 1D, $dV_k\equiv dL_k=dk$.\\
    Para el caso 2D, $dV_k\equiv dS_k=\pi kdk$ para el caso circular.\\
    Para el caso 3D, $dV_k=4\pi k^2dk$ para el caso esférico.
\end{definition}
Esta densidad de estados electrónica la supondremos situada cerca del fondo de la banda, por lo que podremos usar la aproximación de electrones cuasi-libres, cuya energía es
\begin{equation}
    E=\frac{\hbar^2k^2}{2m^*_e}
\end{equation}
Procedemos de la misma forma que con los fonones, por lo que en 3D, tendremos que
\begin{equation}
    \rho_k=\frac{V}{(2\pi)^3};\hspace{6mm}dV_k=4\pi k^2dk
\end{equation}
Primero despejamos $k$ de la energía, tal que
\begin{equation}
    k^2=\frac{2m_e^*E}{\hbar^2}\Rightarrow k=\left(\frac{2m^*_eE}{\hbar^2}\right)^{1/2}
\end{equation}
que tomando el diferencial,
\begin{equation}
    dk=\left(\frac{2m_e^*}{\hbar^2}\right)^{1/2}\frac{1}{2\sqrt{E}}dE
\end{equation}
Por tanto, la densidad de estados electrónica queda
\begin{equation}
    D(E)dE=\frac{V}{(2\pi)^3}4\pi\frac{2m_e^*E}{\hbar}\left(\frac{2m_e^*E}{\hbar^2}\right)^{1/2}\frac{1}{2\sqrt{E}}dE=\frac{V}{4\pi^2}\left(\frac{2m^*_e}{\hbar^2}\right)^{3/2}E^{1/2}dE
\end{equation}
Normalmente tomamos $V=1$, por lo que tendremos la densidad de estados electrónica por unidad de volumen y debemos añadir la degeneración de espín, con lo que nos queda
\begin{equation}
    g(E)=\frac{1}{2\pi^2}\left(\frac{2m_e^*}{\hbar^2}\right)^{3/2}E^{1/2}
\end{equation}

\section{Modelo de Banda Vacía}
Este modelo consiste en tomar un potencial débil. Comenzamos con la red recíproca $(k)$, recordando las funciones de Bloch,
\begin{equation}
    \Psi_{n,k}=u_n(k)e^{i\vec{k}\cdot\vec{r}}
\end{equation}
Como estamos en el modelo de potencial débil, $V\downarrow\downarrow$, haremos un tratamiento perturbativo de la energía, que será
\begin{equation}
    E=\frac{p^2}{2m}=\frac{\hbar^2k^2}{2m}=E(k)
\end{equation}
siendo la aproximación de electrones cuasi-libres sin usar la masa efectiva. Representando,
\begin{Figura}
    \centering
    \includegraphics[width=0.5\linewidth]{Imagenes//Capitulo5/bandavacia.png}
    \captionof{figure}{Representación de la energía frente a $K$.}
    \label{bandavacia}
\end{Figura}
Reducimos todas las soluciones a la primera banda, correspondiente a la PZB. Por tanto, calculamos las correcciones de $E(k)$ en la primera banda, que será
\begin{equation}
    E_1(k)=E_1^{(0)}(k)+\Braket{1,k|V|1,k}+\sum_{k',n}\frac{\left|\Braket{n,k'|V|1,k}\right|^2}{E_1^{(0)}(k)-E_n^{(0)}(k)}
\end{equation}
correspondiente a la energía calculada con método perturbativo. Analizamos el segundo sumando, tal que
\begin{equation}
    \Braket{\Psi_{1,k}^{(0)}|V|\Psi_{1,k}^{(0)}}=\frac{1}{L}\int_{red}e^{-ikx}V(x)e^{ikx}dx=\frac{1}{L}\int_{red}V(x)dx
\end{equation}
Esta integral representa el promedio sobre toda la red del potencial, y por tanto, es un valor constante e independiente de $k$ que no variará el resultado de $E(k)$, por lo que podremos considerarlo nulo.\\ \\
Ahora analizamos el sumatorio. Para ello, consideramos un potencial pequeño, por lo que la diferencia entre bandas es muy pequeña, salvo para bandas consecutivas, por tanto nos queda
\begin{equation}
    \sum_{k',n}\frac{\left|\Braket{n,k'|V|1,k}\right|^2}{E_1^{(0)}(k)-E_n^{(0)}(k)}\approx\sum_{k'}\frac{\left|\Braket{2,k'|V|1,k}\right|^2}{E_1^{(0)}(k)-E_2^{(0)}(k)}
\end{equation}
Vamos a introducir el desarrollo de Fourier en la red recíproca del potencial, es decir,
\begin{equation}
    V(x)=\sum_kV_ke^{ikx}
\end{equation}
donde
\begin{equation}
    V_k=\frac{1}{L}\int_0^LV(x)e^{ikx}dx
\end{equation}
Como $V(x)$ es periódico con la red, solo contribuyen los $\vec{k}\in\vec{G}$, por lo que el factor queda
\begin{equation}
    \Braket{2,k'|V|1,k}=\Braket{k''|V|k}=\frac{1}{L}\int_0^Le^{-ik''x}V(x)e^{ikx}dx=\frac{1}{L}\int_0^Le^{-i(k''-k)x}V(x)dx
\end{equation}
que solo será distinta de cero si $k''-k=\frac{2\pi}{a}m\in\vec{G}$. Para $m=1$ tendremos que
\begin{equation}
    \Braket{2,k'|V|1,k}=\frac{1}{L}\int_0^Le^{-i\left(k+\frac{2\pi}{a}\right)x}V(x)e^{ikx}dx=\frac{1}{L}\int_0^Le^{-i\frac{2\pi}{a}x}V(x)dx\equiv V_1
\end{equation}
Por tanto, la corrección queda,
\begin{equation}
    E_1(k)\approx E_1^{(0)}(k)+\frac{|V_1|^2}{E_1^{(0)}(k)-E_2^{(0)}(k)}
\end{equation}
Como el potencial es débil (pequeño), consideramos que el denominado solo tiene valores apreciables cuando su valor es muy bajo, es decir, donde los puntos de las distintas ramas intersecan, que será en $k=\pm\pi/a$, en relación a la Figura \ref{bandavacia}. Mediante la teoría de perturbaciones llegamos a que
\begin{equation}
    E_{\pm}(k)=\frac{1}{2}\curlybraces{E_1^{(0)}(k)+E_2^{(0)}(k)\pm\left[\left(E_2^{(0)}(k)-E_1^{(0)}(k)\right)^2+4|V_1|^2\right]^{1/2}}
\end{equation}
donde el signo '+' corresponde al nivel superior y el signo '-', al nivel inferior.\\ \\
Para $E_1^{(0)}(\pm\pi/a)=E_2^{(0)}(\pm\pi/a)$ obtenemos que la energía del 'gap' vale
\begin{equation}
    E_g=E_+(\pm\pi/a)-E_-(\pm\pi/a)=2|V_1|
\end{equation}
donde hemos aplicado el Teorema del Virial de forma implícita.\footnote{Si la fuerza entre dos partículas cualesquiera del sistema es producida por una energía potencial $V(r)=ar^n$ que es proporcional a alguna potencia $n$ de la distancia entre las partículas $r$, el teorema del virial adopta la forma $2<T>=n<V>$.}\\ \\
Hemos llegado a la conclusión de que entre bandas de mayor energía, el 'gap' tiende a disminuir, dado que $E_g\sim|V_n|$ y por convergencia del desarrollo de Fourier, las componentes del potencial $V_k$ deben decrecer a medida que aumenta el orden.\\ \\
Para terminar, estudiamos las funciones de onda mediante perturbaciones, tal que
\begin{equation}
    \Psi_{1,k}=\Psi_{1,k}^{(0)}+\frac{V_1}{E_1^{(0)}(k)-E_2^{(0)}(k)}\Psi_{2,k}^{(0)}
\end{equation}

donde $\Psi_{2,k}^{(0)}=\frac{1}{L^{1/2}}e^{i\left(k-\frac{2\pi}{a}\right)x}$. Si obtenemos $\Psi_{\pm}(k)$ tenemos
\begin{equation}
    \Psi_{\pm}(k)=\frac{1}{\sqrt{2}}\left(\Psi_1^{(0)}(\pi/a)\pm\Psi_2^{(0)}(\pi/a)\right)=\frac{1}{(2L)^{1/2}}\left(e^{i\frac{\pi}{a}x}\pm e^{-i\frac{\pi}{a}x}\right)
\end{equation}
que serán ondas estacionarias.
\lhead{\emph{Capítulo 6: Gas de Fermi de electrones libres}}
\chapter{Gas de Fermi de electrones libres}
La hipótesis básica del modelo es que los electrones están atrapados, desde el punto de vista de la mecánica cuántica, en un pozo de potencial que no manifiesta ninguna de las condiciones de periodicidad vistas en capítulos anteriores (electrones libres) y que su función es mantenerlos dentro del cristal, siendo un modelo que se ajusta muy bien al caso de los elementos metálicos (enlace metálico).\\ \\
Con este planteamiento tendremos,
\begin{equation}
    E=\frac{\hbar^2k^2}{2m_e}\label{cap5-1}
\end{equation}
donde $m_e$ es la masa del electrón, que será diferente de $m^*$ (masa efectiva) cuando teníamos en cuenta el potencial periódico.\\ \\
Pasando al espacio de posiciones, la ecuación de Schrödinger será,
\begin{equation}
    \frac{\left(i\hbar\vec{\nabla}\right)^2}{2m_e}\Psi_k=E_k\Psi_k
\end{equation}
donde $\Psi_k=\frac{1}{\sqrt{V}}e^{i\vec{k}\cdot\vec{r}}$, siendo $V$ el volumen, pues estamos en el caso 3D. Además, las $\Psi_k$ son autofunciones tanto del Hamiltoniano como del momento, pues
\begin{equation}
    \vec{p}=-i\hbar\vec{\nabla}\Rightarrow(-i\hbar\vec{\nabla})\Psi_k=\hbar\vec{k}\Psi_k
\end{equation}

Con este punto de partida, el problema se divide en dos partes fundamentales: \textbf{Propiedades térmicas} y \textbf{Propiedades de conducción eléctrica}.
\section{Propiedades térmicas}
Vamos a tratar el problema del cálculo del valor de la componente electrónica de $C_v$, para lo cual plantearemos un procedimiento análogo a lo visto en el capítulo 4 para la componente fonónica de $C_v$. En particular, utilizaremos como eje la función de probabilidad de Fermi-Dirac. Como sabemos,
\begin{equation}
    E_e=\int_0^{\infty}ED(E)f(E,T)dE
\end{equation}
siendo $E_e$ la componente electrónica y el valor de $C_v$ será
\begin{equation}
    C_{v,e}=\frac{\partial E_e}{\partial T}=\int_0^{\infty}ED(E)\frac{\partial f(E,T)}{\partial T}dE
\end{equation}
Vamos a ir viendo los distintos factores que aparecen en estas integrales.
\subsection*{Densidad de estados}
Tal y como vimos en capítulos anteriores, la $D(E)$ representa la densidad de estados, que calculamos utilizando la densidad en el espacio $k$, $\rho_k$, obtenida mediante condiciones de contorno; multiplicando por el diferencial de volumen para el espacio $k$ y añadiendo el factor de la degeneración del espín, tal que
\begin{equation}
    D(E)dE=2\rho_kdV_k=2\frac{V}{(2\pi)^3}4\pi k^2dk=\frac{V}{\pi^2}k^2dk
\end{equation}
Para ponerlo en función de la energía, usamos que la energía de los electrones libres es (\ref{cap5-1}), despejamos $k$ y diferenciamos, tal que
\begin{equation}
    E=\frac{\hbar^2k^2}{2m_e}\Rightarrow k=\frac{\sqrt{2m_eE}}{\hbar}\Rightarrow dk=\frac{dE}{\hbar}\sqrt{\frac{m_e}{2E}}
\end{equation}
Sustituyendo en la densidad de estados tenemos que,
\begin{equation}
    D(E)=\frac{V(2m_e)^{3/2}}{2\pi^2\hbar^3}E^{1/2}\Rightarrow D(E)\propto E^{1/2}
\end{equation}
Luego, la $D(E)$ crece como $\sqrt{E}$ y, cerca de $E_F$, puede aproximare por su valor $D(E_F)$ (aprox. constante).
\subsection*{Estadística de Fermi-Dirac}
La función $f(E,T)$ representa una distribución de probabilidad, que en nuestro caso usaremos la de Fermi-Dirac, pues permite llegar a resultados coherentes con los valores experimentales en este problema, ya que es la que describe a los fermiones (electrones), siendo
\begin{equation}
    f(E,T)=\frac{1}{e^{(E-\mu)/k_BT}+1}
\end{equation}
donde $\mu$ es el potencial químico y $k_B$ es la constante de Boltzmann.\\ \\
Vamos a ver el comportamiento de la estadística de Fermi-Dirac a una temperatura de $0ºK$, puesto que nos aporta un parámetro que utilizaremos a lo largo del capítulo, que es la \textbf{energía de Fermi}, $E_F$.\\ \\
Todos los estados con $E<\mu$ se encuentran ocupados a $T=0ºK$ y para los estados con $E>\mu$, están desocupados, como vemos en la Figura \ref{cap6-fermi}.\\ \\
Definimos la \textbf{energía de Fermi} como la energía del último estado ocupado para $T=0ºK$.
\begin{Figura}
    \centering
    \includegraphics[width=0.5\linewidth]{Imagenes//Capitulo6/fermi.png}
    \captionof{figure}{Representación de la distribución de probabilidad de Fermi-Dirac para $T=0ºK$ y $T=298ºK$.}
    \label{cap6-fermi}
\end{Figura}
En función de la energía de Fermi, definimos la función de probabilidad, teniendo en cuenta que $\mu$ es un parámetro que prácticamente no varía con la temperatura como,
\begin{equation}
    f(E,T)=\frac{1}{e^{(E-E_F)/k_BT}+1}
\end{equation}
Como concepto asociado, tenemos el del \textbf{vector de onda de Fermi}, dado por
\begin{equation}
    E_F=\frac{\hbar^2k_F^2}{2m_e}\Rightarrow k_F=\frac{\sqrt{2m_eE_F}}{\hbar}
\end{equation}
Este parámetro lo usaremos en problemas, referido a la \textbf{esfera de Fermi}, que se genera por rotación de $k_F$ en 3D.
\begin{Figura}
    \centering
    \includegraphics[width=0.5\linewidth]{Imagenes//Capitulo6/esferaFermi.png}
    \captionof{figure}{Representación de la esfera de Fermi. Para $k<k_F$ tenemos los estados ocupados y para $k>k_F$ tenemos los estados vacíos.}
    \label{esferaFermi}
\end{Figura}
De la misma forma que en capítulos anteriores, utilizamos una relación de completitud, asociada al número de electrones, para poder poner $E_F$ y parámetros asociados al sistema en función a características del material. El número de electrones vendrá dado por
\begin{equation}
    N=\int_0^{\infty}D(E)f(E,T)dE
\end{equation}
Si suponemos que el número $N$ de electrones es constante, cosa lógica, entonces podemos evaluar la integral a temperatura $T=0ºK$, tal que
\begin{equation}
    N=\int_0^{E_F}D(E)\left.f(E,T)\right|_{T=0}dE
\end{equation}
donde a $T=0ºK$, la ocupación es un escalón, $f(E,0)=1$ para $E<E_F$ y cero para $E>E_F$, uno por estado cuántico (o dos contando espín). Entonces,
\begin{equation}
    N=\frac{V(2m_e)^{3/2}}{2\pi^2\hbar^3}\int_0^{E_F}E^{1/2}dE=\frac{2}{3}\frac{V(2m_e)^{3/2}}{2\pi^2\hbar^3}E_F^{3/2}
\end{equation}
Despejando la energía de Fermi obtenemos,
\begin{equation}
    E_F=\frac{\hbar^2}{2m_e}\left(3\pi^2n\right)^{2/3}
\end{equation}
donde $n=N/V$ representa la densidad electrónica del material, y aplicando el modelo de electrones libres,
\begin{equation}
    E_F=\frac{\hbar^2k_F^2}{2m_e}\Rightarrow k_F=(3\pi^2n)^{1/3}
\end{equation}
Podemos definir también un parámetro interesante denominado \textbf{temperatura de Fermi}, que vendrá dado por
\begin{equation}
    T_F=\frac{E_F}{k_B}
\end{equation}
tal que
\begin{equation}
    T_F=\frac{\hbar^2}{2m_e}\frac{(3\pi^2n)^{2/3}}{k_B}
\end{equation}
Usando los parámetros asociados a cada material obtenemos la siguiente tabla,
\begin{Figura}
    \centering
    \includegraphics[width=\textwidth]{Imagenes//Capitulo6/tabla.png}
    \label{cap6-tabla}
\end{Figura}
Una vez vistas $D(E)$ y $f(E,T)$, así como sus magnitudes asociadas $E_F, k_F,\dots$; podemos calcular las integrales propuestas al principio del capítulo para $E_e$ y $C_{v,e}$.\\ \\
Para obtener $C_{v,e}$, vamos a tener en cuenta que la densidad de estados, $D(E)$, es independiente de la temperatura y que el número de electrones, $N$, también es independiente de $T$. Luego, aplicando ambas condiciones en la ecuación de completitud (la de $N$), derivando ambos lados respecto la temperatura, obtenemos que
\begin{equation}
    0=\int_0^{\infty}D(E)\frac{\partial f(E,T)}{\partial T}dE
\end{equation}
Esta ecuación se puede multiplicar por la energía de Fermi, tal que
\begin{equation}
    0=\int_0^{\infty}D(E)E_F\frac{\partial f(E,T)}{\partial T}dE
\end{equation}
Luego, restando la integral de $C_{v,e}$ obtenemos que
\begin{equation}
    C_{v,e}=\int_0^{\infty}(E-E_F)D(E)\frac{\partial f(E,T)}{\partial T}dE
\end{equation}
Hemos obtenido una ecuación alternativa para calcular $C_{v,e}$, esto nos permite que al derivar la función de distribución,
\begin{equation}
    \frac{\partial f(E,T)}{\partial T}=\frac{(E-E_F)}{k_BT^2}\frac{\exp\left[\frac{(E-E_F)}{k_BT}\right]}{\left(\exp\left[\frac{(E-E_F)}{k_BT}\right]+1\right)^2}
\end{equation}
de manera muy similar a lo visto en el capítulo 4 con la distribución de Boltzmann, luego haciendo el cambio de variable $x=(E-E_F)/k_BT$, tendremos
\begin{equation}
    C_{v,e}=k_B^2TD(E_F)\int_{-E_F/k_BT}^{\infty}\frac{x^2e^x}{(e^x+1)^2}dx
\end{equation}
Cabe recalcar que hemos sustituido $D(E)$ por $D(E_F)$ porque la derivada solo toma valores considerables en el entorno de $E_F$ a las temperaturas estudiadas.\\ \\
Para terminar, en esta ecuación podemos llevar el límite inferior de la integral a $-\infty$, pues $E_F/k_BT\sim100$ y no hay prácticamente error, luego
\begin{equation}
    \int_{-\infty}^{\infty}\frac{x^2e^x}{(e^x+1)^2}dx=\frac{\pi^2}{3}
\end{equation}
Por tanto,
\begin{equation}
    C_{v,e}=\frac{\pi^2}{3}D(E_F)k_B^2T
\end{equation}
Ahora evaluamos $D(E_F)$, considerando que $D(E)=cte\sqrt{E}$, luego
\begin{equation}
    N=cte\int_0^{E_F}E^{1/2}dE=\frac{2cte}{3}E_F^{3/2}\Rightarrow cte=\frac{3N}{2E_F^{3/2}}
\end{equation}
Por lo tanto, la densidad de estados queda
\begin{equation}
    D(E)=\frac{3N}{2E_F^{3/2}}E^{1/2}
\end{equation}
Luego,
\begin{equation}
    D(E_F)=\frac{3N}{2E_F}
\end{equation}
que al sustituir nos queda,
\begin{equation}
    C_{v,e}=\frac{\pi^2Nk_BT}{2E_F}=\frac{\pi^2Nk_B}{2}\left(\frac{T}{T_F}\right)
\end{equation}
Sus propiedades son las siguientes:
\begin{enumerate}
    \item $C_{v,e}$ es proporcional a $T$.
    \item $C_{v,e}/C_{v,e,clasico}\sim T/T_F$
    \item Para bajas temperaturas tendremos que
    \begin{equation}
        C_{v,total}=\alpha\frac{T}{T_F}+\beta T^3
    \end{equation}
    \item Para altas temperaturas tendremos que
    \begin{equation}
        C_{v,total}=3Nk_B+\alpha\frac{T}{T_F}
    \end{equation}
\end{enumerate}

\section{Propiedades de la conducción eléctrica}
El objetivo de este apartado es estudiar la dependencia con la temperatura del parámetro de resistividad, $\rho$, definido como el inverso de la conductividad $\sigma$,
\begin{equation}
    \rho=\frac{1}{\sigma}
\end{equation}
Lo haremos mediante el uso del modelo de Drude, utilizando los parámetros microscópicos que definen la conductividad. Sabemos que el flujo de electrones viene dado por,
\begin{equation}
    \vec{j}=\sigma\vec{\mathscr{E}}\to\text{ Fenomenológico}
\end{equation}
\begin{equation}
    \vec{j}=-en\vec{v}_d\to\text{ Microscópico}
\end{equation}
siendo $\vec{v}_d$ la velocidad deriva, dada por
\begin{equation}
    \vec{v}_d=-\frac{e\vec{\mathscr{E}}}{m_e}\tau
\end{equation}
con $\tau$ el tiempo entre colisiones dado por el modelo de Drude, utilizando la distribución de probabilidad exponencial, como ya se vio en el capítulo anterior.\\ \\
Sustituyendo la expresión de $\vec{v}_d$ en el flujo y agrupando los términos obtenemos que
\begin{equation}
    \vec{j}=-en\frac{-e\vec{\mathscr{E}}}{m_e}\tau=\frac{e^2n\tau}{m_e}\vec{\mathscr{E}}
\end{equation}
Por tanto, igualando ambos flujos,
\begin{equation}
    \sigma=\frac{ne^2\tau}{m_e}
\end{equation}
Luego, la resistividad será
\begin{equation}
    \rho=\frac{1}{\sigma}=\frac{m_e}{e^2n\tau}
\end{equation}
que corresponde a la definición microscópica de la resistividad. A partir de esta expresión vamos a ver su dependencia con la temperatura, estudiando el valor de $\tau(T)$.\\ \\
En principio, vamos a dividir en dos las contribuciones del valor de $\tau$, tal que
\begin{equation}
    \frac{1}{\tau}=\frac{1}{\tau_f(T)}+\frac{1}{\tau_0}
\end{equation}
donde $\tau_f(T)$ es el tiempo medio de colisiones con los fonones, que depende de la temperatura, y $\tau_0$ es el tiempo medio de colisión con las impurezas, que será independiente de la temperatura.\\ \\
Por lo tanto, podemos distinguir dos términos de resistividad sustituyendo $\tau$, tal que
\begin{equation}
    \rho=\frac{m_e}{e^2n}\left(\frac{1}{\tau_f}+\frac{1}{\tau_0}\right)=\underbrace{\frac{m_e}{e^2n\tau_f(T)}}_{\rho_1}+\underbrace{\frac{m_e}{e^2n\tau_0}}_{\rho_0}
\end{equation}
donde $\rho_1$ se denomina \textbf{resistividad ideal}, pues no incluye las impurezas, y $\rho_0$ representa la \textbf{resistividad residual}.
\begin{Figura}
    \centering
    \includegraphics[width=0.7\linewidth]{Imagenes//Capitulo6/resistividad.png}
    \captionof{figure}{Representación de la resistividad frente a la temperatura.}
    \label{resistividad}
\end{Figura}
Para analizar el comportamiento con la temperatura, partimos del hecho de que $\tau_f$ representa el choque con fonones y que, por tanto, la resistividad asociada $\rho_1$ aumentará al aumentar el número de estos fonones, es decir,
\begin{equation}
    \rho_1(T)\propto n_f\propto\frac{1}{e^{\hbar\omega_D/k_BT}-1}
\end{equation}
pues los fonones se describen por la estadística de Bose-Einstein, y por tanto, podemos establecer dos rangos:
\begin{itemize}
    \item \textbf{Bajas temperaturas:} En este rango de temperaturas, hacemos que $T\to0$, por tanto,
    \begin{equation}
        \lim_{T\to0}\rho_1(T)\propto\lim_{T\to0}\frac{1}{e^{\hbar\omega_D/k_BT}-1}\to\frac{1}{e^{1/0}-1}\to \frac{1}{e^{\infty}-1}\to0
    \end{equation}
    Cosa coherente, pues al no existir excitación térmica de fonones, es lógico que no se produzcan colisiones, y por tanto, desaparezca la resistividad sin impurezas.
    \item \textbf{Altas temperaturas:} Para temperaturas altas hacemos $T\ggg1$, entonces tendremos que
    \begin{equation}
        \rho_1(T)\propto\frac{1}{e^{\hbar\omega_D/k_BT}-1}
    \end{equation}
    Luego, $\frac{\hbar\omega_D}{k_BT}\lll1$, por lo que podremos desarrollar en serie la exponencial, tal que
    \begin{equation}
        e^x\approx\left(1+x+\frac{x^2}{2}\right)
    \end{equation}
    Por tanto,
    \begin{equation}
        \rho_1(T)\propto\frac{1}{\hbar\omega_D/k_BT}\propto T
    \end{equation}
    Luego, la resistividad aumenta de forma lineal con la temperatura en este rango.
\end{itemize}

\section{Propiedades de la conductividad térmica}
Como ya hemos visto en capítulos anteriores, se puede definir la conductividad térmica, $k_T$, según la ecuación de flujo de energía, dada por
\begin{equation}
    \vec{j}_Q=-k_T\nabla T
\end{equation}
En este capítulo, puesto que estamos tratando el modelo de electrones libres, tenemos que tener en cuenta que para estos elementos $k_{T,e}\sim10^2k_{T,f}$, donde $k_{T,e}$ es la conductividad térmica electrónica y $k_{T,f}$, la conductividad térmica fonónica.\\ \\
La dependencia de $k_T$ con la temperatura la hacemos de forma análoga a la que vimos con los fonones, según
\begin{equation}
    k_T=\frac{1}{3}C_vv_Fl
\end{equation}
tomando $C_{v}=c_{v,e}\cdot n_{mol}$, donde $n_{mol}$ es el número de moles por unidad de volumen; hemos visto que $c_{v,e}\propto T$ y podemos poner $C_v$ como
\begin{equation}
    C_v=\frac{\pi^2k_B^2n}{m_ev_F^2}T
\end{equation}
donde $E_F=\frac{1}{2}m_ev_F^2$ para los electrones y estamos usando $C_{v,e}$ por unidad de volumen.\\ \\
Este cambio se hace porque al aparecer $v_F$ en la expresión de $k_T$, podemos ver de forma más clara la dependencia con la temperatura si tenemos en cuenta que $l$ es el recorrido libre medio, dado por $l=v_F\tau$ y que la velocidad de los electrones conductores de energía es $v\approx v_F$, dado por la esfera de Fermi.\\ \\
Por lo tanto, en la expresión final de $k_T$ desaparece el factor $v_F$, quedando
\begin{equation}
    k_T=\frac{\pi^2nk_B^2}{3m_e}\tau T
\end{equation}
donde $\tau$ y $T$ serán los factores de dependencia de $k_T$.\\ \\
A bajas temperaturas sabemos que $\tau$ no depende de la temperatura, por tanto $k\propto T$.\\ \\
A altas temperaturas, sabemos que $\tau\propto1/T$, luego al multiplicar por $T$, la dependencia se pierde, siendo $k_T=cte$ a altas temperaturas.
\begin{Figura}
    \centering
    \includegraphics[width=0.7\linewidth]{Imagenes//Capitulo6/kt.png}
    \captionof{figure}{Dependencia de la conductividad térmica con la temperatura.}
    \label{cap6-kt}
\end{Figura}

\newpage

\begin{appendices}
\chapter{Teoría Cinética de los Gases}\label{cinetica}
\lhead{Apéndice A. \emph{Teoría Cinética de los Gases}}
Si aceptamos la estadística de Maxwell-Boltzmann, una molécula de gas tiene una energía dada por,
\begin{equation}
    \frac{1}{2}mv^2=3\left[\frac{1}{2}k_BT\right]
\end{equation}
La transmisión de energía se da por colisiones entre las moléculas, de forma que, si las moléculas son rápidas, la temperatura sube, luego se mueven y chocan más moléculas, transmitiendo más energía.\\ \\
Se define el flujo de calor,
\begin{equation}
    J=-k_T\frac{dT}{dx}
\end{equation}
donde $k_T$ es la conductividad térmica.\\ \\
Para una molécula con temperatura $T$ que pasa a una zona de temperatura $T-\Delta T$, aporta una energía $c\Delta T$; donde $c$ es la capacidad calorífica por molécula.\\ 
En promedio, para una molécula con velocidad $v$ en la dirección $X$, tenemos que la temperatura $\Delta T$ vale,
\begin{equation}
    \Delta T=\left(\frac{\partial T}{\partial x}\right)v_x\tau
\end{equation}
donde $\tau$ es el tiempo medio entre colisiones.\\ 
El valor del flujo de calor será por tanto,
\begin{equation}
    J=-nv_xcv_x\tau\left(\frac{\partial T}{\partial x}\right)=-ncv_x^2\tau\left(\frac{\partial T}{\partial x}\right)
\end{equation}
donde $n$ es la densidad, $v_x$ es la velocidad en el eje $X$, $c$ es la capacidad calorífica y $\tau$ es el tiempo medio entre colisiones.\\ 
En el caso isotrópico, sabemos que
\begin{equation}
    <v_x^2>=\frac{1}{3}v^2
\end{equation}
Luego,
\begin{equation}
    J=-\frac{1}{3}ncv^2\tau\frac{\partial T}{\partial X}
\end{equation}
Definiendo el recorrido medio como $l=v\tau$ y $C_v=nc$ como la capacidad calorífica por unidad de volumen, tenemos que
\begin{equation}
    J=-\frac{1}{3}C_v\bar{v}l\frac{\partial T}{\partial X}
\end{equation}
Luego, si lo identificamos con la definición del flujo de calor, obtenemos que
\begin{equation}
    k_T=\frac{1}{3}C_v(T)\bar{v}(T)l(T)
\end{equation}
que será la definición de la conductividad térmica para la teoría cinética de los gases.\\ \\
Este resultado lo emplearemos para el caso de los sólidos porque vamos a considerar un gas de fonones.
\end{appendices}



%---------------------------------------------------------------------------
%	BIBLIOGRAPHY
%---------------------------------------------------------------------------

\newpage
\lhead{\emph{Bibliografía}}
% 	\label{Bibliography}
	\lhead{\emph{Bibliograf\'ia}}
        \nocite{*}
	\bibliographystyle{plainnat} 
	\bibliography{librarys/library}

% \addcontentsline{toc}{section}{\textbf{Bibliografía}}

\addcontentsline{toc}{section}{\textbf{Bibliografía}}
\bibliographystyle{plainnat}
\nocite{*}
\bibliography{librarys/library}
% \renewcommand{\refname}{Bibliografía de Figuras}
%%%%%%%%%%%%%%%%%%%%%%%%%%%%%%%%%%%%%%%%%%%%%%%%%%%%%%%%%%
%---------------BIBLIOGRAFÍA DE FIGURAS------------------%
%%%%%%%%%%%%%%%%%%%%%%%%%%%%%%%%%%%%%%%%%%%%%%%%%%%%%%%%%%


% \renewcommand\bibname{Bibliografía de Figuras}
% \begin{thebibliography}{99}
%     \bibitem{BFig1-ApB}
        
%         \label{BFig1-ApB}
% \bibitem{BFig2-ApB}
        
%          \label{BFig2-ApB} 
% \bibitem{BFig3-ApB}
        
%         \label{BFig3-ApB}         
% \bibitem{BFig4-ApB}
        
%         \label{BFig4-ApB} 
% \end{thebibliography}







\addcontentsline{toc}{section}{\textbf{Bibliografía de Figuras}}



\end{document}
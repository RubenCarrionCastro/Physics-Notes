\documentclass[11pt]{article}
%\usepackage[spanish]{babel}
\RequirePackage{etex}
\usepackage[utf8]{inputenc}
\usepackage{braket}
%\usepackage[sc]{mathpazo}
% \linespread{1.5}
%\usepackage[T1]{fontenc}
%\usepackage{heuristica}
%\usepackage[erewhon,vvarbb,bigdelims]{newtxmath}
%\renewcommand*\oldstylenums[1]{\textosf{#1}}
\usepackage{enumitem}
\usepackage{array}
\usepackage{textcomp}
\usepackage{pdfpages}
\usepackage{feynmp-auto}
\usepackage{fancyhdr}
\usepackage{amsmath, amsthm}
\usepackage{slashed}
\usepackage[normalem]{ulem}
\usepackage{amsfonts}
\usepackage{amssymb}
\usepackage{mathtools}
\usepackage{float}
\usepackage{soul}
\usepackage{graphicx}
\usepackage{hyperref}
\usepackage{graphicx}
\usepackage{pstricks-add}
\usepackage{color}
\usepackage{caption}
\usepackage[margin=0.9in]{geometry}
\usepackage{marvosym}
\usepackage{mathtools}
\usepackage{framed}
\usepackage{calrsfs}
\usepackage[mathscr]{euscript}
\usepackage{tensor}
\usepackage{autonum}
\usepackage{cancel}
\usepackage[most]{tcolorbox}

\newtheorem{thm}{Teorema}[section]
\newtheorem{theorem}{Teorema}[section]
\newtheorem{proposition}[thm]{Proposición} 
\newtheorem{lemma}[thm]{Lema}
\newtheorem{corollary}[thm]{Corolario} 
\newtheorem{conv}[thm]{Convención}
\newtheorem{defi}[thm]{Definición}
\newtheorem{definition}[theorem]{Definición}
\newtheorem{notation}[thm]{Notación} 
\newtheorem{exe}[thm]{Ejemplo}
\newtheorem{conjecture}[thm]{Conjetura} 
\newtheorem{prob}[thm]{Problema}
\newtheorem{remark}[thm]{Observación}
\newtheorem{example}[thm]{Ejemplo}
\newtheorem{note}[thm]{Nota}

\newcommand{\brackets}[1]{\left[#1\right]}
\newcommand{\curlybraces}[1]{\left\{#1\right\}}
\newcommand{\qedh}{\hfill\hspace{5mm}\fbox{\phantom{\rule{.5ex}{.5ex}}}}
\newcommand{\scalar}[2]{\langle #1, #2 \rangle}
\newcommand{\ptensor}[2]{#1 \otimes #2}
\newcommand{\pcart}[2]{#1 \times #2}
\newcommand{\voverrightarrowtor}[3]{\begin{pmatrix}#1\\ #2\\ #3\end{pmatrix}}
\newcommand{\cooverrightarrowtor}[3]{\begin{pmatrix}#1 & #2 & #3\end{pmatrix}}
\newcommand{\abss}[1]{\begin{vmatrix}#1\end{vmatrix}^2}

\newtcolorbox[auto counter, number within=section]{mytheorem}[2][]{
  enhanced,
  breakable,
  title=Teorema~\thetcbcounter: #2,
  #1,
}
\newtcolorbox[auto counter, number within=section]{propositionbox}[2][]{
  enhanced,
  breakable,
  title=Proposition~\thetcbcounter: #2,
  #1,
}

\newtcolorbox[auto counter, number within=section]{corollarybox}[2][]{
  enhanced,
  breakable,
  title=Corollary~\thetcbcounter: #2,
  #1,
}

\newtcolorbox[auto counter, number within=section]{remarkbox}[2][]{
  enhanced,
  breakable,
  title=Remark~\thetcbcounter: #2,
  #1,
}

\newtcolorbox[auto counter, number within=section]{notebox}[2][]{
  enhanced,
  breakable,
  title=Note~\thetcbcounter: #2,
  #1,
}


\newenvironment{Figura}
  {\par\medskip\noindent\minipage{\linewidth}}
  {\endminipage\par\medskip}
%\usepackage[spanish]{babel}
\title{\huge{\textbf{Evaluación III. Teoría de Campos y Partículas}}}
\author{\textbf{}\\ \\Rubén Carrión Castro\\}
% \textit{Los Chavales}
\date{Abril 2025}
\begin{document}
\maketitle
\begin{enumerate}
\item \textbf{En primera aproximación, la amplitud de colisión para un proceso }$\mathbf{\varphi\varphi\to\varphi\varphi}$\textbf{ de una partícula sin espín con carga eléctrica }$\mathbf{e}$\textbf{ es}
\[\mathbf{\mathscr{M}_{34,12}=e^2\brackets{\frac{(p^{\mu}_1+p^{\mu}_3)(p_{2\mu}+p_{4\mu})}{t}+\frac{(p_1^{\mu}+p_4^{\mu})(p_{2\mu}+p_{3\mu})}{u}}}\]
\textbf{donde }$\mathbf{e}$\textbf{ es la intensidad de la interacción y }$\mathbf{t,u}$\textbf{ son dos de las variables de Mandelstam. Calcula la sección eficaz diferencial. Debe salir el siguiente resultado:}
\[\mathbf{\frac{d\sigma}{d\Omega}=\frac{\alpha^2}{4s}\brackets{\frac{s-u}{t}+\frac{s-t}{u}}^2}\]
\end{enumerate}

Consideramos un 'scattering' $2\to2$, es decir, colisionan dos partículas y generan otras dos, tal que,
\[\varphi+\varphi\to\varphi+\varphi\]
O bien,
\begin{center}
\begin{fmffile}{diagrama}
\begin{fmfgraph*}(150,100)
  \fmfleft{i1,i2}
  \fmfright{o1,o2}
  \fmf{fermion,label=$\varphi$}{i1,v1}
  \fmf{fermion,label=$\varphi$}{i2,v1}
  \fmf{photon,label=$\gamma$}{v1,v2}
  \fmf{fermion,label=$\varphi$}{v2,o1}
  \fmf{fermion,label=$\varphi$}{v2,o2}
\end{fmfgraph*}
\end{fmffile}
\end{center}
donde la partícula de interacción es el fotón, pues las partículas tienen carga y por tanto, estamos en una interacción electromagnética.\\ \\
Tomamos el sistema de referencia centro de masas, 
    \begin{equation}
        p_1+p_2\to p_3+p_4
    \end{equation}
    con $\vec{p}_1=-\vec{p}_2$, $\vec{p}_3=-\vec{p}_4$ y $E_1+E_2=E_3+E_4=E_{CM}$, donde $E_{CM}$ es la energía total en el sistema centro de masas. Entonces,
    \begin{equation}
        d\Pi_{LIPS}=(2\pi)^4\delta^{(4)}(p_1+p_2-p_3-p_4)\frac{d^3p_3}{(2\pi)^3}\frac{1}{2E_3}\frac{d^3p_4}{(2\pi)^3}\frac{1}{2E_4}
    \end{equation}
    Podemos integrar en $\vec{p}_4$ usando la función delta, tal que
    \begin{equation}
        d\Pi_{LIPS}=\frac{1}{16\pi^2}d\Omega\int dp_f\frac{p_f^2}{E_3}\frac{1}{E_4}\delta(E_3+E_4-E_{CM})
    \end{equation}
    donde $p_f=|\vec{p}_3|=|\vec{p}_4|$ y $E_3=\sqrt{m_3^2+p_f^2}$ y $E_4=\sqrt{m_4^2+p_f^2}$. Hacemos ahora un cambio de variable de $p_f$ a $x(p_f)=E_3(p_f)+E_4(p_f)-E_{CM}$. El Jacobiano es
    \begin{equation}
        \frac{dx}{dp_f}=\frac{d}{dp_f}(E_3+E_4-E_{CM})=\frac{p_f}{E_3}+\frac{p_f}{E_4}=\frac{E_3+E_4}{E_3E_4}p_f
    \end{equation}
    y entonces, usando $E_3+E_4=E_{CM}$, tenemos
    \begin{equation}
        d\Pi_{LIPS}=\frac{1}{16\pi^2}d\Omega\int_{m_3+m_4-E_{CM}}^{\infty}dx\frac{p_f}{E_{CM}}\delta(x)=\frac{1}{16\pi^2}d\Omega\frac{p_f}{E_{CM}}\Theta(E_{CM}-m_3-m_4)
    \end{equation}
    donde $\Theta$ es la función escalón. Luego, la sección eficaz diferencial queda,
    \begin{equation}
        d\sigma=\frac{1}{2E_12E_2\left|\vec{v}_1-\vec{v}_2\right|}\frac{1}{16\pi^2}d\Omega\frac{p_f}{E_{CM}}|\mathscr{M}_{fi}|^2\Theta(E_{CM}-m_3-m_4)
    \end{equation}
    Usando que
    \begin{equation}
        \left|\vec{v}_1-\vec{v}_2\right|=\left|\frac{\left|\vec{p}_1\right|}{E_1}+\frac{\left|\vec{p}_2\right|}{E_2}\right|=\left|\vec{p}_i\right|\frac{E_{CM}}{E_1E_2}
    \end{equation}
    tenemos que
        \begin{equation}
            \left(\frac{d\sigma}{d\Omega}\right)_{CM}=\frac{1}{64\pi^2E_{CM}^2}\frac{\left|\vec{p}_f\right|}{\left|\vec{p}_i\right|}\left|\mathscr{M}_{fi}\right|^2\Theta(E_{CM}-m_3-m_4)
        \end{equation}
        La función escalón nos dice que el proceso es imposible si la energía del estado inicial, $E_{CM}=E_1+E_2$, no es suficiente para crear las demás partículas. Como todas las masas son iguales, tenemos que $|\vec{p}_f|=|\vec{p}_i|$, luego
    \begin{equation}
        \left(\frac{d\sigma}{d\Omega}\right)_{CM}=\frac{1}{64\pi^2E_{CM}^2}|\mathscr{M}_{fi}|^2
    \end{equation}
Luego, para nuestro caso, tendremos
pa
    \begin{equation}
        \left(\frac{d\sigma}{d\Omega}\right)_{CM}=\frac{1}{64\pi^2E_{CM}^2}\left|e^2\brackets{\frac{(p^{\mu}_1+p^{\mu}_3)(p_{2\mu}+p_{4\mu})}{t}+\frac{(p_1^{\mu}+p_4^{\mu})(p_{2\mu}+p_{3\mu})}{u}}\right|^2
    \end{equation}
Por otro lado, la solución depende de $\alpha$, que es la constante de estructura fina, sabemos que viene dada por,
\begin{equation}
    \alpha=\frac{e^2}{4\pi}\approx\frac{1}{137}
\end{equation}
donde la hemos puesto en unidades naturales. También, la solución depende de las variables de Mandelstam, que se definen como,
\begin{equation}
    s=(p_1+p_2)^2=(p_3+p_4)^2;\hspace{6mm}t=(p_1-p_3)^2=(p_2-p_4)^2;\hspace{6mm}u=(p_1-p_4)^2=(p_2-p_3)^2
\end{equation}
Luego, considerando que los cuadrimomentos vienen dados por,
\begin{equation}
    p_1=(E_1,0,0,|\vec{p}_i|);\hspace{4mm}p_2=(E_2,0,0,-|\vec{p}_i|);\hspace{4mm}p_3=(E_3,|\vec{p}_f|\sin\theta,0,|\vec{p}_f|\cos\theta);\hspace{4mm}p_4=(E_4,-|\vec{p}_f|\sin\theta,0,-|\vec{p}_f|\cos\theta)
\end{equation}
Tendremos que,
\begin{equation}
    s=(p_1+p_2)^2=\left(E_1+\cancel{|\vec{p}_i|}+E_2-\cancel{|\vec{p}_i|}\right)^2=E_{CM}^2
\end{equation}
Pues $E_1+E_2=E_{CM}$.\\ \\
Así, la primera parte de la expresión la podemos reescribir como,
\begin{equation}
    \frac{e^4}{64\pi^2E_{CM}^2}=\frac{\alpha^2}{4E_{CM}^2}=\frac{\alpha^2}{4s}
\end{equation}
Si desarrollamos la otra parte de la expresión tendremos productos de cuadrimomentos, que los podemos reescribir en función de las variables de Mandelstam, tal que
\begin{equation}
    s=p_1^2+p_2^2+2p_1p_2=2m_i^2+2p_1p_2\Rightarrow p_1p_2+m_i^2=\frac{s}{2}
\end{equation}
\begin{equation}
    s=p_3^2+p_4^2+2p_3p_4=2m_f^2+2p_3p_4\Rightarrow p_3p_4+m_f^2=\frac{s}{2}
\end{equation}
\begin{equation}
    t=p_1^2+p_3^2-2p_1p_3=m_i^2+m_f^2-2p_1p_2\Rightarrow p_1p_3=\frac{m_i^2+m_f^2-t}{2}
\end{equation}
\begin{equation}
    t=p_2^2+p_4^2-2p_2p_4=m_i^2+m_f^2-2p_2p_4\Rightarrow p_2p_4=\frac{m_i^2+m_f^2-t}{2}
\end{equation}
\begin{equation}
    u=p_1^2+p_4^2-2p_1p_4=m_i^2+m_f^2-2p_1p_4\Rightarrow p_1p_4=\frac{m_i^2+m_f^2-u}{2}
\end{equation}
\begin{equation}
    u=p_2^2+p_3^2-2p_2p_3=m_i^2+m_f^2-2p_2p_3\Rightarrow p_2p_3=\frac{m_i^2+m_f^2-u}{2}
\end{equation}
En resumen,
\begin{equation}
    \begin{matrix}
        m_i^2+(p_1p_2)=m_f^2+(p_3p_4)=\frac{s}{2}\\ \\
        (p_1p_3)=(p_2p_4)=\frac{m_i^2+m_f^2-t}{2}\\ \\
        (p_1p_4)=(p_2p_3)=\frac{m_i^2+m_f^2-u}{2}
    \end{matrix}
\end{equation}
Luego, el numerador del primer sumando queda,
\begin{equation}
    \begin{array}{rl}
        (p_1+p_3)(p_2+p_4) &=p_1p_2+p_1p_4+p_3p_2+p_3p_4= \\ \\
         & =\frac{s}{2}-m_i^2+\frac{m_i^2+m_f^2-u}{2}+\frac{m_i^2+m_f^2-u}{2}+\frac{s}{2}-m_f^2=\\ \\
         &=s-\cancel{m_i^2}+\cancel{m_i^2}-\cancel{m_f^2}+\cancel{m_f^2}-u=s-u
    \end{array}
\end{equation}
El numerador del segundo sumando queda,
\begin{equation}
    \begin{array}{rl}
        (p_1+p_4)(p_2+p_3) &=p_1p_2+p_1p_3+p_4p_2+p_4p_3= \\ \\
         & =\frac{s}{2}-m_i^2+\frac{m_i^2+m_f^2-t}{2}+\frac{m_i^2+m_f^2-t}{2}+\frac{s}{2}-m_f^2=\\ \\
         &=s-\cancel{m_i^2}+\cancel{m_i^2}-\cancel{m_f^2}+\cancel{m_f^2}-t=s-t
    \end{array}
\end{equation}
Así, juntando todo, obtenemos
\begin{equation}
    \left(\frac{d\sigma}{d\Omega}\right)_{CM}=\frac{\alpha^2}{4s}\brackets{\frac{s-u}{t}+\frac{s-t}{u}}^2
\end{equation}
obteniendo el resultado esperado.


\end{document}

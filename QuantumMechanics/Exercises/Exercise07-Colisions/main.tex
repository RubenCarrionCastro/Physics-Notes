\documentclass[11pt]{article}
%\usepackage[spanish]{babel}
\RequirePackage{etex}
\usepackage[utf8]{inputenc}
\usepackage{braket}
%\usepackage[sc]{mathpazo}
% \linespread{1.5}
%\usepackage[T1]{fontenc}
%\usepackage{heuristica}
%\usepackage[erewhon,vvarbb,bigdelims]{newtxmath}
%\renewcommand*\oldstylenums[1]{\textosf{#1}}
\usepackage{enumitem}
\usepackage{array}
\usepackage{textcomp}
\usepackage{fancyhdr}
\usepackage{amsmath, amsthm}
\usepackage{slashed}
\usepackage[normalem]{ulem}
\usepackage{amsfonts}
\usepackage{amssymb}
\usepackage{mathtools}
\usepackage{float}
\usepackage{soul}
\usepackage{graphicx}
\usepackage{hyperref}
\usepackage{graphicx}
\usepackage{pstricks-add}
\usepackage{color}
\usepackage{caption}
\usepackage[margin=0.9in]{geometry}
\usepackage{marvosym}
\usepackage{mathtools}
\usepackage{framed}
\usepackage{calrsfs}
\usepackage[mathscr]{euscript}
\usepackage{tensor}
\usepackage{autonum}
\usepackage{cancel}
\usepackage[most]{tcolorbox}

\newtheorem{thm}{Teorema}[section]
\newtheorem{theorem}{Teorema}[section]
\newtheorem{proposition}[thm]{Proposición} 
\newtheorem{lemma}[thm]{Lema}
\newtheorem{corollary}[thm]{Corolario} 
\newtheorem{conv}[thm]{Convención}
\newtheorem{defi}[thm]{Definición}
\newtheorem{definition}[theorem]{Definición}
\newtheorem{notation}[thm]{Notación} 
\newtheorem{exe}[thm]{Ejemplo}
\newtheorem{conjecture}[thm]{Conjetura} 
\newtheorem{prob}[thm]{Problema}
\newtheorem{remark}[thm]{Observación}
\newtheorem{example}[thm]{Ejemplo}
\newtheorem{note}[thm]{Nota}

\newcommand{\brackets}[1]{\left[#1\right]}
\newcommand{\curlybraces}[1]{\left\{#1\right\}}
\newcommand{\qedh}{\hfill\hspace{5mm}\fbox{\phantom{\rule{.5ex}{.5ex}}}}
\newcommand{\scalar}[2]{\langle #1, #2 \rangle}
\newcommand{\ptensor}[2]{#1 \otimes #2}
\newcommand{\pcart}[2]{#1 \times #2}
\newcommand{\voverrightarrowtor}[3]{\begin{pmatrix}#1\\ #2\\ #3\end{pmatrix}}
\newcommand{\cooverrightarrowtor}[3]{\begin{pmatrix}#1 & #2 & #3\end{pmatrix}}
\newcommand{\abss}[1]{\begin{vmatrix}#1\end{vmatrix}^2}

\newtcolorbox[auto counter, number within=section]{mytheorem}[2][]{
  enhanced,
  breakable,
  title=Teorema~\thetcbcounter: #2,
  #1,
}
\newtcolorbox[auto counter, number within=section]{propositionbox}[2][]{
  enhanced,
  breakable,
  title=Proposition~\thetcbcounter: #2,
  #1,
}

\newtcolorbox[auto counter, number within=section]{corollarybox}[2][]{
  enhanced,
  breakable,
  title=Corollary~\thetcbcounter: #2,
  #1,
}

\newtcolorbox[auto counter, number within=section]{remarkbox}[2][]{
  enhanced,
  breakable,
  title=Remark~\thetcbcounter: #2,
  #1,
}

\newtcolorbox[auto counter, number within=section]{notebox}[2][]{
  enhanced,
  breakable,
  title=Note~\thetcbcounter: #2,
  #1,
}


\newenvironment{Figura}
  {\par\medskip\noindent\minipage{\linewidth}}
  {\endminipage\par\medskip}
%\usepackage[spanish]{babel}
\title{\huge{\textbf{Evaluación VII. Mecánica Cuántica}}}
\author{\textbf{}\\ \\Rubén Carrión Castro\\}
% \textit{Los Chavales}
\date{Diciembre 2024}
\begin{document}
\maketitle
\begin{enumerate}
\item \textbf{La amplitud de dispersión de una partícula con momento $p=200$ MeV/c es}
\[f(p,\theta,\varphi)=\frac{\hbar}{2p}\brackets{\left(1+3\frac{\sqrt{3}}{2}\cos\theta\right)+i\left(1+\frac{9}{2}\cos\theta\right)}\]
\begin{enumerate}
    \item \textbf{Calcula la sección eficaz total en mb (milibarns).}\\
    \textbf{[$\hbar c\approx200$ MeV fm, $1$ fm $=10^{-13}$ cm, $1$ mb$=10^{-27}$ cm$^2$]}
    \item \textbf{¿Se cumple el teorema óptico?}
    \item \textbf{Halla todos los desfajes}
\end{enumerate}
\end{enumerate}

\subsubsection*{Apartado (a)}
Sabemos que la sección eficaz total viene de integrar la sección eficaz diferencial, que es
\[\frac{d\sigma}{d\Omega}=\abss{f(p,\theta,\varphi)}=\abss{f(p,\theta,\varphi)=\frac{\hbar}{2p}\brackets{\left(1+3\frac{\sqrt{3}}{2}\cos\theta\right)+i\left(1+\frac{9}{2}\cos\theta\right)}}=\]
\[=\frac{\hbar^2}{4p^2}\brackets{\left(1+3\frac{\sqrt{3}}{2}\cos\theta\right)^2+\left(1+\frac{9}{2}\cos\theta\right)^2}=\frac{\hbar^2}{4p^2}\brackets{1+\frac{27}{4}\cos^2\theta+3\sqrt{3}\cos\theta+1+\frac{81}{4}\cos^2\theta+9\cos\theta}\]
\[=\frac{\hbar^2}{4p^2}\brackets{2+(9+3\sqrt{3})\cos\theta+27\cos^2\theta}\]

Por tanto, la sección eficaz total viene dada por,
\[\sigma=\int \frac{d\sigma}{d\Omega}d\Omega=\int_{-1}^{1}d(\cos\theta)\int_0^{2\pi}d\varphi\abss{f(p,\theta,\varphi)}=\frac{\cancel{2}\pi\hbar^2}{\cancel{4}p^2}\int_{-1}^{1}\brackets{2+(9+3\sqrt{3})\cos\theta+27\cos^2\theta}d(\cos\theta)=\]
\[=\frac{\hbar^2\pi}{2p^2}\brackets{2\int_{-1}^{1}d(\cos\theta)+(9+3\sqrt{3})\int_{-1}^{1}\cos\theta d(\cos\theta)+27\int_{-1}^1\cos^2\theta d(\cos\theta)}=\]
Vamos a hacer el cambio de variable $x=\cos\theta$.
\[=\frac{\hbar^2\pi}{2p^2}\brackets{2\int_{-1}^{1}dx+(9+3\sqrt{3})\int_{-1}^{1}x\cdot dx+27\int_{-1}^1x^2 dx}=\]
\[=\frac{\hbar^2\pi}{2p^2}\brackets{2\left. x\right|_{-1}^{1}+\frac{9+3\sqrt{3}}{2}\left. x^2\right|_{-1}^{1}+\frac{27}{3}\left.x^3\right|_{-1}^{1}}=\]
\[=\frac{\hbar^2\pi}{2p^2}\brackets{4+18}=\frac{22\hbar^2\pi}{2p^2}=\frac{11\pi(\hbar c)^2}{(pc)^2}\overset{\curlybraces{\begin{matrix}
    \hbar c\approx 200\text{ MeV fm}\\
    p=200 \text{ MeV/c }\\
    \Rightarrow pc=200\text{ MeV}
\end{matrix}}}{=}11\pi \text{ fm}^2\]
Sabiendo que $1$ fm$=10^{-13}$ cm y que $1$ mb$=10^{-27}$ cm$^2$, tenemos que 1 fm$^2=10^{-2}$ b$=10$ mb. Por tanto, la sección eficaz total queda,
\[\sigma=11\pi\text{ fm}^2=11\pi\cdot10\text{ mb}=345,57\text{ mb}\]
\subsubsection*{Apartado (b)}
EL Teorema Óptico establece que,
\[Im[f(p,\theta=0,\varphi)]=\frac{p}{4\pi\hbar}\sigma\]
Por tanto, comprobamos si esto se cumple:
\[Im[f(p,\theta=0\varphi)]=\frac{\hbar}{2p}\left(1+\frac{9}{2}\cos(0)\right)=\frac{11\hbar}{4p}\]
Por otra parte,
\[\frac{p}{4\pi\hbar}\sigma=\frac{\cancel{p}}{4\cancel{\pi}\cancel{\hbar}}\frac{11\cancel{\pi}\hbar^{\cancel{2}}}{p^{\cancel{2}}}=\frac{11\hbar}{4p}\]
Por tanto, vemos que ambas cantidades son iguales, por lo que se cumple el Teorema Óptico.
\subsubsection*{Apartado (c)}
Sabemos que la amplitud de dispersión la podemos reescribir como,
\[f(p,\theta,\varphi)=\frac{\hbar}{2ip}\sum_{l=0}^{\infty}(2l+1)\left(e^{2i\delta_l}-1\right)\mathscr{P}_l(\cos\theta)\]
donde $P_l(\cos\theta)$ son los polinomios de Legendre, tal que
\[\begin{matrix}
    \mathscr{P}_0(\cos\theta)=1\\
    \mathscr{P}_1(\cos\theta)=\cos\theta\\
    \mathscr{P}_2(\cos\theta)=\frac{1}{2}\left(3\cos^2\theta-1\right)\\
    \vdots
\end{matrix}\]
Por comparación con nuestra amplitud de dispersión que es,
\[f(p,\theta,\varphi)=\frac{\hbar}{2p}\brackets{\left(1+3\frac{\sqrt{3}}{2}\cos\theta\right)+i\left(1+\frac{9}{2}\cos\theta\right)}\]
vemos que los términos distintos de cero son $l=0,1$.
Luego, tenemos
\[f(p,\theta,\varphi)=-\frac{i\hbar}{2p}\brackets{\left(\cos(2\delta_0)+i\sin(2\delta_0)-1\right)+3\left(\cos(2\delta_1)+i\sin(2\delta_1)-1\right)\cos\theta}=\]
\[=\frac{\hbar}{2p}\brackets{\left(\sin(2\delta_0)+3\sin(2\delta_1)\cos\theta\right)+i\left(1-\cos(2\delta_0)+(3-3\cos(2\delta_1))\cos\theta\right)}\]
Por tanto, podemos igualar las partes reales e imaginarias (obviando la constante $\hbar/(2p)$, que es la misma),
\[1+\frac{3\sqrt{3}}{2}\cos\theta=\sin(2\delta_0)+3\sin(2\delta_1)\cos\theta\]
y
\[1+\frac{9}{2}\cos\theta=1-\cos(2\delta_0)+(3-3\cos(2\delta_1))\cos\theta\]
Vamos a comparar la igualdad de la parte real,
\[\begin{array}{rl}
    0: & 1=\sin(2\delta_0)\Rightarrow2\delta_0=\arcsin(1)=\frac{(2n+1)\pi}{2};n\in\mathbb{Z}\Rightarrow\delta_0=\frac{(2n+1)\pi}{4};n\in\mathbb{Z} \\ \\
    \cos\theta: & 3\sqrt{3}/2=3\sin(2\delta_1)\Rightarrow2\delta_1=\arcsin(\sqrt{3}/2)=2n\pi\pm\frac{2\pi}{3};n\in\mathbb{Z}\Rightarrow\delta_1=n\pi\pm\frac{\pi}{3};n\in\mathbb{Z}
\end{array}\]
Podemos comprobar los valores en la parte imaginaria, tal que
\[1-\cancelto{1}{\cos\left(\frac{(2n+1)\pi}{2}\right)}+(3-3\cancelto{-1/2}{\cos\left(2n\pi\pm\frac{2\pi}{3}\right)}\cos\theta=1+\frac{9}{2}\cos\theta\]
Por tanto, los posibles desfajes son,
\[\delta_0=\frac{(2n+1)\pi}{4};\hspace{6mm}\delta_1=n\pi\pm\frac{\pi}{3}\]
con $n\in\mathbb{Z}$.
\\ \\
Vamos a quedarnos con $n=0$ y la parte positiva (pues queremos el ángulo antihorario), luego, los desfajes son,
\[\delta_0=\pi/4;\hspace{6mm}\delta_1=\pi/3\]



    
\end{document}

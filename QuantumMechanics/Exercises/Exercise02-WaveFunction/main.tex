\documentclass[11pt]{article}
%\usepackage[spanish]{babel}
\RequirePackage{etex}
\usepackage[utf8]{inputenc}
\usepackage{braket}
%\usepackage[sc]{mathpazo}
% \linespread{1.5}
%\usepackage[T1]{fontenc}
%\usepackage{heuristica}
%\usepackage[erewhon,vvarbb,bigdelims]{newtxmath}
%\renewcommand*\oldstylenums[1]{\textosf{#1}}
\usepackage{enumitem}
\usepackage{array}
\usepackage{textcomp}
\usepackage{fancyhdr}
\usepackage{amsmath, amsthm}
\usepackage{slashed}
\usepackage[normalem]{ulem}
\usepackage{amsfonts}
\usepackage{amssymb}
\usepackage{mathtools}
\usepackage{float}
\usepackage{soul}
\usepackage{graphicx}
\usepackage{hyperref}
\usepackage{graphicx}
\usepackage{pstricks-add}
\usepackage{color}
\usepackage{caption}
\usepackage[margin=0.9in]{geometry}
\usepackage{marvosym}
\usepackage{mathtools}
\usepackage{framed}
\usepackage{calrsfs}
\usepackage[mathscr]{euscript}
\usepackage{tensor}
\usepackage{autonum}
\usepackage{cancel}
\usepackage[most]{tcolorbox}

\newtheorem{thm}{Teorema}[section]
\newtheorem{theorem}{Teorema}[section]
\newtheorem{proposition}[thm]{Proposición} 
\newtheorem{lemma}[thm]{Lema}
\newtheorem{corollary}[thm]{Corolario} 
\newtheorem{conv}[thm]{Convención}
\newtheorem{defi}[thm]{Definición}
\newtheorem{definition}[theorem]{Definición}
\newtheorem{notation}[thm]{Notación} 
\newtheorem{exe}[thm]{Ejemplo}
\newtheorem{conjecture}[thm]{Conjetura} 
\newtheorem{prob}[thm]{Problema}
\newtheorem{remark}[thm]{Observación}
\newtheorem{example}[thm]{Ejemplo}
\newtheorem{note}[thm]{Nota}

\newcommand{\brackets}[1]{\left[#1\right]}
\newcommand{\curlybraces}[1]{\left\{#1\right\}}
\newcommand{\qedh}{\hfill\hspace{5mm}\fbox{\phantom{\rule{.5ex}{.5ex}}}}
\newcommand{\scalar}[2]{\langle #1, #2 \rangle}
\newcommand{\ptensor}[2]{#1 \otimes #2}
\newcommand{\pcart}[2]{#1 \times #2}
\newcommand{\vvector}[3]{\begin{pmatrix}#1\\ #2\\ #3\end{pmatrix}}
\newcommand{\covector}[3]{\begin{pmatrix}#1 & #2 & #3\end{pmatrix}}
\newcommand{\abss}[1]{\begin{vmatrix}#1\end{vmatrix}^2}

\newtcolorbox[auto counter, number within=section]{mytheorem}[2][]{
  enhanced,
  breakable,
  title=Teorema~\thetcbcounter: #2,
  #1,
}
\newtcolorbox[auto counter, number within=section]{propositionbox}[2][]{
  enhanced,
  breakable,
  title=Proposition~\thetcbcounter: #2,
  #1,
}

\newtcolorbox[auto counter, number within=section]{corollarybox}[2][]{
  enhanced,
  breakable,
  title=Corollary~\thetcbcounter: #2,
  #1,
}

\newtcolorbox[auto counter, number within=section]{remarkbox}[2][]{
  enhanced,
  breakable,
  title=Remark~\thetcbcounter: #2,
  #1,
}

\newtcolorbox[auto counter, number within=section]{notebox}[2][]{
  enhanced,
  breakable,
  title=Note~\thetcbcounter: #2,
  #1,
}


\newenvironment{Figura}
  {\par\medskip\noindent\minipage{\linewidth}}
  {\endminipage\par\medskip}
%\usepackage[spanish]{babel}
\title{\huge{\textbf{Evaluación II. Mecánica Cuántica}}}
\author{\textbf{}\\ \\Rubén Carrión Castro\\}
% \textit{Los Chavales}
\date{Octubre 2024}
\begin{document}
\maketitle

\begin{enumerate}
    \item \textbf{Considera una partícula confinada en un pozo de potencial 1-dim de anchura $a$ y altura infinita.}
    \begin{enumerate}
        \item \textbf{Encuentra los valores posibles de la energía de la partícula y sus autoestados correspondiente.}
        \item \textbf{Halla el valor esperado de la posición $X$ y su incertidumbre $\Delta X$ en cualquier estado de energía bien definida.}
        \item \textbf{Encuentra los valores máximo y mínimo de la incertidumbre en la posición de la partícula en cualquier estado de energía bien definida.}
        \item \textbf{Halla el valor esperado del momento $P$ y su incertidumbre $\Delta P$ en cualquier estado de energía bien definida.}
        \item \textbf{Encuentra el producto de incertidumbre $\Delta X\Delta P$ para el estado fundamental y para todos los estados excitados. ¿Se cumple el principio de incertidumbre de Heisenberg?}
    \end{enumerate}
    
\end{enumerate}


Tenemos el problema de una partícula encerrada en un pozo de potencial infinito de anchura $a$. Ver Figure 1.

\begin{Figura}
        \centering
        \includegraphics[width=0.5\linewidth]{image.png}
        \captionof{figure}{Pozo de potencial infinito de anchura $a$.}
        \label{figura 1}
\end{Figura}
donde se considera que el valor mínimo del potencial es $V=0$, pues no nos dicen nada.\\ \\
Sabemos que el Hamiltoniano de una partícula libre sometida a un potencial 1-dim es,
\begin{equation}
    H=\frac{P^2}{2m}+V(x)
\end{equation}

El potencial lo podemos ver como una función a trozos, tal que

\begin{equation}
    V(x)=\left\lbrace\begin{array}{rll}
        0 & \text{si} & 0< x< a\\
        \infty & \text{si no}
    \end{array}\right.
\end{equation}

\subsubsection*{Apartado (a)}

Una vez visto el potencial, tenemos que el Hamiltoniano diverge fuera del pozo, por tanto vamos a trabajar dentro de este, así tenemos que

\begin{equation}
    H=\frac{P^2}{2m}\hspace{7mm}\text{si }0< x< a
\end{equation}

Sabemos que la energía son los autovalores del Hamiltoniano, por tanto, deberemos de resolver la ecuación de autovalores siguiente,
\begin{equation}
    H\ket{\Psi_n}=E_n\ket{\Psi_n}
\end{equation}
donde los $\ket{\Psi_n}$ son los autoestados de las energías. \\ \\
Debemos pasar a representación de coordenadas para poder trabajar más fácilmente, para ello hacemos los cambios siguientes,
\begin{equation}
    \left\lbrace\begin{matrix}
        \ket{\Psi_n} & \to & \Psi_n(x)\\
        P & \to & i\hbar\frac{d}{dx}\\
        P^2 & \to & -\hbar^2\frac{d^2}{dx^2}
    \end{matrix}\right.
\end{equation}
Por tanto, tenemos la ecuación de autovalores como,
\begin{equation}
    -\frac{\hbar^2}{2m}\frac{d^2\Psi_n(x)}{dx^2}=E_n\Psi_n(x)
\end{equation}
Tenemos una ecuación diferencial cuya solución es
\begin{equation}
    \Psi_n(x)=Ae^{\pm ikx}=A_1e^{ikx}+A_2e^{-ikx}
\end{equation}
donde $k^2=\frac{2mE}{\hbar^2}$.\\ \\
Imponemos las condiciones de contorno, que son
\begin{equation}
    \left\lbrace\begin{matrix}
        (i) & \Psi(x=0)=0\\
        (ii) & \Psi(x=a)=0
    \end{matrix}\right.
\end{equation}
Por tanto,
\begin{equation}
    \Psi(x=0)=A_1+A_2=0\Longrightarrow A_1=-A_2
\end{equation}
y también
\begin{equation}
    \Psi(x=a)=A_1e^{ika}+A_2e^{-ika}=A_1e^{ika}-A_1e^{-ika}=2iA_1\sin(ka)=0\Longrightarrow \sin(ka)=0\Longrightarrow ka=n\pi,\hspace{2mm}n\in\mathbb{Z}
\end{equation}
Por tanto, $k=\frac{n\pi}{a}$, pero sabemos que $k=\sqrt{2mE_n}/\hbar$, por tanto los posibles valores de la energía serán,
\begin{equation}
    E_n=\frac{(n\pi\hbar)^2}{2ma^2},\hspace{7mm}n\in\mathbb{N}
\end{equation}
donde ahora $n$ pertenece a los naturales (sin el cero), puesto que si $n$ fuera entero, tendríamos valores dobles de la energía y si fuera cero, tendríamos energía igual a cero, cosa que carece de sentido físico, ya que, de algún modo, violaría el Principio de Incertidumbre, pues tendríamos la posición y la velocidad de la partícula bien definidas.\\
Tenemos que 
\begin{equation}
    \Psi(x)=2iA\sin(kx)\equiv  B\sin(kx)
\end{equation}
Como debe estar normalizada,
\begin{equation}
\begin{array}{ll}
    1=\int_{-\infty}^{+\infty}|\Psi(x)|^2=B^2\int_{0}^{a}dx\sin^2(kx)=B^2\curlybraces{\int_0^adx\frac{1}{2}-\frac{1}{2}\int_0^adx\cos(2kx)}=\\ \\
   = B^2\curlybraces{\frac{a}{2}-\frac{1}{2}\brackets{\cancelto{0}{\sin(2ka)}
    -\cancelto{0}{\sin(0)}}}=B^2\frac{a}{2}=1
\end{array}
\end{equation}
Por tanto, $A=\sqrt{\frac{2}{a}}$. Luego, los autoestados de la energía son,
\begin{equation}
    \Psi(x)=\sqrt{\frac{2}{a}}\sin(kx)
\end{equation}

\subsubsection*{Apartado (b)}
Como estamos en representación de coordenadas, el valor esperado de $X$ vendrá dado por
\begin{equation}
    \braket{X}=\int_{-\infty}^{+\infty} dx|\Psi(x)|^2x=\int_{-\infty}^{+\infty} dx \Psi^*(x)x\Psi(x)
\end{equation}
Por tanto, el valor esperado de la posición será

\begin{equation}
    \begin{array}{rl}
         \braket{X}&=\int_{-\infty}^{+\infty} dx \Psi^*(x)x\Psi(x)=\frac{2}{a}\int_0^adx\sin^2(kx)x=\frac{1}{a}\brackets{\int_0^ax dx-\int_0^ax\cos(2kx)dx}=\\ \\
         &=\frac{1}{a}\brackets{\frac{a^2}{2}-\int_0^ax\cos(2kx)dx}=\curlybraces{\begin{matrix}
             u=x & \to & du=dx\\
             dv=\cos(2kx)dx & \to & v=\frac{\sin(2kx)}{2k}
         \end{matrix}}=\\ \\
         &=\frac{1}{a}\brackets{\frac{a^2}{2}-\left(\left[x\frac{\sin(2kx)}{2k}\right|_{0}^{a}-\int_0^a\frac{\sin(2kx)}{2k}dx\right)}=\frac{1}{a}\brackets{\frac{a^2}{2}-\left(a\frac{\sin(2ka)}{2k} +\frac{\cos(2ka)}{4k^2}-\frac{1}{4k^2}\right)}=\\ \\
         &=\frac{a}{2}-\frac{\sin(2ka)}{2k}+\frac{\cos(2ka)}{4k^2a}-\frac{1}{4k^2a}
    \end{array}
\end{equation}
Usando la condición de contorno, $ka=n\pi$, tenemos que

\begin{equation}
    \braket{X}=\frac{a}{2}-\cancelto{0}{\frac{\sin(2n\pi}{2k}}+\cancelto{0}{\frac{\cos(2n\pi)}{4k^2a}-\frac{1}{4k^2a}}=\frac{a}{2}
\end{equation}
Una vez obtenido el valor esperado de $X$, calculamos su incertidumbre usando que
\begin{equation}
    \Delta X=\sqrt{\braket{X^2}-\braket{X}^2}
\end{equation}
Para calcularla, debemos calcular el valor esperado de $X^2$, tal que

\begin{equation}
    \begin{array}{rl}
         \braket{X^2}&=\int_{-\infty}^{+\infty} dx \Psi^*(x)x^2\Psi(x)=\frac{2}{a}\int_0^adx\sin^2(kx)x^2=\frac{1}{a}\brackets{\int_0^ax^2 dx-\int_0^ax^2\cos(2kx)dx}=\\ \\
         &=\frac{1}{a}\brackets{\frac{a^3}{3}-\int_0^ax^2\cos(2kx)dx}=\curlybraces{\begin{matrix}
             u=x^2 & \to & du=2xdx\\
             dv=\cos(2kx)dx & \to & v=\frac{\sin(2kx)}{2k}
         \end{matrix}}=\\ \\
         &=\frac{1}{a}\brackets{\frac{a^3}{3}-\left(\left[x^2\frac{\sin(2kx)}{2k}\right|_{0}^{a}-\int_0^a\frac{\sin(2kx)}{2k}2xdx\right)}=\frac{1}{a}\brackets{\frac{a^3}{3}-a^2\frac{\sin(2ka)}{2k}+\int_0^a\frac{\sin(2kx)}{2k}2xdx}=\\ \\
         &=\curlybraces{\begin{matrix}
             u=2x & \to & du=2dx\\
             dv=\sin(2kx)dx & \to & v=-\frac{\cos(2kx)}{2k}
         \end{matrix}}=\frac{1}{a}\brackets{\frac{a^3}{3}--a^2\frac{\sin(2ka)}{2k}+\frac{1}{2k}\left(\left.-2x\frac{\cos(2kx)}{2k}\right|_0^a+\int_0^a\frac{\cos(2kx)}{2k}2dx\right)}=\\ \\
         &=\frac{1}{a}\brackets{\frac{a^3}{3}-a^2\frac{\sin(2ka)}{2k}-2a\frac{\cos(2ka)}{4k^2}+\frac{\sin(2ka)}{4k^3}-0}=\frac{a^2}{3}-a\frac{\sin(2ka)}{2k}-\frac{\cos(2ka)}{2k^2}+\frac{\sin(2ka)}{4k^3a}   \\ \\
    \end{array}
\end{equation}
Usando la condición de contorno, $ka=n\pi$, tenemos que
\begin{equation}
    \braket{X^2}=\frac{a^2}{3}-\frac{1}{2k^2}\frac{a^2}{a^2}=\frac{a^2}{3}-\frac{a^2}{2n^2\pi^2}=a^2\left(\frac{1}{3}-\frac{1}{2n^2\pi^2}\right)
\end{equation}
Por tanto, la incertidumbre queda
\begin{equation}
    \Delta X=\sqrt{a^2\left(\frac{1}{3}-\frac{1}{2n^2\pi^2}\right)-\frac{a^2}{4}}=a\sqrt{\left(\frac{1}{12}-\frac{1}{2n^2\pi^2}\right)}
\end{equation}

\subsubsection*{Apartado (c)}

Sabemos que 
\begin{equation}
     \Delta X=a\sqrt{\left(\frac{1}{12}-\frac{1}{2n^2\pi^2}\right)}
\end{equation}
con $n\in\mathbb{N}$.\\ \\
Por tanto, el valor mínimo de la incertidumbre será para n=1, pues el segundo término está restando, quedando como resultado,
\begin{equation}
    \Delta X_{min}=a\sqrt{\frac{1}{12}-\frac{1}{2\pi^2}}\approx 0.18 a
\end{equation}
El valor máximo será para $n\to\infty$, tal que
\begin{equation}
    \Delta X_{max}=\lim_{n\to\infty}a\sqrt{\frac{1}{12}-\frac{1}{2n^2\pi^2}}=a\sqrt{\frac{1}{12}}\approx 0.29 a
\end{equation}

\newpage
\subsubsection*{Apartado (d)}

El valor esperado del momento viene dado por

\begin{equation}
    \braket{P}=\int_{-\infty}^{+\infty}p |\overline{\Psi}(p)|^2dp=\int_{-\infty}^{+\infty}\overline{\Psi}^*(p)p\overline{\Psi}(p)dp
\end{equation}
donde $\overline{\Psi}(p)$ es la función de ondas en la representación de momentos. Podemos pasar a representación de coordenadas, tal que
\begin{equation}
     \braket{P}=\int_{-\infty}^{+\infty}\overline{\Psi}^*(p)p\overline{\Psi}(p)dp=\int_{-\infty}^{+\infty}\Psi^*(x)\left(-i\hbar\frac{d}{dx}\right)\Psi(x)dx
\end{equation}
Usando que $\Psi(x)=\sqrt{\frac{2}{a}}\sin(kx)$, calculamos el valor esperado del momento,
\begin{equation}
    \begin{array}{rl}
          \braket{P}&=\int_{-\infty}^{+\infty}\Psi^*(x)\left(-i\hbar\frac{d}{dx}\right)\Psi(x)dx =-i\hbar\frac{2}{a}\int_0^a\sin(kx)\frac{d}{dx}(\sin(kx))dx=\\ \\
          &=-i\hbar\frac{2k}{a}\int_0^a\sin(kx)\cos(kx)dx=-i\hbar\frac{2k}{a}\frac{\sin^2(ka)}{k}=0
    \end{array}
\end{equation}
donde hemos usado que $ka=n\pi$. Así tenemos que $\braket{P}=0$.\\ \\
Ahora calculamos la incertidumbre de $P$, que viene dada por
\begin{equation}
    \Delta P=\sqrt{\braket{P^2}-\braket{P}^2}=\sqrt{\braket{P^2}}
\end{equation}
Por tanto, calculamos el valor esperado de $P^2$,

\begin{equation}
    \begin{array}{rl}
          \braket{P}&=\int_{-\infty}^{+\infty}\Psi^*(x)\left(\hbar^2\frac{d^2}{dx^2}\right)\Psi(x)dx =\hbar^2\frac{2}{a}\int_0^a\sin(kx)\frac{d^2}{dx^2}(\sin(kx))dx=\\ \\
          &=\hbar^2\frac{2k}{a}\int_0^a\sin(kx)\frac{d}{dx}\cos(kx)dx=-\hbar^2\frac{2k^2}{a}\int_0^a\sin^2(kx)dx=\\ \\
          &=-\frac{2k^2\hbar^2}{a}\int_0^a\brackets{\frac{1}{2}-\frac{\cos(2kx)}{2}}dx=-\frac{2k^2\hbar^2}{a}\brackets{\frac{a}{2}-\frac{\sin(2ka)}{2k}}=k^2\hbar^2=2mE_n
    \end{array}
\end{equation}
donde hemos usado que $ka=n\pi$ y $k=\sqrt{2mE_n}/\hbar$.
Por tanto, la incertidumbre del momento es
\begin{equation}
    \Delta P=\sqrt{2mE_n}=k\hbar
\end{equation}
\newpage
\subsubsection*{Apartado (e)}

Ahora nos piden hacer el producto de ambas incertidumbres para el estado fundamental, es decir, para $n=1$, que sabemos que $E_1=\frac{\pi^2\hbar^2}{2ma^2}$. Así tenemos,
\begin{equation}
    \Delta X\Delta P=\cancel{a}\sqrt{\frac{1}{12}-\frac{1}{2\pi^2}}\sqrt{\cancel{2m}\frac{\pi^2\hbar^2}{\cancel{2m}\cancel{a^2}}}=\sqrt{\frac{\pi^2\hbar^2}{12}-\frac{\cancel{\pi^2}\hbar^2}{2\cancel{\pi^2}}}=\hbar\sqrt{\frac{\pi^2}{12}-\frac{1}{2}}\approx 0.56\hbar
\end{equation}
que como es del orden de $\hbar$ se cumple el Principio de Incertidumbre.\\ \\
Veamos el producto en general,
\begin{equation}
    \Delta X\Delta P=\cancel{a}\sqrt{\frac{1}{12}-\frac{1}{2n^2\pi^2}}\sqrt{\frac{n^2\pi^2\hbar^2}{\cancel{a^2}}}=\sqrt{\frac{n^2\pi^2\hbar^2}{12}-\frac{\cancel{n^2\pi^2}\hbar^2}{2\cancel{n^2\pi^2}}}=\hbar\sqrt{\frac{n^2\pi^2}{12}-\frac{1}{2}}
\end{equation}
Es claro ver, que para cualquier $n\in\mathbb{N}$ (sin el cero) se va a cumplir el Principio de Incertidumbre de Heisenberg. Además, podemos ver que si $n\to\infty$, el producto se va al infinito. Esto tiene sentido, ya que, al considerar un estado tan altamente excitado, o bien la posición se encuentra en el infinito, o bien el momento se vuelve infinito, o incluso ambos.







\end{document}

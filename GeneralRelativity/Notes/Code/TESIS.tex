\documentclass[12pt, oneside]{Thesis}
\usepackage[spanish,mexico]{babel}
\usepackage[sectionbib, square, comma, numbers, sort&compress]{natbib} 
\hypersetup{urlcolor=black, colorlinks=true} 
\usepackage{amsmath} 
\usepackage{mathrsfs}
\usepackage{indentfirst}
\usepackage{enumitem}
\usepackage{url}
\usepackage{braket} 
\usepackage{mathtools}
\usepackage{bibunits} 
\usepackage{float}
\usepackage{blindtext}
% \usepackage{blindtext}
\usepackage{multicol}
\usepackage{hyperref}
\usepackage[colorinlistoftodos, prependcaption, textsize=tiny]{todonotes}
%\setlength{\parindent}{0.5cm}
\newcommand{\tr}{\mathrm{tr}}
\newcommand{\diag}{\mathrm{diag}}
\renewcommand{\qedsymbol}{\rule{0.5em}{0.5em}}
\usepackage[spanish]{cleveref}
\usepackage[spanish]{babelbib}
%\selectbiblanguage{spanish}
\usepackage{siunitx}
\usepackage{booktabs}
\usepackage{bbm}
\usepackage[many]{tcolorbox}
\usepackage[toc]{appendix}
\usepackage{framed}
\usepackage{calrsfs}
\usepackage[mathscr]{euscript}
\usepackage{tensor}
\usepackage{autonum}
\usepackage{cancel}
\newtheorem{observacion}[teorema]{Observaci\'on}
\newtheorem{hipotesis}[teorema]{Hip\'otesis}
\usepackage{xpatch}
\usepackage{xcolor}
\newcommand{\bl}{\color{blue}}
% \newcommand{ }{\color{red}}
\newcommand{\cc}{\color{cyan}}
\newcommand{\cm}{\color{magenta}}
\xpatchcmd{\proof}{\itshape}{\normalfont\proofnamefont}{}{}
% \renewcommand{\refname}{Bibliografía de Figuras}
% \renewcommand{\bibname}{Nuevo Título de Bibliografía}
% \usepackage[usenames,dvipsnames]{color}
%Este pequeño bloque permite que el número de la ecuación permanezca del tamaño del resto del texto
%si se decide hacer small el tamaño de la ecuación.
%%%%%%%%%%%%%%%%%
\makeatletter
\def\maketag@@@#1{\hbox{\m@th\normalfont\normalsize#1}}
\makeatother
%%%%%%%%%%%%%%%%
\newtheorem{propiedad}{Propiedades}[section]
\newtheorem{axioma}{Axioma}[section]
\newtheorem{thm}{Teorema}[section]
\newtheorem{theorem}{Teorema}[section]
\newtheorem{proposition}[thm]{Proposición} 
\newtheorem{lemma}[thm]{Lema}
\newtheorem{corollary}[thm]{Corolario} 
\newtheorem{conv}[thm]{Convención}
\newtheorem{defi}[thm]{Definición}
\newtheorem{definition}[theorem]{Definición}
\newtheorem{notation}[thm]{Notación} 
\newtheorem{exe}[thm]{Ejemplo}
\newtheorem{conjecture}[thm]{Conjetura} 
\newtheorem{prob}[thm]{Problema}
\newtheorem{remark}[thm]{Observación}
\newtheorem{example}[thm]{Ejemplo}
\newtheorem{note}[thm]{Nota}

\newcommand{\brackets}[1]{\left[#1\right]}
\newcommand{\curlybraces}[1]{\left\{#1\right\}}
\newcommand{\qedh}{\hfill\hspace{5mm}\qedsymbol}
\newcommand{\scalar}[2]{\langle #1, #2 \rangle}
\newcommand{\ptensor}[2]{#1 \otimes #2}
\newcommand{\pcart}[2]{#1 \times #2}
\newcommand{\funct}[3]{#1:\hspace{1mm} #2\to #3}
\newcommand{\abs}[1]{|#1|}

\newtcolorbox[auto counter, number within=section]{mytheorem}[2][]{
  enhanced,
  breakable,
  title=Teorema~\thetcbcounter: #2,
  #1,
}
\newtcolorbox[auto counter, number within=section]{propositionbox}[2][]{
  enhanced,
  breakable,
  title=Proposition~\thetcbcounter: #2,
  #1,
}

\newtcolorbox[auto counter, number within=section]{corollarybox}[2][]{
  enhanced,
  breakable,
  title=Corollary~\thetcbcounter: #2,
  #1,
}

\newtcolorbox[auto counter, number within=section]{remarkbox}[2][]{
  enhanced,
  breakable,
  title=Remark~\thetcbcounter: #2,
  #1,
}

\newtcolorbox[auto counter, number within=section]{notebox}[2][]{
  enhanced,
  breakable,
  title=Note~\thetcbcounter: #2,
  #1,
}


\newenvironment{Figura}
  {\par\medskip\noindent\minipage{\linewidth}}
  {\endminipage\par\medskip}


\newcommand{\proofnamefont}{\bfseries}
\setlength {\marginparwidth}{2cm}
\begin{document}


	\frontmatter 
	\setstretch{1.5} 
	\fancyhead{} 
	\rhead{\thepage} 
	\lhead{} 
	\pagestyle{fancy} 
	\newcommand{\HRule}{\rule{\linewidth}{0.5mm}} 
	\hypersetup{pdftitle={\titulo}}
	\hypersetup{pdfauthor=\autor}
%---------------------------------------------------------------------------
	
  \begin{minipage}[t]{1.5cm}
    \vspace{30pt}
      \begin{picture}(50,100)
        %\put(-113,55){\includegraphics[scale=0.15]{Pictures/UGR-Logo.png}
        %\put(-40,-550){\rule[2pt]{1pt}{21cm}}
        %\put(-37,-550){\rule[2pt]{2mm}{21cm}}
        %\put(-29,-550){\rule[2pt]{1pt}{21cm}}
        %\put(-18,-550){\rule[2pt]{1pt}{21cm}}
        %\put(-15,-550){\rule[2pt]{2mm}{21cm}}
        %\put(-7,-550){\rule[2pt]{1pt}{21cm}}
        %\put(-63,53){\includegraphics[scale=0.25]{Pictures/Logo-FS.png}}
        
        \put(-67,60)
        {\includegraphics[scale=0.25]{Logotipo_I_Facultad_de_Ciencias_Fondo_blanco_negativo.png}}
        \put(-40,-550){\rule[2pt]{1pt}{21cm}}
        \put(-37,-550){\rule[2pt]{2mm}{21cm}}
        \put(-29,-550){\rule[2pt]{1pt}{21cm}}
        \put(-18,-550){\rule[2pt]{1pt}{21cm}}
        \put(-15,-550){\rule[2pt]{2mm}{21cm}}
        \put(-7,-550){\rule[2pt]{1pt}{21cm}}
      \end{picture}
  \end{minipage}
  \begin{minipage}[t]{14cm}
  	\vspace{40pt}
   	\begin{picture}(400,100)
   		\put(-25,178){\rule[2pt]{16cm}{1pt}}
   		\put(-25,170){\rule[2pt]{16cm}{2mm}}
   		\put(-25,167){\rule[2pt]{16cm}{1pt}}
   		\put(-25,130){
        
   		%\put(70,100){
        \begin{tabular}[t]{c}
          {\huge\textbf{ UNIVERSIDAD DE CÓRDOBA}}\\
          %{\huge\textbf{UNIVERSIDAD DE CÓRDOBA}}\\
          %{\huge\textbf{FISICA SOCIETY}}\\
             \\
           {\LARGE\textbf{\sc Facultad de Ciencias}}\\
              \\
           {\Large\textbf{Grado de F\'isica}}
        \end{tabular}}
    \end{picture}

  \begin{center}
  	{
    	\Large\bfseries 
     	Apuntes de
    	\par
    }%
    \vskip 3.5em%
    {
    	\begin{tabular}[t]{c}%
    		{\huge \textbf{FÍSICA DEL ESTADO SÓLIDO}}
     	\end{tabular} \par
    }
    \vskip 3em%
\fbox{\parbox{8.1cm}{\Large$\left[-\frac{\hbar^2}{2m}\nabla^2+V(\vec{r})\right]\Psi(\vec{r})=E\Psi(\vec{r})$\strut}}
      \vskip 1.5em %
      {
      	\begin{tabular}[t]{c}%
      		{\Large Autor:}
      	\end{tabular} 
        \par
      }
      \vskip 1.5em
      {
      	\begin{tabular}[t]{c}%
      		{\Large Rubén Carrión Castro}
      	\end{tabular}
      	\par
      }%
      \vskip 1.5em%
	  \vskip 20pt
	  {
		  \Large
			\begin{tabular}{ll}
				Profesores de la asignatura: & Dr. Pedro Rodríguez García   \\
			& Dr. Alberto Jiménez Solano\end{tabular}
		}
   \end{center}
   \vfill
   \begin{center}
     \begin{tabular}{c}
      {
          \large Cursado en la Universidad de Córdoba
      }
          \quad \hfill \quad {\large \today}
     \end{tabular}
   \end{center} \par%
\end{minipage}
{ 

%----------------------------------------------------------------------------------------
%	QUOTATION PAGE
%----------------------------------------------------------------------------------------
\pagestyle{empty}
\null\vfill 
\vfill\vfill\vfill\vfill\vfill\vfill\null 
\cleardoublepage 
	\setstretch{1.3} % Return the line spacing back to 1.3

\pagestyle{empty} % Page style needs to be empty for this page

% \begin{flushright}
% \textit{Para mis compañeros de Córdoba que no tienen Relatividad General \ldots}
% \end{flushright} % Dedication text

\addtocontents{toc}{\vspace{2em}} % Add a gap in the Contents, for aesthetics

	% \input{Agradecimientos}
	%\listoftodos
	\pagestyle{fancy} % The page style headers have been "empty" all this time, now use the "fancy" headers as defined before to bring them back

\lhead{\emph{\'{I}ndice general}} % Set the left side page header to "Contents"
\tableofcontents % Write out the Table of Contents
	\mainmatter 
	\pagestyle{fancy} 
    
    % INTRODUCCIÓN
%\chapter{Introducci\'on} % Main chapter title
\chapter*{Prefacio}
\addcontentsline{toc}{chapter}{Prefacio} 
\label{Introduccion} 
\lhead{\emph{Prefacio}} 
\noindent\textit{Todos somos muy ignorantes. Lo que ocurre es que no todos ignoramos las mismas cosas.} (A. Einstein)
\\ \\
Estos apuntes están hechos para acercar a estudiantes como yo a la Relatividad General, empezaremos con unas nociones de tensores, que hice en la Universidad de Córdoba con el doctor Jónatan Herrera. Seguiremos con la Teoría Especial de la Relatividad, luego un repaso de Geometría Diferencial y finalmente introduciremos la Relatividad General.
    %CAPITULO 1

\chapter{Álgebra de Tensores en espacios vectoriales} % Main chapter title
\noindent\textit{“Toda la ciencia no es más que un refinamiento del pensamiento cotidiano”.}\\(A. Einstein)
\newpage
\label{Capitulo1} 
\lhead{\emph{Álgebra de Tensores en espacios vectoriales}} 
%-------------------------------------------------------------------------------
%------------------------------------------------------------------------------
%-------------------------------------------------------------------------------
Para poder abordar este capítulo, es recomendable hacer una lectura del \textit{\textbf{Apéndice A. Álgebra: Lógica, Conjuntos y Grupos}}.
%SECCION 1
\section{Repaso histórico} % Main chapter title
\label{cap2-sec1} 
%------------------------------------------------------------------------------
	La física clásica, del siglo XIX, era una física bien asentada. La cuál explica la mecánica con el libro de Sir Isaac Newton titulado \textit{Philosophiae Naturalis Principia Mathematica} y el electromagnetismo se explica con el libro de Maxwell titulado \textit{Electricity and Magnetism}.\\
 En 1887, Michelson y Morley iniciaron una revolución en la física con un experimento para medir la velocidad de la luz. El experimento consistía en medir la velocidad de la luz de un rayo paralelo al eje de rotación de la Tierra y de otro rayo perpendicular a este, esperándose obtener resultados diferentes. En cambio, se observó que ambos rayos iban exactamente igual, cosa que no tenía sentido en la época., por tanto, determinaron que la velocidad de la luz no era instantánea, sino que debía ser finita, y llegaron a un resultado de ésta bastante próximo al valor actual de la velocidad de la luz.
 \subsection{Relatividad Galileana}
 El Principio de Relatividad de Galileo establece que,
 \begin{center}
 \textit{''Es imposible determinar a base de experimentos (mecánicos) si un sistema de referencia está en reposo o en movimiento uniforme y rectilíneo''.}
 \end{center}
 Esto se derivó de que en la Relatividad Galileana hay un espacio absoluto en el que las leyes de Newton son ciertas. Definiremos un \textit{sistema de referencia inercial} (SRI) como aquel sistema referencia que se mueve a velocidad constante respecto al espacio absoluto. Además, todos los sistemas de referencia inerciales comparten un tiempo absoluto. Pero con la definición de SRI, el Principio de Relatividad se debe reformular con este concepto, así tenemos el Principio de Relatividad en formulación de equivalencia, que dice que
 \begin{center}
 \textit{''Todos los sistemas inerciales son equivalentes, es decir, todos los observadores inerciales ven la misma física''.}
 \end{center}
 \textbf{Leyes de Newton}\\ \\
 La Ley de Newton por excelencia es $\vec{F}=m\vec{a}=-\nabla V(\vec{r}-\vec{r}_0)$, donde $V$ es la función potencial. Esta ley (y las demás) transforman bien bajo el grupo de transformaciones de Galileo, que son:
 \begin{enumerate}
     \item \textbf{Traslaciones temporales:}
     \[t\to t'=t+t_0\]
     \item \textbf{Traslaciones espaciales:}
     \[\vec{r}\to\vec{r}'=\vec{r}+\vec{r}_i+\vec{v}t\]
     donde $\vec{v}$ es la velocidad relativa de un SRI con respecto al otro, y $\vec{r}_i$ es el vector de posición entre los orígenes de ambos SRI al inicio.
     \item \textbf{Rotaciones espaciales:}
     \[\vec{a}'=R(\theta)\vec{a}\]
     donde $R(\theta)$ es la matriz de rotación.
\end{enumerate}
Se puede ver que las Leyes de Newton no son covariantes, pero sí transforman bien, pues la física se mantiene, esto quiere decir que \textit{las Leyes de Newton de la física transforman de forma covariante}.\\ \\
El grupo de transformaciones de Galileo son simetrías que dan lugar a cantidades conservadas. Por tanto, si tenemos un Lagrangiano que sea invariante bajo traslaciones temporales, tendremos que el sistema conserva energía; si es invariante bajo traslaciones espaciales, conserva momento lineal; y si es invariante bajo rotaciones espaciales; conserva momento angular.\\ \\
El grupo de transformaciones de Galileo NO deja invariante las ecuaciones de Maxwell, que son
\[(i)\hspace{2mm}\nabla\cdot\vec{E}=\rho/\epsilon_0;\hspace{5mm}(iii)\hspace{2mm}\nabla\cdot\vec{B}=0\]
\[(ii)\hspace{2mm}\nabla\times\vec{B}=\partial_t\vec{E}/c^2+\mu_0\vec{J};\hspace{5mm}(iv)\hspace{2mm}\nabla\times\vec{E}=-\partial_t\vec{B}\]
Si $\rho=0$ y $\vec{J}=0$, es decir, estamos en vacío, podemos combinar las ecuaciones de Maxwell en una sola ecuación de ondas que se propaga a velocidad $c=299792,458$ m/s, resultado muy próximo al valor obtenido por Michelson y Morley, que además es independiente del sistema de referencia.
\subsection{Transformaciones de Lorentz}

Las transformaciones de Lorentz hacen que las ecuaciones de Maxwell transformen bien (sean covariantes). Estas transformaciones son:
\[\begin{array}{rcrc}
    (i) & t'=\gamma\left(t-\frac{v}{c^2}x\right); & (iii) & y'=y \\
    (ii) & x'=\gamma\left(x-vt\right); & (iv) & z'=z
\end{array}\]
donde $v$ es la velocidad relativa entre SRI (que suponemos que se mueven en el eje $X$), y $\gamma=\frac{1}{\sqrt{1-\frac{v^2}{c^2}}}$.\\ \\
Como estas transformaciones hacen que las leyes de Maxwell sean covariantes, diremos que las transformaciones de Lorentz sean más fundamentales que las transformaciones de Galileo.\\ \\
Además, vemos que por la transformación $(i)$ el tiempo ya \textbf{no es absoluto}, sino que depende del SRI, por lo que diremos que el tiempo es \textbf{relativo}.
%SECCION 2
\section{Álgebra de Tensores} % Main chapter title
\label{cap1-sec2} 
Llegamos a lo groso del capítulo, el \textbf{Álgebra de Tensores}. En este apartado vamos a ver qué es un tensor de forma matemática y cómo trabajar con ellos. También se mencionará cómo trabajamos los físicos con los tensores.
%------------------------------------------------------------------------------

\subsection{Producto tensorial: caso de dos términos} % Main chapter title
\label{cap1-sec2-subsec1} 
Vamos a ver qué es el \textbf{producto tensorial} y cómo los tensores se definen a partir de este.
\begin{proposition}
    Sea $V$ un $\mathbb{K}$-espacio vectorial, $\scalar{\cdot}{\cdot}$ el producto escalar euclídeo y $B=\curlybraces{v_1,\dots,v_n}$ base de $V$, 
    \[\begin{array}{cccl}
        f_v: & V & \to & V^*\\
         & v & \mapsto & f_v(v)=\scalar{v}{\cdot}
    \end{array}\]
     $f_v$ es una aplicación lineal, concretamente es un isomorfismo.
\end{proposition}
\begin{proof}
    Vemos que $f_v$ es aplicación lineal,
    \[f_v(w_1+w_2)=\scalar{v}{w_1+w_2}=\scalar{v}{w_1}+\scalar{v}{w_2}=f_v(w_1)+f_v(w_2)\checkmark \]
    \[f_v(\lambda\cdot w)=\scalar{v}{\lambda\cdot w}=\lambda\scalar{v}{w}=\lambda f_v(w)\checkmark\]
    para $\forall\lambda\in\mathbb{K}$ y $\forall w_1,w_2,w\in V$. Luego, es aplicación lineal.\\ \\
    Veamos que es isomorfo demostrando que es biyectivo, pues ya hemos visto que es aplicación lineal.\\
    Sabemos que $ker\curlybraces{f_v}=\curlybraces{0}\Leftrightarrow f_v$ es inyectiva. Luego, vemos si $ker\curlybraces{f_v}=\curlybraces{0}$:
    \[ker\curlybraces{f_v}=\curlybraces{w\in V,f_v(w)=0}=\curlybraces{w\in V;\scalar{v}{w}=0\Leftrightarrow w=0}\]
    Por tanto, $ker\curlybraces{f_v}=\curlybraces{0}$ y así, $f_v$ es inyectiva. $\checkmark$\\ \\
    Usando el Primer Teorema de isomorfía, tenemos que $dim(V)=\cancelto{0}{dim(ker\curlybraces{f_v})}+dim(Im f_v)$, pero como la $dim B=dim B^*$, siendo $B$ base de $V$ y $B^*$ base de $V^*$, entonces $dimV=dimV^*$, y por tanto, $dimV=dimImf_v=dimV^*$, luego $Imf_v$ es $V^*$ y por tanto, $f_v$ es sobreyectiva. $\checkmark$\\
    Luego, $f_v$ es un isomorfismo.
\end{proof}
\noindent Veamos cómo se define el producto tensorial y sus propiedades.
\begin{definition}
    Sea $V$ un $\mathbb{K}$-espacio vectorial, $V^*$ el dual de $V$, y $g^1,g^2\in V^*$ aplicaciones lineales, tal que $g^1:V\to\mathbb{K}$ y $g^2:V\to\mathbb{K}$. Así, definimos el producto tensorial como,
    \begin{enumerate}[label=(\roman*)]
        \item Producto tensorial entre dos formas $g^1,g^2\in V^*$,
        \[\begin{array}{cccl}
            \ptensor{g^1}{g^2}: & V\times V & \to & \mathbb{K}\\
            & (v,w) & \mapsto & g^1(v)g^2(w)
        \end{array}\]
        \item Producto tensorial entre dos vectores $v_1,v_2\in V$,
        \[\begin{array}{cccl}
            \ptensor{v_1}{v_2}: & V^*\times V^* & \to & \mathbb{K}\\
             & (f,g) & \mapsto & f(v_1)g(v_2)
        \end{array}\]
        \item Producto tensorial de una forma y un vector $v_1\in V$, $f^1\in V^*$,
        \[\begin{array}{cccl}
            \ptensor{v_1}{f^1}: & V^*\times V & \to & \mathbb{K}\\
             & (g,w) & \mapsto & g(v_1)f^1(w)
        \end{array}\]
    \end{enumerate}
\end{definition}

\begin{proposition}
    Los productos tensoriales definidos anteriormente son formas bilineales.
\end{proposition}
\begin{proof} 
Usando $\forall v_1,v_2,u_1,u_2,v,w,u\in V$, $\forall f^1,f^2,g,p,q\in V^*$ y $\forall \lambda\in\mathbb{K}$,
    \begin{enumerate}[label=(\roman*)]
        \item \[\begin{array}{cccl}
            \ptensor{f^1}{f^2}: & V\times V & \to & \mathbb{K}\\
            & (v,w) & \mapsto & f^1(v)f^2(w)
        \end{array}\]
        siendo $f^1,f^2\in V^*$. Veamos que es forma bilineal,
        \[\begin{array}{lrl} \text{\textbullet)} &(\ptensor{f^1}{f^2})(u_1+u_2,v)=&f^1(u_1+u_2)f^2(v)=\brackets{f^1(u_1)+f^1(u_2)}f^2(v)\\ &=&f^1(u_1)f^2(v)+f^1(u_2)f^2(v)=(\ptensor{f^1}{f^2})(u_1,v)+(\ptensor{f^1}{f^2})(u_2,v),\checkmark\\  \text{\textbullet)} &(\ptensor{f^1}{f^2})(v,u_1+u_2)  =&f^1(v)f^2(u_1+u_2)=f^1(v)\brackets{f^2(u_1)+f^2(u_2)}\\ &=&f^1(v)f^2(u_1)+f^1(v)f^2(u_2)=(\ptensor{f^1}{f^2})(v,u_1)+(\ptensor{f^1}{f^2})(v,u_2)\checkmark\\
             \text{\textbullet)} & (\ptensor{f^1}{f^2})(\lambda v,u) =& f^1(\lambda v)f^2(u)=\lambda f^1(v)f^2(u)=\lambda(\ptensor{f^1}{f^2})(v,u)\checkmark\\
        \text{\textbullet)}&(\ptensor{f^1}{f^2})(u,\lambda v)=&f^1(u)f^2(\lambda v)=\lambda f^1(u)f^2(v)=\lambda(\ptensor{f^1}{f^2})(u,v)\checkmark
          \end{array}\]
        Luego, $\ptensor{f^1}{f^2}$ es una forma bilineal. $\qedh $
        \item \[\begin{array}{cccl}
            \ptensor{v_1}{v_2}: & V^*\times V^* & \to & \mathbb{K}\\
             & (f,g) & \mapsto & f(v_1)g(v_2)
        \end{array}\]
         \[\begin{array}{lrl}
         \text{\textbullet)}&(\ptensor{v_1}{v_2})(f^1+f^2,g)=&(f^1+f^2)(v_1)g(v_2)=\brackets{f^1(v_1)+f^2(v_1)}g(v_2)\\
         &=&f^1(v_1)g(v_2)+f^2(v_1)g(v_2)=(\ptensor{v_1}{v_2})(f^1,g)+(\ptensor{v_1}{v_2})(f^2,g)\checkmark\\
         \text{\textbullet)}&(\ptensor{v_1}{v_2})(g,f^1+f^2)=&g(v_1)(f^1+f^2)(v_2)g=g(v_1)\brackets{f^1(v_2)+f^2(v_2)}\\
         &=&g(v_1)f^1(v_2)+g(v_1)f^2(v_2)=(\ptensor{v_1}{v_2})(g,f^1)+(\ptensor{v_1}{v_2})(g,f^2)\checkmark\\
         \text{\textbullet)}&(\ptensor{v_1}{v_2})(\lambda f,g)=&(\lambda f)(v_1)g(v_2)=\lambda f(v_1)g(v_2)=\lambda(\ptensor{v_1}{v_2})(f,g)\checkmark\\
         \text{\textbullet)}&(\ptensor{v_1}{v_2})(g,\lambda f)=&g(v_1)(\lambda f)(v_2)=\lambda g(v_1)f(v_2)=\lambda(\ptensor{v_1}{v_2})(g,f)\checkmark
         \end{array}\]
        Luego, $\ptensor{v_1}{v_2}$ es una forma bilineal. $\qedh $
        \item \[\begin{array}{cccl}
            \ptensor{v_1}{f^1}: & V^*\times V & \to & \mathbb{K}\\
             & (g,w) & \mapsto & g(v_1)f(w)
        \end{array}\]
        \[\begin{array}{lrl}
        \text{\textbullet)}&(\ptensor{v_1}{f^1})(p+q,w)=&(p+q)(v_1)f^1(w)=\brackets{p(v_1)+q(v_1)}f^1(w)=\\
        &=&p(v_1)f^1(w)+q(v_1)f^1(w)=(\ptensor{v_1}{f^1})(p,w)+(\ptensor{v_1}{f^1})(q,w)\checkmark\\
        \text{\textbullet)}&(\ptensor{v_1}{f^1})(g,u+w)=&g(v_1)f^1(u+w)=g(v_1)\brackets{f^1(u)+f^1(w)}=\\
        &=&g(v_1)f^1(u)+g(v_1)f^1(w)=(\ptensor{v_1}{f^1})(g,u)+(\ptensor{v_1}{f^1})(g,w)\checkmark\\
        \text{\textbullet)}&(\ptensor{v_1}{f^1})(\lambda g,w)=&(\lambda g)(v_1)f^1(w)=\lambda g(v_1)f^1(w)=\lambda(\ptensor{v_1}{f^1})(g,w)\checkmark\\
        \text{\textbullet)}&(\ptensor{v_1}{f^1})(g,\lambda w)=&g(v_1)f^1(\lambda w)=\lambda g(v_1)f^1(w)=\lambda(\ptensor{v_1}{f^1})(g,w)\checkmark
        \end{array}\]
            Luego, $\ptensor{v_1}{f^1}$ es una forma bilineal. \qedhere
    \end{enumerate}
\end{proof}
\noindent El producto tensorial no se da solo entre elementos de los espacios vectoriales o duales, sino que también se puede dar entre espacios, siendo el nuevo espacio generado un \textbf{espacio vectorial}.
\begin{proposition}
    El espacio $\ptensor{V}{V}$ tiene estructura de espacio vectorial.
\end{proposition}
\begin{proof}
    \begin{enumerate}
        \item Vemos que $(\ptensor{V}{V},+)$ es grupo abeliano:
        \begin{enumerate}[label=(\roman*)]
            \item Vemos si la operación $+$ es cerrada:
            \\
            $\forall v,w,z\in V$ con $\ptensor{v}{w},\ptensor{v}{z},\ptensor{w}{z}\in\ptensor{V}{V}$, tenemos que ver si $\ptensor{(v+w)}{z}\in\ptensor{V}{V}$. Sabemos que,
            \[\begin{array}{cccl}
                \ptensor{v}{w}: & \pcart{V^*}{V^*} & \to &\mathbb{R}  \\
                 & (f,g) & \mapsto & f(v)g(w)
            \end{array}\]
            luego,
            \[\begin{array}{cccl}
                \ptensor{(v+w)}{z}: & \pcart{V^*}{V^*} & \to &\mathbb{R}  \\
                 & (f,p) & \mapsto & f(v+w)p(z)
            \end{array}\]
            Entonces,
            \[(\ptensor{(v+w)}{z})(g,p)=f(v+w)p(z)=\brackets{f(v)+f(w)}p(z)=\]\[=f(v)p(z)+f(w)p(z)=(\ptensor{v}{z})(f,p)+(\ptensor{w}{z})(f,p)\]
            Luego, $\ptensor{(v+w)}{z}\in\ptensor{V}{V}$ y así, la operación $+$ es cerrada. $\checkmark$
            \item Asociatividad:
            \\
            Sean $\ptensor{a}{b},\ptensor{c}{d},\ptensor{e}{f}\in\ptensor{V}{V}$, tenemos que ver si $\ptensor{a}{b}+\brackets{\ptensor{c}{d}+\ptensor{e}{f}}=\brackets{\ptensor{a}{b}+\ptensor{c}{d}}+\ptensor{e}{f}$, tal que
            \[(\ptensor{a}{b}+\brackets{\ptensor{c}{d}+\ptensor{e}{f}})(p,q)=p(a)q(b)+\brackets{p(c)q(d)+p(e)q(f)}=p(a)q(b)+p(c+e)q(d+f)=\]
            \[=p(a+c+e)q(b+d+f)=p(a+c)q(b+d)+p(e)q(f)=\brackets{p(a)q(b)+p(c)q(d)}+p(e)q(f)=\]\[=(\brackets{\ptensor{a}{b}+\ptensor{c}{d}}+\ptensor{e}{f})(p,q)\checkmark\]
            \item Elemento neutro:\\
            Sea $\ptensor{e_1}{e_2}\in\ptensor{V}{V}$ el elemento neutro de $\ptensor{V}{V}$, tal que
            \[\ptensor{e_1}{e_2}+\ptensor{v}{w}=\ptensor{v}{w}+\ptensor{e_1}{e_2}=\ptensor{v}{w}\]
            Vemos el valor de este elemento neutro,
            \[(\ptensor{e_1}{e_2}+\ptensor{v}{w})(f,g)=(\ptensor{v}{w})(f,g)\]
            \[f(e_1)g(e_2)+f(v)+g(w)=f(v)g(w)\]
            \[f(e_1+v)g(e_w+w)=f(v)g(w)\Leftrightarrow\left\lbrace\begin{matrix}
                e_1=0\\
                e_2=0
            \end{matrix}\right.\]
            luego, $\ptensor{e_1}{e_2}=0$. $\checkmark$
            \item Elemento simétrico:
            \\
            $\forall\ptensor{v}{u}\in\ptensor{V}{V}$, $\exists\ptensor{\Tilde{v}}{\Tilde{u}}\in\ptensor{V}{V}$, tal que
            \[\ptensor{v}{u}+\ptensor{\Tilde{v}}{\Tilde{u}}=\ptensor{\Tilde{v}}{\Tilde{u}}+\ptensor{v}{u}=\ptensor{e_1}{e_2}=0\]
         Veamos quién es $\ptensor{\Tilde{v}}{\Tilde{u}}$,
        \[(\ptensor{v}{u}+\ptensor{\Tilde{v}}{\Tilde{u}})(f,g)=f(v)g(u)+f(\Tilde{v})g(\Tilde{u})=(\ptensor{0}{0})(f,g)=f(0)g(0)\]
        luego,
        \[v+\Tilde{v}=0\Rightarrow\Tilde{v}=-v\]
        \[u+\Tilde{u}=0\Rightarrow\Tilde{u}=-u\]
        Por tanto, el elemento simétrico de $\ptensor{v}{u}$ es $\ptensor{(-v)}{(-u)}$. $\checkmark$
        \item Conmutabilidad:\\
        Sean $\ptensor{v}{w},\ptensor{u}{z}\in\ptensor{V}{V}$, entonces
        \[(\ptensor{v}{w}+\ptensor{u}{z})(f,g)=f(v)g(w)+f(u)g(z)=f(v+u)g(w+z)=\]\[=f(u+v)g(z+w)=f(u)g(z)+f(v)g(w)=(\ptensor{u}{z}+\ptensor{v}{w})(f,g)\checkmark\]
    Luego, es grupo abeliano. $\checkmark$
         \end{enumerate}
         \item Doble propiedad distributiva:
         \begin{enumerate}
             \item $\forall\lambda,\mu\in\mathbb{R}$, $\forall\ptensor{v}{w}\in\ptensor{V}{V}$,
             \[(\lambda+\mu)\cdot(\ptensor{v}{w})(f,g)=(\lambda+\mu)f(v)g(w)=\]\[=\lambda f(v)g(w)+\mu f(v)g(w)=\lambda(\ptensor{v}{w})(f,g)+\mu(\ptensor{v}{w})(f,g)\checkmark\]
             \item $\forall\lambda\in\mathbb{R}$, $\forall\ptensor{v}{w},\ptensor{u}{z}\in\ptensor{V}{V}$, tenemos que
             \[\lambda(\ptensor{v}{w})(f,g)+\lambda(\ptensor{u}{z})(f,g)=\lambda f(v)g(w)+\lambda f(u)g(z)=\]\[=\lambda\brackets{f(v)g(w)+f(u)g(z)}=\lambda(\ptensor{v}{w}+\ptensor{u}{z})(f,g)\checkmark\]
             \end{enumerate}
             \item Propiedad pseudo-asociativa:\\
             $\forall\lambda,\mu\in\mathbb{R}$; $\forall\ptensor{v}{w}\in\ptensor{V}{V}$, tenemos que
             \[\lambda\cdot\brackets{\mu\cdot(\ptensor{v}{w})(f,g)}=\lambda\brackets{\mu f(v)g(w)}=\lambda f(\mu v)g(\mu w)=\]\[=f(\lambda\mu v)g(\lambda\mu w)=f(\mu\lambda v)g(\mu\lambda w)=\mu\brackets{f(\lambda v)g(\lambda w)}=(\mu\cdot\lambda)f(v)g(w)=(\mu\cdot\lambda)(\ptensor{v}{w})(f,g)\checkmark\]
             \item Elemento unitario del cuerpo: $\forall\ptensor{v}{w}\in\ptensor{V}{V}$; $\Tilde{\mu}\in\mathbb{R}$, entonces $\Tilde{\mu}\cdot\ptensor{v}{w}=\ptensor{v}{w}\cdot\Tilde{\mu}=\ptensor{v}{w}$
             \[(\Tilde{\mu}\cdot\ptensor{v}{w})(f,g)=f(\Tilde{\mu}v)g(\Tilde{\mu}w)=(\ptensor{v}{w})(f,g)=f(v)g(w)\Rightarrow\begin{matrix}
                 \Tilde{\mu}\cdot v=v\\
                 \Tilde{\mu}\cdot w=w
             \end{matrix}\Leftrightarrow\Tilde{\mu}=1\checkmark\]
       \end{enumerate}
       Luego, $(\ptensor{V}{V}, +, \cdot)$ es un $\mathbb{R}$-espacio vectorial.
\end{proof}
\noindent Al igual que cualquier otro espacio vectorial, el espacio $V\otimes V$ deberá tener una \textbf{base}.
\begin{proposition}
    Si tenemos un $V$ espacio vectorial sobre $\mathbb{K}$ con base $B=\curlybraces{v_1,\dots,v_n}$, entonces todo $\ptensor{v}{w}$ será combinación lineal de los elementos de la base de $\ptensor{V}{V}$ dada por $\ptensor{B}{B}=\curlybraces{\ptensor{v_i}{v_j}}_{i,j=1}^{n}$
\end{proposition}
\begin{proof}
    Queremos ver que $\curlybraces{\ptensor{v_i}{}v_j}_{i,j=1}^n$ es base de $\ptensor{V}{V}$. Para ello, tendremos que ver que esta base $\ptensor{B}{B}$ complete el espacio $\ptensor{V}{V}$ y que los vectores de la misma sean linealmente independientes.\\
    Sabemos que $\ptensor{v}{w}\in\ptensor{V}{V}$ y que
    \[\begin{array}{cccl}
        \ptensor{v}{w}: & \pcart{V^*}{V^*} & \to & \mathbb{R}\\
         & (f,g) & \mapsto & f(v)g(w)
    \end{array}\]
    Luego, para que la base $\ptensor{B}{B}$ complete el espacio $\ptensor{V}{V}$, se deberá poder expresar cualquier vector $\ptensor{v}{w}\in\ptensor{V}{V}$ como combinación lineal de los vectores de $\ptensor{B}{B}$. Podemos usar $B=\curlybraces{v_i}_{i=1}^n$ base de $V$, tal que
    \[v=\sum\limits_{i=1}^n\lambda^iv_i=\lambda^iv_i,\hspace{4mm}w=\sum\limits_{j=1}^n\mu^jv_j=\mu^jv_j\]
    Por tanto, usando $f,g\in V^*$, tenemos que
    \[\ptensor{v}{w}(f,g)=f(v)g(w)=f(\lambda^iv_i)g(\mu^jv_j)=\lambda^if(v_i)\mu^jg(v_j)=\lambda^i\mu^jf(v_i)g(v_j)=\lambda^i\mu^j(\ptensor{v_i}{v_j})(f,g)\]
    Luego, hemos expresado un vector del espacio $\ptensor{V}{V}$ como combinación lineal de los vectores de la base $\ptensor{B}{B}$. $\checkmark$\\ \\
    Veamos que son linealmente independientes, para ello, se debe cumplir que,
    \[\sum\limits_{i,j=1}^n\lambda^{ij}(\ptensor{v_i}{v_j})=\lambda^{ij}(\ptensor{v_i}{v_j})=0\Leftrightarrow\lambda^{ij}=0\]
    Sabiendo que la base de $V^*$ es $B^*=\curlybraces{f^1,f^2,\dots,f^n}$, tal que
    \[f^i(v_i)=1\hspace{5mm}f^j(v_i)\overset{i\neq j}{=}0\Rightarrow f^i(v_j)=\delta_{ij}\]
    Podemos evaluar lo anterior en dos elementos arbitrarios de $B^*$, tal que
    \[0=\lambda^{ij}(\ptensor{v_i}{v_j})(f^n,f^m)= \lambda^{ij}f^n(v_i)f^m(v_j)=\lambda_{ij}\delta_{n}^i\delta_{m}^j=\lambda^{nm}\]
    luego, $\lambda^{nm}=0$ y por tanto, los vectores son linealmente independientes. $\checkmark$\\ \\
    Así, hemos demostrado que $\ptensor{B}{B}$ es base de $\ptensor{V}{V}$.
\end{proof}

\begin{note}
    Denotaremos $\ptensor{v}{w}\equiv h$, tal que
    \[
    \begin{array}{cccl}
        h: & \pcart{V^*}{V^*} & \to & \mathbb{R} \\
         & (f^i,f^j) & \mapsto & h(f^i,f^j)=h^{ij}
    \end{array}
    \]
    siendo $f^i,f^j\in B^*$. Por tanto, para dos $p,q\in V^*$ cualesquiera, escribiremos
    \[(\ptensor{v}{w})(p,q)=h(p,q)=h\left(\sum_{i=1}^np_if^i,\sum_{j=1}^nq_jf^j\right)=p_iq_j(f^i,f^j)=h^{ij}p_iq_j\]
\end{note}
\noindent Veamos algunas \textbf{propiedades} del producto tensorial.
\begin{proposition}
    Sea $V$ un $\mathbb{R}$-espacio vectorial,
    \begin{enumerate}[label=(\roman*)]
        \item $\ptensor{(v_1+v_2)}{w}=\ptensor{v_1}{w}+\ptensor{v_2}{w}$; $\forall v_1,v_2,w\in V$.
        \item $\ptensor{w}{(v_1+v_2)}=\ptensor{w}{v_1}+\ptensor{w}{v_2}$, $\forall v_1,v_2,w\in V$.
        \item $\ptensor{(\lambda v)}{w}=\lambda\ptensor{v}{w}$, $\forall v,w\in V$, $\forall\lambda\in\mathbb{R}$.
        \item $\ptensor{w}{(\lambda v)}=\lambda\ptensor{w}{v}$, $\forall v,w\in V$, $\forall \lambda\in\mathbb{R}$.
        \item $\ptensor{v}{w}\neq\ptensor{w}{v}$.
        \item $\ptensor{v}{w}\neq0$ si $v\neq0$ ó $w\neq 0$.
        \item Sea $\ptensor{a}{b}\neq0$, $\ptensor{a}{b}=\ptensor{a'}{b'}\Leftrightarrow a'=\lambda a$ y $b'=\lambda^{-1}b$.
        \item $\ptensor{V}{W}$ es isomorfo con $\ptensor{W}{V}$.
    \end{enumerate}
\end{proposition}
\begin{proof}
    \begin{enumerate}[label=(\roman*)]
        \item $\forall v_1,v_2,w\in V$,
        \[(\ptensor{(v_1+v_2)}{w})(f,g)=f(v_1+v_2)g(w)=\brackets{f(v_1+f(v_2)}g(w)=\]\[=f(v_1)g(w)+f(v_2)g(w)=(\ptensor{v_1}{w})(f,g)+(\ptensor{v_2}{w})(f,g)\qedh\]
        \item $\forall v_1,v_2,w\in V$,
        \[(\ptensor{w}{(v_1+v_2)})(f,g)=f(w)g(v_1+v_2)=f(w)\brackets{g(v_1)+g(v_2)}=\]\[=f(w)g(v_1)+f(w)g(v_2)=(\ptensor{w}{v_1})(f,g)+(\ptensor{w}{v_2})(f,g)\qedh\]
        \item $\forall v,w\in V$ y $\forall\lambda\in\mathbb{R}$,
        \[(\ptensor{(\lambda\cdot v)}{w})(f,g)=f(\lambda\cdot v)g(w)=\lambda\cdot f(v)g(w)=\lambda\cdot(\ptensor{v}{w})(f,g)\qedh\]
        \item $\forall v,w\in V$ y $\forall\mu\in\mathbb{R}$,
        \[(\ptensor{w}{(\lambda\cdot v)})(f,g)=f(w)g(\lambda\cdot v)=\lambda\cdot f(w)g(v)=\lambda\cdot(\ptensor{w}{v})(f,g)\qedh\]
        \item Vemos que, $(\ptensor{v}{w})(f,g)=f(v)g(w)$ y que $(\ptensor{w}{v})(f,g)=f(w)g(v)$, luego estos elementos serían iguales solo si $f\equiv g$. $\qedh$
        \item Sean $v,w\in V$ y $f,g\in V^*$, tales que $f\not\equiv0$ y $g\not\equiv0$, entonces
        \[(\ptensor{v}{w})(f,g)=f(v)g(w)=0\Leftrightarrow\begin{matrix}
            f(v)=0 & \Leftrightarrow v=0\\
            \text{ó} & \\
            g(w)=0 & \Leftrightarrow w=0
        \end{matrix}\qedh\]
        \item \begin{tabular}{c|}
             $\Rightarrow$ \\ \hline
        \end{tabular} 
        Sea $\ptensor{a}{b}=\ptensor{a'}{b'}$ entonces
        \[(\ptensor{a}{b})(f,g)=f(a)g(b)=(\ptensor{a'}{b'})(f,g)=f(a')g(b')\]
        luego,
        \[f(a)g(b)=f(a')g(b')\]
        pero como $a\neq a'$ y $b\neq b'$, debe haber una relación entre ambos, de tal forma que se cumpla la igualdad anterior. Supondremos que $a$ y $a'$ tienen una relación lineal (la más sencilla), tal que $a'=\lambda a+c$, luego 
        \[f(a)g(b)=f(a')g(b')=f(\lambda a+c)g(b')=f(\lambda a)g(b')+f(c)g(b')=\lambda f(a)g(b')+f(c)g(b')\]
        Agrupamos términos de la igualdad, tal que,
        \[0:\hspace{5mm}0=f(c)g(b')\]
        \[f(a):\hspace{5mm}g(b)=\lambda g(b')\]
        Por la propiedad \textit{(vi)}, como $b'\neq0$, entonces $c=0$. Además,
        \[g(b)=\lambda g(b')\Rightarrow g(b')=\lambda^{-1}g(b)\Rightarrow g(b')=g(\lambda^{-1}b)\Rightarrow b'=\lambda^{-1}b\]
        Luego,
        \[\begin{matrix}
            a'=\lambda a\\
            b'=\lambda^{-1}b
        \end{matrix}\hspace{4mm}\checkmark\]
        \begin{tabular}{c|}
             $\Leftarrow$ \\ \hline
        \end{tabular} 
        Sea $a'=\lambda a$ y $b'=\lambda^{-1}b$, entonces
        \[(\ptensor{a'}{b'})(f,g)=f(a')g(b')=f(\lambda a)g(\lambda^{-1}b)=\cancel{\lambda}\cancel{\lambda^{-1}}f(a)g(b)=(\ptensor{a}{b})(f,g)\checkmark\]
        \item Sean $V,W$ espacios vectoriales, tales que
        \[\begin{array}{ccl}
            \ptensor{V}{W} & \to & \ptensor{W}{V}  \\
            \ptensor{v}{w} & \mapsto & \ptensor{w}{v}
        \end{array}\]
        Si suponemos que $dimV=n$ y $dimW=m$, sabemos por tanto que $dim(V\otimes W)=n\cdot m$ y $dim(W\otimes V)=m\cdot n$, luego tienen la misma dimensión y por tanto, son isomorfos. $\checkmark$\\ \\
        También podemos hacerlo sin usar la proposición de que $dim(\ptensor{V}{W})=n\cdot m$. Es claro ver que la aplicación es inyectiva, pues no hay dos elementos con la misma imagen, ya que la imagen se forma al permutar los elementos. Luego, al ser inyectivo, tenemos que $dimKer=0$. Por el Primer Teorema de Isomorfía,
        \[dim(\ptensor{V}{W})=\cancelto{0}{dimKer}+dimIm=dimIm=dim(\ptensor{W}{V})\]
        Luego, como $\ptensor{V}{W}$ y $\ptensor{W}{V}$ tienen la misma dimensión, entonces son isomorfos.
    \end{enumerate}
\end{proof}

\subsection{Aplicaciones lineales} % Main chapter title
\label{cap1-sec1-subsec2} 

Veamos ahora las aplicaciones lineales.
\begin{definition}
    Sean $V$ y $V'$ dos espacios vectoriales sobre el mismo cuerpo $\mathbb{K}$.
    Se dice que en una aplicación $f:V\longrightarrow V'$ es una aplicación lineal, o también llamado homomorfismo de espacios vectoriales, si se verifica:
    \begin{enumerate}[label=(\roman*)]
        \item $f(x+y)=f(x)+f(y),\forall x,y\in V$
        \item $f(\lambda\cdot x)=\lambda\cdot f(x),\forall\lambda\in\mathbb{K},\forall x\in V$
    \end{enumerate}
    Diremos además que $f$ es un isomorfismo lineal si es biyectiva, que $f$ es un endomorfismo si $V=V'$ y que es un automorfismo si es un endomorfismo biyectivo. 
\end{definition}

\noindent Las aplicaciones lineales tienen asociados dos conjuntos cuyas características son de interés, a saber, el núcleo y la imagen.

\begin{definition}
    Sea $f:V\longrightarrow W$ definimos el núcleo o kernel de la aplicación $f$ como
    \[Kerf=\curlybraces{v\in V:f(v)=0}\]
    y la imagen como
    \[Imf=\curlybraces{w\in W:\exists v\in V/f(v)=w}.\]
\end{definition}

\noindent Veamos algunas propiedades básicas de ambos conjuntos.

\begin{proposition}
Sea $f:V\to V'$ una aplicación lineal, se tienen las siguientes propiedades:
\begin{enumerate}[label=(\roman*)]
    \item \label{prop1:item1} $\rm{Im}f$ es un subespacio de $V'$ y que $\rm{Ker}f$ es un subespacio de $V$.
    \item \label{prop1:item2}Si $W$ es un subespacio vectorial de $V$, entonces $f(W):=\curlybraces{f(w): w\in W}$ es un subespacio de $V'$.
    \item \label{prop1:item3}Si $W'$ es un subespacio de $V'$, entonces $f^{-1}(W'):=\curlybraces{v\in V: f(v)\in W'}$ es también un subespacio de $V$.
\end{enumerate}  
\end{proposition}
%\newpage
\begin{proof}
\begin{enumerate}[label=\ref{prop1:item1}]
    \item Por definición, como los elementos de la $\rm{Im}f$ son pertenecientes a $V'$, entonces la $\rm{Im}f$ es subespacio de $V'$. De igual forma ocurre con el $\rm{Ker}f$, pues sus elementos pertenecen a $V$ y por tanto, este es subespacio de $V$.
\end{enumerate}
\begin{enumerate}[label=\ref{prop1:item2}]
    \item Como $W$ es subespacio de $V$, tenemos que $w\in V$ también, por tanto, los $f(w)$ pertenecerán a $V'$, cosa que implica que $f(W)$ es subespacio de $V'$, pues los $f(w)$ de $f(W)$ pertenecen a $V'$.
\end{enumerate}
\begin{enumerate}[label=\ref{prop1:item3}]
    \item Por analogía a $\ref{prop1:item2}$ vemos que $f^{-1}(W)$ es subespacio de $V$.
\end{enumerate}
\end{proof}
\noindent Ahora veamos algunas propiedades esenciales de las aplicaciones lineaales.
\begin{proposition}
    Sea $f:V\longrightarrow V'$ una aplicación lineal,
    \begin{enumerate}[label=(\roman*)]
        \item \label{pro1:item1} entonces $f$ es inyectiva si y solo si $Kerf=\curlybraces{0}$.
        \item \label{pro1:item2} si $G$ es un conjunto generador de $V$, $<G>=V$, entonces $f(G)$ es conjunto generador
        de $Imf$, $<f(G)>=Imf$.
        \item \label{pro1:item3} si $S\subset V$ es un conjunto de vectores linealmente independientes, si $f$ es inyectiva, entonces $f(S)$ es linealmente independiente.
        \item \label{pro1:item4} $f$ es inyectiva $\Leftrightarrow$ conserva la independencia lineal.
   
        \item \label{pro1:item5} si $f$ es biyectiva y $B$ es una base de $V$, entonces $f(B)$ es base de $V'$.

        \item \label{pro1:item6} $f$ es sobreyectiva $\Leftrightarrow$ $Imf=V'$
    \end{enumerate}
\end{proposition}



\begin{proof}
\ref{pro1:item1} \begin{tabular}{c|}
                 $\Rightarrow$ \\ \hline
            \end{tabular}
            Suponiendo que $f$ es inyectiva, sabemos que su Kernel es,
            \[\rm{Ker}f=\curlybraces{v\in V:f(v)=0}\]
            pero como la inyectividad nos implica que la imagen debe provenir de un único vector de entrada, entonces este vector será $v=0$, y por tanto, $ker f=\curlybraces{0}$. $\checkmark$
\\     
            \begin{tabular}{c|}
                 $\Leftarrow$  \\ \hline
            \end{tabular}
            Suponiendo que $ker f=\curlybraces{0}$, esto nos quiere decir que únicamente el vector $v=0$ satisface $f(v)=0$, luego como un vector tiene una única imagen, decimos que $f$ es inyectiva. \qedh
\\ \\
\ref{pro1:item2}  Veamos que el conjunto $f(G)$ es sistema generador de la imagen, es decir,   \[<f(G)>=Imf\Leftrightarrow\forall y\in Imf,\exists\lambda^1,\dots,\lambda^n\in\mathbb{K},y_1,\dots,y_n\in f(G)\text{ tales que }y=\lambda^1y_1+\dots+\lambda^ny_n.\] Sabemos que $G$ es conjunto generador, luego sea $y\in Imf$. Entonces por definición se tiene que existe $x\in V$ tal que $f(x)=y$. Como $<G>=V$, existen $\lambda^1,\dots,\lambda^n\in\mathbb{K}$, $v_1,\dots,v_n\in G$ tales que \[ x= \sum_{i=1}^n \lambda^i v_i.\] Tenemos entonces que:  \[y=f(x)=f(\lambda^1v_1+\dots+\lambda^nv_n)=\lambda^1f(v_1)+\dots+\lambda^nf(v_n).\] Por lo tanto, $y$ es combinación lineal de elementos de $f(G)$, es decir, $<f(G)>=\mathrm{Im}f$. \qedh
\\ \\
\ref{pro1:item3}
            Sea $S$ un conjunto linealmente independiente en $V$. 
            Supongamos que $f$ es inyectiva, vamos a probar que $f(S)$ es linealmente independiente, es decir,
            \[\lambda^1y_1+\dots+\lambda^ny_n=0\Rightarrow\lambda^1=\lambda^2=\dots=\lambda^n=0\hspace{4mm} \forall y_1,\dots,y_n\in f(S),\hspace{2mm}
                \forall\lambda^1,\dots,\lambda^n\in\mathbb{K}\]
            Supongamos $\lambda^1y_1+\dots+\lambda^ny_n=0$.  Como $y_j\in f(S),\exists x_j\in S/f(x_j)=y_j$.
             \[\left.\begin{array}{r}
                \lambda_1f(x_1)+\dots+\lambda_nf(x_n)=0\\
                f(\lambda_1x_1+\dots+\lambda_nx_n)=0
            \end{array}\right\rbrace f(0)=0\Rightarrow\lambda_1x_1+\dots+\lambda_nx_n=0\Rightarrow f\text{ inyectiva}\Rightarrow\lambda_1=\dots=\lambda_n=0\]
                \\ \\
                \ref{pro1:item4} \begin{tabular}{c|}
                 $\Longrightarrow$ \\ \hline
            \end{tabular} 
            Trivial por (iii) $\checkmark$\\
            \begin{tabular}{c|}
                 $\Longleftarrow$ \\ \hline
            \end{tabular} 
            Por reducción al absurdo:\\
            Supongamos que existen $v_1,v_2\in V$ distintos, tales que $f(v_1)=f(v_2)\Leftrightarrow f(v_1)-f(v_2)=0\Leftrightarrow f(v_1-v_2)=0$. 
            Luego, $v=v_1-v_2\neq0$ verifica que $f(v)=0$, $\curlybraces{v}$ es un conjunto linealmente independiente, $f(\curlybraces{v})$ tendría que ser un conjunto l.i. por hipótesis, pero $f(\curlybraces{v})=\curlybraces{0}$ que no es un conjunto l.i. cosa absurda. \qedh
            \\ \\
            \ref{pro1:item5}  Sea una aplicación lineal biyectiva $f:V\to V'$
            y una base de $V$, $B=\curlybraces{v_1,\dots,v_n}$. Entones, si aplicamos
\[f(B)=\curlybraces{f(v_1),\dots,f(v_n)}=\curlybraces{v_1',\dots,v_n'}\]
            y entonces, estos $v_i'\in V'$ van a formar una base de $V'$, pues al ser $f$ biyectiva, los vectores serán linealmente independientes, pues los de $B$ lo son; y además, como tienen la misma dimensión que $V'$, pasan de ser conjunto generador a base. $\qedh$
            \\ \\
            \ref{pro1:item6} \begin{tabular}{c|}
                 $\Rightarrow$  \\ \hline
            \end{tabular}
            Suponiendo que $f$ es sobreyectiva, tendremos que para cada $y\in V'$, existe al menos un $x\in V$, tal que $f(x)=y$. Por consiguiente, cada elemento de $V'$ es la imagen de un elemento de $V$, es decir, $Imf=V'$. $\checkmark$\\
            \begin{tabular}{c|}
                 $\Leftarrow$  \\ \hline
            \end{tabular}
            Suponiendo que $imf=V'$, tenemos que todos los elementos de $V'$ son imagen de los elementos de $V$, siendo esta la propia definición de sobreyectividad, luego $f$ es sobreyectiva.
\end{proof}
\noindent Una vez visto estas propiedades, de $\ref{pro1:item5}$ podemos obtener un resultado interesante, que es la siguiente proposición.
\begin{proposition}
Sea $B=\curlybraces{v_1,\dots,v_n}$ una base de $V$, y sea $f:V\rightarrow V'$ una aplicación lineal. Se tiene entonces que $\curlybraces{f(v_1),\dots,f(v_n)}$ es un sistema generador de la imagen.
\end{proposition}
\begin{proof}
    Supongamos que $B$ es una base y que conocemos $f(v_j),\forall v_j\in B$. 
    Sea $v\in V$, escrito en coordenadas de la base como $v=\lambda^1v_1+\dots+\lambda^nv_n$, con $v_i\in B$, 
 y $\lambda^i\in\mathbb{K}$, entonces  $f(x)=f(\lambda^1v_1+\dots+\lambda^nv_n)=\lambda^1f(v_1)+\dots\lambda^nf(v_n)$, luego hemos puesto $f(x)$ en coordenadas de $\curlybraces{f(v_1,\dots,f(v_n)}$.
\end{proof}

\noindent Ahora vamos a ver un resultado bastante importante, el cuál nos permitirá representar aplicaciones lineales en matrices, denominadas \textbf{matrices asociadas a la aplicación $f$}. Además, este resultado es importante para Física, pues los físicos no solemos trabajar con aplicaciones, sino que trabajamos con sus matrices asociadas, pues se puede decir que "tienen" la misma información que las aplicaciones.

\begin{proposition}  
\label{prop1.4}
    Sean $(V,+,\cdot)$ y $(V',+,\cdot)$ $\mathbb{K}$-espacios vectoriales de dimensión finita con $dimV=n$ y $dimV'=m$. 
    Sea $f:V\longrightarrow V'$ una aplicación lineal, entonces dadas $\left\lbrace\begin{matrix}
        B=\curlybraces{v_1,\dots,v_n}\text{ base de }V\\
        B'=\curlybraces{v_1',\dots,v_n'}\text{ base de }V'
    \end{matrix}\right.$\\
    $f$ se representa en esas bases como una matriz en $\mathcal{M}_{m\times n}(\mathbb{K})$.
\end{proposition}

\begin{proof}
    Como $f$ es lineal, me basta con conocer $f(B)$, para ello, tenemos que conocer $f(v_1),f(v_2),\dots,f(v_n)$, teniendo:
    \[\begin{matrix}
        f(v_1) & = & a_{1}^1v_1'+a_{1}^2v_2'+\dots+a_{1}^mv_m', & a_{1}^i\in\mathbb{K}\\
        f(v_2) & = & a_{2}^1v_1'+a_{2}^2v_2'+\dots+a_{2}^mv_m', & a_{2}^i\in\mathbb{K}\\
        \vdots & & \vdots & \vdots\\
        f(v_n) & = & a_{n}^1v_1'+a_{n}^2v_2'+\dots+a_{n}^mv_m', & a_{n}^i\in\mathbb{K}
    \end{matrix}\]
    Sea $v\in V:v=\lambda^1v_1+\dots+\lambda^nv_n,\hspace{2mm}\lambda^i\in\mathbb{K}$, si le aplicamos $f$ tenemos,
    \[\begin{array}{rll}
        f(v) & = &\lambda^1f(v_1)+\dots+\lambda^nf(v_n) \\
         & = & \lambda^1(a_{1}^1v_1'+\dots+a_{1}^mv_m')+\lambda^2(a_{2}^1v_1'+\dots+a_{2}^mv_m')+\dots+\lambda^n(a_{n}^1v_1'+\dots+a_{n}^mv_m')\\
         & = & (a_{1}^1\lambda^1+a_{2}^1\lambda^2+\dots+a_{n}^1\lambda^n)v_1'+(a_{1}^2\lambda^1+\dots+a_{n}^2\lambda^n)v_2'+\dots+(a_{1}^m\lambda^1+\dots+a_{n}^m\lambda^n)v_m'
    \end{array}
        \]
   Luego,  $f(v)=\mu^1v_1'+\mu^2v_2'+\dots+\mu^mv_m'$, siendo $\mu^i=(a_{1}^i\lambda^1+\dots+a_{n}^i\lambda^n)$, luego, para construir la matriz $A$, ponemos las coordenadas de $v_1$ en la primera columna, las de $v_2$ en la segunda y así sucesivamente, tal que:
    \[\begin{pmatrix}
        \mu^1\\
        \mu^2\\
        \vdots\\
        \mu^m
    \end{pmatrix}=\begin{pmatrix}
        a_{1}^1 & a_{1}^1 & \dots & a_{1}^m\\
        a_{2}^1 & a_{2}^2 & \dots & a_{2}^m\\
        \vdots & \vdots & \ddots & \vdots\\
        a_{n}^1 & a_{n}^2 & \dots & a_{n}^m
    \end{pmatrix}\begin{pmatrix}
        \lambda^1\\
        \lambda^2\\
        \vdots\\
        \lambda^n
    \end{pmatrix}\Rightarrow \mu=A\cdot\lambda\]
\end{proof}

\noindent Vamos a introducir ahora el concepto de \textbf{rango} de una aplicación lineal, que puede extenderse al rango de su matriz asociada.

\begin{definition}
    Se llama rango de una aplicación lineal (matriz) a la dimensión de su imagen y se denota por $rg()$.
\end{definition}

\noindent Como un mismo espacio vectorial puede estar generado por varias bases, es lógico pensar que debe haber una relación entre estas bases o al menos una forma de cambiar de una base a otra, lo que se conoce como \textbf{cambio de base}. Esto es posible y una forma sencilla de hacerlo es mediante las matrices asociadas.

\begin{proposition}
    -Sean $V$ y $V'$ dos espacios vectoriales en $\mathbb{K}$, sea $f:V\longrightarrow V'$ lineal.\\
    -Sea $B_1=\curlybraces{v_1,\dots,v_n}$ base de $V$, $B_1'=\curlybraces{v_1',\dots,v_m'}$ base de $V'$.\\
    -Sea $A\in\mathcal{M}_{m\times n}(\mathbb{K})$ la matriz que representa a $f$ en $B_1,B_1'$.\\
    -Sea $B_2=\curlybraces{u_1,\dots,u_n}$ base de $V$, $B_2'=\curlybraces{u_1',\dots,u_m'}$ base de $V'$.\\
    -Sea $\tilde{A}\in\mathcal{M}_{m\times n}(\mathbb{K})$ la matriz que representa a $f$ en $B_2,B_2'$.\\
    -Sea $P$ la matriz de cambio de base de $B_1$ en $B_2$.\\
    -Sea $Q$ la matriz de cambio de base de $B_1'$ en $B_2'$.\\
    Entonces $\tilde{A}=Q^{-1}\cdot A\cdot P$.
\end{proposition}
\begin{proof}
    Sea $f:V\to V'$ una aplicación lineal con $n=dim V$ y $m=dim V'$. Si $A$ y $\tilde{A}$ son las matrices asociadas a $f$ respecto de distintas bases, entonces
    \[rg(A)=dim(Imf)=rg(\tilde{A})\]
    Luego $A$ y $\Tilde{A}$ tienen igual rango, y por tanto, son matrices equivalentes. Concretemos más esta situación:\\
    Sean $B_1$ y $B_2$ bases de $V$ con cambio de base de $B_1$ a $B_2$ dado por $X_1=PX_2$ y sean $B_1'$ y $B_2'$ bases de $V'$, con cambio de $B_1'$ a $B_2'$ dado por $Y_1=QY_2$.\\
    Consideremos la matriz asociada a $f$ respecto de $B_1$ y $B_1'$, $A\in\mathcal{M}_{m\times n}(\mathcal{K})$, tal que $A=\mathcal{M}_{B_1,B_1'}(f)$ y la ecuación matricial
    \[Y_1=AX_1\]
    De igual forma, sea $\tilde{A}\in\mathcal{M}_{m\times n}(\mathbb{K})$ la matriz asociada a $f$ respecto de $B_2$ y $B_2'$, tal que $\tilde{A}=\mathcal{M}_{B_2,B_2'}(\mathbb{K})$ y la ecuación matricial de $f$ respecto de estas bases,
    \[Y_2=\tilde{A}X_2\]
    Gráficamente,
    \[\begin{matrix}
    & V & \to & V' & \\
            & & A & &\\
            & B_1 & \longrightarrow & B_1' & \\
            P & \uparrow & & \uparrow & Q\\
             & & \tilde{A} & & \\
             & B_2 & \longrightarrow & B_2' &
        \end{matrix}\]
        Entonces,
        \[Y_2=\left\lbrace \begin{array}{l}
        \tilde{A}X_2\\    Q^{-1}Y_1=Q^{-1}AX_1=Q^{-1}APX_2\end{array}\right.\]
        y en consecuente,
        \[\tilde{A}=Q^{-1}AP\]
        O bien,
        \[X_2=\left\lbrace\begin{array}{l}
            \tilde{A}^{-1}Y_2\\
            P^{-1}X_1=P^{-1}A^{-1}Y_1=P^{-1}A^{-1}QY_2
        \end{array}\right.\]
        y en consecuente,
        \[\tilde{A}^{-1}=P^{-1}A^{-1}Q\]
\end{proof}

Ahora vamos a enunciar el \textbf{Primer Teorema de Isomorfía}, del que obtendremos un Corolario muy importante a la hora de trabajar con aplicaciones lineales. Este teorema no se va a demostrar (si se quiere ver la prueba consultar  \cite[Chapter 6, Theorem 6.5, Page 77]{IntroducciónTeoríaDeGrupos}).

\begin{theorem}[Primer teorema de isomorfismo de Noether]
    Sea $f:V\longrightarrow V'$ una aplicación lineal, entonces:
    \begin{enumerate}[label=(\roman*)]
        \item Existe una aplicación lineal sobreyectiva $\pi:V\longrightarrow V/Kerf$
        \item Existe un isomorfismo $\bar{f}:V/Kerf\longrightarrow Imf$
        \item Existe una aplicación lineal inyectiva $i:Imf\longrightarrow V'$, tales que $f=i\circ\bar{f}\circ\pi$, tal que
        \[\begin{matrix}
            & & f & \\
            & V & \longrightarrow & V' & \\
            \pi & \downarrow & & \uparrow & i\\
             & & \bar{f} & & \\
             & V/Kerf & \longrightarrow & Imf &
        \end{matrix}\]
    \end{enumerate}
   
\end{theorem}
\begin{corollary}
     Si además $V$ es finitamente generado,
    \[dimV=dim(Kerf)+dim(Imf)\]
\end{corollary}
\subsection{Contrtacción de tensores} % Main chapter title
\label{cap1-sec2-subsec3} 

 Una vez que hemos visto cómo subir y bajar índices, podemos definir una operación denominada \textbf{contracción} de tensores, la cuál encoge un tensor $(r,s)$ a uno $(r-1,s-1)$. La definición general se obtiene a partir del siguiente caso especial.

\begin{lemma}
    Hay una única aplicación lineal
    $C:\Omega_1^1\to\mathbb{R}$
    llamada \textit{contracción (1,1)}, tal que
    \[\begin{array}{rlll}
        C: & \Omega_1^1 (V)& \to & \mathbb{R} \\
         & \ptensor{v}{f} & \mapsto & C(\ptensor{v}{f})=f(v)
    \end{array}\]
    para todo $v\in V$ y $f\in V^*$.
\end{lemma}
\begin{proof} (Esta demostración usa el concepto de matrices de cambio de base, por lo que se recomienda ver la sección \ref{CambioBasesTensores(1,1)})\\ 
    Tomando $B=\curlybraces{v^1,v^2,\dots,v^n}$ base de $V$ y $B^*= \curlybraces{f_1,f_2,\dots,f_n}$ base de $V^*$, podemos escribir un tensor de tipo $(1,1)$ como
    \[A\equiv\sum A^i_j\ptensor{f_i}{v^j}\]
    Como $C(\ptensor{f_i}{v^j})=f_i(v^j)=\delta^j_i$, por la condición de base dual, no nos queda otra opción, más que definir,
    \[C(A)=\sum A_i^i=\sum A(f_i,v^i)\]
    Entonces, $C$ tiene las propiedades requeridas en las bases $B,B^*$. Luego, para obtener la función general requerida es suficiente con mostrar que esta definición es independiente de la elección del sistema de coordenadas. Así, tomando una nueva base de $V$, $B'=\curlybraces{w^1,w^2,\dots,w^n}$ y otra de $V^*$, $B^{*'}=\curlybraces{q_1,q_2,\dots,q_n}$, tenemos
    \[\begin{array}{rrl}
        C(A) & = & \sum\limits_mA(q_m,w^m)= \sum\limits_mA\left(\sum\limits_i a_i^mf_m,\sum_jb_m^jv^m\right)\\
         & = & \sum\limits_{i,j,m}a_i^mb_m^jA(f_i,v^j)=\sum\limits_{i,j}\delta^j_iA(f_i,v^j)\\
         & = & \sum\limits_iA(f_i,v^j)
    \end{array}\]
\end{proof}
\noindent Para extender las contracciones $(1,1)$, $C$, a un tensor de un tipo mayor, el esquema es especificar una componente covariante y otra contravariante y aplicar $C$ a estos.\\

\noindent Suponemos un tensor $A\in\Omega_r^s(V)$ y $1\leq r$ y $1\leq j \leq s$. Fijamos las formas $p_1,p_2,\dots,p_{r-1}$ y los vectores $u_1,u_2,\dots ,u_{s-1}$. Entonces la función
\[(p,u) \to A(p_1, \dots, \underbrace{p_{i}}_{\mathclap{i\text{-ésima componente contravariante}}}, \dots, p_{r-1}, u^{1}, \dots, \overbrace{u^{j}}^{\mathclap{j\text{-ésima componente covariante}}}, \ldots, u^{s-1})\]
es un tensor $(1,1)$ que puede escribirse como

\[A(p_1,\dots,\cdot,\dots,p_{r-1},u^1,\dots,\cdot,\dots,u^{s-1})\]
Aplicando la contracción $(1,1)$ a este tensor, produce una función de valor real denotada por

\[\left(C_j^iA\right)\left(p_1,\dots,p_{r-1},u^1,\dots,u^{s-1}\right)\]
Siendo $C_j^iA$ una función multilineal. Por tanto, esto es un tensor de tipo $(r-1,s-1)$ llamado \textit{la contracción de }$A$\textit{ sobre }$i,j$.

%\begin{definition}
 %   La contracción de un tensor $A$ de tipo $(r,s)$ con respecto al índice contravariante $p$ $(p\leq r)$ y al índice covariante $q$ $(q\leq s)$ es el tensor de tipo $(r-1,s-1)$, teniendo las componentes,
  %  \[B^{i_1\dots i_{r-1}}_{j_1\dots j_{s-1}}=A^{i_1\dots i_{p-1}ki_p\dots i_{r-1}}_{j_1\dots j_{q-1}kj_q\dots j_{s-1}}\]
%\end{definition}

\begin{note}
    Para poder contraer tensores, debemos tener superíndices y subíndices, así, podemos usar primero la métrica para subir o bajar índices y luego aplicar la contracción.
   \end{note} 
\begin{example}
        Si tenemos un tensor de tipo (0,2),
    $S\equiv S_{\alpha\beta}$, podemos hacer,
    \[\begin{array}{rllll}
        S_{\alpha\beta} & \to & g^{\gamma\alpha}S_{\alpha\beta}=S^{\alpha}_{\beta} & \to & C^1_1S^{\gamma}_{\beta}=S^{\beta}_{\beta} \\
        \text{Tensor (0,2)} & \to & \text{Tensor (1,1)} & \to &\text{Escalar}
    \end{array}\]
    cosa que se puede simplificar simplemente usando,
    \[S\equiv S_{\alpha\beta}\to g^{\beta\alpha}S_{\alpha\beta}=S^{\beta}_{\beta}\]
    es decir, podemos contraer tensores con la propia métrica.
\end{example}

\begin{example}
    Si
    \[U^j_i=T^{kj}_{ik}\]
    entonces
    \[U'^{j'}_{i'}=T'^{k'j'}_{i'k'}=S^i_{i'}S^l_{k'}R^{k'}_kR^{j'}_jT^{kj}_{il}=S^i_{i'}\delta^l_kR^{j'}_jT^{kj}_{il}=S^i_{i'}R^{j'}_jT^{kj}_{ik}=S^i_{i'}R^{j'}_jU^j_i\]
    donde hemos utilizado $S^l_{k'}R^{k'}_k=\delta^l_k$. Vemos que se transforma como un tensor (1,1).\\

\noindent    Así, dado un tensor $T^{ij}_{kl}$ de tipo (2,2), serán posible las 4 contracciones
    \[T^{kj}_{ki},\hspace{3mm}T^{jk}_{ik},\hspace{3mm}T^{kj}_{ik},\hspace{3mm}T^{jk}_{ki}\]
    que originan 4 tensores de tipo (1,1). Por otro lado, las dos posibles contracciones dobles que dan lugar a un escalar (tensor de tipo (0,0)) son
    \[T^{kj}_{kj},\hspace{3mm}T^{jk}_{kj}\]
\end{example}
\begin{note}
    El producto escalar $(\mathbb{R}^n,g_{ij})$ también se puede contraer. Pues $g_{ij}$ es un tensor de tipo (0,2), al cual le podemos aplicar una contracción 1,1, pero primero lo pasamos a un tensor de tipo (1,1), variando sus índices, tal que
    \[C^1_1\left(g^{ki}g_{ij}\right)=C^1_1(g^k_j)=g^j_j=n\]
    donde sabemos que vale $n$, pues al ser un espacio de dimensión $n$, la matriz asociada a $g$ será $G\in\mathcal{M}_{n\times n}$ y por tanto, la traza será la suma de $n$-elementos. Sabemos que estos elementos son el 1, porque la traza es invariante frente a los cambios de base (cosa que veremos más adelante), por tanto, si cogemos el producto escalar usual en la base usual, la matriz asociada es la matriz de Gram, cuyos elementos son todos nulos, salvo la diagonal que está formada por 1.
\end{note}
\subsection{Notación de Einstein} % Main chapter title
\label{cap1-sec1-subsec4} 

La notación de Einstein va a servir para facilitarnos la escritura, pues cada vez que tengamos un vector o una forma escrita como combinación lineal, vamos a poder redefinirlos como
\[w=\sum\limits_{i=1}^n\lambda^iv_i\equiv\lambda^iv_i\]
esto para un vector. Para una forma, tendremos
\[p=\sum\limits_{i=1}^n\mu_if^i\equiv\mu_i f^i\]
Además, para simplificar aún más la notación y dejarnos de tantas letras, vamos a identificar los escalares de $w$ como 
\[\lambda^i\equiv w^i\]
Así, los vectores como combinación lineal de otros vectores, los escribiremos como
\[w=w^iv_i\]
Y para las formas, haremos la identificación
\[\mu_i\equiv p_i\]
Así, las formas como combinación lineal de otras formas se escribirán como
\[p=p_if^i\]
\begin{example}
Un ejemplo de ello, será a la hora de identificar un vector en los términos de su base, pues suponiendo un $V$ espacio vectorial sobre el cuerpo $\mathbb{K}$ y cuya base sea $B=\curlybraces{v_1,v_2,\dots,v_n}$, tomando un $u\in V$, lo denotaremos como,
\[u=u^iv_i\]
\end{example}
\begin{example}
    Otro ejemplo será a la hora de identificar una forma en términos de la base dual, pues suponiendo un $V^*$ espacio dual de $V$, cuya base dual es $B^*=\curlybraces{f^1,f^2,\dots,f^n}$, tomando un $q\in V^*$, lo denotaremos como,
    \[q=q_if^i\]
\end{example}
\begin{note}
    En un artículo físico, se identifica directamente el escalar con el vector, es decir,
    \[w^i\equiv w\]
    pues se presupone que existe una base donde $w$ está bien definido. Así, los físicos usaremos de forma indistinguible los vectores y sus componentes respecto de una base fijada.
\end{note}
\subsection{Invariantes} % Main chapter title
\label{cap1-sec2-subsec5} 

Dado que los tensores suelen describirse en términos respecto de ciertas bases, cuando estos términos no dependen de la base empleada, los tensores se llamarán \textbf{invariantes}. O en otras palabras, los tensores que no se transforman frente a un cambio de base, serán los que llamaremos \textbf{invariantes}.\\ \\
Vamos a intentar ilustrar este concepto definiendo un tensor invariante de tipo (1,1), denominado \textit{traza}, que es un invariante conocido de las matrices. Si tenemos un tensor $A=A_j^i\ptensor{e_i}{f^j}$ que definimos como
\[\text{traza de }A=\rm{tr}A=A^i_i\]
siendo la suma de los elementos de la diagonal principal de la matriz $(A^i_j)$. No es a priori evidente que hayamos definido algo que depende únicamente de $A$, ya que los $A_j^i$ dependen no solo de $A$ sino también de la base $\curlybraces{e_i}$. Para mostrar que $\rm{tr} A$ es un número determinado enteramente por $A$ mismo y no por los $e_i$ también, debemos demostrar la invariancia; es decir, si $A$ se expresa en términos de otra base ${\tilde{e}_i}$, entonces la fórmula correspondiente en los nuevos componentes da el mismo número que antes. Así, escribimos $A=\tilde{A}^i_j\ptensor{\tilde{e}_i}{\tilde{f}^j}=A^i_j\ptensor{e_i}{f^j}$ y veremos que $A^i_j=\tilde{A}^i_j$. Usando la misma notación de cambios de base que hemos visto en el apartado anterior, tenemos la ley de transformación siguiente,
\[\tilde{A}^n_m=A^i_ja_m^jb_i^n\]
de lo cual se obtiene
\[\tilde{A}^i_i=A^p_ja^j_ib^i_p=A^p_j\delta^j_p=A^i_i\]
Queda demostrado. Luego, tenemos la proposición,
\begin{proposition}
    La traza de un tensor de tipo (1,1) es un invariante.
\end{proposition}
Para ver que no todas las expresiones en términos de las componentes de un tensor necesariamente serán un invariante, veamos el siguiente ejemplo. 
\begin{example}
    Supongamos $d=2$ y $A=\ptensor{e_1}{e_1}+\ptensor{e_1}{e_2}$, un tensor de tipo (0,2). La expresión de $A_{ii}$ en este caso será $A_{11}+A_{22}$=1+0=1. Ahora consideramos una nueva base dada por $e_1=\tilde{e}_1+\tilde{e}_2$ y $e_2=\tilde{e}_2$, entonces
    \[\begin{array}{rrl}
        A & = & (\tilde{e}_1+\tilde{e}_2)\otimes(\tilde{e}_1+\tilde{e}_2)+(\tilde{e}_1+\tilde{e}_2)\otimes\tilde{e}_2 \\
         & = & \tilde{e}_1\otimes\tilde{e}_1+2\tilde{e}_1\otimes\tilde{e}_2+\tilde{e}_2\otimes\tilde{e}_1+2\tilde{e}_2\otimes\tilde{e}_2
    \end{array}\]
    de la cuál se obtiene que $\tilde{A}_{ii}=\tilde{A}_{11}+\tilde{A}_{22}=1+2=3$. Por tanto es diferente a la base primera, luego no es un invariante.
\end{example}
\subsubsection*{Nota Final}
    Finalmente diremos que un tensor es todo aquel objeto matemático que satisfaga los cambios de base, o en otras palabras: \textit{Un tensor es todo objeto matemático que transforma como un tensor}.

%------------------------------------------------------------------------------

    \chapter{Teoría de la Relatividad Especial} %
\label{Capitulo2} %
\noindent\textit{“Hay una fuerza motriz más poderosa que el vapor, la electricidad y la energía atómica: la voluntad”.}\\(A. Einstein)
\newpage
\lhead{\emph{Teoría de la Relatividad Especial}}
%-------------------------------------------------------------------------------
%SECCION 1
\section{Repaso histórico} % Main chapter title
\label{cap2-sec1} 
%------------------------------------------------------------------------------
	La física clásica, del siglo XIX, era una física bien asentada. La cuál explica la mecánica con el libro de Sir Isaac Newton titulado \textit{Philosophiae Naturalis Principia Mathematica} y el electromagnetismo se explica con el libro de Maxwell titulado \textit{Electricity and Magnetism}.\\
 En 1887, Michelson y Morley iniciaron una revolución en la física con un experimento para medir la velocidad de la luz. El experimento consistía en medir la velocidad de la luz de un rayo paralelo al eje de rotación de la Tierra y de otro rayo perpendicular a este, esperándose obtener resultados diferentes. En cambio, se observó que ambos rayos iban exactamente igual, cosa que no tenía sentido en la época., por tanto, determinaron que la velocidad de la luz no era instantánea, sino que debía ser finita, y llegaron a un resultado de ésta bastante próximo al valor actual de la velocidad de la luz.
 \subsection{Relatividad Galileana}
 El Principio de Relatividad de Galileo establece que,
 \begin{center}
 \textit{''Es imposible determinar a base de experimentos (mecánicos) si un sistema de referencia está en reposo o en movimiento uniforme y rectilíneo''.}
 \end{center}
 Esto se derivó de que en la Relatividad Galileana hay un espacio absoluto en el que las leyes de Newton son ciertas. Definiremos un \textit{sistema de referencia inercial} (SRI) como aquel sistema referencia que se mueve a velocidad constante respecto al espacio absoluto. Además, todos los sistemas de referencia inerciales comparten un tiempo absoluto. Pero con la definición de SRI, el Principio de Relatividad se debe reformular con este concepto, así tenemos el Principio de Relatividad en formulación de equivalencia, que dice que
 \begin{center}
 \textit{''Todos los sistemas inerciales son equivalentes, es decir, todos los observadores inerciales ven la misma física''.}
 \end{center}
 \textbf{Leyes de Newton}\\ \\
 La Ley de Newton por excelencia es $\vec{F}=m\vec{a}=-\nabla V(\vec{r}-\vec{r}_0)$, donde $V$ es la función potencial. Esta ley (y las demás) transforman bien bajo el grupo de transformaciones de Galileo, que son:
 \begin{enumerate}
     \item \textbf{Traslaciones temporales:}
     \[t\to t'=t+t_0\]
     \item \textbf{Traslaciones espaciales:}
     \[\vec{r}\to\vec{r}'=\vec{r}+\vec{r}_i+\vec{v}t\]
     donde $\vec{v}$ es la velocidad relativa de un SRI con respecto al otro, y $\vec{r}_i$ es el vector de posición entre los orígenes de ambos SRI al inicio.
     \item \textbf{Rotaciones espaciales:}
     \[\vec{a}'=R(\theta)\vec{a}\]
     donde $R(\theta)$ es la matriz de rotación.
\end{enumerate}
Se puede ver que las Leyes de Newton no son covariantes, pero sí transforman bien, pues la física se mantiene, esto quiere decir que \textit{las Leyes de Newton de la física transforman de forma covariante}.\\ \\
El grupo de transformaciones de Galileo son simetrías que dan lugar a cantidades conservadas. Por tanto, si tenemos un Lagrangiano que sea invariante bajo traslaciones temporales, tendremos que el sistema conserva energía; si es invariante bajo traslaciones espaciales, conserva momento lineal; y si es invariante bajo rotaciones espaciales; conserva momento angular.\\ \\
El grupo de transformaciones de Galileo NO deja invariante las ecuaciones de Maxwell, que son
\[(i)\hspace{2mm}\nabla\cdot\vec{E}=\rho/\epsilon_0;\hspace{5mm}(iii)\hspace{2mm}\nabla\cdot\vec{B}=0\]
\[(ii)\hspace{2mm}\nabla\times\vec{B}=\partial_t\vec{E}/c^2+\mu_0\vec{J};\hspace{5mm}(iv)\hspace{2mm}\nabla\times\vec{E}=-\partial_t\vec{B}\]
Si $\rho=0$ y $\vec{J}=0$, es decir, estamos en vacío, podemos combinar las ecuaciones de Maxwell en una sola ecuación de ondas que se propaga a velocidad $c=299792,458$ m/s, resultado muy próximo al valor obtenido por Michelson y Morley, que además es independiente del sistema de referencia.
\subsection{Transformaciones de Lorentz}

Las transformaciones de Lorentz hacen que las ecuaciones de Maxwell transformen bien (sean covariantes). Estas transformaciones son:
\[\begin{array}{rcrc}
    (i) & t'=\gamma\left(t-\frac{v}{c^2}x\right); & (iii) & y'=y \\
    (ii) & x'=\gamma\left(x-vt\right); & (iv) & z'=z
\end{array}\]
donde $v$ es la velocidad relativa entre SRI (que suponemos que se mueven en el eje $X$), y $\gamma=\frac{1}{\sqrt{1-\frac{v^2}{c^2}}}$.\\ \\
Como estas transformaciones hacen que las leyes de Maxwell sean covariantes, diremos que las transformaciones de Lorentz sean más fundamentales que las transformaciones de Galileo.\\ \\
Además, vemos que por la transformación $(i)$ el tiempo ya \textbf{no es absoluto}, sino que depende del SRI, por lo que diremos que el tiempo es \textbf{relativo}.
%SECCION 2
\section{Álgebra de Tensores} % Main chapter title
\label{cap1-sec2} 
Llegamos a lo groso del capítulo, el \textbf{Álgebra de Tensores}. En este apartado vamos a ver qué es un tensor de forma matemática y cómo trabajar con ellos. También se mencionará cómo trabajamos los físicos con los tensores.
%------------------------------------------------------------------------------

\subsection{Producto tensorial: caso de dos términos} % Main chapter title
\label{cap1-sec2-subsec1} 
Vamos a ver qué es el \textbf{producto tensorial} y cómo los tensores se definen a partir de este.
\begin{proposition}
    Sea $V$ un $\mathbb{K}$-espacio vectorial, $\scalar{\cdot}{\cdot}$ el producto escalar euclídeo y $B=\curlybraces{v_1,\dots,v_n}$ base de $V$, 
    \[\begin{array}{cccl}
        f_v: & V & \to & V^*\\
         & v & \mapsto & f_v(v)=\scalar{v}{\cdot}
    \end{array}\]
     $f_v$ es una aplicación lineal, concretamente es un isomorfismo.
\end{proposition}
\begin{proof}
    Vemos que $f_v$ es aplicación lineal,
    \[f_v(w_1+w_2)=\scalar{v}{w_1+w_2}=\scalar{v}{w_1}+\scalar{v}{w_2}=f_v(w_1)+f_v(w_2)\checkmark \]
    \[f_v(\lambda\cdot w)=\scalar{v}{\lambda\cdot w}=\lambda\scalar{v}{w}=\lambda f_v(w)\checkmark\]
    para $\forall\lambda\in\mathbb{K}$ y $\forall w_1,w_2,w\in V$. Luego, es aplicación lineal.\\ \\
    Veamos que es isomorfo demostrando que es biyectivo, pues ya hemos visto que es aplicación lineal.\\
    Sabemos que $ker\curlybraces{f_v}=\curlybraces{0}\Leftrightarrow f_v$ es inyectiva. Luego, vemos si $ker\curlybraces{f_v}=\curlybraces{0}$:
    \[ker\curlybraces{f_v}=\curlybraces{w\in V,f_v(w)=0}=\curlybraces{w\in V;\scalar{v}{w}=0\Leftrightarrow w=0}\]
    Por tanto, $ker\curlybraces{f_v}=\curlybraces{0}$ y así, $f_v$ es inyectiva. $\checkmark$\\ \\
    Usando el Primer Teorema de isomorfía, tenemos que $dim(V)=\cancelto{0}{dim(ker\curlybraces{f_v})}+dim(Im f_v)$, pero como la $dim B=dim B^*$, siendo $B$ base de $V$ y $B^*$ base de $V^*$, entonces $dimV=dimV^*$, y por tanto, $dimV=dimImf_v=dimV^*$, luego $Imf_v$ es $V^*$ y por tanto, $f_v$ es sobreyectiva. $\checkmark$\\
    Luego, $f_v$ es un isomorfismo.
\end{proof}
\noindent Veamos cómo se define el producto tensorial y sus propiedades.
\begin{definition}
    Sea $V$ un $\mathbb{K}$-espacio vectorial, $V^*$ el dual de $V$, y $g^1,g^2\in V^*$ aplicaciones lineales, tal que $g^1:V\to\mathbb{K}$ y $g^2:V\to\mathbb{K}$. Así, definimos el producto tensorial como,
    \begin{enumerate}[label=(\roman*)]
        \item Producto tensorial entre dos formas $g^1,g^2\in V^*$,
        \[\begin{array}{cccl}
            \ptensor{g^1}{g^2}: & V\times V & \to & \mathbb{K}\\
            & (v,w) & \mapsto & g^1(v)g^2(w)
        \end{array}\]
        \item Producto tensorial entre dos vectores $v_1,v_2\in V$,
        \[\begin{array}{cccl}
            \ptensor{v_1}{v_2}: & V^*\times V^* & \to & \mathbb{K}\\
             & (f,g) & \mapsto & f(v_1)g(v_2)
        \end{array}\]
        \item Producto tensorial de una forma y un vector $v_1\in V$, $f^1\in V^*$,
        \[\begin{array}{cccl}
            \ptensor{v_1}{f^1}: & V^*\times V & \to & \mathbb{K}\\
             & (g,w) & \mapsto & g(v_1)f^1(w)
        \end{array}\]
    \end{enumerate}
\end{definition}

\begin{proposition}
    Los productos tensoriales definidos anteriormente son formas bilineales.
\end{proposition}
\begin{proof} 
Usando $\forall v_1,v_2,u_1,u_2,v,w,u\in V$, $\forall f^1,f^2,g,p,q\in V^*$ y $\forall \lambda\in\mathbb{K}$,
    \begin{enumerate}[label=(\roman*)]
        \item \[\begin{array}{cccl}
            \ptensor{f^1}{f^2}: & V\times V & \to & \mathbb{K}\\
            & (v,w) & \mapsto & f^1(v)f^2(w)
        \end{array}\]
        siendo $f^1,f^2\in V^*$. Veamos que es forma bilineal,
        \[\begin{array}{lrl} \text{\textbullet)} &(\ptensor{f^1}{f^2})(u_1+u_2,v)=&f^1(u_1+u_2)f^2(v)=\brackets{f^1(u_1)+f^1(u_2)}f^2(v)\\ &=&f^1(u_1)f^2(v)+f^1(u_2)f^2(v)=(\ptensor{f^1}{f^2})(u_1,v)+(\ptensor{f^1}{f^2})(u_2,v),\checkmark\\  \text{\textbullet)} &(\ptensor{f^1}{f^2})(v,u_1+u_2)  =&f^1(v)f^2(u_1+u_2)=f^1(v)\brackets{f^2(u_1)+f^2(u_2)}\\ &=&f^1(v)f^2(u_1)+f^1(v)f^2(u_2)=(\ptensor{f^1}{f^2})(v,u_1)+(\ptensor{f^1}{f^2})(v,u_2)\checkmark\\
             \text{\textbullet)} & (\ptensor{f^1}{f^2})(\lambda v,u) =& f^1(\lambda v)f^2(u)=\lambda f^1(v)f^2(u)=\lambda(\ptensor{f^1}{f^2})(v,u)\checkmark\\
        \text{\textbullet)}&(\ptensor{f^1}{f^2})(u,\lambda v)=&f^1(u)f^2(\lambda v)=\lambda f^1(u)f^2(v)=\lambda(\ptensor{f^1}{f^2})(u,v)\checkmark
          \end{array}\]
        Luego, $\ptensor{f^1}{f^2}$ es una forma bilineal. $\qedh $
        \item \[\begin{array}{cccl}
            \ptensor{v_1}{v_2}: & V^*\times V^* & \to & \mathbb{K}\\
             & (f,g) & \mapsto & f(v_1)g(v_2)
        \end{array}\]
         \[\begin{array}{lrl}
         \text{\textbullet)}&(\ptensor{v_1}{v_2})(f^1+f^2,g)=&(f^1+f^2)(v_1)g(v_2)=\brackets{f^1(v_1)+f^2(v_1)}g(v_2)\\
         &=&f^1(v_1)g(v_2)+f^2(v_1)g(v_2)=(\ptensor{v_1}{v_2})(f^1,g)+(\ptensor{v_1}{v_2})(f^2,g)\checkmark\\
         \text{\textbullet)}&(\ptensor{v_1}{v_2})(g,f^1+f^2)=&g(v_1)(f^1+f^2)(v_2)g=g(v_1)\brackets{f^1(v_2)+f^2(v_2)}\\
         &=&g(v_1)f^1(v_2)+g(v_1)f^2(v_2)=(\ptensor{v_1}{v_2})(g,f^1)+(\ptensor{v_1}{v_2})(g,f^2)\checkmark\\
         \text{\textbullet)}&(\ptensor{v_1}{v_2})(\lambda f,g)=&(\lambda f)(v_1)g(v_2)=\lambda f(v_1)g(v_2)=\lambda(\ptensor{v_1}{v_2})(f,g)\checkmark\\
         \text{\textbullet)}&(\ptensor{v_1}{v_2})(g,\lambda f)=&g(v_1)(\lambda f)(v_2)=\lambda g(v_1)f(v_2)=\lambda(\ptensor{v_1}{v_2})(g,f)\checkmark
         \end{array}\]
        Luego, $\ptensor{v_1}{v_2}$ es una forma bilineal. $\qedh $
        \item \[\begin{array}{cccl}
            \ptensor{v_1}{f^1}: & V^*\times V & \to & \mathbb{K}\\
             & (g,w) & \mapsto & g(v_1)f(w)
        \end{array}\]
        \[\begin{array}{lrl}
        \text{\textbullet)}&(\ptensor{v_1}{f^1})(p+q,w)=&(p+q)(v_1)f^1(w)=\brackets{p(v_1)+q(v_1)}f^1(w)=\\
        &=&p(v_1)f^1(w)+q(v_1)f^1(w)=(\ptensor{v_1}{f^1})(p,w)+(\ptensor{v_1}{f^1})(q,w)\checkmark\\
        \text{\textbullet)}&(\ptensor{v_1}{f^1})(g,u+w)=&g(v_1)f^1(u+w)=g(v_1)\brackets{f^1(u)+f^1(w)}=\\
        &=&g(v_1)f^1(u)+g(v_1)f^1(w)=(\ptensor{v_1}{f^1})(g,u)+(\ptensor{v_1}{f^1})(g,w)\checkmark\\
        \text{\textbullet)}&(\ptensor{v_1}{f^1})(\lambda g,w)=&(\lambda g)(v_1)f^1(w)=\lambda g(v_1)f^1(w)=\lambda(\ptensor{v_1}{f^1})(g,w)\checkmark\\
        \text{\textbullet)}&(\ptensor{v_1}{f^1})(g,\lambda w)=&g(v_1)f^1(\lambda w)=\lambda g(v_1)f^1(w)=\lambda(\ptensor{v_1}{f^1})(g,w)\checkmark
        \end{array}\]
            Luego, $\ptensor{v_1}{f^1}$ es una forma bilineal. \qedhere
    \end{enumerate}
\end{proof}
\noindent El producto tensorial no se da solo entre elementos de los espacios vectoriales o duales, sino que también se puede dar entre espacios, siendo el nuevo espacio generado un \textbf{espacio vectorial}.
\begin{proposition}
    El espacio $\ptensor{V}{V}$ tiene estructura de espacio vectorial.
\end{proposition}
\begin{proof}
    \begin{enumerate}
        \item Vemos que $(\ptensor{V}{V},+)$ es grupo abeliano:
        \begin{enumerate}[label=(\roman*)]
            \item Vemos si la operación $+$ es cerrada:
            \\
            $\forall v,w,z\in V$ con $\ptensor{v}{w},\ptensor{v}{z},\ptensor{w}{z}\in\ptensor{V}{V}$, tenemos que ver si $\ptensor{(v+w)}{z}\in\ptensor{V}{V}$. Sabemos que,
            \[\begin{array}{cccl}
                \ptensor{v}{w}: & \pcart{V^*}{V^*} & \to &\mathbb{R}  \\
                 & (f,g) & \mapsto & f(v)g(w)
            \end{array}\]
            luego,
            \[\begin{array}{cccl}
                \ptensor{(v+w)}{z}: & \pcart{V^*}{V^*} & \to &\mathbb{R}  \\
                 & (f,p) & \mapsto & f(v+w)p(z)
            \end{array}\]
            Entonces,
            \[(\ptensor{(v+w)}{z})(g,p)=f(v+w)p(z)=\brackets{f(v)+f(w)}p(z)=\]\[=f(v)p(z)+f(w)p(z)=(\ptensor{v}{z})(f,p)+(\ptensor{w}{z})(f,p)\]
            Luego, $\ptensor{(v+w)}{z}\in\ptensor{V}{V}$ y así, la operación $+$ es cerrada. $\checkmark$
            \item Asociatividad:
            \\
            Sean $\ptensor{a}{b},\ptensor{c}{d},\ptensor{e}{f}\in\ptensor{V}{V}$, tenemos que ver si $\ptensor{a}{b}+\brackets{\ptensor{c}{d}+\ptensor{e}{f}}=\brackets{\ptensor{a}{b}+\ptensor{c}{d}}+\ptensor{e}{f}$, tal que
            \[(\ptensor{a}{b}+\brackets{\ptensor{c}{d}+\ptensor{e}{f}})(p,q)=p(a)q(b)+\brackets{p(c)q(d)+p(e)q(f)}=p(a)q(b)+p(c+e)q(d+f)=\]
            \[=p(a+c+e)q(b+d+f)=p(a+c)q(b+d)+p(e)q(f)=\brackets{p(a)q(b)+p(c)q(d)}+p(e)q(f)=\]\[=(\brackets{\ptensor{a}{b}+\ptensor{c}{d}}+\ptensor{e}{f})(p,q)\checkmark\]
            \item Elemento neutro:\\
            Sea $\ptensor{e_1}{e_2}\in\ptensor{V}{V}$ el elemento neutro de $\ptensor{V}{V}$, tal que
            \[\ptensor{e_1}{e_2}+\ptensor{v}{w}=\ptensor{v}{w}+\ptensor{e_1}{e_2}=\ptensor{v}{w}\]
            Vemos el valor de este elemento neutro,
            \[(\ptensor{e_1}{e_2}+\ptensor{v}{w})(f,g)=(\ptensor{v}{w})(f,g)\]
            \[f(e_1)g(e_2)+f(v)+g(w)=f(v)g(w)\]
            \[f(e_1+v)g(e_w+w)=f(v)g(w)\Leftrightarrow\left\lbrace\begin{matrix}
                e_1=0\\
                e_2=0
            \end{matrix}\right.\]
            luego, $\ptensor{e_1}{e_2}=0$. $\checkmark$
            \item Elemento simétrico:
            \\
            $\forall\ptensor{v}{u}\in\ptensor{V}{V}$, $\exists\ptensor{\Tilde{v}}{\Tilde{u}}\in\ptensor{V}{V}$, tal que
            \[\ptensor{v}{u}+\ptensor{\Tilde{v}}{\Tilde{u}}=\ptensor{\Tilde{v}}{\Tilde{u}}+\ptensor{v}{u}=\ptensor{e_1}{e_2}=0\]
         Veamos quién es $\ptensor{\Tilde{v}}{\Tilde{u}}$,
        \[(\ptensor{v}{u}+\ptensor{\Tilde{v}}{\Tilde{u}})(f,g)=f(v)g(u)+f(\Tilde{v})g(\Tilde{u})=(\ptensor{0}{0})(f,g)=f(0)g(0)\]
        luego,
        \[v+\Tilde{v}=0\Rightarrow\Tilde{v}=-v\]
        \[u+\Tilde{u}=0\Rightarrow\Tilde{u}=-u\]
        Por tanto, el elemento simétrico de $\ptensor{v}{u}$ es $\ptensor{(-v)}{(-u)}$. $\checkmark$
        \item Conmutabilidad:\\
        Sean $\ptensor{v}{w},\ptensor{u}{z}\in\ptensor{V}{V}$, entonces
        \[(\ptensor{v}{w}+\ptensor{u}{z})(f,g)=f(v)g(w)+f(u)g(z)=f(v+u)g(w+z)=\]\[=f(u+v)g(z+w)=f(u)g(z)+f(v)g(w)=(\ptensor{u}{z}+\ptensor{v}{w})(f,g)\checkmark\]
    Luego, es grupo abeliano. $\checkmark$
         \end{enumerate}
         \item Doble propiedad distributiva:
         \begin{enumerate}
             \item $\forall\lambda,\mu\in\mathbb{R}$, $\forall\ptensor{v}{w}\in\ptensor{V}{V}$,
             \[(\lambda+\mu)\cdot(\ptensor{v}{w})(f,g)=(\lambda+\mu)f(v)g(w)=\]\[=\lambda f(v)g(w)+\mu f(v)g(w)=\lambda(\ptensor{v}{w})(f,g)+\mu(\ptensor{v}{w})(f,g)\checkmark\]
             \item $\forall\lambda\in\mathbb{R}$, $\forall\ptensor{v}{w},\ptensor{u}{z}\in\ptensor{V}{V}$, tenemos que
             \[\lambda(\ptensor{v}{w})(f,g)+\lambda(\ptensor{u}{z})(f,g)=\lambda f(v)g(w)+\lambda f(u)g(z)=\]\[=\lambda\brackets{f(v)g(w)+f(u)g(z)}=\lambda(\ptensor{v}{w}+\ptensor{u}{z})(f,g)\checkmark\]
             \end{enumerate}
             \item Propiedad pseudo-asociativa:\\
             $\forall\lambda,\mu\in\mathbb{R}$; $\forall\ptensor{v}{w}\in\ptensor{V}{V}$, tenemos que
             \[\lambda\cdot\brackets{\mu\cdot(\ptensor{v}{w})(f,g)}=\lambda\brackets{\mu f(v)g(w)}=\lambda f(\mu v)g(\mu w)=\]\[=f(\lambda\mu v)g(\lambda\mu w)=f(\mu\lambda v)g(\mu\lambda w)=\mu\brackets{f(\lambda v)g(\lambda w)}=(\mu\cdot\lambda)f(v)g(w)=(\mu\cdot\lambda)(\ptensor{v}{w})(f,g)\checkmark\]
             \item Elemento unitario del cuerpo: $\forall\ptensor{v}{w}\in\ptensor{V}{V}$; $\Tilde{\mu}\in\mathbb{R}$, entonces $\Tilde{\mu}\cdot\ptensor{v}{w}=\ptensor{v}{w}\cdot\Tilde{\mu}=\ptensor{v}{w}$
             \[(\Tilde{\mu}\cdot\ptensor{v}{w})(f,g)=f(\Tilde{\mu}v)g(\Tilde{\mu}w)=(\ptensor{v}{w})(f,g)=f(v)g(w)\Rightarrow\begin{matrix}
                 \Tilde{\mu}\cdot v=v\\
                 \Tilde{\mu}\cdot w=w
             \end{matrix}\Leftrightarrow\Tilde{\mu}=1\checkmark\]
       \end{enumerate}
       Luego, $(\ptensor{V}{V}, +, \cdot)$ es un $\mathbb{R}$-espacio vectorial.
\end{proof}
\noindent Al igual que cualquier otro espacio vectorial, el espacio $V\otimes V$ deberá tener una \textbf{base}.
\begin{proposition}
    Si tenemos un $V$ espacio vectorial sobre $\mathbb{K}$ con base $B=\curlybraces{v_1,\dots,v_n}$, entonces todo $\ptensor{v}{w}$ será combinación lineal de los elementos de la base de $\ptensor{V}{V}$ dada por $\ptensor{B}{B}=\curlybraces{\ptensor{v_i}{v_j}}_{i,j=1}^{n}$
\end{proposition}
\begin{proof}
    Queremos ver que $\curlybraces{\ptensor{v_i}{}v_j}_{i,j=1}^n$ es base de $\ptensor{V}{V}$. Para ello, tendremos que ver que esta base $\ptensor{B}{B}$ complete el espacio $\ptensor{V}{V}$ y que los vectores de la misma sean linealmente independientes.\\
    Sabemos que $\ptensor{v}{w}\in\ptensor{V}{V}$ y que
    \[\begin{array}{cccl}
        \ptensor{v}{w}: & \pcart{V^*}{V^*} & \to & \mathbb{R}\\
         & (f,g) & \mapsto & f(v)g(w)
    \end{array}\]
    Luego, para que la base $\ptensor{B}{B}$ complete el espacio $\ptensor{V}{V}$, se deberá poder expresar cualquier vector $\ptensor{v}{w}\in\ptensor{V}{V}$ como combinación lineal de los vectores de $\ptensor{B}{B}$. Podemos usar $B=\curlybraces{v_i}_{i=1}^n$ base de $V$, tal que
    \[v=\sum\limits_{i=1}^n\lambda^iv_i=\lambda^iv_i,\hspace{4mm}w=\sum\limits_{j=1}^n\mu^jv_j=\mu^jv_j\]
    Por tanto, usando $f,g\in V^*$, tenemos que
    \[\ptensor{v}{w}(f,g)=f(v)g(w)=f(\lambda^iv_i)g(\mu^jv_j)=\lambda^if(v_i)\mu^jg(v_j)=\lambda^i\mu^jf(v_i)g(v_j)=\lambda^i\mu^j(\ptensor{v_i}{v_j})(f,g)\]
    Luego, hemos expresado un vector del espacio $\ptensor{V}{V}$ como combinación lineal de los vectores de la base $\ptensor{B}{B}$. $\checkmark$\\ \\
    Veamos que son linealmente independientes, para ello, se debe cumplir que,
    \[\sum\limits_{i,j=1}^n\lambda^{ij}(\ptensor{v_i}{v_j})=\lambda^{ij}(\ptensor{v_i}{v_j})=0\Leftrightarrow\lambda^{ij}=0\]
    Sabiendo que la base de $V^*$ es $B^*=\curlybraces{f^1,f^2,\dots,f^n}$, tal que
    \[f^i(v_i)=1\hspace{5mm}f^j(v_i)\overset{i\neq j}{=}0\Rightarrow f^i(v_j)=\delta_{ij}\]
    Podemos evaluar lo anterior en dos elementos arbitrarios de $B^*$, tal que
    \[0=\lambda^{ij}(\ptensor{v_i}{v_j})(f^n,f^m)= \lambda^{ij}f^n(v_i)f^m(v_j)=\lambda_{ij}\delta_{n}^i\delta_{m}^j=\lambda^{nm}\]
    luego, $\lambda^{nm}=0$ y por tanto, los vectores son linealmente independientes. $\checkmark$\\ \\
    Así, hemos demostrado que $\ptensor{B}{B}$ es base de $\ptensor{V}{V}$.
\end{proof}

\begin{note}
    Denotaremos $\ptensor{v}{w}\equiv h$, tal que
    \[
    \begin{array}{cccl}
        h: & \pcart{V^*}{V^*} & \to & \mathbb{R} \\
         & (f^i,f^j) & \mapsto & h(f^i,f^j)=h^{ij}
    \end{array}
    \]
    siendo $f^i,f^j\in B^*$. Por tanto, para dos $p,q\in V^*$ cualesquiera, escribiremos
    \[(\ptensor{v}{w})(p,q)=h(p,q)=h\left(\sum_{i=1}^np_if^i,\sum_{j=1}^nq_jf^j\right)=p_iq_j(f^i,f^j)=h^{ij}p_iq_j\]
\end{note}
\noindent Veamos algunas \textbf{propiedades} del producto tensorial.
\begin{proposition}
    Sea $V$ un $\mathbb{R}$-espacio vectorial,
    \begin{enumerate}[label=(\roman*)]
        \item $\ptensor{(v_1+v_2)}{w}=\ptensor{v_1}{w}+\ptensor{v_2}{w}$; $\forall v_1,v_2,w\in V$.
        \item $\ptensor{w}{(v_1+v_2)}=\ptensor{w}{v_1}+\ptensor{w}{v_2}$, $\forall v_1,v_2,w\in V$.
        \item $\ptensor{(\lambda v)}{w}=\lambda\ptensor{v}{w}$, $\forall v,w\in V$, $\forall\lambda\in\mathbb{R}$.
        \item $\ptensor{w}{(\lambda v)}=\lambda\ptensor{w}{v}$, $\forall v,w\in V$, $\forall \lambda\in\mathbb{R}$.
        \item $\ptensor{v}{w}\neq\ptensor{w}{v}$.
        \item $\ptensor{v}{w}\neq0$ si $v\neq0$ ó $w\neq 0$.
        \item Sea $\ptensor{a}{b}\neq0$, $\ptensor{a}{b}=\ptensor{a'}{b'}\Leftrightarrow a'=\lambda a$ y $b'=\lambda^{-1}b$.
        \item $\ptensor{V}{W}$ es isomorfo con $\ptensor{W}{V}$.
    \end{enumerate}
\end{proposition}
\begin{proof}
    \begin{enumerate}[label=(\roman*)]
        \item $\forall v_1,v_2,w\in V$,
        \[(\ptensor{(v_1+v_2)}{w})(f,g)=f(v_1+v_2)g(w)=\brackets{f(v_1+f(v_2)}g(w)=\]\[=f(v_1)g(w)+f(v_2)g(w)=(\ptensor{v_1}{w})(f,g)+(\ptensor{v_2}{w})(f,g)\qedh\]
        \item $\forall v_1,v_2,w\in V$,
        \[(\ptensor{w}{(v_1+v_2)})(f,g)=f(w)g(v_1+v_2)=f(w)\brackets{g(v_1)+g(v_2)}=\]\[=f(w)g(v_1)+f(w)g(v_2)=(\ptensor{w}{v_1})(f,g)+(\ptensor{w}{v_2})(f,g)\qedh\]
        \item $\forall v,w\in V$ y $\forall\lambda\in\mathbb{R}$,
        \[(\ptensor{(\lambda\cdot v)}{w})(f,g)=f(\lambda\cdot v)g(w)=\lambda\cdot f(v)g(w)=\lambda\cdot(\ptensor{v}{w})(f,g)\qedh\]
        \item $\forall v,w\in V$ y $\forall\mu\in\mathbb{R}$,
        \[(\ptensor{w}{(\lambda\cdot v)})(f,g)=f(w)g(\lambda\cdot v)=\lambda\cdot f(w)g(v)=\lambda\cdot(\ptensor{w}{v})(f,g)\qedh\]
        \item Vemos que, $(\ptensor{v}{w})(f,g)=f(v)g(w)$ y que $(\ptensor{w}{v})(f,g)=f(w)g(v)$, luego estos elementos serían iguales solo si $f\equiv g$. $\qedh$
        \item Sean $v,w\in V$ y $f,g\in V^*$, tales que $f\not\equiv0$ y $g\not\equiv0$, entonces
        \[(\ptensor{v}{w})(f,g)=f(v)g(w)=0\Leftrightarrow\begin{matrix}
            f(v)=0 & \Leftrightarrow v=0\\
            \text{ó} & \\
            g(w)=0 & \Leftrightarrow w=0
        \end{matrix}\qedh\]
        \item \begin{tabular}{c|}
             $\Rightarrow$ \\ \hline
        \end{tabular} 
        Sea $\ptensor{a}{b}=\ptensor{a'}{b'}$ entonces
        \[(\ptensor{a}{b})(f,g)=f(a)g(b)=(\ptensor{a'}{b'})(f,g)=f(a')g(b')\]
        luego,
        \[f(a)g(b)=f(a')g(b')\]
        pero como $a\neq a'$ y $b\neq b'$, debe haber una relación entre ambos, de tal forma que se cumpla la igualdad anterior. Supondremos que $a$ y $a'$ tienen una relación lineal (la más sencilla), tal que $a'=\lambda a+c$, luego 
        \[f(a)g(b)=f(a')g(b')=f(\lambda a+c)g(b')=f(\lambda a)g(b')+f(c)g(b')=\lambda f(a)g(b')+f(c)g(b')\]
        Agrupamos términos de la igualdad, tal que,
        \[0:\hspace{5mm}0=f(c)g(b')\]
        \[f(a):\hspace{5mm}g(b)=\lambda g(b')\]
        Por la propiedad \textit{(vi)}, como $b'\neq0$, entonces $c=0$. Además,
        \[g(b)=\lambda g(b')\Rightarrow g(b')=\lambda^{-1}g(b)\Rightarrow g(b')=g(\lambda^{-1}b)\Rightarrow b'=\lambda^{-1}b\]
        Luego,
        \[\begin{matrix}
            a'=\lambda a\\
            b'=\lambda^{-1}b
        \end{matrix}\hspace{4mm}\checkmark\]
        \begin{tabular}{c|}
             $\Leftarrow$ \\ \hline
        \end{tabular} 
        Sea $a'=\lambda a$ y $b'=\lambda^{-1}b$, entonces
        \[(\ptensor{a'}{b'})(f,g)=f(a')g(b')=f(\lambda a)g(\lambda^{-1}b)=\cancel{\lambda}\cancel{\lambda^{-1}}f(a)g(b)=(\ptensor{a}{b})(f,g)\checkmark\]
        \item Sean $V,W$ espacios vectoriales, tales que
        \[\begin{array}{ccl}
            \ptensor{V}{W} & \to & \ptensor{W}{V}  \\
            \ptensor{v}{w} & \mapsto & \ptensor{w}{v}
        \end{array}\]
        Si suponemos que $dimV=n$ y $dimW=m$, sabemos por tanto que $dim(V\otimes W)=n\cdot m$ y $dim(W\otimes V)=m\cdot n$, luego tienen la misma dimensión y por tanto, son isomorfos. $\checkmark$\\ \\
        También podemos hacerlo sin usar la proposición de que $dim(\ptensor{V}{W})=n\cdot m$. Es claro ver que la aplicación es inyectiva, pues no hay dos elementos con la misma imagen, ya que la imagen se forma al permutar los elementos. Luego, al ser inyectivo, tenemos que $dimKer=0$. Por el Primer Teorema de Isomorfía,
        \[dim(\ptensor{V}{W})=\cancelto{0}{dimKer}+dimIm=dimIm=dim(\ptensor{W}{V})\]
        Luego, como $\ptensor{V}{W}$ y $\ptensor{W}{V}$ tienen la misma dimensión, entonces son isomorfos.
    \end{enumerate}
\end{proof}

\subsection{Aplicaciones lineales} % Main chapter title
\label{cap1-sec1-subsec2} 

Veamos ahora las aplicaciones lineales.
\begin{definition}
    Sean $V$ y $V'$ dos espacios vectoriales sobre el mismo cuerpo $\mathbb{K}$.
    Se dice que en una aplicación $f:V\longrightarrow V'$ es una aplicación lineal, o también llamado homomorfismo de espacios vectoriales, si se verifica:
    \begin{enumerate}[label=(\roman*)]
        \item $f(x+y)=f(x)+f(y),\forall x,y\in V$
        \item $f(\lambda\cdot x)=\lambda\cdot f(x),\forall\lambda\in\mathbb{K},\forall x\in V$
    \end{enumerate}
    Diremos además que $f$ es un isomorfismo lineal si es biyectiva, que $f$ es un endomorfismo si $V=V'$ y que es un automorfismo si es un endomorfismo biyectivo. 
\end{definition}

\noindent Las aplicaciones lineales tienen asociados dos conjuntos cuyas características son de interés, a saber, el núcleo y la imagen.

\begin{definition}
    Sea $f:V\longrightarrow W$ definimos el núcleo o kernel de la aplicación $f$ como
    \[Kerf=\curlybraces{v\in V:f(v)=0}\]
    y la imagen como
    \[Imf=\curlybraces{w\in W:\exists v\in V/f(v)=w}.\]
\end{definition}

\noindent Veamos algunas propiedades básicas de ambos conjuntos.

\begin{proposition}
Sea $f:V\to V'$ una aplicación lineal, se tienen las siguientes propiedades:
\begin{enumerate}[label=(\roman*)]
    \item \label{prop1:item1} $\rm{Im}f$ es un subespacio de $V'$ y que $\rm{Ker}f$ es un subespacio de $V$.
    \item \label{prop1:item2}Si $W$ es un subespacio vectorial de $V$, entonces $f(W):=\curlybraces{f(w): w\in W}$ es un subespacio de $V'$.
    \item \label{prop1:item3}Si $W'$ es un subespacio de $V'$, entonces $f^{-1}(W'):=\curlybraces{v\in V: f(v)\in W'}$ es también un subespacio de $V$.
\end{enumerate}  
\end{proposition}
%\newpage
\begin{proof}
\begin{enumerate}[label=\ref{prop1:item1}]
    \item Por definición, como los elementos de la $\rm{Im}f$ son pertenecientes a $V'$, entonces la $\rm{Im}f$ es subespacio de $V'$. De igual forma ocurre con el $\rm{Ker}f$, pues sus elementos pertenecen a $V$ y por tanto, este es subespacio de $V$.
\end{enumerate}
\begin{enumerate}[label=\ref{prop1:item2}]
    \item Como $W$ es subespacio de $V$, tenemos que $w\in V$ también, por tanto, los $f(w)$ pertenecerán a $V'$, cosa que implica que $f(W)$ es subespacio de $V'$, pues los $f(w)$ de $f(W)$ pertenecen a $V'$.
\end{enumerate}
\begin{enumerate}[label=\ref{prop1:item3}]
    \item Por analogía a $\ref{prop1:item2}$ vemos que $f^{-1}(W)$ es subespacio de $V$.
\end{enumerate}
\end{proof}
\noindent Ahora veamos algunas propiedades esenciales de las aplicaciones lineaales.
\begin{proposition}
    Sea $f:V\longrightarrow V'$ una aplicación lineal,
    \begin{enumerate}[label=(\roman*)]
        \item \label{pro1:item1} entonces $f$ es inyectiva si y solo si $Kerf=\curlybraces{0}$.
        \item \label{pro1:item2} si $G$ es un conjunto generador de $V$, $<G>=V$, entonces $f(G)$ es conjunto generador
        de $Imf$, $<f(G)>=Imf$.
        \item \label{pro1:item3} si $S\subset V$ es un conjunto de vectores linealmente independientes, si $f$ es inyectiva, entonces $f(S)$ es linealmente independiente.
        \item \label{pro1:item4} $f$ es inyectiva $\Leftrightarrow$ conserva la independencia lineal.
   
        \item \label{pro1:item5} si $f$ es biyectiva y $B$ es una base de $V$, entonces $f(B)$ es base de $V'$.

        \item \label{pro1:item6} $f$ es sobreyectiva $\Leftrightarrow$ $Imf=V'$
    \end{enumerate}
\end{proposition}



\begin{proof}
\ref{pro1:item1} \begin{tabular}{c|}
                 $\Rightarrow$ \\ \hline
            \end{tabular}
            Suponiendo que $f$ es inyectiva, sabemos que su Kernel es,
            \[\rm{Ker}f=\curlybraces{v\in V:f(v)=0}\]
            pero como la inyectividad nos implica que la imagen debe provenir de un único vector de entrada, entonces este vector será $v=0$, y por tanto, $ker f=\curlybraces{0}$. $\checkmark$
\\     
            \begin{tabular}{c|}
                 $\Leftarrow$  \\ \hline
            \end{tabular}
            Suponiendo que $ker f=\curlybraces{0}$, esto nos quiere decir que únicamente el vector $v=0$ satisface $f(v)=0$, luego como un vector tiene una única imagen, decimos que $f$ es inyectiva. \qedh
\\ \\
\ref{pro1:item2}  Veamos que el conjunto $f(G)$ es sistema generador de la imagen, es decir,   \[<f(G)>=Imf\Leftrightarrow\forall y\in Imf,\exists\lambda^1,\dots,\lambda^n\in\mathbb{K},y_1,\dots,y_n\in f(G)\text{ tales que }y=\lambda^1y_1+\dots+\lambda^ny_n.\] Sabemos que $G$ es conjunto generador, luego sea $y\in Imf$. Entonces por definición se tiene que existe $x\in V$ tal que $f(x)=y$. Como $<G>=V$, existen $\lambda^1,\dots,\lambda^n\in\mathbb{K}$, $v_1,\dots,v_n\in G$ tales que \[ x= \sum_{i=1}^n \lambda^i v_i.\] Tenemos entonces que:  \[y=f(x)=f(\lambda^1v_1+\dots+\lambda^nv_n)=\lambda^1f(v_1)+\dots+\lambda^nf(v_n).\] Por lo tanto, $y$ es combinación lineal de elementos de $f(G)$, es decir, $<f(G)>=\mathrm{Im}f$. \qedh
\\ \\
\ref{pro1:item3}
            Sea $S$ un conjunto linealmente independiente en $V$. 
            Supongamos que $f$ es inyectiva, vamos a probar que $f(S)$ es linealmente independiente, es decir,
            \[\lambda^1y_1+\dots+\lambda^ny_n=0\Rightarrow\lambda^1=\lambda^2=\dots=\lambda^n=0\hspace{4mm} \forall y_1,\dots,y_n\in f(S),\hspace{2mm}
                \forall\lambda^1,\dots,\lambda^n\in\mathbb{K}\]
            Supongamos $\lambda^1y_1+\dots+\lambda^ny_n=0$.  Como $y_j\in f(S),\exists x_j\in S/f(x_j)=y_j$.
             \[\left.\begin{array}{r}
                \lambda_1f(x_1)+\dots+\lambda_nf(x_n)=0\\
                f(\lambda_1x_1+\dots+\lambda_nx_n)=0
            \end{array}\right\rbrace f(0)=0\Rightarrow\lambda_1x_1+\dots+\lambda_nx_n=0\Rightarrow f\text{ inyectiva}\Rightarrow\lambda_1=\dots=\lambda_n=0\]
                \\ \\
                \ref{pro1:item4} \begin{tabular}{c|}
                 $\Longrightarrow$ \\ \hline
            \end{tabular} 
            Trivial por (iii) $\checkmark$\\
            \begin{tabular}{c|}
                 $\Longleftarrow$ \\ \hline
            \end{tabular} 
            Por reducción al absurdo:\\
            Supongamos que existen $v_1,v_2\in V$ distintos, tales que $f(v_1)=f(v_2)\Leftrightarrow f(v_1)-f(v_2)=0\Leftrightarrow f(v_1-v_2)=0$. 
            Luego, $v=v_1-v_2\neq0$ verifica que $f(v)=0$, $\curlybraces{v}$ es un conjunto linealmente independiente, $f(\curlybraces{v})$ tendría que ser un conjunto l.i. por hipótesis, pero $f(\curlybraces{v})=\curlybraces{0}$ que no es un conjunto l.i. cosa absurda. \qedh
            \\ \\
            \ref{pro1:item5}  Sea una aplicación lineal biyectiva $f:V\to V'$
            y una base de $V$, $B=\curlybraces{v_1,\dots,v_n}$. Entones, si aplicamos
\[f(B)=\curlybraces{f(v_1),\dots,f(v_n)}=\curlybraces{v_1',\dots,v_n'}\]
            y entonces, estos $v_i'\in V'$ van a formar una base de $V'$, pues al ser $f$ biyectiva, los vectores serán linealmente independientes, pues los de $B$ lo son; y además, como tienen la misma dimensión que $V'$, pasan de ser conjunto generador a base. $\qedh$
            \\ \\
            \ref{pro1:item6} \begin{tabular}{c|}
                 $\Rightarrow$  \\ \hline
            \end{tabular}
            Suponiendo que $f$ es sobreyectiva, tendremos que para cada $y\in V'$, existe al menos un $x\in V$, tal que $f(x)=y$. Por consiguiente, cada elemento de $V'$ es la imagen de un elemento de $V$, es decir, $Imf=V'$. $\checkmark$\\
            \begin{tabular}{c|}
                 $\Leftarrow$  \\ \hline
            \end{tabular}
            Suponiendo que $imf=V'$, tenemos que todos los elementos de $V'$ son imagen de los elementos de $V$, siendo esta la propia definición de sobreyectividad, luego $f$ es sobreyectiva.
\end{proof}
\noindent Una vez visto estas propiedades, de $\ref{pro1:item5}$ podemos obtener un resultado interesante, que es la siguiente proposición.
\begin{proposition}
Sea $B=\curlybraces{v_1,\dots,v_n}$ una base de $V$, y sea $f:V\rightarrow V'$ una aplicación lineal. Se tiene entonces que $\curlybraces{f(v_1),\dots,f(v_n)}$ es un sistema generador de la imagen.
\end{proposition}
\begin{proof}
    Supongamos que $B$ es una base y que conocemos $f(v_j),\forall v_j\in B$. 
    Sea $v\in V$, escrito en coordenadas de la base como $v=\lambda^1v_1+\dots+\lambda^nv_n$, con $v_i\in B$, 
 y $\lambda^i\in\mathbb{K}$, entonces  $f(x)=f(\lambda^1v_1+\dots+\lambda^nv_n)=\lambda^1f(v_1)+\dots\lambda^nf(v_n)$, luego hemos puesto $f(x)$ en coordenadas de $\curlybraces{f(v_1,\dots,f(v_n)}$.
\end{proof}

\noindent Ahora vamos a ver un resultado bastante importante, el cuál nos permitirá representar aplicaciones lineales en matrices, denominadas \textbf{matrices asociadas a la aplicación $f$}. Además, este resultado es importante para Física, pues los físicos no solemos trabajar con aplicaciones, sino que trabajamos con sus matrices asociadas, pues se puede decir que "tienen" la misma información que las aplicaciones.

\begin{proposition}  
\label{prop1.4}
    Sean $(V,+,\cdot)$ y $(V',+,\cdot)$ $\mathbb{K}$-espacios vectoriales de dimensión finita con $dimV=n$ y $dimV'=m$. 
    Sea $f:V\longrightarrow V'$ una aplicación lineal, entonces dadas $\left\lbrace\begin{matrix}
        B=\curlybraces{v_1,\dots,v_n}\text{ base de }V\\
        B'=\curlybraces{v_1',\dots,v_n'}\text{ base de }V'
    \end{matrix}\right.$\\
    $f$ se representa en esas bases como una matriz en $\mathcal{M}_{m\times n}(\mathbb{K})$.
\end{proposition}

\begin{proof}
    Como $f$ es lineal, me basta con conocer $f(B)$, para ello, tenemos que conocer $f(v_1),f(v_2),\dots,f(v_n)$, teniendo:
    \[\begin{matrix}
        f(v_1) & = & a_{1}^1v_1'+a_{1}^2v_2'+\dots+a_{1}^mv_m', & a_{1}^i\in\mathbb{K}\\
        f(v_2) & = & a_{2}^1v_1'+a_{2}^2v_2'+\dots+a_{2}^mv_m', & a_{2}^i\in\mathbb{K}\\
        \vdots & & \vdots & \vdots\\
        f(v_n) & = & a_{n}^1v_1'+a_{n}^2v_2'+\dots+a_{n}^mv_m', & a_{n}^i\in\mathbb{K}
    \end{matrix}\]
    Sea $v\in V:v=\lambda^1v_1+\dots+\lambda^nv_n,\hspace{2mm}\lambda^i\in\mathbb{K}$, si le aplicamos $f$ tenemos,
    \[\begin{array}{rll}
        f(v) & = &\lambda^1f(v_1)+\dots+\lambda^nf(v_n) \\
         & = & \lambda^1(a_{1}^1v_1'+\dots+a_{1}^mv_m')+\lambda^2(a_{2}^1v_1'+\dots+a_{2}^mv_m')+\dots+\lambda^n(a_{n}^1v_1'+\dots+a_{n}^mv_m')\\
         & = & (a_{1}^1\lambda^1+a_{2}^1\lambda^2+\dots+a_{n}^1\lambda^n)v_1'+(a_{1}^2\lambda^1+\dots+a_{n}^2\lambda^n)v_2'+\dots+(a_{1}^m\lambda^1+\dots+a_{n}^m\lambda^n)v_m'
    \end{array}
        \]
   Luego,  $f(v)=\mu^1v_1'+\mu^2v_2'+\dots+\mu^mv_m'$, siendo $\mu^i=(a_{1}^i\lambda^1+\dots+a_{n}^i\lambda^n)$, luego, para construir la matriz $A$, ponemos las coordenadas de $v_1$ en la primera columna, las de $v_2$ en la segunda y así sucesivamente, tal que:
    \[\begin{pmatrix}
        \mu^1\\
        \mu^2\\
        \vdots\\
        \mu^m
    \end{pmatrix}=\begin{pmatrix}
        a_{1}^1 & a_{1}^1 & \dots & a_{1}^m\\
        a_{2}^1 & a_{2}^2 & \dots & a_{2}^m\\
        \vdots & \vdots & \ddots & \vdots\\
        a_{n}^1 & a_{n}^2 & \dots & a_{n}^m
    \end{pmatrix}\begin{pmatrix}
        \lambda^1\\
        \lambda^2\\
        \vdots\\
        \lambda^n
    \end{pmatrix}\Rightarrow \mu=A\cdot\lambda\]
\end{proof}

\noindent Vamos a introducir ahora el concepto de \textbf{rango} de una aplicación lineal, que puede extenderse al rango de su matriz asociada.

\begin{definition}
    Se llama rango de una aplicación lineal (matriz) a la dimensión de su imagen y se denota por $rg()$.
\end{definition}

\noindent Como un mismo espacio vectorial puede estar generado por varias bases, es lógico pensar que debe haber una relación entre estas bases o al menos una forma de cambiar de una base a otra, lo que se conoce como \textbf{cambio de base}. Esto es posible y una forma sencilla de hacerlo es mediante las matrices asociadas.

\begin{proposition}
    -Sean $V$ y $V'$ dos espacios vectoriales en $\mathbb{K}$, sea $f:V\longrightarrow V'$ lineal.\\
    -Sea $B_1=\curlybraces{v_1,\dots,v_n}$ base de $V$, $B_1'=\curlybraces{v_1',\dots,v_m'}$ base de $V'$.\\
    -Sea $A\in\mathcal{M}_{m\times n}(\mathbb{K})$ la matriz que representa a $f$ en $B_1,B_1'$.\\
    -Sea $B_2=\curlybraces{u_1,\dots,u_n}$ base de $V$, $B_2'=\curlybraces{u_1',\dots,u_m'}$ base de $V'$.\\
    -Sea $\tilde{A}\in\mathcal{M}_{m\times n}(\mathbb{K})$ la matriz que representa a $f$ en $B_2,B_2'$.\\
    -Sea $P$ la matriz de cambio de base de $B_1$ en $B_2$.\\
    -Sea $Q$ la matriz de cambio de base de $B_1'$ en $B_2'$.\\
    Entonces $\tilde{A}=Q^{-1}\cdot A\cdot P$.
\end{proposition}
\begin{proof}
    Sea $f:V\to V'$ una aplicación lineal con $n=dim V$ y $m=dim V'$. Si $A$ y $\tilde{A}$ son las matrices asociadas a $f$ respecto de distintas bases, entonces
    \[rg(A)=dim(Imf)=rg(\tilde{A})\]
    Luego $A$ y $\Tilde{A}$ tienen igual rango, y por tanto, son matrices equivalentes. Concretemos más esta situación:\\
    Sean $B_1$ y $B_2$ bases de $V$ con cambio de base de $B_1$ a $B_2$ dado por $X_1=PX_2$ y sean $B_1'$ y $B_2'$ bases de $V'$, con cambio de $B_1'$ a $B_2'$ dado por $Y_1=QY_2$.\\
    Consideremos la matriz asociada a $f$ respecto de $B_1$ y $B_1'$, $A\in\mathcal{M}_{m\times n}(\mathcal{K})$, tal que $A=\mathcal{M}_{B_1,B_1'}(f)$ y la ecuación matricial
    \[Y_1=AX_1\]
    De igual forma, sea $\tilde{A}\in\mathcal{M}_{m\times n}(\mathbb{K})$ la matriz asociada a $f$ respecto de $B_2$ y $B_2'$, tal que $\tilde{A}=\mathcal{M}_{B_2,B_2'}(\mathbb{K})$ y la ecuación matricial de $f$ respecto de estas bases,
    \[Y_2=\tilde{A}X_2\]
    Gráficamente,
    \[\begin{matrix}
    & V & \to & V' & \\
            & & A & &\\
            & B_1 & \longrightarrow & B_1' & \\
            P & \uparrow & & \uparrow & Q\\
             & & \tilde{A} & & \\
             & B_2 & \longrightarrow & B_2' &
        \end{matrix}\]
        Entonces,
        \[Y_2=\left\lbrace \begin{array}{l}
        \tilde{A}X_2\\    Q^{-1}Y_1=Q^{-1}AX_1=Q^{-1}APX_2\end{array}\right.\]
        y en consecuente,
        \[\tilde{A}=Q^{-1}AP\]
        O bien,
        \[X_2=\left\lbrace\begin{array}{l}
            \tilde{A}^{-1}Y_2\\
            P^{-1}X_1=P^{-1}A^{-1}Y_1=P^{-1}A^{-1}QY_2
        \end{array}\right.\]
        y en consecuente,
        \[\tilde{A}^{-1}=P^{-1}A^{-1}Q\]
\end{proof}

Ahora vamos a enunciar el \textbf{Primer Teorema de Isomorfía}, del que obtendremos un Corolario muy importante a la hora de trabajar con aplicaciones lineales. Este teorema no se va a demostrar (si se quiere ver la prueba consultar  \cite[Chapter 6, Theorem 6.5, Page 77]{IntroducciónTeoríaDeGrupos}).

\begin{theorem}[Primer teorema de isomorfismo de Noether]
    Sea $f:V\longrightarrow V'$ una aplicación lineal, entonces:
    \begin{enumerate}[label=(\roman*)]
        \item Existe una aplicación lineal sobreyectiva $\pi:V\longrightarrow V/Kerf$
        \item Existe un isomorfismo $\bar{f}:V/Kerf\longrightarrow Imf$
        \item Existe una aplicación lineal inyectiva $i:Imf\longrightarrow V'$, tales que $f=i\circ\bar{f}\circ\pi$, tal que
        \[\begin{matrix}
            & & f & \\
            & V & \longrightarrow & V' & \\
            \pi & \downarrow & & \uparrow & i\\
             & & \bar{f} & & \\
             & V/Kerf & \longrightarrow & Imf &
        \end{matrix}\]
    \end{enumerate}
   
\end{theorem}
\begin{corollary}
     Si además $V$ es finitamente generado,
    \[dimV=dim(Kerf)+dim(Imf)\]
\end{corollary}
\subsection{Contrtacción de tensores} % Main chapter title
\label{cap1-sec2-subsec3} 

 Una vez que hemos visto cómo subir y bajar índices, podemos definir una operación denominada \textbf{contracción} de tensores, la cuál encoge un tensor $(r,s)$ a uno $(r-1,s-1)$. La definición general se obtiene a partir del siguiente caso especial.

\begin{lemma}
    Hay una única aplicación lineal
    $C:\Omega_1^1\to\mathbb{R}$
    llamada \textit{contracción (1,1)}, tal que
    \[\begin{array}{rlll}
        C: & \Omega_1^1 (V)& \to & \mathbb{R} \\
         & \ptensor{v}{f} & \mapsto & C(\ptensor{v}{f})=f(v)
    \end{array}\]
    para todo $v\in V$ y $f\in V^*$.
\end{lemma}
\begin{proof} (Esta demostración usa el concepto de matrices de cambio de base, por lo que se recomienda ver la sección \ref{CambioBasesTensores(1,1)})\\ 
    Tomando $B=\curlybraces{v^1,v^2,\dots,v^n}$ base de $V$ y $B^*= \curlybraces{f_1,f_2,\dots,f_n}$ base de $V^*$, podemos escribir un tensor de tipo $(1,1)$ como
    \[A\equiv\sum A^i_j\ptensor{f_i}{v^j}\]
    Como $C(\ptensor{f_i}{v^j})=f_i(v^j)=\delta^j_i$, por la condición de base dual, no nos queda otra opción, más que definir,
    \[C(A)=\sum A_i^i=\sum A(f_i,v^i)\]
    Entonces, $C$ tiene las propiedades requeridas en las bases $B,B^*$. Luego, para obtener la función general requerida es suficiente con mostrar que esta definición es independiente de la elección del sistema de coordenadas. Así, tomando una nueva base de $V$, $B'=\curlybraces{w^1,w^2,\dots,w^n}$ y otra de $V^*$, $B^{*'}=\curlybraces{q_1,q_2,\dots,q_n}$, tenemos
    \[\begin{array}{rrl}
        C(A) & = & \sum\limits_mA(q_m,w^m)= \sum\limits_mA\left(\sum\limits_i a_i^mf_m,\sum_jb_m^jv^m\right)\\
         & = & \sum\limits_{i,j,m}a_i^mb_m^jA(f_i,v^j)=\sum\limits_{i,j}\delta^j_iA(f_i,v^j)\\
         & = & \sum\limits_iA(f_i,v^j)
    \end{array}\]
\end{proof}
\noindent Para extender las contracciones $(1,1)$, $C$, a un tensor de un tipo mayor, el esquema es especificar una componente covariante y otra contravariante y aplicar $C$ a estos.\\

\noindent Suponemos un tensor $A\in\Omega_r^s(V)$ y $1\leq r$ y $1\leq j \leq s$. Fijamos las formas $p_1,p_2,\dots,p_{r-1}$ y los vectores $u_1,u_2,\dots ,u_{s-1}$. Entonces la función
\[(p,u) \to A(p_1, \dots, \underbrace{p_{i}}_{\mathclap{i\text{-ésima componente contravariante}}}, \dots, p_{r-1}, u^{1}, \dots, \overbrace{u^{j}}^{\mathclap{j\text{-ésima componente covariante}}}, \ldots, u^{s-1})\]
es un tensor $(1,1)$ que puede escribirse como

\[A(p_1,\dots,\cdot,\dots,p_{r-1},u^1,\dots,\cdot,\dots,u^{s-1})\]
Aplicando la contracción $(1,1)$ a este tensor, produce una función de valor real denotada por

\[\left(C_j^iA\right)\left(p_1,\dots,p_{r-1},u^1,\dots,u^{s-1}\right)\]
Siendo $C_j^iA$ una función multilineal. Por tanto, esto es un tensor de tipo $(r-1,s-1)$ llamado \textit{la contracción de }$A$\textit{ sobre }$i,j$.

%\begin{definition}
 %   La contracción de un tensor $A$ de tipo $(r,s)$ con respecto al índice contravariante $p$ $(p\leq r)$ y al índice covariante $q$ $(q\leq s)$ es el tensor de tipo $(r-1,s-1)$, teniendo las componentes,
  %  \[B^{i_1\dots i_{r-1}}_{j_1\dots j_{s-1}}=A^{i_1\dots i_{p-1}ki_p\dots i_{r-1}}_{j_1\dots j_{q-1}kj_q\dots j_{s-1}}\]
%\end{definition}

\begin{note}
    Para poder contraer tensores, debemos tener superíndices y subíndices, así, podemos usar primero la métrica para subir o bajar índices y luego aplicar la contracción.
   \end{note} 
\begin{example}
        Si tenemos un tensor de tipo (0,2),
    $S\equiv S_{\alpha\beta}$, podemos hacer,
    \[\begin{array}{rllll}
        S_{\alpha\beta} & \to & g^{\gamma\alpha}S_{\alpha\beta}=S^{\alpha}_{\beta} & \to & C^1_1S^{\gamma}_{\beta}=S^{\beta}_{\beta} \\
        \text{Tensor (0,2)} & \to & \text{Tensor (1,1)} & \to &\text{Escalar}
    \end{array}\]
    cosa que se puede simplificar simplemente usando,
    \[S\equiv S_{\alpha\beta}\to g^{\beta\alpha}S_{\alpha\beta}=S^{\beta}_{\beta}\]
    es decir, podemos contraer tensores con la propia métrica.
\end{example}

\begin{example}
    Si
    \[U^j_i=T^{kj}_{ik}\]
    entonces
    \[U'^{j'}_{i'}=T'^{k'j'}_{i'k'}=S^i_{i'}S^l_{k'}R^{k'}_kR^{j'}_jT^{kj}_{il}=S^i_{i'}\delta^l_kR^{j'}_jT^{kj}_{il}=S^i_{i'}R^{j'}_jT^{kj}_{ik}=S^i_{i'}R^{j'}_jU^j_i\]
    donde hemos utilizado $S^l_{k'}R^{k'}_k=\delta^l_k$. Vemos que se transforma como un tensor (1,1).\\

\noindent    Así, dado un tensor $T^{ij}_{kl}$ de tipo (2,2), serán posible las 4 contracciones
    \[T^{kj}_{ki},\hspace{3mm}T^{jk}_{ik},\hspace{3mm}T^{kj}_{ik},\hspace{3mm}T^{jk}_{ki}\]
    que originan 4 tensores de tipo (1,1). Por otro lado, las dos posibles contracciones dobles que dan lugar a un escalar (tensor de tipo (0,0)) son
    \[T^{kj}_{kj},\hspace{3mm}T^{jk}_{kj}\]
\end{example}
\begin{note}
    El producto escalar $(\mathbb{R}^n,g_{ij})$ también se puede contraer. Pues $g_{ij}$ es un tensor de tipo (0,2), al cual le podemos aplicar una contracción 1,1, pero primero lo pasamos a un tensor de tipo (1,1), variando sus índices, tal que
    \[C^1_1\left(g^{ki}g_{ij}\right)=C^1_1(g^k_j)=g^j_j=n\]
    donde sabemos que vale $n$, pues al ser un espacio de dimensión $n$, la matriz asociada a $g$ será $G\in\mathcal{M}_{n\times n}$ y por tanto, la traza será la suma de $n$-elementos. Sabemos que estos elementos son el 1, porque la traza es invariante frente a los cambios de base (cosa que veremos más adelante), por tanto, si cogemos el producto escalar usual en la base usual, la matriz asociada es la matriz de Gram, cuyos elementos son todos nulos, salvo la diagonal que está formada por 1.
\end{note}
\subsection{Notación de Einstein} % Main chapter title
\label{cap1-sec1-subsec4} 

La notación de Einstein va a servir para facilitarnos la escritura, pues cada vez que tengamos un vector o una forma escrita como combinación lineal, vamos a poder redefinirlos como
\[w=\sum\limits_{i=1}^n\lambda^iv_i\equiv\lambda^iv_i\]
esto para un vector. Para una forma, tendremos
\[p=\sum\limits_{i=1}^n\mu_if^i\equiv\mu_i f^i\]
Además, para simplificar aún más la notación y dejarnos de tantas letras, vamos a identificar los escalares de $w$ como 
\[\lambda^i\equiv w^i\]
Así, los vectores como combinación lineal de otros vectores, los escribiremos como
\[w=w^iv_i\]
Y para las formas, haremos la identificación
\[\mu_i\equiv p_i\]
Así, las formas como combinación lineal de otras formas se escribirán como
\[p=p_if^i\]
\begin{example}
Un ejemplo de ello, será a la hora de identificar un vector en los términos de su base, pues suponiendo un $V$ espacio vectorial sobre el cuerpo $\mathbb{K}$ y cuya base sea $B=\curlybraces{v_1,v_2,\dots,v_n}$, tomando un $u\in V$, lo denotaremos como,
\[u=u^iv_i\]
\end{example}
\begin{example}
    Otro ejemplo será a la hora de identificar una forma en términos de la base dual, pues suponiendo un $V^*$ espacio dual de $V$, cuya base dual es $B^*=\curlybraces{f^1,f^2,\dots,f^n}$, tomando un $q\in V^*$, lo denotaremos como,
    \[q=q_if^i\]
\end{example}
\begin{note}
    En un artículo físico, se identifica directamente el escalar con el vector, es decir,
    \[w^i\equiv w\]
    pues se presupone que existe una base donde $w$ está bien definido. Así, los físicos usaremos de forma indistinguible los vectores y sus componentes respecto de una base fijada.
\end{note}
\subsection{Invariantes} % Main chapter title
\label{cap1-sec2-subsec5} 

Dado que los tensores suelen describirse en términos respecto de ciertas bases, cuando estos términos no dependen de la base empleada, los tensores se llamarán \textbf{invariantes}. O en otras palabras, los tensores que no se transforman frente a un cambio de base, serán los que llamaremos \textbf{invariantes}.\\ \\
Vamos a intentar ilustrar este concepto definiendo un tensor invariante de tipo (1,1), denominado \textit{traza}, que es un invariante conocido de las matrices. Si tenemos un tensor $A=A_j^i\ptensor{e_i}{f^j}$ que definimos como
\[\text{traza de }A=\rm{tr}A=A^i_i\]
siendo la suma de los elementos de la diagonal principal de la matriz $(A^i_j)$. No es a priori evidente que hayamos definido algo que depende únicamente de $A$, ya que los $A_j^i$ dependen no solo de $A$ sino también de la base $\curlybraces{e_i}$. Para mostrar que $\rm{tr} A$ es un número determinado enteramente por $A$ mismo y no por los $e_i$ también, debemos demostrar la invariancia; es decir, si $A$ se expresa en términos de otra base ${\tilde{e}_i}$, entonces la fórmula correspondiente en los nuevos componentes da el mismo número que antes. Así, escribimos $A=\tilde{A}^i_j\ptensor{\tilde{e}_i}{\tilde{f}^j}=A^i_j\ptensor{e_i}{f^j}$ y veremos que $A^i_j=\tilde{A}^i_j$. Usando la misma notación de cambios de base que hemos visto en el apartado anterior, tenemos la ley de transformación siguiente,
\[\tilde{A}^n_m=A^i_ja_m^jb_i^n\]
de lo cual se obtiene
\[\tilde{A}^i_i=A^p_ja^j_ib^i_p=A^p_j\delta^j_p=A^i_i\]
Queda demostrado. Luego, tenemos la proposición,
\begin{proposition}
    La traza de un tensor de tipo (1,1) es un invariante.
\end{proposition}
Para ver que no todas las expresiones en términos de las componentes de un tensor necesariamente serán un invariante, veamos el siguiente ejemplo. 
\begin{example}
    Supongamos $d=2$ y $A=\ptensor{e_1}{e_1}+\ptensor{e_1}{e_2}$, un tensor de tipo (0,2). La expresión de $A_{ii}$ en este caso será $A_{11}+A_{22}$=1+0=1. Ahora consideramos una nueva base dada por $e_1=\tilde{e}_1+\tilde{e}_2$ y $e_2=\tilde{e}_2$, entonces
    \[\begin{array}{rrl}
        A & = & (\tilde{e}_1+\tilde{e}_2)\otimes(\tilde{e}_1+\tilde{e}_2)+(\tilde{e}_1+\tilde{e}_2)\otimes\tilde{e}_2 \\
         & = & \tilde{e}_1\otimes\tilde{e}_1+2\tilde{e}_1\otimes\tilde{e}_2+\tilde{e}_2\otimes\tilde{e}_1+2\tilde{e}_2\otimes\tilde{e}_2
    \end{array}\]
    de la cuál se obtiene que $\tilde{A}_{ii}=\tilde{A}_{11}+\tilde{A}_{22}=1+2=3$. Por tanto es diferente a la base primera, luego no es un invariante.
\end{example}
\subsubsection*{Nota Final}
    Finalmente diremos que un tensor es todo aquel objeto matemático que satisfaga los cambios de base, o en otras palabras: \textit{Un tensor es todo objeto matemático que transforma como un tensor}.
%SECCION 3
\section{Dilatación temporal} % Main chapter title
\label{cap2-sec3} 
%------------------------------------------------------------------------------
La dilatación temporal es una causa directa de los postulados de Einstein. Veámoslo con un esquema,
\begin{multicols}{2}
    \begin{Figura}
        \centering
        \includegraphics[width=0.8\textwidth]{Capitulos/Capitulo2/Seccion3/Lrep.png}
        \captionof{figure}{Espejos en reposo.}
        \label{fig2.1}
    \end{Figura}
    \begin{Figura}
        \centering
        \includegraphics[width=0.8\textwidth]{Capitulos/Capitulo2/Seccion3/Lmov.png}
        \captionof{figure}{Espejos en movimiento.}
        \label{fig2.2}
    \end{Figura}
\end{multicols}
Si nos fijamos en la Figura \ref{fig2.1}, al estar los espejos en reposo, el rayo de luz que sale de la linterna vuelve en un tiempo $\Delta t=\frac{2l_0}{c}$. En cambio, suponiendo que los espejos se mueven a velocidad $\vec{v}$, y que la distancia de los brazos del rayo es $D$, entonces ahora el tiempo que tarda el rayo en ir y volver es $\Delta t'=\frac{2D}{c}$. Usando el Teorema de Pitágoras podemos calcular $D$, tal que
\[D^2=l_0^2+\left(\frac{\Delta t'v}{2}\right)^2\]
Sustituyendo $D$ y $l_0$ de las ecuaciones de $\Delta t$ y $\Delta t'$, tenemos
\[\left(\frac{\Delta t'c}{2}\right)^2=\left(\frac{\delta t'v}{2}\right)^2+\left(\frac{\Delta tc}{2}\right)^2\]
Por tanto, tenemos que el tiempo se dilata de la forma,
\begin{equation}
    \Delta t'=\gamma\Delta t
\end{equation}
y como $\gamma>1$ siempre, entonces $\Delta t'>\Delta t$, por eso se dilata el tiempo.\\ \\
Vemos que en el SRI $S'$ los relojes van más lento que en el SRI $S$, pues si consideramos como reloj el rebote de los fotones en los espejos, entonces en $S$ los fotones van más rápido que los fotones en $S'$.

%SECCION 4
\section{Contracción de longitudes} % Main chapter title
\label{cap2-sec4} 
%------------------------------------------------------------------------------
Tomamos dos eventos del espacio-tiempo, tal que
\[\Delta t'=\gamma\left(\Delta t-\frac{v}{c^2}\Delta x\right)\]
\[\Delta x'=\gamma\left(\Delta x-v\Delta t\right)\]
Asumimos que tomamos eventos que no están separados temporalmente, es decir, como si en $S'$ tomásemos una foto, así, $\Delta t'=0$. Por tanto, tendremos que $\Delta x'=L'$ y $\Delta x=L$. Luego, sustituyendo tenemos que
\[\Delta x'=\frac{\Delta x}{\gamma}\Longrightarrow L'=\frac{L}{\gamma}\]
Además, como $\gamma>1$, tendremos que $L>L'$, por tanto, se habla de contracción de longitudes; donde $L$ se conoce como \textbf{longitud propia}, que es la longitud del objeto respecto a un SRI en reposo respecto al objeto, es decir, el SRI centro de masas del objeto.
\begin{note}
    Las transformaciones de Lorentz dejan invariante las distancias espacio-temporales, pues dados dos eventos $(t_1,x_1)$ y $(t_2,x_2)$ en $S$, y los eventos correspondientes $(t_1',x_1')$ y $(t_2',x_2')$ en $S'$, entonces
    \[-c^2(t_2-t_1)^2+(x_2-x_1)^2=-c^2(t_2'-t_1')+(x_2'-x_1')^2\]
    por tanto, tenemos una cantidad que es invariante al SRI.
\end{note}
%SECCION 3
\section{Derivada de Lie} % Main chapter title
\label{cap3-sec5} 
%------------------------------------------------------------------------------
La derivada de Lie nos dice cómo derivar funciones escalares, campos vectoriales y campos tensoriales de forma general. Además, nos dice cómo conectar espacios tangentes $T_p$ y $T_q$ de dos puntos $p,q\in\mathscr{M}$ con $p\neq q$.\\ \\
Para ello, necesitamos un campo vectorial $\vec{\xi}$, que define un conjunto de difeomorfismos. En concreto, por cada punto $p\in\mathscr{M}$ pasa una curva $\gamma_p(t)$ tal que
\[\dot{\gamma}_p(t)=\xi^{\mu}(p)\]
Esta curva define el difeomorfismo siguiente,
\[\varphi_t(p)=\gamma_p(t);\hspace{4mm}\varphi_t:U\to\varphi_t(U)\]
La derivada de Lie de una función escalar $f$ a lo largo de $\vec{\xi}$ se define como,
\begin{equation}
    \mathscr{L}_{\vec{\xi}}f|_p=\vec{\xi}(f)\equiv\xi^{\mu}\partial_{\mu}f
\end{equation}
La derivada de Lie de un campo vectorial se define como,
\begin{equation}
    \mathscr{L}_{\vec{\xi}}\vec{v}|_p=\lim_{t\to0}\frac{\varphi^*_t(\vec{v}(\varphi_t(p)))-\vec{v}|_p}{t}
\end{equation}
Se puede demostrar que
\[\mathscr{L}_{\vec{\xi}}\vec{v}|_p=\brackets{\vec{\xi},\vec{v}}=-\mathscr{L}_{\vec{v}}\vec{\xi}|_p\]
En coordenadas se escribe,
\[(\mathscr{L}_{\vec{\xi}\vec{v}})^{\mu}=\xi^{\nu}\partial_{\nu}v^{\mu}-v^{\nu}\partial_{\nu}\xi^{\mu}\]
Sus propiedades son:
\begin{enumerate}
    \item La derivada de Lie preserva el tipo tensorial, las simetrías y las operaciones.
    \item Es una aplicación lineal.
    \item Satisface la regla de Leibtniz,
    \[\mathscr{L}_{\vec{\xi}}(\vec{u}\otimes\vec{v})=(\mathscr{L}_{\vec{\xi}}\vec{u})\otimes\vec{v}+\vec{u}\otimes(\mathscr{L}_{\vec{\xi}}\vec{v})\]
    \item Cumple que
    \[\left<\vec{f},\mathscr{L}_{\vec{\xi}}\vec{v}\right>=\mathscr{L}_{\vec{\xi}}(<\vec{f},\vec{v}>)-\left<\mathscr{L}_{\vec{\xi}}\vec{f},\vec{v}\right>\]
\end{enumerate}
La derivada de un tensor se define como
\begin{equation}
    \begin{array}{rl}
    \mathscr{L}_{\vec{\xi}}\left(T^{\mu_1\mu_2\dots\mu_r}_{\nu_1\nu_2\dots\nu_s}\right)&=\xi^{\mu}\partial_{\mu}T^{\mu_1\mu_2\dots\mu_r}_{\nu_1\nu_2\dots\nu_s}-T^{\nu\mu_2\dots\mu_r}_{\nu_1\nu_2\dots\nu_s}\partial_{\nu}\xi^{\mu_1}-T^{\mu_1\nu\dots\mu_r}_{\nu_1\nu_2\dots\nu_s}\partial_{\nu}\xi^{\mu_2}-\dots-T^{\mu_1\mu_2\dots\mu_{r-1}\nu}_{\nu_1\nu_2\dots\nu_s}\partial_{\nu}\xi^{\mu_r}+\\
    &+T^{\mu_1\mu_2\dots\mu_r}_{\nu\nu_2\dots\nu_s}\partial_{\nu_1}\xi^{\nu}+\dots+T^{\mu_1\mu_2\dots\mu_r}_{\nu_1\nu_2\dots\nu_{s-1}\nu}\partial_{\nu_s}\xi^{\nu}
    \end{array}
\end{equation}
La derivada de Lie nos permite saber cuándo hay una simetría, pues si $\mathscr{L}_{\vec{\xi}}\vec{v}=0$, entonces $\vec{v}$ es simétrico bajo $\varphi_t(p)$.\\ \\
La derivada de Lie necesita conocer los campos vectoriales y sus derivadas en las direcciones no tangenciales a las curvas que unen $p$ y $q$. Como la derivada de Lie necesita mucha información de primeras, puesto que hay infinitos campos fuera de la curva, introduciremos la noción de \textbf{derivada covariante}. Esta derivada, por otro lado, solo necesitará información sobre la curva y las direcciones tangentes a ella.
\subsection{Derivada covariante}
La derivada covariante es una derivada que está enteramente definida en $T_p$ y al actuar sobre un tensor, $T^{\mu_1\mu_2\dots\mu_r}_{\nu_1\nu_2\dots\nu_s}$, nos devolverá otro tensor de tipo $(r,s+1)$, tal que
\begin{equation}
    \nabla_{\nu_{s+1}}T^{\mu_1\mu_2\dots\mu_r}_{\nu_1\nu_2\dots\nu_s}\equiv T^{\mu_1\mu_2\dots\mu_r}_{\nu_1\nu_2\dots\nu_s;\nu_{s+1}}
\end{equation}
Sus propiedades son las siguientes:
\begin{enumerate}
    \item Es lineal,
    \[\nabla_{\nu}\left(\alpha T^{\mu_1\mu_2\dots\mu_r}_{\nu_1\nu_2\dots\nu_s}+\beta S^{\mu_1\mu_2\dots\mu_r}_{\nu_1\nu_2\dots\nu_s}\right)=\alpha\nabla_{\nu}T^{\mu_1\mu_2\dots\mu_r}_{\nu_1\nu_2\dots\nu_s}+\beta\nabla_{\nu}S^{\mu_1\mu_2\dots\mu_r}_{\nu_1\nu_2\dots\nu_s}\]
    con $\alpha,\beta\in\mathbb{R}$.
    \item Satisface la regla de Leibtniz (regla de la cadena),
    \[\nabla_{\nu}\left(T^{\mu_1\mu_2\dots\mu_r}_{\nu_1\nu_2\dots\nu_s}S^{\rho_1\rho_2\dots\rho_{r'}}_{\sigma_1\sigma_2\dots\sigma_{s'}}\right)=S^{\rho_1\rho_2\dots\rho_{r'}}_{\sigma_1\sigma_2\dots\sigma_{s'}}\nabla_{\nu}T^{\mu_1\mu_2\dots\mu_r}_{\nu_1\nu_2\dots\nu_s}+T^{\mu_1\mu_2\dots\mu_r}_{\nu_1\nu_2\dots\nu_s}\nabla_{\nu}S^{\rho_1\rho_2\dots\rho_{r'}}_{\sigma_1\sigma_2\dots\sigma_{s'}}\]
    \item Conmuta con la contracción,
    \[\nabla_{\nu}\left(T^{\mu\mu_2\dots\mu_r}_{\mu\nu_2\dots\nu_s}\right)=\nabla_{\nu}T^{\mu\mu_2\dots\mu_r}_{\mu\nu_2\dots\nu_s}\]
    \item Sobre funciones actúa como,
    \[\nabla_{\mu}f=(df)_{\mu}\]
    En coordenadas,
    \[(df)_{\mu}=\partial_{\mu}f\]
\end{enumerate}
De la propiedad 4. tenemos que 
\[\nabla_{\mu}f=(d\vec{f})_{\mu}\partial_{\mu}f\]
Sobre $<\vec{f},\vec{v}>=f_{\mu}v^{\mu}$ actúa $\nabla_{\mu}(f_{\nu}v^{\mu})=f_{\nu}\partial_{\mu}v^{\nu}$.
\begin{remark}
$\hspace{5mm}$
    \begin{itemize}
        \item $\partial_{\mu}v^{\nu}$ no es un tensor, pues
        \[\partial'_{\mu}v^{'\nu}=\frac{\partial x^{\rho}}{\partial x^{'\mu}}\partial_{\rho}\left(\frac{\partial x^{'\nu}}{\partial x^{\sigma}}v^{\sigma}\right)=\frac{\partial x^{\rho}}{\partial x^{'\mu}}\frac{\partial x^{'\nu}}{\partial x^{\sigma}}\partial_{\rho}v^{\sigma}+\underbrace{\frac{\partial x^{\rho}}{\partial x^{'\mu}}\partial_{\rho}\left(\frac{\partial x^{'\nu}}{\partial x^{\sigma}}\right)v^{\sigma}}\]
        donde el término señalado hace que no sea un tensor, pues hace que no transforme como un tensor.
        \item $\nabla_{\mu}v^{\nu}$ es un tensor, pues
        \[\begin{array}{cl}
            \nabla_{\mu}(f_{\nu}v^{\nu}) & =\partial_{\mu}(f_{\nu}v^{\nu})=f_{\nu}\partial_{\mu}v^{\nu}+v^{\nu}\partial_{\mu}f^{\nu} \\
            || & \\
             f_{\nu}\nabla_{\mu}v^{\nu}+v^{\nu}\nabla_{\mu}f_{\nu}&=f_{\nu}\left(\partial_{\mu}v^{\nu}+\Gamma_{\mu\rho}^{\rho}v^{\rho}\right)+v^{\nu}\left(\partial_{\mu}f_{\nu}+\overline{\Gamma}_{\mu\nu}^{\rho}f_{\rho}\right)=\\
             &=f_{\nu}\partial_{\mu}v^{\nu}+v^{\nu}\partial_{\mu}f_{\nu}+\underbrace{\Gamma_{\mu\sigma}^{\rho}v^{\sigma}f_{\rho}+\overline{\Gamma}_{\mu\rho}^{\sigma}v^{\rho}f_{\sigma}}_{\begin{matrix}
                 ||\\
                 0
             \end{matrix}}
        \end{array}\]
        donde el último término se anula porque $\overline{\Gamma}_{\mu\rho}^{\sigma}=-\Gamma_{\mu\rho}^{\sigma}$.
    \end{itemize}
\end{remark}
En resumen, tenemos que
\begin{equation}
    \nabla_{\mu}v^{\nu}=\partial_{\mu}v^{\nu}+\Gamma_{\mu\rho}^{\nu}v^{\rho}
\end{equation}
es la derivada covariante de vectores.
\begin{equation}
    \nabla_{\mu}f_{\nu}=\partial_{\mu}f_{\nu}-\Gamma_{\mu\nu}^{\rho}f_{\rho}
\end{equation}
es la derivada covariante de formas.\\ \\
Notemos que $\nabla_{\mu}v^{\nu}$ es un tensor, pero $\partial_{\mu}v^{\nu}$ no es un tensor, por tanto $\Gamma_{\mu\nu}^{\rho}$ no transforma como un tensor, pues
\[\Gamma_{\mu\rho}^{'\nu}=\frac{\partial x^{'\nu}}{\partial x^{\gamma}}\frac{\partial x^{\sigma}}{\partial x^{'\mu}}\frac{\partial x^{\delta}}{\partial x^{'\rho}}\Gamma_{\sigma\delta}^{\gamma}-\frac{\partial x^{\gamma}}{\partial x^{'\nu}}\frac{\partial x^{\sigma}}{\partial x^{'\rho}}\partial_{\gamma}\left(\frac{\partial x^{'\nu}}{\partial x^{\sigma}}\right)\]
En cambio, la combinación $\partial_{\mu}v^{\nu}+\Gamma_{\mu\rho}^{\nu}v^{\rho}$ sí es un tensor.\\ \\
La derivada covariante de un tensor tipo $(r,s)$ viene dada por,
\begin{equation}
    \begin{array}{rl}
        \nabla_{\mu}T^{\mu_1\mu_2\dots\mu_r}_{\nu_1\nu_2\dots\nu_s} & =\partial_{\mu}T^{\mu_1\mu_2\dots\mu_r}_{\nu_1\nu_2\dots\nu_s}+\Gamma_{\mu\rho}^{\mu_1}T^{\rho\mu_2\dots\mu_r}_{\nu_1\nu_2\dots\nu_s}+\dots+\Gamma_{\mu\rho}^{\mu_r}T^{\mu_1\mu_2\dots\mu_{r-1}\rho}_{\nu_1\nu_2\dots\nu_s}- \\
         & -\Gamma_{\mu\nu_1}^{\rho}T^{\mu_1\mu_2\dots\mu_r}_{\rho\nu_2\dots\nu_s}-\dots-\Gamma_{\mu\nu_s}^{\rho}T^{\mu_1\mu_2\dots\mu_r}_{\nu_1\nu_2\dots\nu_{s-1}\rho}
    \end{array}
\end{equation}
En una base no coordenada cualquiera, tenemos que la diferencia de dos conexiones en un tensor de tipo $(1,2)$ se cumple que
\[\left\lbrace\begin{array}{l}
     (\nabla_a-\overline{\nabla}_a)f_b=-C_{ab}^{c}f_c \\
     (\nabla_a-\overline{\nabla}_a)v^b=C_{ac}^bv^c 
\end{array}\right.\hspace{4mm}\text{con}\hspace{3mm}C_{bc}^a=\Gamma_{bc}^a-\overline{\Gamma}_{bc}^a\text{, que es un tensor.}\]
De entre todas las conexiones posibles, vamos a tomar aquellas conexiones que son simétricas, es decir, que $\nabla_{\mu}\nabla_{\nu}f=\nabla_{\nu}\nabla_{\mu}f$. Por tanto, tendremos que los símbolos de Christoffel, en una base coordenada, son simétricos,
\[\Gamma_{\mu\nu}^{\rho}=\Gamma_{\nu\mu}^{\rho}=\Gamma_{(\mu\nu)}^{\rho}\]
Como consecuencia, tenemos que
\[\left(\mathscr{L}_{\vec{\xi}}\vec{v}\right)^{\mu}=\xi^{\nu}\partial_{\nu}v^{\mu}-v^{\nu}\partial_{\nu}\xi^{\mu}=\xi^{\nu}\nabla_{\nu}v^{\mu}-v^{\nu}\nabla_{\nu}\xi^{\mu}\]
Por tanto, podemos sustituir $\partial_{\mu}\to\nabla_{\mu}$ en la derivada de Lie.
\section{Conexión de Levi-Civita}
Es la única conexión simétrica y compatible con la métrica, es decir, 
\[\nabla_{\mu}g_{\nu\rho}=\partial_{\mu}g_{\nu\rho}-\Gamma_{\mu\nu}^{\sigma}g_{\sigma\rho}-\Gamma_{\mu\rho}^{\sigma}g_{\nu\sigma}=0\]
Por tanto, podemos llegar a una definición de los símbolos de Christoffel en la conexión de Levi-Civita que solo depende de la métrica, tal que
\begin{equation}
    \Gamma_{\nu\rho}^{\mu}=\frac{1}{2}g^{\mu\sigma}\left(g_{\sigma\nu,\rho}+g_{\sigma\rho,\nu}-g_{\nu\rho,\sigma}\right)
\end{equation}
La conexión de Levi-Civita preserva las normas y los ángulos bajo el transporte paralelo (que se explicará en la siguiente sección).\\ \\
Además, para la conexión de Levi-Civita se cumple que
\begin{equation}
    \left(\mathscr{L}_{\vec{\xi}}g_{\mu\nu}\right)=\nabla_{\mu}\xi_{\nu}+\nabla_{\nu}\xi_{\mu}
\end{equation}
Podemos calcular las simetrías de nuestra variedad resolviendo $\left(\mathscr{L}_{\vec{\xi}}g_{\mu\nu}\right)=\nabla_{\mu}\xi_{\nu}+\nabla_{\nu}\xi_{\mu}=0$, obteniendo así las cantidades simétricas de nuestra variedad con la métrica escogida.
\section{Transporte paralelo}
Es una forma 'barata' de movernos de un punto $p$ a un punto $q$ de la variedad. Decimos que es 'barata' porque solo necesitamos una curva $\gamma_p(t)$ que pase por $p$ y $q$, y una conexión $\nabla_a$.\\ \\
Sea $\vec{v}\in T_p$ y sea $t^a$ el vector tangente a $\gamma_p(t)$. Dada una conexión $\nabla_a$, se define el transporte paralelo de $\vec{v}$ sobre $\gamma_p(t)$ como
\begin{equation}
    t^a\nabla_av^b=0
\end{equation}
teniendo una solución única. En coordenadas tenemos que  $t^{\mu}=\dot{x}^{\mu}=\frac{dx^{\mu}}{dt}$, tal que
\begin{equation}
    t^{\mu}\partial_{\mu}v^{\nu}+\Gamma_{\mu\rho}^{\nu}t^{\mu}v^{\rho}=0
\end{equation}
que equivale a
\begin{equation}
    \dot{x}^{\mu}\partial_{\mu}v^{\nu}+\Gamma_{\mu\rho}^{\nu}\frac{dx^{\mu}}{dt}v^{\rho}=\frac{dv^{\nu}}{dt}+\Gamma_{\mu\rho}^{\nu}\frac{dx^{\mu}}{dt}v^{\rho}=0
\end{equation}
Tenemos un conjunto de $n-$ecuaciones diferenciales ordinarias de primer orden y lineales en $v^{\mu}$. Existe solución y es única. Además, cualquier combinación lineal de vectores $v^{\mu}$ también es un vector de transporte paralelamente.\\ \\
El transporte paralelo induce un isomorfismo entre los espacios tangentes $T_p$ y $T_q$. Este isomorfismo depende de $\nabla_{\mu}$ y de $\gamma_p(t)$.
\begin{note}
    Si la curvatura asociada a $\nabla_a$ es cero, entonces el transporte paralelo no dependerá de $\gamma_p(t)$.
\end{note}
Para la conexión de Levi-Civita, dados $v^{\mu}$ y $u^{\mu}$, que satisfacen $t^{\mu}\nabla_{\mu}v^{\nu}=0$ y $t^{\mu}\nabla_{\mu}u^{\nu}=0$, con $t^{\mu}$ tangente a $\gamma_p(t)$, entonces
\begin{equation}
    t^{\mu}\nabla_{\mu}\left(v^{\nu}u^{\rho}g_{\nu\rho}\right)=u^{\rho}g_{\nu\rho}\cancelto{0}{t^{\mu}\nabla_{\mu}v^{\nu}}+v^{\nu}g_{\nu\rho}\cancelto{0}{t^{\mu}\nabla_{\mu}u^{\rho}}+v^{\nu}u^{\rho}t^{\mu}\cancelto{0}{\nabla_{\mu}g_{\nu\rho}}=0
\end{equation}
Por tanto, los ángulos y las normas de los vectores se preservan al ser transportados paralelamente con la conexión de Levi-Civita, independientemente de $\gamma_p(t)$.
\section{Geodésicas}
Una curva $\gamma_p(s)$, donde usamos el parámetro afín $s$, se dice que es geodésica si su vector tangente cumple que,
\begin{equation}
    \frac{dv^{\mu}}{ds}=0\Longleftrightarrow v^{\mu}\nabla_{\mu}v^{\nu}=0
\end{equation}
Este parámetro afín es único salvo multiplicación y adición de una constante. En coordenadas tenemos,
\[\begin{array}{c}
     v^{\mu}\partial_{\mu}v^{\nu}+\Gamma_{\mu\rho}^{\nu}v^{\mu}v^{\rho}=0  \\
      ||| \\
      \frac{dx^{\mu}}{ds}\partial_{\mu}\left(\frac{dx^{\nu}}{ds}\right)+\Gamma_{\mu\rho}^{\nu}\frac{dx^{\mu}}{ds}\frac{dx^{\rho}}{ds}=0
\end{array}\]
Por tanto, la ecuación de la geodésica queda
\begin{equation}
    \frac{d^2x^{\nu}}{ds^2}+\Gamma_{\mu\rho}^{\nu}\frac{dx^{\mu}}{ds}\frac{dx^{\rho}}{ds}=0
\end{equation}
teniendo así un sistema de $n-$ecuaciones diferenciales ordinarias de segundo orden no lineales. Estas ecuaciones diferenciales tienen solución y es única, pues satisfacen teoremas de existencia y unicidad.\\ \\
Localmente, podemos encontrar coordenadas normales $x=x(x')$, donde la ecuación de las geodésicas queda como,
\begin{equation}
    \frac{d^2x^{'\mu}}{ds^2}=0
\end{equation}
\subsection{Geodésicas como Principio Variacional}
Postulamos una acción,
\begin{equation}
    S=\int\sqrt{ds^2}=\int\sqrt{g_{\mu\nu}(x)\frac{dx^{\mu}}{d\lambda}\frac{dx^{\nu}}{d\lambda}}d\lambda=\int\mathcal{L}d\lambda
\end{equation}
donde si $\mathcal{L}^2>0$, entonces tenemos vectores espaciales.\\ \\
Si interpretamos $\mathcal{L}$ como un Lagrangiano, podemos obtener las ecuaciones de Euler-Lagrange, tal que
\[\frac{\partial\mathcal{L}}{\partial x^{\mu}}-\frac{d}{d\lambda}\left(\frac{\partial\mathcal{L}}{\partial\dot{x}^{\mu}}\right)=0\]
o bien,
\[\frac{\partial(\mathcal{L}^2)}{dx^{\mu}}-\frac{d}{d\lambda}\left(\frac{\partial(\mathcal{L}^2)}{\partial \dot{x}^{\mu}}\right)=-2\frac{\partial\mathcal{L}}{\partial\dot{x}^{\mu}}\dot{\mathcal{L}}\]
donde $\dot{x}^{\mu}=\frac{\partial x^{\mu}}{\partial\lambda}$. Lo calculamos,
\[\begin{array}{l}
     \frac{\partial(\mathcal{L}^2)}{\partial x^{\mu}}=(\partial_{\mu}g_{\rho\nu})\frac{dx^{\rho}}{d\lambda}\frac{dx^{\nu}}{d\lambda}  \\ \\
     \frac{\partial(\mathcal{L}^2)}{\partial\dot{x}^{\mu}}=2g_{\mu\nu}\frac{dx^{\nu}}{d\lambda} \\ \\
     \frac{d}{d\lambda}\left(\frac{\partial (\mathcal{L}^2)}{\partial\dot{x}^{\mu}}\right)=2g_{\mu\nu}\frac{d^2x^{\nu}}{d\lambda^2}+2(\partial_{\sigma}g_{\mu\nu})\frac{dx^{\sigma}}{d\lambda}\frac{dx^{\nu}}{d\lambda}
\end{array}\]
Por tanto,
\[\frac{\partial(\mathcal{L}^2)}{\partial x^{\mu}}-\frac{d}{d\lambda}\left(\frac{\partial(\mathcal{L}^2)}{\partial\dot{x}^{\mu}}\right)=-2g_{\mu\nu}\left(\frac{d^2x^{\nu}}{d\lambda^2}+\Gamma_{\rho\sigma}^{\nu}\dot{x}^{\rho}\dot{x}^{\sigma}\right)=-2\frac{\partial\mathcal{L}}{\partial\dot{x}^{\mu}}\dot{\mathcal{L}}\]
Entonces, la ecuación de la geodésica generalizada queda,
\begin{equation}
    \frac{d^2x^{\mu}}{d\lambda^2}+\Gamma_{\rho\sigma}^{\mu}\dot{x}^{\rho}\dot{x}^{\sigma}=\dot{x}^{\mu}\frac{\dot{\mathcal{L}}}{\mathcal{L}}
\end{equation}
Para llegar a la ecuación de la geodésica debemos usar el parámetro afín, pues el término de la derecha de la igualdad no se anula debido a que $\lambda$ no es el parámetro afín. Luego, redefinimos $\mathcal{L}d\lambda\equiv ds$, teniendo así la ecuación de la geodésica,
\begin{equation}
    \frac{d^2x^{\mu}}{ds^2}+\Gamma_{\rho\sigma}^{\mu}\frac{dx^{\rho}}{ds}\frac{dx^{\sigma}}{ds}=0
\end{equation}
\subsection{Derivación de los términos de Christoffel mediante las geodésicas}
Tomamos como Lagrangiano $\Tilde{\mathcal{L}}=g_{\mu\nu}\dot{x}^{\mu}\dot{x}^{\nu}$. Veamos un ejemplo de cómo derivar los símbolos de Christoffel usando la ecuación de la geodésica.
\begin{example}
    Tomamos la métrica en coordenadas cilíndricas,
    \[ds^2=dr^2+r^2d\theta^2+dz^2\]
    por tanto, el Lagrangiano queda
    \[\Tilde{\mathcal{L}}=\dot{r}^2+r^2\dot{\theta}^2+\dot{z}^2\]
    resolvemos las ecuaciones de Euler-Lagrange, \\
    para el eje $z$:
    \[\frac{\partial\Tilde{\mathcal{L}}}{\partial z}=0;\hspace{3mm}\frac{\partial\mathcal{L}}{\partial\dot{z}}=2\dot{z}\Rightarrow 0-\frac{d}{ds}(2\dot{z})=0\Rightarrow\ddot{z}=0\]
    luego, las ecuaciones de la geodésica para el eje $z$ quedan,
    \[\ddot{z}+\Gamma_{\mu\nu}^z\dot{x}^{\mu}\dot{x}^{\nu}=0\]
    por tanto $\Gamma_{\mu\nu}^z=0$.\\
    Para el eje $r$:
    \[\frac{\partial\Tilde{L}}{\partial r}=2r\dot{\theta}^2;\hspace{3mm}\frac{\partial\Tilde{\mathcal{L}}}{\partial\dot{r}}=2\dot{r}\Rightarrow 2r\theta^2-\frac{d}{ds}(2\dot{r})=0\Rightarrow\ddot{r}=r\dot{\theta}\]
    luego, las ecuaciones de la geodésica para el eje $r$ quedan,
    \[\ddot{r}+\Gamma_{\mu\nu}^r\dot{x}^{\mu}\dot{x}^{\nu}=0\]
    por tanto,
    \[\Gamma
    _{\theta\theta}^r=-r;\hspace{3mm}\Gamma_{r\theta}^r=0=\Gamma_{rz}^r=\Gamma_{zr}^r=\Gamma_{\theta r}^r\]
    Para el eje $\theta$:
    \[\frac{\partial\Tilde{\mathcal{L}}}{\partial\theta}=0;\hspace{3mm}\frac{\partial\Tilde{\mathcal{L}}}{\partial\dot{\theta}}=2r^2\dot{\theta}\Rightarrow0-\frac{d}{ds}(2r^2\dot{\theta})=0\]
    luego, las ecuaciones de la geodésica para el eje $\theta$ quedan,
    \[\ddot{\theta}+2\frac{1}{r}\dot{r}\dot{\theta}=0\]
    por tanto,
    \[\Gamma_{r\theta}^{\theta}=\frac{1}{r}=\Gamma_{\theta r}^{\theta}\]
    donde no aparece el 2 porque es simétrico.
\end{example}
\subsection{Densidad tensorial}
Sea un tensor $T^{\mu\nu\dots}_{\rho\sigma\dots}$ multiplicando a $(\sqrt{g})^{\omega}$, con $\omega=\dots,-2,-1,0,1,2,\dots$, su derivada covariante se define como
\begin{equation}
    \nabla_{\mu}\left(\sqrt{g}^{\omega}T^{\mu_1\mu_2\dots}_{\nu_1\nu_2\dots}\right)=(\sqrt{-g})^{\omega}\nabla_{\mu}T^{\mu_1\mu_2\dots}_{\nu_1\nu_2\dots}-\frac{\omega}{2}\Gamma_{\nu\mu}^{\nu}(\sqrt{g})^{\omega}T^{\mu_1\mu_2\dots}_{\nu_1\nu_2\dots}
\end{equation}
de forma que $\nabla_{\mu}(\sqrt{g})=0$.
\section{Tensores de Curvatura}
Dada una conexión $\nabla_a$ y una uno-forma, el operador $(\nabla_a\nabla_b-\nabla_b\nabla_a)f_c$ es lineal, es decir, sea $h$ una función escalar, entonces
\[(\nabla_a\nabla_b-\nabla_b\nabla_a)(hf_c)=h(\nabla_a\nabla_b-\nabla_b\nabla_a)f_c\]
Esto implica que
\begin{equation}
    (\nabla_a\nabla_b-\nabla_b\nabla_a)f_c=\mathscr{R}_{abc}^df_d
\end{equation}
donde $\mathscr{R}_{abc}^d$ es el \textbf{tensor de Riemann}, se puede interpretar como que $\nabla_a\nabla_b\equiv\rightarrow\uparrow$ y $\nabla_b\nabla_a\equiv\uparrow\rightarrow$, sacando así la curvatura de la variedad.\\ \\
Para derivar $(\nabla_a\nabla_b-\nabla_b\nabla_a)v^c$, recordemos que la conexión es simétrica, por tanto $(\nabla_a\nabla_b-\nabla_b\nabla_a)(f_cv^c)=0$, luego
\begin{equation}
    (\nabla_a\nabla_b-\nabla_b\nabla_a)v^c=\mathscr{R}_{abd}^cv^d
\end{equation}
Las propiedades del tensor de Riemann son:
\begin{enumerate}
    \item Es antisimétrico, $\mathscr{R}_{abc}^d=-\mathscr{R}_{bac}^d$.
    \item $\mathscr{R}_{[abc]}^d=0$, puesto que $\nabla_{[a}\nabla_{b}f_{c]}=0$.
    \item Cumple la identidad de Bianchi,
    \[\nabla_{[a}\mathscr{R}_{bc]d}^e=0\]
    puesto que $\nabla_{[a}\nabla_{b}\nabla_{c]}f_d=0$.
    \item Si la conexión es de Levi-Civita, entonces tenemos que
    \[(\nabla_a\nabla_b-\nabla_b\nabla_a)T^{a_1a_2\dots}_{b_1b_2\dots}=-\mathscr{R}_{abc}^{a_1}T^{ca_2\dots}_{b_1b_2\dots}-\dots+\mathscr{R}_{abb_1}^cT^{a_1a_2\dots}_{cb_2\dots}+\dots\]
    Luego, en una base coordenada en la conexión de Levi-Civita, podemos definir el tensor de Riemann como,
    \begin{equation}
        \mathscr{R}_{\mu\nu\rho}^{\sigma}=\partial_{\nu}\Gamma_{\mu\rho}^{\sigma}-\partial_{\mu}\Gamma_{\nu\rho}^{\sigma}+\Gamma_{\mu\rho}^{\lambda}\Gamma_{\lambda\nu}^{\sigma}-\Gamma_{\nu\rho}^{\lambda}\Gamma_{\lambda\mu}^{\sigma}
    \end{equation}
\end{enumerate}
El \textbf{tensor de Ricci} se define como,
    \begin{equation}
        \mathscr{R}_{ab}=\mathscr{R}_{acb}^c
    \end{equation}
El \textbf{escalar de Ricci} se define como,
\begin{equation}
    \mathscr{R}=\mathscr{R}_{ab}g^{ab}
\end{equation}
Para la conexión de Levi-Civita, tenemos que el tensor de Ricci es simétrico, es decir, $\mathscr{R}_{ab}=\mathscr{R}_{ba}$.
%SECCION 1
\section{Ecuaciones de Einstein y Principio de Mínima Acción} % Main chapter title
\label{cap4-sec6} 

La acción de la formulación Lagrangiana para la relatividad general es formulada por Hilbert, pero éste dejó que Einstein lo publicara, y así, la acción de la relatividad general se denomina \textit{acción de ''Hilbert-Einstein''}. Se construye con cantidades invariantes bajo transformaciones generales de coordenadas (difeomorfismos). Se define como,
\begin{equation}
    S_{HE}=\int d^4x\sqrt{|g|}R
\end{equation}
donde $\sqrt{|g|}$ es el Jacobiano y $R$ el escalar de Ricci. que involucra derivadas segundas de $g_{\mu\nu}$, pero se pueden eliminar integrando por partes.\\ \\
Consideramos variaciones $\delta g_{\mu\nu}$ que se anulan en la frontera, es decir, $\left.\delta g_{\mu\nu}\right|_{\partial\mathscr{M}}=0$. Así, podemos variar la acción, tal que
\begin{equation}
    \delta S_{HE}=\int d^4x\brackets{\sqrt{|g|}g^{\mu\nu}\delta R_{\mu\nu}+\sqrt{|g|}(\delta g^{\mu\nu})R_{\mu\nu}+(\delta\sqrt{|g|})g^{\mu\nu}R_{\mu\nu}} 
\end{equation}
donde, asumiendo que la conexión es de Levi-Civita, tenemos
\[\delta\sqrt{|g|}=-\frac{1}{2}\sqrt{|g|}g_{\mu\nu}\delta g^{\mu\nu};\hspace{5mm}\delta R_{\mu\nu\rho}^{\sigma}=\nabla_{\nu}(\delta\Gamma_{\mu\rho}^{\sigma})-\nabla_{\mu}(\delta\Gamma_{\nu\rho}^{\sigma})\]
recordando que $\delta\partial()=\partial\delta()$ y que $\Gamma$ no es un tensor, pero $\delta\Gamma$ sí lo es.
\begin{proof}
    (...)
\end{proof}
Por tanto,
\begin{equation}
    g^{\mu\nu}\delta R_{\mu\nu}=\nabla_{\nu}\left(g^{\mu\rho}\delta\Gamma_{\mu\rho}^{\nu}-g^{\nu\rho}\delta\Gamma_{\sigma\rho}^{\sigma}\right)=\nabla_{\nu}v^{\nu}
\end{equation}
es decir, es igual a la divergencia de un vector. Por tanto,
\begin{equation}
    \int_{\mathscr{M}}d^4x\sqrt{|g|}g^{\mu\nu}\delta R_{\mu\nu}=\int_{\mathscr{M}}d^4x\sqrt{|g|}\nabla_{\nu}v^{\nu}=\int_{\partial\mathscr{M}}d\sigma n_{\nu}v^{\nu}=0
\end{equation}
donde hemos usado el Teorema de Stokes.\\ \\
Si acoplamos materia,
\[S=\frac{1}{\kappa}S_{HE}+S_{\mathscr{M}}\]
y tomamos variaciones con respecto a $g_{\mu\nu}$,
\[\begin{array}{rl}
     \delta_gS&=\frac{1}{\kappa}\delta_gS_{HE}+\delta_gS_{\mathscr{M}}=0  \\
     & =\frac{1}{\kappa}\int d^4x\sqrt{|g|}(R_{\mu\nu}-\frac{1}{2}gR-\kappa T_{\mu\nu})\delta g^{\mu\nu}=0
\end{array}\]
donde hemos definido el tensor de energía-impulso como $T_{\mu\nu}=-\frac{1}{\sqrt{g}}\frac{\delta S_{\mathscr{M}}}{\delta g^{\mu\nu}}$ y $T^{\mu\nu}=\frac{1}{\sqrt{|g|}}\frac{\delta S_{\mathscr{M}}}{\delta g_{\mu\nu}}$.\\ \\
En términos de la densidad lagrangiana (o Lagrangiano), tenemos
\[S_{\mathscr{M}}=\int d^4x\sqrt{|g|}\mathscr{L}_{\mathscr{M}}=\int d^4x\mathcal{L}_{\mathscr{M}}\]
donde $\mathscr{L}_{\mathscr{M}}$ es el Lagrangiano y $\mathcal{L}_{\mathscr{M}}$ es la densidad lagrangiana. Así, el tensor de energía-impulso se reescribe como 
\[T^{\mu\nu}=\frac{2}{\sqrt{|g|}}\frac{\partial\mathcal{L}_{\mathscr{M}}}{\partial g_{\mu\nu}}=2\frac{\partial\mathscr{L}_{\mathscr{M}}}{\partial g_{\mu\nu}}+g^{\mu\nu}\mathscr{L}_{\mathscr{M}}\]
\[T_{\mu\nu}=-\frac{2}{\sqrt{|g|}}\frac{\partial\mathcal{L}_{\mathscr{M}}}{\partial g^{\mu\nu}}=-2\frac{\partial\mathscr{L}_{\mathscr{M}}}{\partial g^{\mu\nu}}+g_{\mu\nu}\mathscr{L}_{\mathscr{M}}\]
El tensor $T_{\mu\nu}$ es conservado como consecuencia de la invariancia bajo transformaciones generales de coordenadas (difeomorfismos) de la acción,
\begin{equation}
    S_{\mathscr{M}}=\int_{\mathcal{M}}d^4x\mathscr{L}(\phi,\nabla\phi,g)
\end{equation}
Dado una densidad lagrangiana, aplicamos un difeomorfismo y vemos como transforman los campos.
Consideramos un difeomorfismo $\xi^{\mu}$ que se anula en frontera $\partial\mathscr{M}$. Tal que
\[\delta_{\xi}\phi=-\mathcal{L}_{\xi}\phi;\hspace{5mm}\delta_{\xi}g_{\mu\nu}=-\mathcal{L}_{\xi}g_{\mu\nu}=-2\nabla_{(\mu}\xi_{\nu)}\]
teniendo así una variación de la acción tal que
\[\delta_{\xi}S_{\mathscr{M}}=-\int_{\mathscr{M}}d^4x(\delta_{\phi}\mathcal{L}_{\mathscr{M}}\mathcal{L}_{\xi}\phi+\frac{\delta\mathcal{L}_{\mathscr{M}}}{\delta g_{\mu\nu}}\Delta_{\mu}\xi_{\nu})\]
donde $\delta_{\phi}\mathscr{L}_{\mathscr{M}}\mathscr{L}_{\xi}\phi=0$, pues se cumplen las ecuaciones del movimiento. Así,
\[\delta_{\xi}S_{\mathscr{M}}=-\int d^4x\sqrt{|g|}(T^{\mu\nu}\nabla_{\mu}\xi_{\nu})=-\int d^4x\sqrt{|g|}\brackets{\underbrace{\nabla_{\mu}(T^{\mu\nu}\xi_{\nu})}_{\int_{\partial\mathscr{M}}(...)=0\text{( }\xi^{\mu}\text{ es cero en }\partial\mathscr{M})}-\xi_{\nu}\nabla_{\mu}T^{\mu\nu}}=0\]
Derivando $T_{\mu\nu}$ tenemos las acciones,
\[\text{Campo escalar}:\hspace{4mm}S_{\phi}=-\frac{1}{2}\int d^4x\sqrt{|g|}(\nabla_{\mu}\phi\nabla_{\nu}\phi g^{\mu\nu}+m^2\phi^2)\]
\[\text{Campo electromagnético}:\hspace{4mm}S_A=-\frac{1}{16\pi}\int d^4x\sqrt{|g|}F_{\mu\nu}F^{\mu\nu};\hspace{3mm}F_{\mu\nu}=2\nabla_{[\mu,}A_{\nu]}\]






































%SECCION 5
\section{Repaso de álgebra} % Main chapter title
\label{cap2-sec7} 
Un vector $\vec{v}\in V$, siendo $V$ un espacio vectorial, podemos escribir $\vec{v} $ en función de una base de $V$, de la forma,
\[\vec{v}=v^{\mu}\hat{e}_{\mu}=v^0\hat{e}_0+v^1\hat{e}_1+\dots+v^s\hat{e}_s\]
donde $v^{\mu}$ son las componentes de $\vec{v}$ en la base $\curlybraces{\hat{e}_{\mu}}$, siendo un vector columna.\\ \\
Los vectores duales son aplicaciones lineales, tal que $\vec{f}:V\to\mathbb{R}$, formando un espacio vectorial dual $V^*$. Además, $\vec{f}=f_{\mu}\hat{e}^{\mu}$, siendo $f_{\mu}$ un vector fila. Es decir, podemos representar $\vec{f}$ en las componentes de la base dual $\curlybraces{\hat{e}^{\mu}}$, que es la única base que cumple que $<\hat{e}^{\mu},\hat{e}_{\nu}>=\delta^{\mu}_{\nu}$.\\ \\
Dada una base dual, podemos extraer las componentes de $\vec{v}\in V$ usando esta base, es decir,
\[<\hat{e}^{\mu},\vec{v}>=<\hat{e}^{\mu},v^{\nu}\hat{e}_{\nu}>=v^{\nu}<\hat{e}^{\mu},\hat{e}_{\nu}>=v^{\nu}\delta_{\mu}^{\nu}=v^{\mu}\]
Podemos hacer lo mismo con los vectores duales dada una base vectorial, tal que
\[<\hat{e}_{\mu},\vec{f}>=<\hat{e}_{\mu},f_{\nu}\hat{e}^{\nu}>=f_{\nu}<\hat{e}_{\mu},\hat{e}^{\nu}>=f_{\nu}\delta^{\mu}_{\nu}=f_{\mu}\]
También podemos hacer el producto escalar entre vectores y vectores duales, tal que
\[<\vec{f},\vec{v}>=<f_{\mu}\hat{e}^{\mu},v^{\nu}\hat{e}_{\nu}>=f_{\mu}v^{\nu}<\hat{e}^{\mu},\hat{e}_{\nu}>=f_{\mu}v^{\nu}\delta_{\nu}^{\mu}=f_{\mu}v^{\mu}\]
Si aplicamos una transformación pasiva $M_{\mu}^{\nu}$, es decir, dejamos el vector fijo y movemos el sistema de referencia, entonces vamos a decir que $v^{'\mu}=M_{\nu}^{\mu}v^{\nu}$ y que $\hat{e}_{\mu}'=(M^{-1})_{\mu}^{\nu}\hat{e}_{\nu}$, pues
\[\vec{v}'=v^{'\mu}\hat{e}'_{\mu}=M_{\nu}^{\mu}v^{\nu}(M^{-1})^{\rho}_{\mu}\hat{e}_{\rho}=\delta_{\nu}^{\rho} v^{\nu}\hat{e}_{\rho}=v^{\nu}\hat{e}_{\nu}=\vec{v}\]
es decir, vemos que el vector no se ve afectado por las transformaciones pasivas, pero sus componentes sí se alteran.\\ \\
Diremos que la métrica de Minkowski nos permite mapear vectores de $V$ a vectores duales de $V^*$, pues dado un vector $v^{\mu}\hat{e}_{\mu}\in V$, vemos que
\[\begin{array}{rlcl}
    \eta_{\mu\nu}: & \hat{e}_{\rho} & \to & \hat{e}^{\rho}=\eta^{\rho\sigma}\hat{e}_{\sigma} \\
     & v^{\rho} & \mapsto & v_{\rho}=\eta_{\rho\sigma}v^{\sigma}
\end{array}\]
es decir, la métrica $\eta_{\mu\nu}:\vec{v}\to\vec{v}^*$ transforma los vectores tal que,
\[\begin{array}{ccl}
    v_0=-v^0; & v_i=v^i; & i=1,2,3 \\
    \hat{e}^0=-\hat{e}_0; & \hat{e}^i=\hat{e}_i &
\end{array}\]
También podemos usar $\eta_{\mu\nu}$ para mapear vectores duales a vectores.\\ \\
La métrica nos da un producto escalar, tal que
\[<\vec{v}^*,\vec{v}>=<v_{\mu}\hat{e}^{\mu},v^{\nu}\hat{e}_{\nu}>=v_{\mu}v^{\nu}<\hat{e}^{\mu},\hat{e}_{\nu}>=v_{\mu}v^{\nu}\delta_{\nu}^{\mu}=v_{\mu}v^{\mu}=\eta_{\mu\nu}v^{\mu}v^{\nu}\]
La métrica $\eta_{\mu\nu}$ nos permite subir y bajar índices, es decir, contraer índices. \\ \\
La métrica transforma como un tensor de rango $(0,2)$, es decir,
\begin{equation}
    \eta_{\mu\nu}^{(M)}=(M^{-1})^{\rho}_{\mu}(M^{-1})_{\nu}^{\sigma}\eta_{\rho\sigma}
\end{equation}
Denotaremos a $\vec{v}$ y $v^{\mu}$ indistintamente, y diremos que son \textbf{vectores contravariantes}. A los vectores del espacio dual los denotaremos por $\vec{f}$ y $f_{\mu}$ indistintamente, denominados \textbf{vectores covariantes}.\\ \\
Un tensor de rango $(r,s)$ es una aplicación multilineal, tal que
\[T_{\nu_1\nu_2\dots\nu_s}^{\mu_1\mu_2\dots\mu_r}:V_1^*\otimes V_2^*\otimes\dots\otimes V_r^*\otimes V_1\otimes V_2\otimes\dots\otimes V_s\to\mathbb{R}\]
Los tensores transforman como,
\begin{equation}
    T_{\nu_1\nu_2\dots\nu_s}^{'\mu_1\mu_2\dots\mu_r}=M^{\mu_1}_{\rho1}M^{\mu_2}_{\rho_2}\dots M^{\mu_r}_{\rho_r}(M^{-1})^{\sigma_1}_{\nu_1}(M^{-1})_{\nu_2}^{\sigma_2}\dots(M^{-1})^{\sigma_s}_{\nu_s}T_{\sigma_1\sigma_2\dots\sigma_s}^{\rho_1\rho_2\dots\rho_r}
\end{equation}
Con la métrica $\eta_{\mu\nu}$ podemos calcular la traza de un tensor, tal que para un tensor $T_{\mu\nu}$, su traza será $T=T_{\mu\nu}\eta^{\mu\nu}$, es decir, hemos contraído todos los índices. Pero si tenemos $T_{\mu}^{\nu}$, su traza será $T=T_{\mu}^{\nu}\delta_{\nu}^{\mu}$ y si tenemos $T^{\mu\nu}$, su traza será $T=T^{\mu\nu}\eta_{\mu\nu}$. O bien, podemos transformar los tensores y aplicar la primera definición, tal que $T_{\mu}^{\nu}=T_{\mu\nu}\eta^{\sigma\nu}$ y $T^{\mu\nu}=T_{\rho\sigma}\eta^{\rho\mu}\eta^{\sigma\nu}$.\\ \\
La métrica está relacionada con el elemento de línea, llegando a tener también la misma clasificación, tal que
\begin{itemize}
    \item Si $\eta_{\mu\nu}v^{\mu}v^{\nu}>0$, entonces diremos que $v^{\mu}$ es un vector espacial.
    \item Si $\eta_{\mu\nu}v^{\mu}v^{\nu}<0$, entonces diremos que $v^{\mu}$ es un vector temporal.
    \item Si $\eta_{\mu\nu}v^{\mu}v^{\nu}=0$, entonces diremos que $v^{\mu}$ es un vector nulo.
\end{itemize}
Además, se cumple que un vector ortogonal a uno de tipo tiempo es espacial y un vector ortogonal a uno de tipo espacio o nulo, no tiene por qué ser temporal, es decir, es de cualquier género.
\subsection{Tensor simétrico y antisimétrico}
Un tensor simétrico será aquel que si se le permutan dos índices, permanece invariante, es decir, $T_{\mu\nu}=T_{\nu\mu}$. La parte simétrica de un tensor es
\begin{equation}
    T_{(\mu\nu)}=\frac{1}{2}(T_{\mu\nu}+T_{\nu\mu})
\end{equation}
Un tensor antisimétrico es aquel que si se le permutan dos índices, cambia de signo, es decir, $T_{\mu\nu}=-T_{\nu\mu}$. La parte antisimétrica de un tensor es
\[T_{[\mu\nu]}=\frac{1}{2}(T_{\mu\nu}-T_{\nu\mu})\]
Un tensor cualquiera siempre se puede descomponer en la suma de su parte simétrica y su parte antisimétrica, es decir,
\[R_{\mu\nu}=R_{(\mu\nu)}+R_{[\mu\nu]}\]
\subsection{Transformaciones de Lorentz. Versión covariante}
Recordemos que las transformaciones de Lorentz son,
\[\begin{array}{rcrc}
    (i) & dt'=\gamma\left(dt-\frac{v}{c^2}dx\right); & (iii) & dy'=dy \\
    (ii) & dx'=\gamma\left(dx-vdt\right); & (iv) & dz'=dz
\end{array}\]
Por tanto, usando la notación covariante, podemos agruparlas todas en una sola ecuación, tal que
\begin{equation}
    dx^{'\mu}=\Lambda_{\nu}^{\mu}dx^{\nu}
\end{equation}
donde 
\[\Lambda_{\nu}^{\mu}=\begin{pmatrix}
    \gamma & -\gamma\frac{v}{c^2} & 0 & 0\\
    -\gamma v & \gamma & 0 & 0 \\
    0 & 0 & 1 & 0\\
    0 & 0 & 0 & 1
\end{pmatrix}\]
es la matriz de Lorentz.\\ \\
Recordando que $ds^2=(ds')^2$, vemos que la métrica de Minkowski y la matriz de Lorentz se pueden relacionar, tal que
\[\begin{array}{cllc}
    (ds')^2 & = & \eta_{\mu\nu}dx^{'\mu}dx^{'\nu}&=\eta_{\mu\nu}\Lambda_{\rho}^{\mu}dx^{\rho}\Lambda_{\sigma}^{\nu}dx^{\sigma} \\
    || & & & || \\
    ds^2 & = & \eta_{\mu\nu}dx^{\mu}dx^{\nu}&=\eta_{\mu\nu}\Lambda_{\rho}^{\mu}\Lambda_{\sigma}^{\nu}dx^{\rho}dx^{\sigma}
\end{array}\]
Por tanto,
\begin{equation}
    \eta_{\rho\sigma}=\eta_{\mu\nu}\Lambda_{\rho}^{\mu}\Lambda_{\sigma}^{\nu}
\end{equation}
Es decir, la métrica de Minkowski permanece invariante bajo transformaciones de Lorentz.\\ \\
Además, la inversa de la matriz de Lorentz también está definida, $(\Lambda^{-1})_{\nu}^{\mu}$, que también deja invariante la inversa de la métrica de Minkowski, y cumple que
\begin{equation}
    \Lambda_{\nu}^{\mu}(\Lambda^{-1})_{\rho}^{\nu}=\delta_{\rho}^{\mu}
\end{equation}
Para pasar de la matriz de Lorentz $\Lambda_{\mu}^{\nu}$ a su inversa $(\Lambda^{-1})_{\nu}^{\mu}$, teniendo un movimiento con velocidad $v$ a lo largo del eje $X$, solo debemos cambiar $v$ por $-v$; igual para rotaciones. Las propiedades de esta matriz son:
\begin{enumerate}
    \item La traspuesta de la matriz de Lorentz es $(\Lambda^{-1})_{\nu}^{\mu}=\Lambda_{\mu}^{\nu}$.
    \item Las componentes de la matriz de Lorentz para boosts de forma general son,
    \[\Lambda_0^0=\gamma;\hspace{3mm}\Lambda_i^0=-\gamma\frac{v^i}{c};\hspace{3mm}\Lambda_0^i=-\gamma\frac{v^i}{c};\hspace{3mm}\Lambda_j^i=\delta_j^i+(\gamma-1)\frac{v^iv^j}{\vec{v}^2}\]
    \item Para las rotaciones de ángulo $\theta$ alrededor de un eje $\hat{\omega}^i$, con $\hat{\omega}^{\mu}=(0,\hat{\omega}^i)$ y $\hat{\omega}^{\mu}\hat{\omega}_{\mu}=1$, las componentes de la matriz de Lorentz serán,
    \[\Lambda_0^0=0;\hspace{3mm}\Lambda_0^i=0=\Lambda_i^0;\hspace{3mm}\Lambda_j^i=\cos\theta\delta_j^i+(1-\cos\theta)\hat{\omega}^i\hat{\omega}_j-\sin\theta\mathscr{E}_{jk}^i\hat{\omega}^k\]
    donde $\mathscr{E}_{jk}^i$ es el tensor de Levi-Civita espacial, que actúa parecido a la delta de Kronceker, tal que
    \[\mathscr{E}_{ijk}=\left\lbrace
    \begin{array}{lcl}
        +1 & \text{si} & i\neq j\neq k\text{ y la permutación es par} \\
        -1 & \text{si} & i\neq j\neq k\text{ y la permutación es impar}\\
        0 & \text{en}&\text{otro caso}
    \end{array}\right.\]
    Para la métrica de Minkowski, $\eta_{\mu\nu}=diag[-1,1,1,1]$, tenemos que el tensor de Levi-Civita cumple que
    \[\mathscr{E}_{ijk}=\mathscr{E}^{ijk}=\mathscr{E}_{jk}^i\]
\end{enumerate}

%SECCION 5
\section{Cinemática relativista} % Main chapter title
\label{cap2-sec8} 
Para tratar la cinemática de forma relativista, debemos formalizar los conceptos de la cinemática clásica en formulación covariante.
\subsection{Vector cuadrivelocidad}
Representa la velocidad espacio-temporal de una partícula puntual y se obtiene derivando respecto al tiempo propio de la partícula su cuadrivector, tal que
\begin{equation}
    u^{\mu}=\frac{dx^{\mu}}{d\tau}=\dot{x}^{\mu}
\end{equation}
Si estamos en el sistema de referencia comóvil de la partícula, entonces la cuadrivelocidad será $u^{\mu}=(c,0,0,0)$, pues en este sistema de referencia, la partícula se encuentra en reposo espacial relativo; no existe el reposo relativo temporal.\\ \\
Si estamos en un SRI cualquiera donde la partícula se mueve a velocidad $\vec{v}=\frac{d\vec{x}}{dt}$, entonces la cuadrivelocidad será $u^{\mu}=\gamma(c,\vec{v})$, pues debemos reemplazar el $dt$ por $d\tau$, recordando que $dt=\gamma d\tau$.\\ \\
Además, sabemos que el producto escalar es un invariante, por lo que usando el sistema de referencia comóvil de la partícula, obtenemos que $u^{\mu}u_{\mu}=-c^2$.
\subsection{Vector cuadrimomento}
Representa el momento espacio-temporal de una partícula, y se obtiene multiplicando la masa en reposo de la partícula por su cuadrivelocidad. Así,
\[p^{\mu}=m_0u^{\mu}\Rightarrow\left\lbrace\begin{array}{l}
      p^0=m_0\gamma c\Rightarrow E=cp^0=m_0\gamma c^2 \\
     p^i=m_0\gamma v^i\Rightarrow \vec{p}=m_0\gamma\vec{v}
\end{array}\right.\]
Por tanto, el cuadrimomento es
\begin{equation}
    p^{\mu}=(E/c,p^1,p^2,p^3)
\end{equation}
Podemos calcular también el producto escalar de cuadrimomentos, tal que $p^{\mu}p_{\mu}=-m^2c^2$.
\subsection{Vector cuadriaceleración}
Representa la aceleración espacio-temporal de una partícula puntual. Se calcula derivando respecto al tiempo propio el vector cuadrivelocidad de la partícula, o derivando dos veces el cuadrivector de la partícula respecto al tiempo propio, tal que
\[b^{\mu}=\frac{d^2x^{\mu}}{d\tau^2}=\frac{du^{\mu}}{d\tau}=\dot{u}^{\mu}=\Ddot{x}^{\mu}\]
La cuadriaceleración de una partícula puntual que se mueve a velocidad $v$ y aceleración $a$ en un SRI cualquiera es,
\begin{equation}
    b^{\mu}=\left(\gamma^4\frac{\vec{v}\cdot\vec{a}}{d\tau^2},\gamma^4\frac{(\vec{v}\cdot\vec{a})\vec{v}}{c^2}+\gamma^2\vec{a}\right)
\end{equation}
donde $\vec{a}=\frac{d\vec{v}}{dt}$ y $\frac{d\gamma}{dt}=\gamma^3\frac{\vec{v}\cdot\vec{a}}{c^2}$.\\ \\
Las propiedades de la cuadriaceleración son:
\begin{enumerate}
    \item $u^{\mu}b_{\mu}=0$
    \item $b^{\mu}b_{\mu}=\gamma^4\left(\gamma^2\frac{\vec{a}\cdot\vec{v}}{c^2}+\vec{a}\cdot\vec{a}\right)\geq0$. Por tanto, esto implica que $b^{\mu}$ es de género espacio.
\end{enumerate}
\subsection{Derivación}
Al igual que tenemos un gradiente tridimensional, podemos construir un gradiente cuadridimensional, tal que
\begin{equation}
    \partial_{\mu}f=f_{,\mu}=\frac{\partial f}{\partial x^{\mu}}=\left(\frac{1}{c}\frac{\partial f}{\partial t},\nabla f\right)
\end{equation}
siendo un vector covariante, por lo que podemos construir su versión contravariante, tal que
\begin{equation}
    \partial^{\mu}f=\eta^{\mu\nu}\partial_{\nu}f=\left(-1\frac{1}{c}\frac{\partial f}{\partial t},\nabla f\right)
\end{equation}
\subsection{Operador D'Alembertiano}
Se define como el Laplaciano cuadridimensional, tal que
\begin{equation}
    \square f=\eta^{\mu\nu}\partial_{\mu}\partial_{\nu}f=-c^2\partial_t^2f+\nabla^2f
\end{equation}
Es invariante bajo transformaciones de Lorentz, pues si tenemos dos SRI $S'$ y $S$, los operadores D'Alembertianos de cada SRI cumplirán que $\square'=\square$.
\subsection{Tensor de Levi-Civita}
Es un tensor totalmente antisimétrico. se define como,
\begin{equation}
    \mathscr{E}^{\mu\nu\rho\sigma}=\left\lbrace\begin{array}{ll}
        +1 &\text{si }\mu\neq\nu\neq\rho\neq\sigma \text{y la perturbación es par} \\
        -1 & \text{si }\mu\neq\nu\neq\rho\neq\sigma \text{y la perturbación es imppar}\\
        0 & \text{en otro caso}
    \end{array}\right.
\end{equation}
Este tensor cumple que,
\begin{enumerate}
    \item $\mathscr{E}_{0123}=-\mathscr{E}^{0123}$.
    \item $\mathscr{E}^{\mu\nu\rho\sigma}\mathscr{E}_{\mu\nu\rho\sigma}=-4!$.
    \item $\mathscr{E}^{\mu\nu\rho\sigma}\mathscr{E}_{\mu\nu\rho\gamma}=-3!\delta_{\gamma}^{\sigma}$.
    \item $\mathscr{E}^{\mu\nu\rho\sigma}\mathscr{E}_{\alpha\beta\rho\sigma}=-2!\delta_{\mu}^{[\alpha}\delta_{\beta]}^{\nu}$.
    \item $\mathscr{E}^{\mu\nu\sigma\rho}\mathscr{E}_{\alpha\beta\gamma\rho}=-1!\delta_{[\alpha}^{\mu}\delta_{\beta}^{\nu}\delta_{\gamma]}^{\sigma}$.
    \item $\mathscr{E}^{\mu\nu\sigma\rho}\mathscr{E}_{\alpha\beta\gamma\delta}=\delta_{[\alpha}^{\mu}\delta_{\beta}^{\nu}\delta_{\gamma}^{\sigma}\delta_{\delta]}^{\rho}$.
\end{enumerate}
%SECCION 5
\section{Grupo de Poincaré} % Main chapter title
\label{cap2-sec9} 
Este grupo es el grupo de transformaciones que deja invariante el elemento de línea. Tenemos,
\begin{itemize}
    \item \textbf{Traslaciones espaciotemporales:} son del tipo $x^{'\mu}=x^{\mu}+\alpha^{\mu}$, donde $\alpha^{\mu}=cte$.\\
    El operador momento en mecánica cuántica se define como $\hat{P}_{\mu}=-i\partial_{\mu}=-i\frac{\partial}{\partial x^{\mu}}$, pero si definimos el operador $\hat{U}_{\alpha^{\mu}}=\exp[i\alpha^{\mu}\hat{P}_{\mu}]$, que podemos expandirlo en serie de potencias, tal que
    \[\hat{U}_{\alpha^{\mu}}f(x^{\mu})=\left[1+\alpha^{\mu}(\partial_{\mu})+\frac{1}{2|}\left(\alpha^{\mu}(\partial_{\mu})^2\right)+\dots\right]f(x^{\mu})=f(x^{\mu}+\alpha^{\mu})=f(x^{'\mu})\]
    por tanto, este operador nos permite hacer que $\hat{U}_{\alpha^{\mu}}x^{\mu}=x^{\mu}+\alpha^{\mu}=x^{'\mu}$.\\ \\
    Al operador $\hat{P}^{\mu}$ se le denomina \textit{generador de traslaciones}. Podemos calcular el conmutador, tal que $\brackets{\hat{P}_{\mu},\hat{P}_{\nu}}=0$, si y solo si el grupo de traslaciones es un grupo abeliano, por lo que las traslaciones conmutan.
    \item \textbf{Transformaciones de Lorentz:} están formadas por las rotaciones espaciales y las transformaciones de Lorentz puras, los boosts.\\ \\
    Sabemos que $\det(\Lambda_{nu}^{\mu})=\pm1$, pero nos vamos a quedar con el subgrupo propio de las transformaciones con $\det(\Lambda_{\nu}^{\mu})=+1$ y dentro de este subgrupo, nos quedamos con el subgrupo ortocrono, es decir, el subgrupo donde se mantiene la dirección temporal, $\Lambda_{0}^{0}\geq1$. Si consideramos una transformación de Lorentz infinitesimal, tendremos que
    \begin{equation}
    \Lambda_{\nu}^{\mu}=\delta_{\nu}^{\mu}+\delta\omega^{\mu}_{\nu}
    \end{equation}
    Recordamos que $\eta_{\mu\nu}=\eta^{\rho}_{\mu}\Lambda_{\nu}^{\sigma}\eta_{\rho\sigma}$, por lo que sustituyendo tenemos que,
    \begin{equation}
        \delta\omega_{\mu\nu}+\delta\omega_{\nu\mu}=0
    \end{equation}
    a primer orden. Es decir, $\delta\omega_{\mu\nu}$ es un tensor antisimétrico.\\ \\
    Al ser un tensor antisimétrico, podemos hacer la siguiente clasificación:
    \[\delta\omega_{0i}=-\delta\omega_{i0}=\frac{\delta v_i}{c}=\delta\xi_i\]
    que se hace cargo de los boosts.
    \[\delta\omega_{ij}=-\delta\omega_{ji}=-\mathscr{E}_{ijk}\delta\theta^k\]
    que se hace cargo de las rotaciones.\\ \\
    Decimos que los $\xi_i$ y $\theta^k$ son los vectores de Killing de las traslaciones espaciales y rotaciones, respectivamente.
\end{itemize}
Como el grupo de Lorentz tiene 4 traslaciones espaciotemporales, 3 boosts y 3 rotaciones; decimos que este grupo tiene 10 grados de libertad.
\\ \\
Veamos cómo cambian los vectores,
\[x^{'\mu}=x^{\mu}+\delta\ x^{\mu}\]
con
\[\delta x^{\mu}=\delta\omega^{\mu}_{\nu}x^{\nu}=-\frac{i}{2}\delta\omega^{\rho\sigma}\hat{L}_{\rho\sigma}x^{\mu}\]
donde 
\[\hat{L}_{\mu\nu}=\hat{X}_{\mu}(-i\partial_{\nu})-\hat{X}_{\nu}(-i\partial_{\mu})=\hat{X}_{\mu}\hat{P}_{\nu}-\hat{X}_{\nu}\hat{P}_{\mu}\]
Definimos el momento angular espacial tal que
\begin{equation}
    \hat{L}^i=\frac{1}{2}\mathscr{E}^{ijk}\hat{L}_{jk}
\end{equation}
Definimos $\hat{K}_i=\hat{L}_{0i}$ como el generador de los boosts, tal que
\begin{equation}
    \delta x^{\mu}=i\left(\delta\vec{\theta}\cdot\vec{L}+\delta\vec{\xi}\cdot\hat{\vec{K}}\right)x^{\mu}
\end{equation}
Vemos que
\[[\hat{L}_i,\hat{L}_j]f(x)=i\mathscr{E}_{ij}^k\hat{L}_k(f(x))\]
\[[\hat{L}_i,\hat{K}_j]=i\mathscr{E}_{ij}^k\hat{K}_k\]
Luego, no conmutan, pues no es lo mismo rotar y hacer un boost, que hacer un boost y luego rotar.\\ \\
Además, si hacemos dos transformaciones de Lorentz en dos direcciones diferentes, tampoco conmutan, pues $[\hat{K}_i,\hat{K}_j]=-i\mathscr{E}_{ij}^k\hat{L}_k$. Esto es lo que se conoce como \textit{Rotación de Weigner}, que se traduce en: 'boost + boost = boost + rotación'.\\ \\
Para boosts finitos, es decir, velocidad finita, podemos identificar el vector $\vec{\xi}$ con la velocidad, tal que
\begin{equation}
    \vec{\xi}=\vec{v}\cdot\arctan\left(\frac{|\vec{v}|}{c}\right)
\end{equation}
La versión matricial de $\hat{L}_i$ y $\hat{K}_i$ son,
\[\begin{array}{ccc}
    (L_1)^{\mu}_{\nu}=\begin{pmatrix}
        0 & 0 & 0 & 0\\
        0 & 0 & 0 & 0\\
        0 & 0 & 0 & -i\\
        0 & 0 & i & 0
    \end{pmatrix}; & (L_2)_{\nu}^{\mu}=\begin{pmatrix}
        0 & 0 & 0 & 0 \\
        0 & 0 & 0 & i \\
        0 & 0 & 0 & 0 \\
        0 & -i & 0 & 0
    \end{pmatrix}; & (L_3)_{\nu}^{\mu}=\begin{pmatrix}
        0 & 0 & 0 & 0 \\
        0 & 0 & -i & 0 \\
        0 & i & 0 & 0 \\
        0 & 0 & 0 & 0
    \end{pmatrix} \\ \\
    (K_1)_{\nu}^{\mu}=\begin{pmatrix}
        0 & i & 0 & 0 \\
        i & 0 & 0 & 0 \\
        0 & 0 & 0 & 0 \\
        0 & 0 & 0 & 0
    \end{pmatrix}; & (K_2)_{\nu}^{\mu}=\begin{pmatrix}
        0 & 0 & i & 0 \\
        0 & 0 & 0 & 0 \\
        i & 0 & 0 & 0 \\
        0 & 0 & 0 & 0 
    \end{pmatrix}; & (K_3)_{\nu}^{\mu}=\begin{pmatrix}
        0 & 0 & 0 & i \\
        0 & 0 & 0 & 0 \\
        0 & 0 & 0 & 0 \\
        i & 0 & 0 & 0
    \end{pmatrix}
\end{array}\]
%SECCION 5
\section{Dinámica relativista} % Main chapter title
\label{cap2-sec10}
Sabemos que la dinámica viene descrita por el Lagrangiano y que las ecuaciones del movimiento se hallan minimizando la acción. La acción se define en términos del Lagrangiano como,
\begin{equation}
    S=\int_{\tau_1}^{\tau_2}d\tau\mathscr{L}(\tau,x^{\mu},\dot{x}^{\mu})
\end{equation}
donde usamos el tiempo propio $\tau$.\\ \\
Dada una trayectoria $x^{\mu}(\tau)$, consideramos variaciones de la trayectoria, de la forma $x^{\mu}(\tau)+\delta x^{\mu}(\tau)$, tal que $\left.\delta x^{\mu}\right|_{\tau_1,\tau_2}=0$.
\subsection{Principio variacional de acción estacionaria}
Las trayectorias físicas son aquellas que dejan invariante la acción estacionaria (minimizan o maximizan la acción). Esto se traduce en,
\[\delta S=\int_{\tau_1}^{\tau_2}\left(\frac{\partial\mathscr{L}}{\partial x^{\mu}}\delta x^{\mu}+\frac{\partial\mathscr{L}}{\partial\dot{x}^{\mu}}\delta\dot{x}^{\mu}\right)d\tau=\int_{\tau_2}^{\tau_1}\left[\left(\frac{\partial\mathscr{L}}{\partial x^{\mu}}-\frac{d}{d\tau}\left(\frac{\partial\mathscr{L}}{\partial x^{\mu}}\right)\right)\delta x^{\mu}+\frac{d}{d\tau}\left(\frac{\partial\mathscr{L}}{\partial\dot{x}^{\mu}}\delta x^{\mu}\right)\right]d\tau=0\]
donde $\left.\frac{\partial\mathscr{L}}{\partial\dot{x}^{\mu}}\delta x^{\mu}\right|_{\tau_1,\tau_2}=0$, pues $\delta x^{\mu}|_{\tau_1,\tau_2}=0$. Usando que $\delta S=0$, obtenemos las ecuaciones de Euler-Lagrange, tal que
\[\frac{\partial\mathscr{L}}{\partial x^{\mu}}-\frac{d}{d\tau}\left(\frac{\partial\mathscr{L}}{\partial \dot{x}^{\mu}}\right)=0\]
\subsection{Cantidades Conservadas}
\begin{theorem}{Teorema de Noether}
    Si tomamos variaciones de las trayectorias que no se anulan en los extremos y resulta que $\delta S=0$, entonces para esta variación existe una simetría, es decir, existe una cantidad conservada.
\end{theorem}
Por tanto, a esta variación le podemos asociar una ley de conservación, tal que
\[\frac{d}{\tau}\left(\frac{\partial\mathscr{L}}{\partial\dot{x}^{\mu}}\delta x^{\mu}\right)=0\Longrightarrow\delta Q_{\delta x}\equiv \frac{\partial\mathscr{L}}{\partial\dot{x}^{\mu}}\delta x^{\mu}=cte\]
\subsection{Partícula libre relativista}
Construimos la acción usando el elemento de línea, tal que
\[S=-mc\int_{s_1}^{s_2}\sqrt{-ds^2}=-mc^2\int_{\tau_1}^{\tau_2}\sqrt{\dot{t}^2-\dot{\vec{x}}^2/c^2}d\tau=-mc^2\int_{t_1}^{t_2}\sqrt{1-\frac{\vec{v}^2}{c^2}}dt=-mc^2\int_{t_1}^{t_2}\frac{dt}{\gamma}\]
Por tanto, el Lagrangiano relativista de una partícula libre será
\begin{equation}
    \mathscr{L}(\tau,x^{\mu},\dot{x}^{\mu})=\sqrt{\dot{t}^2-\frac{\dot{x}^i\dot{x}_i}{c^2}}
\end{equation}
Aplicamos las ecuaciones de Euler-Lagrange, tal que
\[\text{Eje 0:}\hspace{5mm}\frac{\partial\mathscr{L}}{\partial x^0}-\frac{d}{d\tau}\left(\frac{\partial\mathscr{L}}{\partial\dot{x}^0}\right)=0\Longrightarrow \frac{d}{d\tau}\left(\frac{\dot{t}}{\sqrt{\dot{t}^2-\frac{\dot{x}^i\dot{x}_i}{c^2}}}\right)=0\]
En resumen,
\[\frac{d}{d\tau}(mu^{\mu})=0=\frac{d}{d\tau}(m\dot{x}^{\mu})\]
La acción es invariante de Lorentz (de Poincaré) y a velocidades pequeñas $(|\vec{v}|<<c)$, recuperando el límite no relativista del Lagrangiano,
\[S=-mc^2\int\sqrt{1-\frac{|\vec{v}|^2}{c^2}}dt\approx-mc^2\int dt+\int dt\left(\frac{1}{2}m|\vec{v}|^2\right)+\cancelto{0}{\mathscr{O}(|\vec{v}|^4/c^4)}\]
El cuadrimomento de la partícula será $p_{\mu}=mu_{\mu}=m\gamma(-c,\vec{v})$, siendo una cantidad conservada. La simetría asociada es la invariancia bajo traslaciones. Luego, dada una traslación infinitesimal $\delta x^{\mu}=\delta\alpha^{\mu}=cte$, tenemos que
\[\delta Q=\frac{\partial\mathscr{L}}{\partial\dot{x}^{\mu}}\delta x^{\mu}=\frac{\partial\mathscr{L}}{\partial\dot{\alpha}^{\mu}}\delta\alpha^{\mu}=\frac{(mu_{\mu})}{\sqrt{-u^{\mu}u_{\mu}}}\delta\alpha^{\mu}=p_{\mu}\delta\alpha^{\mu}\]
como $\delta Q=cte$ y $\delta\alpha^{\mu}=cte$, entonces
\[\frac{d(\delta Q)}{d\tau}=0\Rightarrow\frac{d}{d\tau}(mu_{\mu})=0\Rightarrow \frac{dp_{\mu}}{d\tau}=0\Rightarrow p_{\mu}=cte\]
siendo la cantidad conservada de las traslaciones.\\ \\
Por otro lado, si hacemos una transformada de Lorentz, $\delta x^{\mu}=\delta\omega_{\nu}^{\mu}x^{\nu}$, la cantidad conservada será,
\[\delta Q=\frac{\partial\mathscr{L}}{\partial\dot{x}^{\mu}}\delta x^{\mu}=p_{\mu}\delta\omega_{\nu}^{\mu}x^{\nu}=\frac{1}{2}\delta\omega_{\mu\nu}(x^{\mu}p^{\nu}-x^{\nu}p^{\mu})=\frac{1}{2}M^{\mu\nu}\delta\omega_{\nu\mu}\]
donde $M^{\mu\nu}=(x^{\mu}p^{\nu}-x^{\nu}p^{\mu})$ representa el momento angular. Además, vemos que $\dot{M}^{\mu\nu}=0$, por lo que $M^{\mu\nu}=cte$, entonces la cantidad conservada de las transformaciones de Lorentz es el momento angular.
\subsection{Sistema de N-partículas}
Si tenemos N-partículas, de forma general, tendremos un Lagrangiano tal que
\[\mathscr{L}(\tau,x_1{\mu},x_2^{\mu},\dots,x_N^{\mu},\dot{x}_1^{\mu},\dot{x}_2^{\mu},\dots,\dot{x}_N^{\mu})=\mathscr{L}(\tau,x_n^{\mu},\dot{x}_n^{\mu})\]
Análogamente al caso de una partícula, tendremos las ecuaciones de Euler-Lagrange,
\[\frac{\partial\mathscr{L}}{\partial x^{\mu}_n}-\frac{d}{d\tau}\left(\frac{\partial\mathscr{L}}{\partial \dot{x}^{\mu}_n}\right)=0\]
que serán N-ecuaciones.\\ \\
Para las cantidades conservadas, tendremos que
\begin{equation}
    \delta Q_{\delta x}=\sum_{n=1}^N\frac{\partial\mathscr{L}}{\partial\dot{x}_n^{\mu}}\delta x_n^{\mu}
\end{equation}
Para una transformación infinitesimal $(\delta x_1^{\mu},\dots,x_N^{\mu})$ que deje invariante el Lagrangiano, tendremos que $\delta Q_{\delta x}$ es conservado.\\ \\
En general tendremos,
\begin{itemize}
    \item Para la invariancia bajo traslaciones, se conserva el cuadrimomento.
    \begin{itemize}
        \item Si tenemos el Lagrangiano, $\mathscr{L}=\mathscr{L}(\tau,\dot{x}_n^{\mu})$; si hacemos traslaciones $\delta x_n^{\mu}$, el Lagrangiano queda invariante, luego se conserva el momento $p_n^{\mu}=\eta^{\mu\nu}\frac{\partial\mathscr{L}}{\partial\dot{x}^{\mu}}$ de la partícula.
    \item Si tenemos el Lagrangiano, $\mathscr{L}=\mathscr{L}(\tau,x_n^{\mu},\dot{x}_n^{\mu})$ que sea invariante bajo una traslación global cuya perturbación sea $\delta x_n^{\mu}=\delta\alpha^{\mu}=cte$, entonces el momento total se conserva, 
    \[P^{\mu}=\sum_{n=1}^Np_n^{\mu}\]
    pero no se conservan los momentos individuales. Donde $P^0$ representa la energía total del sistema (masas + energía cinética + energía potencial) y $P^i$ representa el momento lineal del sistema.
    \end{itemize}
    \item Para la invariancia bajo transformaciones de Lorentz tendremos dos casos,
    \begin{itemize}
        \item Si las partículas no interaccionan entre ellas, se conserva el cuadrimomento angular, 
        \[M^{\mu\nu}_n=x_n^{\mu}p_n^{\nu}-x_n^{\nu}p_n^{\mu}\]
        \item Si las partículas interaccionan, pero el sistema está aislado, entonces se conserva el cuadrimomento angular total, 
        \[M^{\mu\nu}=\sum_{n=1}^NM_n^{\mu\nu}\]
        pero no los individuales.
    \end{itemize}
\end{itemize}
%SECCION 5
\section{Campos relativistas} % Main chapter title
\label{cap2-sec11}
Un campo escalar relativista será,
\[\begin{array}{rlcl}
     \phi(x):&\mathbb{R}^4&\to&\mathbb{R}  \\
     & x^{\mu} & \mapsto & \lambda
\end{array}\]
Su acción es una función de la forma,
\[S=\int_{\mathscr{M}}d^4x\mathcal{L}(\phi,\partial_{\mu}\phi)\]
donde $\mathcal{L}$ es la densidad lagrangiana. Aplicamos el principio variacional se acción estacionaria bajo variaciones del campo que se anulan en la frontera $\partial\mathscr{M}$ de la variedad $\mathscr{M}$. Luego, tenemos la condición de que $\delta\phi|_{\partial\mathscr{M}}=0$, tal que
\[\delta S=\int_{\mathscr{M}}d^4x\left(\frac{\partial\mathcal{L}}{\partial\phi}\delta\phi+\frac{\partial\mathcal{L}}{\partial(\partial_{\mu}\phi)}\delta(\partial_{\mu}\phi)\right)\]
Imponemos que las variaciones sean del tipo $\delta(\partial_{\mu}\phi)=\partial_{\mu}(\delta\phi)$, pues solo estamos variando el campo y no las coordenadas, así la acción se transforma en,
\[\delta S=\int_{\mathscr{M}}d^4x\left(\frac{\partial\mathcal{L}}{\partial\phi}\delta\phi+\frac{\partial\mathcal{L}}{\partial(\partial_{\mu}\phi)}\partial_{\mu}(\delta\phi)\right)=\int_{\mathscr{M}}d^4x\left[\left(\frac{\partial\mathcal{L}}{\partial\phi}-\partial_{\mu}\left(\frac{\partial\mathcal{L}}{\partial(\partial_{\mu}\phi)}\right)\right)\delta\phi+\underbrace{\partial_{\mu}\left(\frac{\partial\mathcal{L}}{\partial(\partial_{\mu}\phi)}\delta\phi\right)}_{\int_{\partial\mathscr{M}}(...)\delta\phi=0}\right]=0\]
Por tanto, vemos que se cumplen las ecuaciones  de Euler-Lagrange para un campo escalar relativista,
\[\frac{\partial\mathcal{L}}{\partial\phi}-\partial_{\mu}\left(\frac{\partial\mathcal{L}}{\partial(\partial_{\mu}\phi)}\right)=0\]
\begin{example}
    Si tenemos un campo escalar sin masa cuya densidad lagrangiana es $\mathcal{L}=\partial_{\mu}\phi\partial^{\mu}\phi=\eta^{\mu\nu}\partial_{\mu}\phi\partial_{\nu}\phi$, tenemos que
    \[\frac{\partial\mathcal{L}}{\partial(\partial_{\mu}\phi)}=\frac{\partial}{\partial(\partial_{\mu}\phi)}\left(\partial_{\rho}\phi\partial_{\sigma}\phi\eta^{\rho\sigma}\right)=\delta_{\rho}^{\mu}\partial_{\sigma}\phi\eta^{\rho\sigma}+\delta_{\sigma}^{\mu}\partial_{\rho}\phi\eta^{\rho\sigma}=2\partial^{\mu}\phi\]
    \[\frac{\partial\mathcal{L}}{\partial\phi}=0\]
    Luego, aplicando la ecuación de Euler-Lagrange, tenemos 
    \[-\partial_{\mu}(\partial^{\mu}\phi)=0\]
    es decir, tenemos $\square\phi=0$.
\end{example}
\subsection{Cantidades conservadas}
Consideremos variaciones de la forma $x^{'\mu}=x^{\mu}+\delta x^{\mu}$, que induce $\phi'(x')=\phi(x)$. Entonces,
\[\phi'(x')=\phi'(x)+\delta x^{\mu}\partial_{\mu}\phi'(x)=\phi(x)\]
\[\phi'(x)-\phi(x)=-\delta x^{\mu}\partial_{\mu}\phi'(x)=-\delta x^{\mu}\partial_{\mu}\phi(x)\]
Si hacemos una transformación de coordenadas, vemos que
\[d^4x'=d^4x(1+\partial_{\mu}\delta x^{\mu})\]
y que
\[\mathcal{L}'=\mathcal{L}+\delta x^{\mu}\partial_{\mu}\mathcal{L}+\frac{\partial\mathcal{L}}{\partial\phi}(\cancelto{-\delta x^{\mu}\partial_{\mu}\phi}{\phi'(x)-\phi(x)})+\frac{\partial\mathcal{L}}{\partial(\partial_{\mu}\phi)}(\cancelto{-(\delta x^{\nu}\partial_{\nu})(\partial_{\mu}\phi)}{\partial_{\mu}\phi'-\partial_{\mu}\phi})\]
Si introducimos todo en la acción, tenemos
\[\begin{array}{rl}
    \delta S &=\int d^4x\left[\mathcal{L}(\partial_{\mu}\delta x^{\mu})+\delta x^{\mu}\partial_{\mu}\mathcal{L}-\frac{\partial\mathcal{L}}{\partial\phi}(\delta x^{\mu}\partial_{\mu}\phi)-\frac{\partial\mathcal{L}}{\partial(\partial_{\mu}\phi)}(\delta x^{\nu}\partial_{\nu})(\partial_{\mu}\phi)\right]=  \\
     & =\int d^4x\left[\mathcal{L}(\partial_{\mu}\delta x^{\mu})+\delta x^{\mu}\partial_{\mu}\mathcal{L}-\partial_{\mu}\left(\frac{\partial\mathcal{L}}{\partial(\partial_{\mu}\phi)}\right)(\delta x^{\mu}\partial_{\mu}\phi)-\frac{\partial\mathcal{L}}{\partial(\partial_{\mu}\phi)}(\delta x^{\nu}\partial_{\nu})(\partial_{\mu}\phi)\right]= \\
     & =\int d^4x\partial_{\mu}\left[\left(\delta_{\nu}^{\mu}\mathcal{L}-\frac{\partial\mathcal{L}}{\partial(\partial_{\mu}\phi)}\partial_{\nu}\phi\right)\delta x^{\nu}\right]     
\end{array}\]
Si $\delta x^{\mu}$ es una traslación constante, $\delta x^{\mu}=\delta\alpha^{\mu}$, entonces tenemos que
\[T_{\nu}^{\mu}=\delta_{\nu}^{\mu}\mathcal{L}-\frac{\partial\mathcal{L}}{\partial(\partial_{\mu}\phi)}\partial_{\nu}\phi\]
es conservado y se denomina \textit{tensor de energía-impulso}, que tiene el contenido energético del campo de forma local.\\ \\
Además, como $\delta S=0$, entonces tendremos que $\partial_{\mu}T_{\nu}^{\mu}=0$, por lo que es una ley de conservación. Este tensor es simétrico y toda densidad lagrangiana, con suficiente simetría, tiene asociado un tensor de energía-impulso que cumple que $\partial_{\mu}T_{\nu}^{\mu}=0$.







    \chapter{Geometría Diferenciable} %
\label{Capitulo3} %
\lhead{\emph{Geometría Diferenciable}}
\textit{“Primero tienes que aprender las reglas del juego, y después jugar mejor que nadie”.}\\
(A. Einstein)
\newpage
%-------------------------------------------------------------------------------
%SECCION 1
\section{Repaso histórico} % Main chapter title
\label{cap2-sec1} 
%------------------------------------------------------------------------------
	La física clásica, del siglo XIX, era una física bien asentada. La cuál explica la mecánica con el libro de Sir Isaac Newton titulado \textit{Philosophiae Naturalis Principia Mathematica} y el electromagnetismo se explica con el libro de Maxwell titulado \textit{Electricity and Magnetism}.\\
 En 1887, Michelson y Morley iniciaron una revolución en la física con un experimento para medir la velocidad de la luz. El experimento consistía en medir la velocidad de la luz de un rayo paralelo al eje de rotación de la Tierra y de otro rayo perpendicular a este, esperándose obtener resultados diferentes. En cambio, se observó que ambos rayos iban exactamente igual, cosa que no tenía sentido en la época., por tanto, determinaron que la velocidad de la luz no era instantánea, sino que debía ser finita, y llegaron a un resultado de ésta bastante próximo al valor actual de la velocidad de la luz.
 \subsection{Relatividad Galileana}
 El Principio de Relatividad de Galileo establece que,
 \begin{center}
 \textit{''Es imposible determinar a base de experimentos (mecánicos) si un sistema de referencia está en reposo o en movimiento uniforme y rectilíneo''.}
 \end{center}
 Esto se derivó de que en la Relatividad Galileana hay un espacio absoluto en el que las leyes de Newton son ciertas. Definiremos un \textit{sistema de referencia inercial} (SRI) como aquel sistema referencia que se mueve a velocidad constante respecto al espacio absoluto. Además, todos los sistemas de referencia inerciales comparten un tiempo absoluto. Pero con la definición de SRI, el Principio de Relatividad se debe reformular con este concepto, así tenemos el Principio de Relatividad en formulación de equivalencia, que dice que
 \begin{center}
 \textit{''Todos los sistemas inerciales son equivalentes, es decir, todos los observadores inerciales ven la misma física''.}
 \end{center}
 \textbf{Leyes de Newton}\\ \\
 La Ley de Newton por excelencia es $\vec{F}=m\vec{a}=-\nabla V(\vec{r}-\vec{r}_0)$, donde $V$ es la función potencial. Esta ley (y las demás) transforman bien bajo el grupo de transformaciones de Galileo, que son:
 \begin{enumerate}
     \item \textbf{Traslaciones temporales:}
     \[t\to t'=t+t_0\]
     \item \textbf{Traslaciones espaciales:}
     \[\vec{r}\to\vec{r}'=\vec{r}+\vec{r}_i+\vec{v}t\]
     donde $\vec{v}$ es la velocidad relativa de un SRI con respecto al otro, y $\vec{r}_i$ es el vector de posición entre los orígenes de ambos SRI al inicio.
     \item \textbf{Rotaciones espaciales:}
     \[\vec{a}'=R(\theta)\vec{a}\]
     donde $R(\theta)$ es la matriz de rotación.
\end{enumerate}
Se puede ver que las Leyes de Newton no son covariantes, pero sí transforman bien, pues la física se mantiene, esto quiere decir que \textit{las Leyes de Newton de la física transforman de forma covariante}.\\ \\
El grupo de transformaciones de Galileo son simetrías que dan lugar a cantidades conservadas. Por tanto, si tenemos un Lagrangiano que sea invariante bajo traslaciones temporales, tendremos que el sistema conserva energía; si es invariante bajo traslaciones espaciales, conserva momento lineal; y si es invariante bajo rotaciones espaciales; conserva momento angular.\\ \\
El grupo de transformaciones de Galileo NO deja invariante las ecuaciones de Maxwell, que son
\[(i)\hspace{2mm}\nabla\cdot\vec{E}=\rho/\epsilon_0;\hspace{5mm}(iii)\hspace{2mm}\nabla\cdot\vec{B}=0\]
\[(ii)\hspace{2mm}\nabla\times\vec{B}=\partial_t\vec{E}/c^2+\mu_0\vec{J};\hspace{5mm}(iv)\hspace{2mm}\nabla\times\vec{E}=-\partial_t\vec{B}\]
Si $\rho=0$ y $\vec{J}=0$, es decir, estamos en vacío, podemos combinar las ecuaciones de Maxwell en una sola ecuación de ondas que se propaga a velocidad $c=299792,458$ m/s, resultado muy próximo al valor obtenido por Michelson y Morley, que además es independiente del sistema de referencia.
\subsection{Transformaciones de Lorentz}

Las transformaciones de Lorentz hacen que las ecuaciones de Maxwell transformen bien (sean covariantes). Estas transformaciones son:
\[\begin{array}{rcrc}
    (i) & t'=\gamma\left(t-\frac{v}{c^2}x\right); & (iii) & y'=y \\
    (ii) & x'=\gamma\left(x-vt\right); & (iv) & z'=z
\end{array}\]
donde $v$ es la velocidad relativa entre SRI (que suponemos que se mueven en el eje $X$), y $\gamma=\frac{1}{\sqrt{1-\frac{v^2}{c^2}}}$.\\ \\
Como estas transformaciones hacen que las leyes de Maxwell sean covariantes, diremos que las transformaciones de Lorentz sean más fundamentales que las transformaciones de Galileo.\\ \\
Además, vemos que por la transformación $(i)$ el tiempo ya \textbf{no es absoluto}, sino que depende del SRI, por lo que diremos que el tiempo es \textbf{relativo}.
%SECCION 2
\section{Álgebra de Tensores} % Main chapter title
\label{cap1-sec2} 
Llegamos a lo groso del capítulo, el \textbf{Álgebra de Tensores}. En este apartado vamos a ver qué es un tensor de forma matemática y cómo trabajar con ellos. También se mencionará cómo trabajamos los físicos con los tensores.
%------------------------------------------------------------------------------

\subsection{Producto tensorial: caso de dos términos} % Main chapter title
\label{cap1-sec2-subsec1} 
Vamos a ver qué es el \textbf{producto tensorial} y cómo los tensores se definen a partir de este.
\begin{proposition}
    Sea $V$ un $\mathbb{K}$-espacio vectorial, $\scalar{\cdot}{\cdot}$ el producto escalar euclídeo y $B=\curlybraces{v_1,\dots,v_n}$ base de $V$, 
    \[\begin{array}{cccl}
        f_v: & V & \to & V^*\\
         & v & \mapsto & f_v(v)=\scalar{v}{\cdot}
    \end{array}\]
     $f_v$ es una aplicación lineal, concretamente es un isomorfismo.
\end{proposition}
\begin{proof}
    Vemos que $f_v$ es aplicación lineal,
    \[f_v(w_1+w_2)=\scalar{v}{w_1+w_2}=\scalar{v}{w_1}+\scalar{v}{w_2}=f_v(w_1)+f_v(w_2)\checkmark \]
    \[f_v(\lambda\cdot w)=\scalar{v}{\lambda\cdot w}=\lambda\scalar{v}{w}=\lambda f_v(w)\checkmark\]
    para $\forall\lambda\in\mathbb{K}$ y $\forall w_1,w_2,w\in V$. Luego, es aplicación lineal.\\ \\
    Veamos que es isomorfo demostrando que es biyectivo, pues ya hemos visto que es aplicación lineal.\\
    Sabemos que $ker\curlybraces{f_v}=\curlybraces{0}\Leftrightarrow f_v$ es inyectiva. Luego, vemos si $ker\curlybraces{f_v}=\curlybraces{0}$:
    \[ker\curlybraces{f_v}=\curlybraces{w\in V,f_v(w)=0}=\curlybraces{w\in V;\scalar{v}{w}=0\Leftrightarrow w=0}\]
    Por tanto, $ker\curlybraces{f_v}=\curlybraces{0}$ y así, $f_v$ es inyectiva. $\checkmark$\\ \\
    Usando el Primer Teorema de isomorfía, tenemos que $dim(V)=\cancelto{0}{dim(ker\curlybraces{f_v})}+dim(Im f_v)$, pero como la $dim B=dim B^*$, siendo $B$ base de $V$ y $B^*$ base de $V^*$, entonces $dimV=dimV^*$, y por tanto, $dimV=dimImf_v=dimV^*$, luego $Imf_v$ es $V^*$ y por tanto, $f_v$ es sobreyectiva. $\checkmark$\\
    Luego, $f_v$ es un isomorfismo.
\end{proof}
\noindent Veamos cómo se define el producto tensorial y sus propiedades.
\begin{definition}
    Sea $V$ un $\mathbb{K}$-espacio vectorial, $V^*$ el dual de $V$, y $g^1,g^2\in V^*$ aplicaciones lineales, tal que $g^1:V\to\mathbb{K}$ y $g^2:V\to\mathbb{K}$. Así, definimos el producto tensorial como,
    \begin{enumerate}[label=(\roman*)]
        \item Producto tensorial entre dos formas $g^1,g^2\in V^*$,
        \[\begin{array}{cccl}
            \ptensor{g^1}{g^2}: & V\times V & \to & \mathbb{K}\\
            & (v,w) & \mapsto & g^1(v)g^2(w)
        \end{array}\]
        \item Producto tensorial entre dos vectores $v_1,v_2\in V$,
        \[\begin{array}{cccl}
            \ptensor{v_1}{v_2}: & V^*\times V^* & \to & \mathbb{K}\\
             & (f,g) & \mapsto & f(v_1)g(v_2)
        \end{array}\]
        \item Producto tensorial de una forma y un vector $v_1\in V$, $f^1\in V^*$,
        \[\begin{array}{cccl}
            \ptensor{v_1}{f^1}: & V^*\times V & \to & \mathbb{K}\\
             & (g,w) & \mapsto & g(v_1)f^1(w)
        \end{array}\]
    \end{enumerate}
\end{definition}

\begin{proposition}
    Los productos tensoriales definidos anteriormente son formas bilineales.
\end{proposition}
\begin{proof} 
Usando $\forall v_1,v_2,u_1,u_2,v,w,u\in V$, $\forall f^1,f^2,g,p,q\in V^*$ y $\forall \lambda\in\mathbb{K}$,
    \begin{enumerate}[label=(\roman*)]
        \item \[\begin{array}{cccl}
            \ptensor{f^1}{f^2}: & V\times V & \to & \mathbb{K}\\
            & (v,w) & \mapsto & f^1(v)f^2(w)
        \end{array}\]
        siendo $f^1,f^2\in V^*$. Veamos que es forma bilineal,
        \[\begin{array}{lrl} \text{\textbullet)} &(\ptensor{f^1}{f^2})(u_1+u_2,v)=&f^1(u_1+u_2)f^2(v)=\brackets{f^1(u_1)+f^1(u_2)}f^2(v)\\ &=&f^1(u_1)f^2(v)+f^1(u_2)f^2(v)=(\ptensor{f^1}{f^2})(u_1,v)+(\ptensor{f^1}{f^2})(u_2,v),\checkmark\\  \text{\textbullet)} &(\ptensor{f^1}{f^2})(v,u_1+u_2)  =&f^1(v)f^2(u_1+u_2)=f^1(v)\brackets{f^2(u_1)+f^2(u_2)}\\ &=&f^1(v)f^2(u_1)+f^1(v)f^2(u_2)=(\ptensor{f^1}{f^2})(v,u_1)+(\ptensor{f^1}{f^2})(v,u_2)\checkmark\\
             \text{\textbullet)} & (\ptensor{f^1}{f^2})(\lambda v,u) =& f^1(\lambda v)f^2(u)=\lambda f^1(v)f^2(u)=\lambda(\ptensor{f^1}{f^2})(v,u)\checkmark\\
        \text{\textbullet)}&(\ptensor{f^1}{f^2})(u,\lambda v)=&f^1(u)f^2(\lambda v)=\lambda f^1(u)f^2(v)=\lambda(\ptensor{f^1}{f^2})(u,v)\checkmark
          \end{array}\]
        Luego, $\ptensor{f^1}{f^2}$ es una forma bilineal. $\qedh $
        \item \[\begin{array}{cccl}
            \ptensor{v_1}{v_2}: & V^*\times V^* & \to & \mathbb{K}\\
             & (f,g) & \mapsto & f(v_1)g(v_2)
        \end{array}\]
         \[\begin{array}{lrl}
         \text{\textbullet)}&(\ptensor{v_1}{v_2})(f^1+f^2,g)=&(f^1+f^2)(v_1)g(v_2)=\brackets{f^1(v_1)+f^2(v_1)}g(v_2)\\
         &=&f^1(v_1)g(v_2)+f^2(v_1)g(v_2)=(\ptensor{v_1}{v_2})(f^1,g)+(\ptensor{v_1}{v_2})(f^2,g)\checkmark\\
         \text{\textbullet)}&(\ptensor{v_1}{v_2})(g,f^1+f^2)=&g(v_1)(f^1+f^2)(v_2)g=g(v_1)\brackets{f^1(v_2)+f^2(v_2)}\\
         &=&g(v_1)f^1(v_2)+g(v_1)f^2(v_2)=(\ptensor{v_1}{v_2})(g,f^1)+(\ptensor{v_1}{v_2})(g,f^2)\checkmark\\
         \text{\textbullet)}&(\ptensor{v_1}{v_2})(\lambda f,g)=&(\lambda f)(v_1)g(v_2)=\lambda f(v_1)g(v_2)=\lambda(\ptensor{v_1}{v_2})(f,g)\checkmark\\
         \text{\textbullet)}&(\ptensor{v_1}{v_2})(g,\lambda f)=&g(v_1)(\lambda f)(v_2)=\lambda g(v_1)f(v_2)=\lambda(\ptensor{v_1}{v_2})(g,f)\checkmark
         \end{array}\]
        Luego, $\ptensor{v_1}{v_2}$ es una forma bilineal. $\qedh $
        \item \[\begin{array}{cccl}
            \ptensor{v_1}{f^1}: & V^*\times V & \to & \mathbb{K}\\
             & (g,w) & \mapsto & g(v_1)f(w)
        \end{array}\]
        \[\begin{array}{lrl}
        \text{\textbullet)}&(\ptensor{v_1}{f^1})(p+q,w)=&(p+q)(v_1)f^1(w)=\brackets{p(v_1)+q(v_1)}f^1(w)=\\
        &=&p(v_1)f^1(w)+q(v_1)f^1(w)=(\ptensor{v_1}{f^1})(p,w)+(\ptensor{v_1}{f^1})(q,w)\checkmark\\
        \text{\textbullet)}&(\ptensor{v_1}{f^1})(g,u+w)=&g(v_1)f^1(u+w)=g(v_1)\brackets{f^1(u)+f^1(w)}=\\
        &=&g(v_1)f^1(u)+g(v_1)f^1(w)=(\ptensor{v_1}{f^1})(g,u)+(\ptensor{v_1}{f^1})(g,w)\checkmark\\
        \text{\textbullet)}&(\ptensor{v_1}{f^1})(\lambda g,w)=&(\lambda g)(v_1)f^1(w)=\lambda g(v_1)f^1(w)=\lambda(\ptensor{v_1}{f^1})(g,w)\checkmark\\
        \text{\textbullet)}&(\ptensor{v_1}{f^1})(g,\lambda w)=&g(v_1)f^1(\lambda w)=\lambda g(v_1)f^1(w)=\lambda(\ptensor{v_1}{f^1})(g,w)\checkmark
        \end{array}\]
            Luego, $\ptensor{v_1}{f^1}$ es una forma bilineal. \qedhere
    \end{enumerate}
\end{proof}
\noindent El producto tensorial no se da solo entre elementos de los espacios vectoriales o duales, sino que también se puede dar entre espacios, siendo el nuevo espacio generado un \textbf{espacio vectorial}.
\begin{proposition}
    El espacio $\ptensor{V}{V}$ tiene estructura de espacio vectorial.
\end{proposition}
\begin{proof}
    \begin{enumerate}
        \item Vemos que $(\ptensor{V}{V},+)$ es grupo abeliano:
        \begin{enumerate}[label=(\roman*)]
            \item Vemos si la operación $+$ es cerrada:
            \\
            $\forall v,w,z\in V$ con $\ptensor{v}{w},\ptensor{v}{z},\ptensor{w}{z}\in\ptensor{V}{V}$, tenemos que ver si $\ptensor{(v+w)}{z}\in\ptensor{V}{V}$. Sabemos que,
            \[\begin{array}{cccl}
                \ptensor{v}{w}: & \pcart{V^*}{V^*} & \to &\mathbb{R}  \\
                 & (f,g) & \mapsto & f(v)g(w)
            \end{array}\]
            luego,
            \[\begin{array}{cccl}
                \ptensor{(v+w)}{z}: & \pcart{V^*}{V^*} & \to &\mathbb{R}  \\
                 & (f,p) & \mapsto & f(v+w)p(z)
            \end{array}\]
            Entonces,
            \[(\ptensor{(v+w)}{z})(g,p)=f(v+w)p(z)=\brackets{f(v)+f(w)}p(z)=\]\[=f(v)p(z)+f(w)p(z)=(\ptensor{v}{z})(f,p)+(\ptensor{w}{z})(f,p)\]
            Luego, $\ptensor{(v+w)}{z}\in\ptensor{V}{V}$ y así, la operación $+$ es cerrada. $\checkmark$
            \item Asociatividad:
            \\
            Sean $\ptensor{a}{b},\ptensor{c}{d},\ptensor{e}{f}\in\ptensor{V}{V}$, tenemos que ver si $\ptensor{a}{b}+\brackets{\ptensor{c}{d}+\ptensor{e}{f}}=\brackets{\ptensor{a}{b}+\ptensor{c}{d}}+\ptensor{e}{f}$, tal que
            \[(\ptensor{a}{b}+\brackets{\ptensor{c}{d}+\ptensor{e}{f}})(p,q)=p(a)q(b)+\brackets{p(c)q(d)+p(e)q(f)}=p(a)q(b)+p(c+e)q(d+f)=\]
            \[=p(a+c+e)q(b+d+f)=p(a+c)q(b+d)+p(e)q(f)=\brackets{p(a)q(b)+p(c)q(d)}+p(e)q(f)=\]\[=(\brackets{\ptensor{a}{b}+\ptensor{c}{d}}+\ptensor{e}{f})(p,q)\checkmark\]
            \item Elemento neutro:\\
            Sea $\ptensor{e_1}{e_2}\in\ptensor{V}{V}$ el elemento neutro de $\ptensor{V}{V}$, tal que
            \[\ptensor{e_1}{e_2}+\ptensor{v}{w}=\ptensor{v}{w}+\ptensor{e_1}{e_2}=\ptensor{v}{w}\]
            Vemos el valor de este elemento neutro,
            \[(\ptensor{e_1}{e_2}+\ptensor{v}{w})(f,g)=(\ptensor{v}{w})(f,g)\]
            \[f(e_1)g(e_2)+f(v)+g(w)=f(v)g(w)\]
            \[f(e_1+v)g(e_w+w)=f(v)g(w)\Leftrightarrow\left\lbrace\begin{matrix}
                e_1=0\\
                e_2=0
            \end{matrix}\right.\]
            luego, $\ptensor{e_1}{e_2}=0$. $\checkmark$
            \item Elemento simétrico:
            \\
            $\forall\ptensor{v}{u}\in\ptensor{V}{V}$, $\exists\ptensor{\Tilde{v}}{\Tilde{u}}\in\ptensor{V}{V}$, tal que
            \[\ptensor{v}{u}+\ptensor{\Tilde{v}}{\Tilde{u}}=\ptensor{\Tilde{v}}{\Tilde{u}}+\ptensor{v}{u}=\ptensor{e_1}{e_2}=0\]
         Veamos quién es $\ptensor{\Tilde{v}}{\Tilde{u}}$,
        \[(\ptensor{v}{u}+\ptensor{\Tilde{v}}{\Tilde{u}})(f,g)=f(v)g(u)+f(\Tilde{v})g(\Tilde{u})=(\ptensor{0}{0})(f,g)=f(0)g(0)\]
        luego,
        \[v+\Tilde{v}=0\Rightarrow\Tilde{v}=-v\]
        \[u+\Tilde{u}=0\Rightarrow\Tilde{u}=-u\]
        Por tanto, el elemento simétrico de $\ptensor{v}{u}$ es $\ptensor{(-v)}{(-u)}$. $\checkmark$
        \item Conmutabilidad:\\
        Sean $\ptensor{v}{w},\ptensor{u}{z}\in\ptensor{V}{V}$, entonces
        \[(\ptensor{v}{w}+\ptensor{u}{z})(f,g)=f(v)g(w)+f(u)g(z)=f(v+u)g(w+z)=\]\[=f(u+v)g(z+w)=f(u)g(z)+f(v)g(w)=(\ptensor{u}{z}+\ptensor{v}{w})(f,g)\checkmark\]
    Luego, es grupo abeliano. $\checkmark$
         \end{enumerate}
         \item Doble propiedad distributiva:
         \begin{enumerate}
             \item $\forall\lambda,\mu\in\mathbb{R}$, $\forall\ptensor{v}{w}\in\ptensor{V}{V}$,
             \[(\lambda+\mu)\cdot(\ptensor{v}{w})(f,g)=(\lambda+\mu)f(v)g(w)=\]\[=\lambda f(v)g(w)+\mu f(v)g(w)=\lambda(\ptensor{v}{w})(f,g)+\mu(\ptensor{v}{w})(f,g)\checkmark\]
             \item $\forall\lambda\in\mathbb{R}$, $\forall\ptensor{v}{w},\ptensor{u}{z}\in\ptensor{V}{V}$, tenemos que
             \[\lambda(\ptensor{v}{w})(f,g)+\lambda(\ptensor{u}{z})(f,g)=\lambda f(v)g(w)+\lambda f(u)g(z)=\]\[=\lambda\brackets{f(v)g(w)+f(u)g(z)}=\lambda(\ptensor{v}{w}+\ptensor{u}{z})(f,g)\checkmark\]
             \end{enumerate}
             \item Propiedad pseudo-asociativa:\\
             $\forall\lambda,\mu\in\mathbb{R}$; $\forall\ptensor{v}{w}\in\ptensor{V}{V}$, tenemos que
             \[\lambda\cdot\brackets{\mu\cdot(\ptensor{v}{w})(f,g)}=\lambda\brackets{\mu f(v)g(w)}=\lambda f(\mu v)g(\mu w)=\]\[=f(\lambda\mu v)g(\lambda\mu w)=f(\mu\lambda v)g(\mu\lambda w)=\mu\brackets{f(\lambda v)g(\lambda w)}=(\mu\cdot\lambda)f(v)g(w)=(\mu\cdot\lambda)(\ptensor{v}{w})(f,g)\checkmark\]
             \item Elemento unitario del cuerpo: $\forall\ptensor{v}{w}\in\ptensor{V}{V}$; $\Tilde{\mu}\in\mathbb{R}$, entonces $\Tilde{\mu}\cdot\ptensor{v}{w}=\ptensor{v}{w}\cdot\Tilde{\mu}=\ptensor{v}{w}$
             \[(\Tilde{\mu}\cdot\ptensor{v}{w})(f,g)=f(\Tilde{\mu}v)g(\Tilde{\mu}w)=(\ptensor{v}{w})(f,g)=f(v)g(w)\Rightarrow\begin{matrix}
                 \Tilde{\mu}\cdot v=v\\
                 \Tilde{\mu}\cdot w=w
             \end{matrix}\Leftrightarrow\Tilde{\mu}=1\checkmark\]
       \end{enumerate}
       Luego, $(\ptensor{V}{V}, +, \cdot)$ es un $\mathbb{R}$-espacio vectorial.
\end{proof}
\noindent Al igual que cualquier otro espacio vectorial, el espacio $V\otimes V$ deberá tener una \textbf{base}.
\begin{proposition}
    Si tenemos un $V$ espacio vectorial sobre $\mathbb{K}$ con base $B=\curlybraces{v_1,\dots,v_n}$, entonces todo $\ptensor{v}{w}$ será combinación lineal de los elementos de la base de $\ptensor{V}{V}$ dada por $\ptensor{B}{B}=\curlybraces{\ptensor{v_i}{v_j}}_{i,j=1}^{n}$
\end{proposition}
\begin{proof}
    Queremos ver que $\curlybraces{\ptensor{v_i}{}v_j}_{i,j=1}^n$ es base de $\ptensor{V}{V}$. Para ello, tendremos que ver que esta base $\ptensor{B}{B}$ complete el espacio $\ptensor{V}{V}$ y que los vectores de la misma sean linealmente independientes.\\
    Sabemos que $\ptensor{v}{w}\in\ptensor{V}{V}$ y que
    \[\begin{array}{cccl}
        \ptensor{v}{w}: & \pcart{V^*}{V^*} & \to & \mathbb{R}\\
         & (f,g) & \mapsto & f(v)g(w)
    \end{array}\]
    Luego, para que la base $\ptensor{B}{B}$ complete el espacio $\ptensor{V}{V}$, se deberá poder expresar cualquier vector $\ptensor{v}{w}\in\ptensor{V}{V}$ como combinación lineal de los vectores de $\ptensor{B}{B}$. Podemos usar $B=\curlybraces{v_i}_{i=1}^n$ base de $V$, tal que
    \[v=\sum\limits_{i=1}^n\lambda^iv_i=\lambda^iv_i,\hspace{4mm}w=\sum\limits_{j=1}^n\mu^jv_j=\mu^jv_j\]
    Por tanto, usando $f,g\in V^*$, tenemos que
    \[\ptensor{v}{w}(f,g)=f(v)g(w)=f(\lambda^iv_i)g(\mu^jv_j)=\lambda^if(v_i)\mu^jg(v_j)=\lambda^i\mu^jf(v_i)g(v_j)=\lambda^i\mu^j(\ptensor{v_i}{v_j})(f,g)\]
    Luego, hemos expresado un vector del espacio $\ptensor{V}{V}$ como combinación lineal de los vectores de la base $\ptensor{B}{B}$. $\checkmark$\\ \\
    Veamos que son linealmente independientes, para ello, se debe cumplir que,
    \[\sum\limits_{i,j=1}^n\lambda^{ij}(\ptensor{v_i}{v_j})=\lambda^{ij}(\ptensor{v_i}{v_j})=0\Leftrightarrow\lambda^{ij}=0\]
    Sabiendo que la base de $V^*$ es $B^*=\curlybraces{f^1,f^2,\dots,f^n}$, tal que
    \[f^i(v_i)=1\hspace{5mm}f^j(v_i)\overset{i\neq j}{=}0\Rightarrow f^i(v_j)=\delta_{ij}\]
    Podemos evaluar lo anterior en dos elementos arbitrarios de $B^*$, tal que
    \[0=\lambda^{ij}(\ptensor{v_i}{v_j})(f^n,f^m)= \lambda^{ij}f^n(v_i)f^m(v_j)=\lambda_{ij}\delta_{n}^i\delta_{m}^j=\lambda^{nm}\]
    luego, $\lambda^{nm}=0$ y por tanto, los vectores son linealmente independientes. $\checkmark$\\ \\
    Así, hemos demostrado que $\ptensor{B}{B}$ es base de $\ptensor{V}{V}$.
\end{proof}

\begin{note}
    Denotaremos $\ptensor{v}{w}\equiv h$, tal que
    \[
    \begin{array}{cccl}
        h: & \pcart{V^*}{V^*} & \to & \mathbb{R} \\
         & (f^i,f^j) & \mapsto & h(f^i,f^j)=h^{ij}
    \end{array}
    \]
    siendo $f^i,f^j\in B^*$. Por tanto, para dos $p,q\in V^*$ cualesquiera, escribiremos
    \[(\ptensor{v}{w})(p,q)=h(p,q)=h\left(\sum_{i=1}^np_if^i,\sum_{j=1}^nq_jf^j\right)=p_iq_j(f^i,f^j)=h^{ij}p_iq_j\]
\end{note}
\noindent Veamos algunas \textbf{propiedades} del producto tensorial.
\begin{proposition}
    Sea $V$ un $\mathbb{R}$-espacio vectorial,
    \begin{enumerate}[label=(\roman*)]
        \item $\ptensor{(v_1+v_2)}{w}=\ptensor{v_1}{w}+\ptensor{v_2}{w}$; $\forall v_1,v_2,w\in V$.
        \item $\ptensor{w}{(v_1+v_2)}=\ptensor{w}{v_1}+\ptensor{w}{v_2}$, $\forall v_1,v_2,w\in V$.
        \item $\ptensor{(\lambda v)}{w}=\lambda\ptensor{v}{w}$, $\forall v,w\in V$, $\forall\lambda\in\mathbb{R}$.
        \item $\ptensor{w}{(\lambda v)}=\lambda\ptensor{w}{v}$, $\forall v,w\in V$, $\forall \lambda\in\mathbb{R}$.
        \item $\ptensor{v}{w}\neq\ptensor{w}{v}$.
        \item $\ptensor{v}{w}\neq0$ si $v\neq0$ ó $w\neq 0$.
        \item Sea $\ptensor{a}{b}\neq0$, $\ptensor{a}{b}=\ptensor{a'}{b'}\Leftrightarrow a'=\lambda a$ y $b'=\lambda^{-1}b$.
        \item $\ptensor{V}{W}$ es isomorfo con $\ptensor{W}{V}$.
    \end{enumerate}
\end{proposition}
\begin{proof}
    \begin{enumerate}[label=(\roman*)]
        \item $\forall v_1,v_2,w\in V$,
        \[(\ptensor{(v_1+v_2)}{w})(f,g)=f(v_1+v_2)g(w)=\brackets{f(v_1+f(v_2)}g(w)=\]\[=f(v_1)g(w)+f(v_2)g(w)=(\ptensor{v_1}{w})(f,g)+(\ptensor{v_2}{w})(f,g)\qedh\]
        \item $\forall v_1,v_2,w\in V$,
        \[(\ptensor{w}{(v_1+v_2)})(f,g)=f(w)g(v_1+v_2)=f(w)\brackets{g(v_1)+g(v_2)}=\]\[=f(w)g(v_1)+f(w)g(v_2)=(\ptensor{w}{v_1})(f,g)+(\ptensor{w}{v_2})(f,g)\qedh\]
        \item $\forall v,w\in V$ y $\forall\lambda\in\mathbb{R}$,
        \[(\ptensor{(\lambda\cdot v)}{w})(f,g)=f(\lambda\cdot v)g(w)=\lambda\cdot f(v)g(w)=\lambda\cdot(\ptensor{v}{w})(f,g)\qedh\]
        \item $\forall v,w\in V$ y $\forall\mu\in\mathbb{R}$,
        \[(\ptensor{w}{(\lambda\cdot v)})(f,g)=f(w)g(\lambda\cdot v)=\lambda\cdot f(w)g(v)=\lambda\cdot(\ptensor{w}{v})(f,g)\qedh\]
        \item Vemos que, $(\ptensor{v}{w})(f,g)=f(v)g(w)$ y que $(\ptensor{w}{v})(f,g)=f(w)g(v)$, luego estos elementos serían iguales solo si $f\equiv g$. $\qedh$
        \item Sean $v,w\in V$ y $f,g\in V^*$, tales que $f\not\equiv0$ y $g\not\equiv0$, entonces
        \[(\ptensor{v}{w})(f,g)=f(v)g(w)=0\Leftrightarrow\begin{matrix}
            f(v)=0 & \Leftrightarrow v=0\\
            \text{ó} & \\
            g(w)=0 & \Leftrightarrow w=0
        \end{matrix}\qedh\]
        \item \begin{tabular}{c|}
             $\Rightarrow$ \\ \hline
        \end{tabular} 
        Sea $\ptensor{a}{b}=\ptensor{a'}{b'}$ entonces
        \[(\ptensor{a}{b})(f,g)=f(a)g(b)=(\ptensor{a'}{b'})(f,g)=f(a')g(b')\]
        luego,
        \[f(a)g(b)=f(a')g(b')\]
        pero como $a\neq a'$ y $b\neq b'$, debe haber una relación entre ambos, de tal forma que se cumpla la igualdad anterior. Supondremos que $a$ y $a'$ tienen una relación lineal (la más sencilla), tal que $a'=\lambda a+c$, luego 
        \[f(a)g(b)=f(a')g(b')=f(\lambda a+c)g(b')=f(\lambda a)g(b')+f(c)g(b')=\lambda f(a)g(b')+f(c)g(b')\]
        Agrupamos términos de la igualdad, tal que,
        \[0:\hspace{5mm}0=f(c)g(b')\]
        \[f(a):\hspace{5mm}g(b)=\lambda g(b')\]
        Por la propiedad \textit{(vi)}, como $b'\neq0$, entonces $c=0$. Además,
        \[g(b)=\lambda g(b')\Rightarrow g(b')=\lambda^{-1}g(b)\Rightarrow g(b')=g(\lambda^{-1}b)\Rightarrow b'=\lambda^{-1}b\]
        Luego,
        \[\begin{matrix}
            a'=\lambda a\\
            b'=\lambda^{-1}b
        \end{matrix}\hspace{4mm}\checkmark\]
        \begin{tabular}{c|}
             $\Leftarrow$ \\ \hline
        \end{tabular} 
        Sea $a'=\lambda a$ y $b'=\lambda^{-1}b$, entonces
        \[(\ptensor{a'}{b'})(f,g)=f(a')g(b')=f(\lambda a)g(\lambda^{-1}b)=\cancel{\lambda}\cancel{\lambda^{-1}}f(a)g(b)=(\ptensor{a}{b})(f,g)\checkmark\]
        \item Sean $V,W$ espacios vectoriales, tales que
        \[\begin{array}{ccl}
            \ptensor{V}{W} & \to & \ptensor{W}{V}  \\
            \ptensor{v}{w} & \mapsto & \ptensor{w}{v}
        \end{array}\]
        Si suponemos que $dimV=n$ y $dimW=m$, sabemos por tanto que $dim(V\otimes W)=n\cdot m$ y $dim(W\otimes V)=m\cdot n$, luego tienen la misma dimensión y por tanto, son isomorfos. $\checkmark$\\ \\
        También podemos hacerlo sin usar la proposición de que $dim(\ptensor{V}{W})=n\cdot m$. Es claro ver que la aplicación es inyectiva, pues no hay dos elementos con la misma imagen, ya que la imagen se forma al permutar los elementos. Luego, al ser inyectivo, tenemos que $dimKer=0$. Por el Primer Teorema de Isomorfía,
        \[dim(\ptensor{V}{W})=\cancelto{0}{dimKer}+dimIm=dimIm=dim(\ptensor{W}{V})\]
        Luego, como $\ptensor{V}{W}$ y $\ptensor{W}{V}$ tienen la misma dimensión, entonces son isomorfos.
    \end{enumerate}
\end{proof}

\subsection{Aplicaciones lineales} % Main chapter title
\label{cap1-sec1-subsec2} 

Veamos ahora las aplicaciones lineales.
\begin{definition}
    Sean $V$ y $V'$ dos espacios vectoriales sobre el mismo cuerpo $\mathbb{K}$.
    Se dice que en una aplicación $f:V\longrightarrow V'$ es una aplicación lineal, o también llamado homomorfismo de espacios vectoriales, si se verifica:
    \begin{enumerate}[label=(\roman*)]
        \item $f(x+y)=f(x)+f(y),\forall x,y\in V$
        \item $f(\lambda\cdot x)=\lambda\cdot f(x),\forall\lambda\in\mathbb{K},\forall x\in V$
    \end{enumerate}
    Diremos además que $f$ es un isomorfismo lineal si es biyectiva, que $f$ es un endomorfismo si $V=V'$ y que es un automorfismo si es un endomorfismo biyectivo. 
\end{definition}

\noindent Las aplicaciones lineales tienen asociados dos conjuntos cuyas características son de interés, a saber, el núcleo y la imagen.

\begin{definition}
    Sea $f:V\longrightarrow W$ definimos el núcleo o kernel de la aplicación $f$ como
    \[Kerf=\curlybraces{v\in V:f(v)=0}\]
    y la imagen como
    \[Imf=\curlybraces{w\in W:\exists v\in V/f(v)=w}.\]
\end{definition}

\noindent Veamos algunas propiedades básicas de ambos conjuntos.

\begin{proposition}
Sea $f:V\to V'$ una aplicación lineal, se tienen las siguientes propiedades:
\begin{enumerate}[label=(\roman*)]
    \item \label{prop1:item1} $\rm{Im}f$ es un subespacio de $V'$ y que $\rm{Ker}f$ es un subespacio de $V$.
    \item \label{prop1:item2}Si $W$ es un subespacio vectorial de $V$, entonces $f(W):=\curlybraces{f(w): w\in W}$ es un subespacio de $V'$.
    \item \label{prop1:item3}Si $W'$ es un subespacio de $V'$, entonces $f^{-1}(W'):=\curlybraces{v\in V: f(v)\in W'}$ es también un subespacio de $V$.
\end{enumerate}  
\end{proposition}
%\newpage
\begin{proof}
\begin{enumerate}[label=\ref{prop1:item1}]
    \item Por definición, como los elementos de la $\rm{Im}f$ son pertenecientes a $V'$, entonces la $\rm{Im}f$ es subespacio de $V'$. De igual forma ocurre con el $\rm{Ker}f$, pues sus elementos pertenecen a $V$ y por tanto, este es subespacio de $V$.
\end{enumerate}
\begin{enumerate}[label=\ref{prop1:item2}]
    \item Como $W$ es subespacio de $V$, tenemos que $w\in V$ también, por tanto, los $f(w)$ pertenecerán a $V'$, cosa que implica que $f(W)$ es subespacio de $V'$, pues los $f(w)$ de $f(W)$ pertenecen a $V'$.
\end{enumerate}
\begin{enumerate}[label=\ref{prop1:item3}]
    \item Por analogía a $\ref{prop1:item2}$ vemos que $f^{-1}(W)$ es subespacio de $V$.
\end{enumerate}
\end{proof}
\noindent Ahora veamos algunas propiedades esenciales de las aplicaciones lineaales.
\begin{proposition}
    Sea $f:V\longrightarrow V'$ una aplicación lineal,
    \begin{enumerate}[label=(\roman*)]
        \item \label{pro1:item1} entonces $f$ es inyectiva si y solo si $Kerf=\curlybraces{0}$.
        \item \label{pro1:item2} si $G$ es un conjunto generador de $V$, $<G>=V$, entonces $f(G)$ es conjunto generador
        de $Imf$, $<f(G)>=Imf$.
        \item \label{pro1:item3} si $S\subset V$ es un conjunto de vectores linealmente independientes, si $f$ es inyectiva, entonces $f(S)$ es linealmente independiente.
        \item \label{pro1:item4} $f$ es inyectiva $\Leftrightarrow$ conserva la independencia lineal.
   
        \item \label{pro1:item5} si $f$ es biyectiva y $B$ es una base de $V$, entonces $f(B)$ es base de $V'$.

        \item \label{pro1:item6} $f$ es sobreyectiva $\Leftrightarrow$ $Imf=V'$
    \end{enumerate}
\end{proposition}



\begin{proof}
\ref{pro1:item1} \begin{tabular}{c|}
                 $\Rightarrow$ \\ \hline
            \end{tabular}
            Suponiendo que $f$ es inyectiva, sabemos que su Kernel es,
            \[\rm{Ker}f=\curlybraces{v\in V:f(v)=0}\]
            pero como la inyectividad nos implica que la imagen debe provenir de un único vector de entrada, entonces este vector será $v=0$, y por tanto, $ker f=\curlybraces{0}$. $\checkmark$
\\     
            \begin{tabular}{c|}
                 $\Leftarrow$  \\ \hline
            \end{tabular}
            Suponiendo que $ker f=\curlybraces{0}$, esto nos quiere decir que únicamente el vector $v=0$ satisface $f(v)=0$, luego como un vector tiene una única imagen, decimos que $f$ es inyectiva. \qedh
\\ \\
\ref{pro1:item2}  Veamos que el conjunto $f(G)$ es sistema generador de la imagen, es decir,   \[<f(G)>=Imf\Leftrightarrow\forall y\in Imf,\exists\lambda^1,\dots,\lambda^n\in\mathbb{K},y_1,\dots,y_n\in f(G)\text{ tales que }y=\lambda^1y_1+\dots+\lambda^ny_n.\] Sabemos que $G$ es conjunto generador, luego sea $y\in Imf$. Entonces por definición se tiene que existe $x\in V$ tal que $f(x)=y$. Como $<G>=V$, existen $\lambda^1,\dots,\lambda^n\in\mathbb{K}$, $v_1,\dots,v_n\in G$ tales que \[ x= \sum_{i=1}^n \lambda^i v_i.\] Tenemos entonces que:  \[y=f(x)=f(\lambda^1v_1+\dots+\lambda^nv_n)=\lambda^1f(v_1)+\dots+\lambda^nf(v_n).\] Por lo tanto, $y$ es combinación lineal de elementos de $f(G)$, es decir, $<f(G)>=\mathrm{Im}f$. \qedh
\\ \\
\ref{pro1:item3}
            Sea $S$ un conjunto linealmente independiente en $V$. 
            Supongamos que $f$ es inyectiva, vamos a probar que $f(S)$ es linealmente independiente, es decir,
            \[\lambda^1y_1+\dots+\lambda^ny_n=0\Rightarrow\lambda^1=\lambda^2=\dots=\lambda^n=0\hspace{4mm} \forall y_1,\dots,y_n\in f(S),\hspace{2mm}
                \forall\lambda^1,\dots,\lambda^n\in\mathbb{K}\]
            Supongamos $\lambda^1y_1+\dots+\lambda^ny_n=0$.  Como $y_j\in f(S),\exists x_j\in S/f(x_j)=y_j$.
             \[\left.\begin{array}{r}
                \lambda_1f(x_1)+\dots+\lambda_nf(x_n)=0\\
                f(\lambda_1x_1+\dots+\lambda_nx_n)=0
            \end{array}\right\rbrace f(0)=0\Rightarrow\lambda_1x_1+\dots+\lambda_nx_n=0\Rightarrow f\text{ inyectiva}\Rightarrow\lambda_1=\dots=\lambda_n=0\]
                \\ \\
                \ref{pro1:item4} \begin{tabular}{c|}
                 $\Longrightarrow$ \\ \hline
            \end{tabular} 
            Trivial por (iii) $\checkmark$\\
            \begin{tabular}{c|}
                 $\Longleftarrow$ \\ \hline
            \end{tabular} 
            Por reducción al absurdo:\\
            Supongamos que existen $v_1,v_2\in V$ distintos, tales que $f(v_1)=f(v_2)\Leftrightarrow f(v_1)-f(v_2)=0\Leftrightarrow f(v_1-v_2)=0$. 
            Luego, $v=v_1-v_2\neq0$ verifica que $f(v)=0$, $\curlybraces{v}$ es un conjunto linealmente independiente, $f(\curlybraces{v})$ tendría que ser un conjunto l.i. por hipótesis, pero $f(\curlybraces{v})=\curlybraces{0}$ que no es un conjunto l.i. cosa absurda. \qedh
            \\ \\
            \ref{pro1:item5}  Sea una aplicación lineal biyectiva $f:V\to V'$
            y una base de $V$, $B=\curlybraces{v_1,\dots,v_n}$. Entones, si aplicamos
\[f(B)=\curlybraces{f(v_1),\dots,f(v_n)}=\curlybraces{v_1',\dots,v_n'}\]
            y entonces, estos $v_i'\in V'$ van a formar una base de $V'$, pues al ser $f$ biyectiva, los vectores serán linealmente independientes, pues los de $B$ lo son; y además, como tienen la misma dimensión que $V'$, pasan de ser conjunto generador a base. $\qedh$
            \\ \\
            \ref{pro1:item6} \begin{tabular}{c|}
                 $\Rightarrow$  \\ \hline
            \end{tabular}
            Suponiendo que $f$ es sobreyectiva, tendremos que para cada $y\in V'$, existe al menos un $x\in V$, tal que $f(x)=y$. Por consiguiente, cada elemento de $V'$ es la imagen de un elemento de $V$, es decir, $Imf=V'$. $\checkmark$\\
            \begin{tabular}{c|}
                 $\Leftarrow$  \\ \hline
            \end{tabular}
            Suponiendo que $imf=V'$, tenemos que todos los elementos de $V'$ son imagen de los elementos de $V$, siendo esta la propia definición de sobreyectividad, luego $f$ es sobreyectiva.
\end{proof}
\noindent Una vez visto estas propiedades, de $\ref{pro1:item5}$ podemos obtener un resultado interesante, que es la siguiente proposición.
\begin{proposition}
Sea $B=\curlybraces{v_1,\dots,v_n}$ una base de $V$, y sea $f:V\rightarrow V'$ una aplicación lineal. Se tiene entonces que $\curlybraces{f(v_1),\dots,f(v_n)}$ es un sistema generador de la imagen.
\end{proposition}
\begin{proof}
    Supongamos que $B$ es una base y que conocemos $f(v_j),\forall v_j\in B$. 
    Sea $v\in V$, escrito en coordenadas de la base como $v=\lambda^1v_1+\dots+\lambda^nv_n$, con $v_i\in B$, 
 y $\lambda^i\in\mathbb{K}$, entonces  $f(x)=f(\lambda^1v_1+\dots+\lambda^nv_n)=\lambda^1f(v_1)+\dots\lambda^nf(v_n)$, luego hemos puesto $f(x)$ en coordenadas de $\curlybraces{f(v_1,\dots,f(v_n)}$.
\end{proof}

\noindent Ahora vamos a ver un resultado bastante importante, el cuál nos permitirá representar aplicaciones lineales en matrices, denominadas \textbf{matrices asociadas a la aplicación $f$}. Además, este resultado es importante para Física, pues los físicos no solemos trabajar con aplicaciones, sino que trabajamos con sus matrices asociadas, pues se puede decir que "tienen" la misma información que las aplicaciones.

\begin{proposition}  
\label{prop1.4}
    Sean $(V,+,\cdot)$ y $(V',+,\cdot)$ $\mathbb{K}$-espacios vectoriales de dimensión finita con $dimV=n$ y $dimV'=m$. 
    Sea $f:V\longrightarrow V'$ una aplicación lineal, entonces dadas $\left\lbrace\begin{matrix}
        B=\curlybraces{v_1,\dots,v_n}\text{ base de }V\\
        B'=\curlybraces{v_1',\dots,v_n'}\text{ base de }V'
    \end{matrix}\right.$\\
    $f$ se representa en esas bases como una matriz en $\mathcal{M}_{m\times n}(\mathbb{K})$.
\end{proposition}

\begin{proof}
    Como $f$ es lineal, me basta con conocer $f(B)$, para ello, tenemos que conocer $f(v_1),f(v_2),\dots,f(v_n)$, teniendo:
    \[\begin{matrix}
        f(v_1) & = & a_{1}^1v_1'+a_{1}^2v_2'+\dots+a_{1}^mv_m', & a_{1}^i\in\mathbb{K}\\
        f(v_2) & = & a_{2}^1v_1'+a_{2}^2v_2'+\dots+a_{2}^mv_m', & a_{2}^i\in\mathbb{K}\\
        \vdots & & \vdots & \vdots\\
        f(v_n) & = & a_{n}^1v_1'+a_{n}^2v_2'+\dots+a_{n}^mv_m', & a_{n}^i\in\mathbb{K}
    \end{matrix}\]
    Sea $v\in V:v=\lambda^1v_1+\dots+\lambda^nv_n,\hspace{2mm}\lambda^i\in\mathbb{K}$, si le aplicamos $f$ tenemos,
    \[\begin{array}{rll}
        f(v) & = &\lambda^1f(v_1)+\dots+\lambda^nf(v_n) \\
         & = & \lambda^1(a_{1}^1v_1'+\dots+a_{1}^mv_m')+\lambda^2(a_{2}^1v_1'+\dots+a_{2}^mv_m')+\dots+\lambda^n(a_{n}^1v_1'+\dots+a_{n}^mv_m')\\
         & = & (a_{1}^1\lambda^1+a_{2}^1\lambda^2+\dots+a_{n}^1\lambda^n)v_1'+(a_{1}^2\lambda^1+\dots+a_{n}^2\lambda^n)v_2'+\dots+(a_{1}^m\lambda^1+\dots+a_{n}^m\lambda^n)v_m'
    \end{array}
        \]
   Luego,  $f(v)=\mu^1v_1'+\mu^2v_2'+\dots+\mu^mv_m'$, siendo $\mu^i=(a_{1}^i\lambda^1+\dots+a_{n}^i\lambda^n)$, luego, para construir la matriz $A$, ponemos las coordenadas de $v_1$ en la primera columna, las de $v_2$ en la segunda y así sucesivamente, tal que:
    \[\begin{pmatrix}
        \mu^1\\
        \mu^2\\
        \vdots\\
        \mu^m
    \end{pmatrix}=\begin{pmatrix}
        a_{1}^1 & a_{1}^1 & \dots & a_{1}^m\\
        a_{2}^1 & a_{2}^2 & \dots & a_{2}^m\\
        \vdots & \vdots & \ddots & \vdots\\
        a_{n}^1 & a_{n}^2 & \dots & a_{n}^m
    \end{pmatrix}\begin{pmatrix}
        \lambda^1\\
        \lambda^2\\
        \vdots\\
        \lambda^n
    \end{pmatrix}\Rightarrow \mu=A\cdot\lambda\]
\end{proof}

\noindent Vamos a introducir ahora el concepto de \textbf{rango} de una aplicación lineal, que puede extenderse al rango de su matriz asociada.

\begin{definition}
    Se llama rango de una aplicación lineal (matriz) a la dimensión de su imagen y se denota por $rg()$.
\end{definition}

\noindent Como un mismo espacio vectorial puede estar generado por varias bases, es lógico pensar que debe haber una relación entre estas bases o al menos una forma de cambiar de una base a otra, lo que se conoce como \textbf{cambio de base}. Esto es posible y una forma sencilla de hacerlo es mediante las matrices asociadas.

\begin{proposition}
    -Sean $V$ y $V'$ dos espacios vectoriales en $\mathbb{K}$, sea $f:V\longrightarrow V'$ lineal.\\
    -Sea $B_1=\curlybraces{v_1,\dots,v_n}$ base de $V$, $B_1'=\curlybraces{v_1',\dots,v_m'}$ base de $V'$.\\
    -Sea $A\in\mathcal{M}_{m\times n}(\mathbb{K})$ la matriz que representa a $f$ en $B_1,B_1'$.\\
    -Sea $B_2=\curlybraces{u_1,\dots,u_n}$ base de $V$, $B_2'=\curlybraces{u_1',\dots,u_m'}$ base de $V'$.\\
    -Sea $\tilde{A}\in\mathcal{M}_{m\times n}(\mathbb{K})$ la matriz que representa a $f$ en $B_2,B_2'$.\\
    -Sea $P$ la matriz de cambio de base de $B_1$ en $B_2$.\\
    -Sea $Q$ la matriz de cambio de base de $B_1'$ en $B_2'$.\\
    Entonces $\tilde{A}=Q^{-1}\cdot A\cdot P$.
\end{proposition}
\begin{proof}
    Sea $f:V\to V'$ una aplicación lineal con $n=dim V$ y $m=dim V'$. Si $A$ y $\tilde{A}$ son las matrices asociadas a $f$ respecto de distintas bases, entonces
    \[rg(A)=dim(Imf)=rg(\tilde{A})\]
    Luego $A$ y $\Tilde{A}$ tienen igual rango, y por tanto, son matrices equivalentes. Concretemos más esta situación:\\
    Sean $B_1$ y $B_2$ bases de $V$ con cambio de base de $B_1$ a $B_2$ dado por $X_1=PX_2$ y sean $B_1'$ y $B_2'$ bases de $V'$, con cambio de $B_1'$ a $B_2'$ dado por $Y_1=QY_2$.\\
    Consideremos la matriz asociada a $f$ respecto de $B_1$ y $B_1'$, $A\in\mathcal{M}_{m\times n}(\mathcal{K})$, tal que $A=\mathcal{M}_{B_1,B_1'}(f)$ y la ecuación matricial
    \[Y_1=AX_1\]
    De igual forma, sea $\tilde{A}\in\mathcal{M}_{m\times n}(\mathbb{K})$ la matriz asociada a $f$ respecto de $B_2$ y $B_2'$, tal que $\tilde{A}=\mathcal{M}_{B_2,B_2'}(\mathbb{K})$ y la ecuación matricial de $f$ respecto de estas bases,
    \[Y_2=\tilde{A}X_2\]
    Gráficamente,
    \[\begin{matrix}
    & V & \to & V' & \\
            & & A & &\\
            & B_1 & \longrightarrow & B_1' & \\
            P & \uparrow & & \uparrow & Q\\
             & & \tilde{A} & & \\
             & B_2 & \longrightarrow & B_2' &
        \end{matrix}\]
        Entonces,
        \[Y_2=\left\lbrace \begin{array}{l}
        \tilde{A}X_2\\    Q^{-1}Y_1=Q^{-1}AX_1=Q^{-1}APX_2\end{array}\right.\]
        y en consecuente,
        \[\tilde{A}=Q^{-1}AP\]
        O bien,
        \[X_2=\left\lbrace\begin{array}{l}
            \tilde{A}^{-1}Y_2\\
            P^{-1}X_1=P^{-1}A^{-1}Y_1=P^{-1}A^{-1}QY_2
        \end{array}\right.\]
        y en consecuente,
        \[\tilde{A}^{-1}=P^{-1}A^{-1}Q\]
\end{proof}

Ahora vamos a enunciar el \textbf{Primer Teorema de Isomorfía}, del que obtendremos un Corolario muy importante a la hora de trabajar con aplicaciones lineales. Este teorema no se va a demostrar (si se quiere ver la prueba consultar  \cite[Chapter 6, Theorem 6.5, Page 77]{IntroducciónTeoríaDeGrupos}).

\begin{theorem}[Primer teorema de isomorfismo de Noether]
    Sea $f:V\longrightarrow V'$ una aplicación lineal, entonces:
    \begin{enumerate}[label=(\roman*)]
        \item Existe una aplicación lineal sobreyectiva $\pi:V\longrightarrow V/Kerf$
        \item Existe un isomorfismo $\bar{f}:V/Kerf\longrightarrow Imf$
        \item Existe una aplicación lineal inyectiva $i:Imf\longrightarrow V'$, tales que $f=i\circ\bar{f}\circ\pi$, tal que
        \[\begin{matrix}
            & & f & \\
            & V & \longrightarrow & V' & \\
            \pi & \downarrow & & \uparrow & i\\
             & & \bar{f} & & \\
             & V/Kerf & \longrightarrow & Imf &
        \end{matrix}\]
    \end{enumerate}
   
\end{theorem}
\begin{corollary}
     Si además $V$ es finitamente generado,
    \[dimV=dim(Kerf)+dim(Imf)\]
\end{corollary}
\subsection{Contrtacción de tensores} % Main chapter title
\label{cap1-sec2-subsec3} 

 Una vez que hemos visto cómo subir y bajar índices, podemos definir una operación denominada \textbf{contracción} de tensores, la cuál encoge un tensor $(r,s)$ a uno $(r-1,s-1)$. La definición general se obtiene a partir del siguiente caso especial.

\begin{lemma}
    Hay una única aplicación lineal
    $C:\Omega_1^1\to\mathbb{R}$
    llamada \textit{contracción (1,1)}, tal que
    \[\begin{array}{rlll}
        C: & \Omega_1^1 (V)& \to & \mathbb{R} \\
         & \ptensor{v}{f} & \mapsto & C(\ptensor{v}{f})=f(v)
    \end{array}\]
    para todo $v\in V$ y $f\in V^*$.
\end{lemma}
\begin{proof} (Esta demostración usa el concepto de matrices de cambio de base, por lo que se recomienda ver la sección \ref{CambioBasesTensores(1,1)})\\ 
    Tomando $B=\curlybraces{v^1,v^2,\dots,v^n}$ base de $V$ y $B^*= \curlybraces{f_1,f_2,\dots,f_n}$ base de $V^*$, podemos escribir un tensor de tipo $(1,1)$ como
    \[A\equiv\sum A^i_j\ptensor{f_i}{v^j}\]
    Como $C(\ptensor{f_i}{v^j})=f_i(v^j)=\delta^j_i$, por la condición de base dual, no nos queda otra opción, más que definir,
    \[C(A)=\sum A_i^i=\sum A(f_i,v^i)\]
    Entonces, $C$ tiene las propiedades requeridas en las bases $B,B^*$. Luego, para obtener la función general requerida es suficiente con mostrar que esta definición es independiente de la elección del sistema de coordenadas. Así, tomando una nueva base de $V$, $B'=\curlybraces{w^1,w^2,\dots,w^n}$ y otra de $V^*$, $B^{*'}=\curlybraces{q_1,q_2,\dots,q_n}$, tenemos
    \[\begin{array}{rrl}
        C(A) & = & \sum\limits_mA(q_m,w^m)= \sum\limits_mA\left(\sum\limits_i a_i^mf_m,\sum_jb_m^jv^m\right)\\
         & = & \sum\limits_{i,j,m}a_i^mb_m^jA(f_i,v^j)=\sum\limits_{i,j}\delta^j_iA(f_i,v^j)\\
         & = & \sum\limits_iA(f_i,v^j)
    \end{array}\]
\end{proof}
\noindent Para extender las contracciones $(1,1)$, $C$, a un tensor de un tipo mayor, el esquema es especificar una componente covariante y otra contravariante y aplicar $C$ a estos.\\

\noindent Suponemos un tensor $A\in\Omega_r^s(V)$ y $1\leq r$ y $1\leq j \leq s$. Fijamos las formas $p_1,p_2,\dots,p_{r-1}$ y los vectores $u_1,u_2,\dots ,u_{s-1}$. Entonces la función
\[(p,u) \to A(p_1, \dots, \underbrace{p_{i}}_{\mathclap{i\text{-ésima componente contravariante}}}, \dots, p_{r-1}, u^{1}, \dots, \overbrace{u^{j}}^{\mathclap{j\text{-ésima componente covariante}}}, \ldots, u^{s-1})\]
es un tensor $(1,1)$ que puede escribirse como

\[A(p_1,\dots,\cdot,\dots,p_{r-1},u^1,\dots,\cdot,\dots,u^{s-1})\]
Aplicando la contracción $(1,1)$ a este tensor, produce una función de valor real denotada por

\[\left(C_j^iA\right)\left(p_1,\dots,p_{r-1},u^1,\dots,u^{s-1}\right)\]
Siendo $C_j^iA$ una función multilineal. Por tanto, esto es un tensor de tipo $(r-1,s-1)$ llamado \textit{la contracción de }$A$\textit{ sobre }$i,j$.

%\begin{definition}
 %   La contracción de un tensor $A$ de tipo $(r,s)$ con respecto al índice contravariante $p$ $(p\leq r)$ y al índice covariante $q$ $(q\leq s)$ es el tensor de tipo $(r-1,s-1)$, teniendo las componentes,
  %  \[B^{i_1\dots i_{r-1}}_{j_1\dots j_{s-1}}=A^{i_1\dots i_{p-1}ki_p\dots i_{r-1}}_{j_1\dots j_{q-1}kj_q\dots j_{s-1}}\]
%\end{definition}

\begin{note}
    Para poder contraer tensores, debemos tener superíndices y subíndices, así, podemos usar primero la métrica para subir o bajar índices y luego aplicar la contracción.
   \end{note} 
\begin{example}
        Si tenemos un tensor de tipo (0,2),
    $S\equiv S_{\alpha\beta}$, podemos hacer,
    \[\begin{array}{rllll}
        S_{\alpha\beta} & \to & g^{\gamma\alpha}S_{\alpha\beta}=S^{\alpha}_{\beta} & \to & C^1_1S^{\gamma}_{\beta}=S^{\beta}_{\beta} \\
        \text{Tensor (0,2)} & \to & \text{Tensor (1,1)} & \to &\text{Escalar}
    \end{array}\]
    cosa que se puede simplificar simplemente usando,
    \[S\equiv S_{\alpha\beta}\to g^{\beta\alpha}S_{\alpha\beta}=S^{\beta}_{\beta}\]
    es decir, podemos contraer tensores con la propia métrica.
\end{example}

\begin{example}
    Si
    \[U^j_i=T^{kj}_{ik}\]
    entonces
    \[U'^{j'}_{i'}=T'^{k'j'}_{i'k'}=S^i_{i'}S^l_{k'}R^{k'}_kR^{j'}_jT^{kj}_{il}=S^i_{i'}\delta^l_kR^{j'}_jT^{kj}_{il}=S^i_{i'}R^{j'}_jT^{kj}_{ik}=S^i_{i'}R^{j'}_jU^j_i\]
    donde hemos utilizado $S^l_{k'}R^{k'}_k=\delta^l_k$. Vemos que se transforma como un tensor (1,1).\\

\noindent    Así, dado un tensor $T^{ij}_{kl}$ de tipo (2,2), serán posible las 4 contracciones
    \[T^{kj}_{ki},\hspace{3mm}T^{jk}_{ik},\hspace{3mm}T^{kj}_{ik},\hspace{3mm}T^{jk}_{ki}\]
    que originan 4 tensores de tipo (1,1). Por otro lado, las dos posibles contracciones dobles que dan lugar a un escalar (tensor de tipo (0,0)) son
    \[T^{kj}_{kj},\hspace{3mm}T^{jk}_{kj}\]
\end{example}
\begin{note}
    El producto escalar $(\mathbb{R}^n,g_{ij})$ también se puede contraer. Pues $g_{ij}$ es un tensor de tipo (0,2), al cual le podemos aplicar una contracción 1,1, pero primero lo pasamos a un tensor de tipo (1,1), variando sus índices, tal que
    \[C^1_1\left(g^{ki}g_{ij}\right)=C^1_1(g^k_j)=g^j_j=n\]
    donde sabemos que vale $n$, pues al ser un espacio de dimensión $n$, la matriz asociada a $g$ será $G\in\mathcal{M}_{n\times n}$ y por tanto, la traza será la suma de $n$-elementos. Sabemos que estos elementos son el 1, porque la traza es invariante frente a los cambios de base (cosa que veremos más adelante), por tanto, si cogemos el producto escalar usual en la base usual, la matriz asociada es la matriz de Gram, cuyos elementos son todos nulos, salvo la diagonal que está formada por 1.
\end{note}
\subsection{Notación de Einstein} % Main chapter title
\label{cap1-sec1-subsec4} 

La notación de Einstein va a servir para facilitarnos la escritura, pues cada vez que tengamos un vector o una forma escrita como combinación lineal, vamos a poder redefinirlos como
\[w=\sum\limits_{i=1}^n\lambda^iv_i\equiv\lambda^iv_i\]
esto para un vector. Para una forma, tendremos
\[p=\sum\limits_{i=1}^n\mu_if^i\equiv\mu_i f^i\]
Además, para simplificar aún más la notación y dejarnos de tantas letras, vamos a identificar los escalares de $w$ como 
\[\lambda^i\equiv w^i\]
Así, los vectores como combinación lineal de otros vectores, los escribiremos como
\[w=w^iv_i\]
Y para las formas, haremos la identificación
\[\mu_i\equiv p_i\]
Así, las formas como combinación lineal de otras formas se escribirán como
\[p=p_if^i\]
\begin{example}
Un ejemplo de ello, será a la hora de identificar un vector en los términos de su base, pues suponiendo un $V$ espacio vectorial sobre el cuerpo $\mathbb{K}$ y cuya base sea $B=\curlybraces{v_1,v_2,\dots,v_n}$, tomando un $u\in V$, lo denotaremos como,
\[u=u^iv_i\]
\end{example}
\begin{example}
    Otro ejemplo será a la hora de identificar una forma en términos de la base dual, pues suponiendo un $V^*$ espacio dual de $V$, cuya base dual es $B^*=\curlybraces{f^1,f^2,\dots,f^n}$, tomando un $q\in V^*$, lo denotaremos como,
    \[q=q_if^i\]
\end{example}
\begin{note}
    En un artículo físico, se identifica directamente el escalar con el vector, es decir,
    \[w^i\equiv w\]
    pues se presupone que existe una base donde $w$ está bien definido. Así, los físicos usaremos de forma indistinguible los vectores y sus componentes respecto de una base fijada.
\end{note}
\subsection{Invariantes} % Main chapter title
\label{cap1-sec2-subsec5} 

Dado que los tensores suelen describirse en términos respecto de ciertas bases, cuando estos términos no dependen de la base empleada, los tensores se llamarán \textbf{invariantes}. O en otras palabras, los tensores que no se transforman frente a un cambio de base, serán los que llamaremos \textbf{invariantes}.\\ \\
Vamos a intentar ilustrar este concepto definiendo un tensor invariante de tipo (1,1), denominado \textit{traza}, que es un invariante conocido de las matrices. Si tenemos un tensor $A=A_j^i\ptensor{e_i}{f^j}$ que definimos como
\[\text{traza de }A=\rm{tr}A=A^i_i\]
siendo la suma de los elementos de la diagonal principal de la matriz $(A^i_j)$. No es a priori evidente que hayamos definido algo que depende únicamente de $A$, ya que los $A_j^i$ dependen no solo de $A$ sino también de la base $\curlybraces{e_i}$. Para mostrar que $\rm{tr} A$ es un número determinado enteramente por $A$ mismo y no por los $e_i$ también, debemos demostrar la invariancia; es decir, si $A$ se expresa en términos de otra base ${\tilde{e}_i}$, entonces la fórmula correspondiente en los nuevos componentes da el mismo número que antes. Así, escribimos $A=\tilde{A}^i_j\ptensor{\tilde{e}_i}{\tilde{f}^j}=A^i_j\ptensor{e_i}{f^j}$ y veremos que $A^i_j=\tilde{A}^i_j$. Usando la misma notación de cambios de base que hemos visto en el apartado anterior, tenemos la ley de transformación siguiente,
\[\tilde{A}^n_m=A^i_ja_m^jb_i^n\]
de lo cual se obtiene
\[\tilde{A}^i_i=A^p_ja^j_ib^i_p=A^p_j\delta^j_p=A^i_i\]
Queda demostrado. Luego, tenemos la proposición,
\begin{proposition}
    La traza de un tensor de tipo (1,1) es un invariante.
\end{proposition}
Para ver que no todas las expresiones en términos de las componentes de un tensor necesariamente serán un invariante, veamos el siguiente ejemplo. 
\begin{example}
    Supongamos $d=2$ y $A=\ptensor{e_1}{e_1}+\ptensor{e_1}{e_2}$, un tensor de tipo (0,2). La expresión de $A_{ii}$ en este caso será $A_{11}+A_{22}$=1+0=1. Ahora consideramos una nueva base dada por $e_1=\tilde{e}_1+\tilde{e}_2$ y $e_2=\tilde{e}_2$, entonces
    \[\begin{array}{rrl}
        A & = & (\tilde{e}_1+\tilde{e}_2)\otimes(\tilde{e}_1+\tilde{e}_2)+(\tilde{e}_1+\tilde{e}_2)\otimes\tilde{e}_2 \\
         & = & \tilde{e}_1\otimes\tilde{e}_1+2\tilde{e}_1\otimes\tilde{e}_2+\tilde{e}_2\otimes\tilde{e}_1+2\tilde{e}_2\otimes\tilde{e}_2
    \end{array}\]
    de la cuál se obtiene que $\tilde{A}_{ii}=\tilde{A}_{11}+\tilde{A}_{22}=1+2=3$. Por tanto es diferente a la base primera, luego no es un invariante.
\end{example}
\subsubsection*{Nota Final}
    Finalmente diremos que un tensor es todo aquel objeto matemático que satisfaga los cambios de base, o en otras palabras: \textit{Un tensor es todo objeto matemático que transforma como un tensor}.
%SECCION 3
\section{Dilatación temporal} % Main chapter title
\label{cap2-sec3} 
%------------------------------------------------------------------------------
La dilatación temporal es una causa directa de los postulados de Einstein. Veámoslo con un esquema,
\begin{multicols}{2}
    \begin{Figura}
        \centering
        \includegraphics[width=0.8\textwidth]{Capitulos/Capitulo2/Seccion3/Lrep.png}
        \captionof{figure}{Espejos en reposo.}
        \label{fig2.1}
    \end{Figura}
    \begin{Figura}
        \centering
        \includegraphics[width=0.8\textwidth]{Capitulos/Capitulo2/Seccion3/Lmov.png}
        \captionof{figure}{Espejos en movimiento.}
        \label{fig2.2}
    \end{Figura}
\end{multicols}
Si nos fijamos en la Figura \ref{fig2.1}, al estar los espejos en reposo, el rayo de luz que sale de la linterna vuelve en un tiempo $\Delta t=\frac{2l_0}{c}$. En cambio, suponiendo que los espejos se mueven a velocidad $\vec{v}$, y que la distancia de los brazos del rayo es $D$, entonces ahora el tiempo que tarda el rayo en ir y volver es $\Delta t'=\frac{2D}{c}$. Usando el Teorema de Pitágoras podemos calcular $D$, tal que
\[D^2=l_0^2+\left(\frac{\Delta t'v}{2}\right)^2\]
Sustituyendo $D$ y $l_0$ de las ecuaciones de $\Delta t$ y $\Delta t'$, tenemos
\[\left(\frac{\Delta t'c}{2}\right)^2=\left(\frac{\delta t'v}{2}\right)^2+\left(\frac{\Delta tc}{2}\right)^2\]
Por tanto, tenemos que el tiempo se dilata de la forma,
\begin{equation}
    \Delta t'=\gamma\Delta t
\end{equation}
y como $\gamma>1$ siempre, entonces $\Delta t'>\Delta t$, por eso se dilata el tiempo.\\ \\
Vemos que en el SRI $S'$ los relojes van más lento que en el SRI $S$, pues si consideramos como reloj el rebote de los fotones en los espejos, entonces en $S$ los fotones van más rápido que los fotones en $S'$.

%SECCION 4
\section{Contracción de longitudes} % Main chapter title
\label{cap2-sec4} 
%------------------------------------------------------------------------------
Tomamos dos eventos del espacio-tiempo, tal que
\[\Delta t'=\gamma\left(\Delta t-\frac{v}{c^2}\Delta x\right)\]
\[\Delta x'=\gamma\left(\Delta x-v\Delta t\right)\]
Asumimos que tomamos eventos que no están separados temporalmente, es decir, como si en $S'$ tomásemos una foto, así, $\Delta t'=0$. Por tanto, tendremos que $\Delta x'=L'$ y $\Delta x=L$. Luego, sustituyendo tenemos que
\[\Delta x'=\frac{\Delta x}{\gamma}\Longrightarrow L'=\frac{L}{\gamma}\]
Además, como $\gamma>1$, tendremos que $L>L'$, por tanto, se habla de contracción de longitudes; donde $L$ se conoce como \textbf{longitud propia}, que es la longitud del objeto respecto a un SRI en reposo respecto al objeto, es decir, el SRI centro de masas del objeto.
\begin{note}
    Las transformaciones de Lorentz dejan invariante las distancias espacio-temporales, pues dados dos eventos $(t_1,x_1)$ y $(t_2,x_2)$ en $S$, y los eventos correspondientes $(t_1',x_1')$ y $(t_2',x_2')$ en $S'$, entonces
    \[-c^2(t_2-t_1)^2+(x_2-x_1)^2=-c^2(t_2'-t_1')+(x_2'-x_1')^2\]
    por tanto, tenemos una cantidad que es invariante al SRI.
\end{note}
%SECCION 3
\section{Derivada de Lie} % Main chapter title
\label{cap3-sec5} 
%------------------------------------------------------------------------------
La derivada de Lie nos dice cómo derivar funciones escalares, campos vectoriales y campos tensoriales de forma general. Además, nos dice cómo conectar espacios tangentes $T_p$ y $T_q$ de dos puntos $p,q\in\mathscr{M}$ con $p\neq q$.\\ \\
Para ello, necesitamos un campo vectorial $\vec{\xi}$, que define un conjunto de difeomorfismos. En concreto, por cada punto $p\in\mathscr{M}$ pasa una curva $\gamma_p(t)$ tal que
\[\dot{\gamma}_p(t)=\xi^{\mu}(p)\]
Esta curva define el difeomorfismo siguiente,
\[\varphi_t(p)=\gamma_p(t);\hspace{4mm}\varphi_t:U\to\varphi_t(U)\]
La derivada de Lie de una función escalar $f$ a lo largo de $\vec{\xi}$ se define como,
\begin{equation}
    \mathscr{L}_{\vec{\xi}}f|_p=\vec{\xi}(f)\equiv\xi^{\mu}\partial_{\mu}f
\end{equation}
La derivada de Lie de un campo vectorial se define como,
\begin{equation}
    \mathscr{L}_{\vec{\xi}}\vec{v}|_p=\lim_{t\to0}\frac{\varphi^*_t(\vec{v}(\varphi_t(p)))-\vec{v}|_p}{t}
\end{equation}
Se puede demostrar que
\[\mathscr{L}_{\vec{\xi}}\vec{v}|_p=\brackets{\vec{\xi},\vec{v}}=-\mathscr{L}_{\vec{v}}\vec{\xi}|_p\]
En coordenadas se escribe,
\[(\mathscr{L}_{\vec{\xi}\vec{v}})^{\mu}=\xi^{\nu}\partial_{\nu}v^{\mu}-v^{\nu}\partial_{\nu}\xi^{\mu}\]
Sus propiedades son:
\begin{enumerate}
    \item La derivada de Lie preserva el tipo tensorial, las simetrías y las operaciones.
    \item Es una aplicación lineal.
    \item Satisface la regla de Leibtniz,
    \[\mathscr{L}_{\vec{\xi}}(\vec{u}\otimes\vec{v})=(\mathscr{L}_{\vec{\xi}}\vec{u})\otimes\vec{v}+\vec{u}\otimes(\mathscr{L}_{\vec{\xi}}\vec{v})\]
    \item Cumple que
    \[\left<\vec{f},\mathscr{L}_{\vec{\xi}}\vec{v}\right>=\mathscr{L}_{\vec{\xi}}(<\vec{f},\vec{v}>)-\left<\mathscr{L}_{\vec{\xi}}\vec{f},\vec{v}\right>\]
\end{enumerate}
La derivada de un tensor se define como
\begin{equation}
    \begin{array}{rl}
    \mathscr{L}_{\vec{\xi}}\left(T^{\mu_1\mu_2\dots\mu_r}_{\nu_1\nu_2\dots\nu_s}\right)&=\xi^{\mu}\partial_{\mu}T^{\mu_1\mu_2\dots\mu_r}_{\nu_1\nu_2\dots\nu_s}-T^{\nu\mu_2\dots\mu_r}_{\nu_1\nu_2\dots\nu_s}\partial_{\nu}\xi^{\mu_1}-T^{\mu_1\nu\dots\mu_r}_{\nu_1\nu_2\dots\nu_s}\partial_{\nu}\xi^{\mu_2}-\dots-T^{\mu_1\mu_2\dots\mu_{r-1}\nu}_{\nu_1\nu_2\dots\nu_s}\partial_{\nu}\xi^{\mu_r}+\\
    &+T^{\mu_1\mu_2\dots\mu_r}_{\nu\nu_2\dots\nu_s}\partial_{\nu_1}\xi^{\nu}+\dots+T^{\mu_1\mu_2\dots\mu_r}_{\nu_1\nu_2\dots\nu_{s-1}\nu}\partial_{\nu_s}\xi^{\nu}
    \end{array}
\end{equation}
La derivada de Lie nos permite saber cuándo hay una simetría, pues si $\mathscr{L}_{\vec{\xi}}\vec{v}=0$, entonces $\vec{v}$ es simétrico bajo $\varphi_t(p)$.\\ \\
La derivada de Lie necesita conocer los campos vectoriales y sus derivadas en las direcciones no tangenciales a las curvas que unen $p$ y $q$. Como la derivada de Lie necesita mucha información de primeras, puesto que hay infinitos campos fuera de la curva, introduciremos la noción de \textbf{derivada covariante}. Esta derivada, por otro lado, solo necesitará información sobre la curva y las direcciones tangentes a ella.
\subsection{Derivada covariante}
La derivada covariante es una derivada que está enteramente definida en $T_p$ y al actuar sobre un tensor, $T^{\mu_1\mu_2\dots\mu_r}_{\nu_1\nu_2\dots\nu_s}$, nos devolverá otro tensor de tipo $(r,s+1)$, tal que
\begin{equation}
    \nabla_{\nu_{s+1}}T^{\mu_1\mu_2\dots\mu_r}_{\nu_1\nu_2\dots\nu_s}\equiv T^{\mu_1\mu_2\dots\mu_r}_{\nu_1\nu_2\dots\nu_s;\nu_{s+1}}
\end{equation}
Sus propiedades son las siguientes:
\begin{enumerate}
    \item Es lineal,
    \[\nabla_{\nu}\left(\alpha T^{\mu_1\mu_2\dots\mu_r}_{\nu_1\nu_2\dots\nu_s}+\beta S^{\mu_1\mu_2\dots\mu_r}_{\nu_1\nu_2\dots\nu_s}\right)=\alpha\nabla_{\nu}T^{\mu_1\mu_2\dots\mu_r}_{\nu_1\nu_2\dots\nu_s}+\beta\nabla_{\nu}S^{\mu_1\mu_2\dots\mu_r}_{\nu_1\nu_2\dots\nu_s}\]
    con $\alpha,\beta\in\mathbb{R}$.
    \item Satisface la regla de Leibtniz (regla de la cadena),
    \[\nabla_{\nu}\left(T^{\mu_1\mu_2\dots\mu_r}_{\nu_1\nu_2\dots\nu_s}S^{\rho_1\rho_2\dots\rho_{r'}}_{\sigma_1\sigma_2\dots\sigma_{s'}}\right)=S^{\rho_1\rho_2\dots\rho_{r'}}_{\sigma_1\sigma_2\dots\sigma_{s'}}\nabla_{\nu}T^{\mu_1\mu_2\dots\mu_r}_{\nu_1\nu_2\dots\nu_s}+T^{\mu_1\mu_2\dots\mu_r}_{\nu_1\nu_2\dots\nu_s}\nabla_{\nu}S^{\rho_1\rho_2\dots\rho_{r'}}_{\sigma_1\sigma_2\dots\sigma_{s'}}\]
    \item Conmuta con la contracción,
    \[\nabla_{\nu}\left(T^{\mu\mu_2\dots\mu_r}_{\mu\nu_2\dots\nu_s}\right)=\nabla_{\nu}T^{\mu\mu_2\dots\mu_r}_{\mu\nu_2\dots\nu_s}\]
    \item Sobre funciones actúa como,
    \[\nabla_{\mu}f=(df)_{\mu}\]
    En coordenadas,
    \[(df)_{\mu}=\partial_{\mu}f\]
\end{enumerate}
De la propiedad 4. tenemos que 
\[\nabla_{\mu}f=(d\vec{f})_{\mu}\partial_{\mu}f\]
Sobre $<\vec{f},\vec{v}>=f_{\mu}v^{\mu}$ actúa $\nabla_{\mu}(f_{\nu}v^{\mu})=f_{\nu}\partial_{\mu}v^{\nu}$.
\begin{remark}
$\hspace{5mm}$
    \begin{itemize}
        \item $\partial_{\mu}v^{\nu}$ no es un tensor, pues
        \[\partial'_{\mu}v^{'\nu}=\frac{\partial x^{\rho}}{\partial x^{'\mu}}\partial_{\rho}\left(\frac{\partial x^{'\nu}}{\partial x^{\sigma}}v^{\sigma}\right)=\frac{\partial x^{\rho}}{\partial x^{'\mu}}\frac{\partial x^{'\nu}}{\partial x^{\sigma}}\partial_{\rho}v^{\sigma}+\underbrace{\frac{\partial x^{\rho}}{\partial x^{'\mu}}\partial_{\rho}\left(\frac{\partial x^{'\nu}}{\partial x^{\sigma}}\right)v^{\sigma}}\]
        donde el término señalado hace que no sea un tensor, pues hace que no transforme como un tensor.
        \item $\nabla_{\mu}v^{\nu}$ es un tensor, pues
        \[\begin{array}{cl}
            \nabla_{\mu}(f_{\nu}v^{\nu}) & =\partial_{\mu}(f_{\nu}v^{\nu})=f_{\nu}\partial_{\mu}v^{\nu}+v^{\nu}\partial_{\mu}f^{\nu} \\
            || & \\
             f_{\nu}\nabla_{\mu}v^{\nu}+v^{\nu}\nabla_{\mu}f_{\nu}&=f_{\nu}\left(\partial_{\mu}v^{\nu}+\Gamma_{\mu\rho}^{\rho}v^{\rho}\right)+v^{\nu}\left(\partial_{\mu}f_{\nu}+\overline{\Gamma}_{\mu\nu}^{\rho}f_{\rho}\right)=\\
             &=f_{\nu}\partial_{\mu}v^{\nu}+v^{\nu}\partial_{\mu}f_{\nu}+\underbrace{\Gamma_{\mu\sigma}^{\rho}v^{\sigma}f_{\rho}+\overline{\Gamma}_{\mu\rho}^{\sigma}v^{\rho}f_{\sigma}}_{\begin{matrix}
                 ||\\
                 0
             \end{matrix}}
        \end{array}\]
        donde el último término se anula porque $\overline{\Gamma}_{\mu\rho}^{\sigma}=-\Gamma_{\mu\rho}^{\sigma}$.
    \end{itemize}
\end{remark}
En resumen, tenemos que
\begin{equation}
    \nabla_{\mu}v^{\nu}=\partial_{\mu}v^{\nu}+\Gamma_{\mu\rho}^{\nu}v^{\rho}
\end{equation}
es la derivada covariante de vectores.
\begin{equation}
    \nabla_{\mu}f_{\nu}=\partial_{\mu}f_{\nu}-\Gamma_{\mu\nu}^{\rho}f_{\rho}
\end{equation}
es la derivada covariante de formas.\\ \\
Notemos que $\nabla_{\mu}v^{\nu}$ es un tensor, pero $\partial_{\mu}v^{\nu}$ no es un tensor, por tanto $\Gamma_{\mu\nu}^{\rho}$ no transforma como un tensor, pues
\[\Gamma_{\mu\rho}^{'\nu}=\frac{\partial x^{'\nu}}{\partial x^{\gamma}}\frac{\partial x^{\sigma}}{\partial x^{'\mu}}\frac{\partial x^{\delta}}{\partial x^{'\rho}}\Gamma_{\sigma\delta}^{\gamma}-\frac{\partial x^{\gamma}}{\partial x^{'\nu}}\frac{\partial x^{\sigma}}{\partial x^{'\rho}}\partial_{\gamma}\left(\frac{\partial x^{'\nu}}{\partial x^{\sigma}}\right)\]
En cambio, la combinación $\partial_{\mu}v^{\nu}+\Gamma_{\mu\rho}^{\nu}v^{\rho}$ sí es un tensor.\\ \\
La derivada covariante de un tensor tipo $(r,s)$ viene dada por,
\begin{equation}
    \begin{array}{rl}
        \nabla_{\mu}T^{\mu_1\mu_2\dots\mu_r}_{\nu_1\nu_2\dots\nu_s} & =\partial_{\mu}T^{\mu_1\mu_2\dots\mu_r}_{\nu_1\nu_2\dots\nu_s}+\Gamma_{\mu\rho}^{\mu_1}T^{\rho\mu_2\dots\mu_r}_{\nu_1\nu_2\dots\nu_s}+\dots+\Gamma_{\mu\rho}^{\mu_r}T^{\mu_1\mu_2\dots\mu_{r-1}\rho}_{\nu_1\nu_2\dots\nu_s}- \\
         & -\Gamma_{\mu\nu_1}^{\rho}T^{\mu_1\mu_2\dots\mu_r}_{\rho\nu_2\dots\nu_s}-\dots-\Gamma_{\mu\nu_s}^{\rho}T^{\mu_1\mu_2\dots\mu_r}_{\nu_1\nu_2\dots\nu_{s-1}\rho}
    \end{array}
\end{equation}
En una base no coordenada cualquiera, tenemos que la diferencia de dos conexiones en un tensor de tipo $(1,2)$ se cumple que
\[\left\lbrace\begin{array}{l}
     (\nabla_a-\overline{\nabla}_a)f_b=-C_{ab}^{c}f_c \\
     (\nabla_a-\overline{\nabla}_a)v^b=C_{ac}^bv^c 
\end{array}\right.\hspace{4mm}\text{con}\hspace{3mm}C_{bc}^a=\Gamma_{bc}^a-\overline{\Gamma}_{bc}^a\text{, que es un tensor.}\]
De entre todas las conexiones posibles, vamos a tomar aquellas conexiones que son simétricas, es decir, que $\nabla_{\mu}\nabla_{\nu}f=\nabla_{\nu}\nabla_{\mu}f$. Por tanto, tendremos que los símbolos de Christoffel, en una base coordenada, son simétricos,
\[\Gamma_{\mu\nu}^{\rho}=\Gamma_{\nu\mu}^{\rho}=\Gamma_{(\mu\nu)}^{\rho}\]
Como consecuencia, tenemos que
\[\left(\mathscr{L}_{\vec{\xi}}\vec{v}\right)^{\mu}=\xi^{\nu}\partial_{\nu}v^{\mu}-v^{\nu}\partial_{\nu}\xi^{\mu}=\xi^{\nu}\nabla_{\nu}v^{\mu}-v^{\nu}\nabla_{\nu}\xi^{\mu}\]
Por tanto, podemos sustituir $\partial_{\mu}\to\nabla_{\mu}$ en la derivada de Lie.
\section{Conexión de Levi-Civita}
Es la única conexión simétrica y compatible con la métrica, es decir, 
\[\nabla_{\mu}g_{\nu\rho}=\partial_{\mu}g_{\nu\rho}-\Gamma_{\mu\nu}^{\sigma}g_{\sigma\rho}-\Gamma_{\mu\rho}^{\sigma}g_{\nu\sigma}=0\]
Por tanto, podemos llegar a una definición de los símbolos de Christoffel en la conexión de Levi-Civita que solo depende de la métrica, tal que
\begin{equation}
    \Gamma_{\nu\rho}^{\mu}=\frac{1}{2}g^{\mu\sigma}\left(g_{\sigma\nu,\rho}+g_{\sigma\rho,\nu}-g_{\nu\rho,\sigma}\right)
\end{equation}
La conexión de Levi-Civita preserva las normas y los ángulos bajo el transporte paralelo (que se explicará en la siguiente sección).\\ \\
Además, para la conexión de Levi-Civita se cumple que
\begin{equation}
    \left(\mathscr{L}_{\vec{\xi}}g_{\mu\nu}\right)=\nabla_{\mu}\xi_{\nu}+\nabla_{\nu}\xi_{\mu}
\end{equation}
Podemos calcular las simetrías de nuestra variedad resolviendo $\left(\mathscr{L}_{\vec{\xi}}g_{\mu\nu}\right)=\nabla_{\mu}\xi_{\nu}+\nabla_{\nu}\xi_{\mu}=0$, obteniendo así las cantidades simétricas de nuestra variedad con la métrica escogida.
\section{Transporte paralelo}
Es una forma 'barata' de movernos de un punto $p$ a un punto $q$ de la variedad. Decimos que es 'barata' porque solo necesitamos una curva $\gamma_p(t)$ que pase por $p$ y $q$, y una conexión $\nabla_a$.\\ \\
Sea $\vec{v}\in T_p$ y sea $t^a$ el vector tangente a $\gamma_p(t)$. Dada una conexión $\nabla_a$, se define el transporte paralelo de $\vec{v}$ sobre $\gamma_p(t)$ como
\begin{equation}
    t^a\nabla_av^b=0
\end{equation}
teniendo una solución única. En coordenadas tenemos que  $t^{\mu}=\dot{x}^{\mu}=\frac{dx^{\mu}}{dt}$, tal que
\begin{equation}
    t^{\mu}\partial_{\mu}v^{\nu}+\Gamma_{\mu\rho}^{\nu}t^{\mu}v^{\rho}=0
\end{equation}
que equivale a
\begin{equation}
    \dot{x}^{\mu}\partial_{\mu}v^{\nu}+\Gamma_{\mu\rho}^{\nu}\frac{dx^{\mu}}{dt}v^{\rho}=\frac{dv^{\nu}}{dt}+\Gamma_{\mu\rho}^{\nu}\frac{dx^{\mu}}{dt}v^{\rho}=0
\end{equation}
Tenemos un conjunto de $n-$ecuaciones diferenciales ordinarias de primer orden y lineales en $v^{\mu}$. Existe solución y es única. Además, cualquier combinación lineal de vectores $v^{\mu}$ también es un vector de transporte paralelamente.\\ \\
El transporte paralelo induce un isomorfismo entre los espacios tangentes $T_p$ y $T_q$. Este isomorfismo depende de $\nabla_{\mu}$ y de $\gamma_p(t)$.
\begin{note}
    Si la curvatura asociada a $\nabla_a$ es cero, entonces el transporte paralelo no dependerá de $\gamma_p(t)$.
\end{note}
Para la conexión de Levi-Civita, dados $v^{\mu}$ y $u^{\mu}$, que satisfacen $t^{\mu}\nabla_{\mu}v^{\nu}=0$ y $t^{\mu}\nabla_{\mu}u^{\nu}=0$, con $t^{\mu}$ tangente a $\gamma_p(t)$, entonces
\begin{equation}
    t^{\mu}\nabla_{\mu}\left(v^{\nu}u^{\rho}g_{\nu\rho}\right)=u^{\rho}g_{\nu\rho}\cancelto{0}{t^{\mu}\nabla_{\mu}v^{\nu}}+v^{\nu}g_{\nu\rho}\cancelto{0}{t^{\mu}\nabla_{\mu}u^{\rho}}+v^{\nu}u^{\rho}t^{\mu}\cancelto{0}{\nabla_{\mu}g_{\nu\rho}}=0
\end{equation}
Por tanto, los ángulos y las normas de los vectores se preservan al ser transportados paralelamente con la conexión de Levi-Civita, independientemente de $\gamma_p(t)$.
\section{Geodésicas}
Una curva $\gamma_p(s)$, donde usamos el parámetro afín $s$, se dice que es geodésica si su vector tangente cumple que,
\begin{equation}
    \frac{dv^{\mu}}{ds}=0\Longleftrightarrow v^{\mu}\nabla_{\mu}v^{\nu}=0
\end{equation}
Este parámetro afín es único salvo multiplicación y adición de una constante. En coordenadas tenemos,
\[\begin{array}{c}
     v^{\mu}\partial_{\mu}v^{\nu}+\Gamma_{\mu\rho}^{\nu}v^{\mu}v^{\rho}=0  \\
      ||| \\
      \frac{dx^{\mu}}{ds}\partial_{\mu}\left(\frac{dx^{\nu}}{ds}\right)+\Gamma_{\mu\rho}^{\nu}\frac{dx^{\mu}}{ds}\frac{dx^{\rho}}{ds}=0
\end{array}\]
Por tanto, la ecuación de la geodésica queda
\begin{equation}
    \frac{d^2x^{\nu}}{ds^2}+\Gamma_{\mu\rho}^{\nu}\frac{dx^{\mu}}{ds}\frac{dx^{\rho}}{ds}=0
\end{equation}
teniendo así un sistema de $n-$ecuaciones diferenciales ordinarias de segundo orden no lineales. Estas ecuaciones diferenciales tienen solución y es única, pues satisfacen teoremas de existencia y unicidad.\\ \\
Localmente, podemos encontrar coordenadas normales $x=x(x')$, donde la ecuación de las geodésicas queda como,
\begin{equation}
    \frac{d^2x^{'\mu}}{ds^2}=0
\end{equation}
\subsection{Geodésicas como Principio Variacional}
Postulamos una acción,
\begin{equation}
    S=\int\sqrt{ds^2}=\int\sqrt{g_{\mu\nu}(x)\frac{dx^{\mu}}{d\lambda}\frac{dx^{\nu}}{d\lambda}}d\lambda=\int\mathcal{L}d\lambda
\end{equation}
donde si $\mathcal{L}^2>0$, entonces tenemos vectores espaciales.\\ \\
Si interpretamos $\mathcal{L}$ como un Lagrangiano, podemos obtener las ecuaciones de Euler-Lagrange, tal que
\[\frac{\partial\mathcal{L}}{\partial x^{\mu}}-\frac{d}{d\lambda}\left(\frac{\partial\mathcal{L}}{\partial\dot{x}^{\mu}}\right)=0\]
o bien,
\[\frac{\partial(\mathcal{L}^2)}{dx^{\mu}}-\frac{d}{d\lambda}\left(\frac{\partial(\mathcal{L}^2)}{\partial \dot{x}^{\mu}}\right)=-2\frac{\partial\mathcal{L}}{\partial\dot{x}^{\mu}}\dot{\mathcal{L}}\]
donde $\dot{x}^{\mu}=\frac{\partial x^{\mu}}{\partial\lambda}$. Lo calculamos,
\[\begin{array}{l}
     \frac{\partial(\mathcal{L}^2)}{\partial x^{\mu}}=(\partial_{\mu}g_{\rho\nu})\frac{dx^{\rho}}{d\lambda}\frac{dx^{\nu}}{d\lambda}  \\ \\
     \frac{\partial(\mathcal{L}^2)}{\partial\dot{x}^{\mu}}=2g_{\mu\nu}\frac{dx^{\nu}}{d\lambda} \\ \\
     \frac{d}{d\lambda}\left(\frac{\partial (\mathcal{L}^2)}{\partial\dot{x}^{\mu}}\right)=2g_{\mu\nu}\frac{d^2x^{\nu}}{d\lambda^2}+2(\partial_{\sigma}g_{\mu\nu})\frac{dx^{\sigma}}{d\lambda}\frac{dx^{\nu}}{d\lambda}
\end{array}\]
Por tanto,
\[\frac{\partial(\mathcal{L}^2)}{\partial x^{\mu}}-\frac{d}{d\lambda}\left(\frac{\partial(\mathcal{L}^2)}{\partial\dot{x}^{\mu}}\right)=-2g_{\mu\nu}\left(\frac{d^2x^{\nu}}{d\lambda^2}+\Gamma_{\rho\sigma}^{\nu}\dot{x}^{\rho}\dot{x}^{\sigma}\right)=-2\frac{\partial\mathcal{L}}{\partial\dot{x}^{\mu}}\dot{\mathcal{L}}\]
Entonces, la ecuación de la geodésica generalizada queda,
\begin{equation}
    \frac{d^2x^{\mu}}{d\lambda^2}+\Gamma_{\rho\sigma}^{\mu}\dot{x}^{\rho}\dot{x}^{\sigma}=\dot{x}^{\mu}\frac{\dot{\mathcal{L}}}{\mathcal{L}}
\end{equation}
Para llegar a la ecuación de la geodésica debemos usar el parámetro afín, pues el término de la derecha de la igualdad no se anula debido a que $\lambda$ no es el parámetro afín. Luego, redefinimos $\mathcal{L}d\lambda\equiv ds$, teniendo así la ecuación de la geodésica,
\begin{equation}
    \frac{d^2x^{\mu}}{ds^2}+\Gamma_{\rho\sigma}^{\mu}\frac{dx^{\rho}}{ds}\frac{dx^{\sigma}}{ds}=0
\end{equation}
\subsection{Derivación de los términos de Christoffel mediante las geodésicas}
Tomamos como Lagrangiano $\Tilde{\mathcal{L}}=g_{\mu\nu}\dot{x}^{\mu}\dot{x}^{\nu}$. Veamos un ejemplo de cómo derivar los símbolos de Christoffel usando la ecuación de la geodésica.
\begin{example}
    Tomamos la métrica en coordenadas cilíndricas,
    \[ds^2=dr^2+r^2d\theta^2+dz^2\]
    por tanto, el Lagrangiano queda
    \[\Tilde{\mathcal{L}}=\dot{r}^2+r^2\dot{\theta}^2+\dot{z}^2\]
    resolvemos las ecuaciones de Euler-Lagrange, \\
    para el eje $z$:
    \[\frac{\partial\Tilde{\mathcal{L}}}{\partial z}=0;\hspace{3mm}\frac{\partial\mathcal{L}}{\partial\dot{z}}=2\dot{z}\Rightarrow 0-\frac{d}{ds}(2\dot{z})=0\Rightarrow\ddot{z}=0\]
    luego, las ecuaciones de la geodésica para el eje $z$ quedan,
    \[\ddot{z}+\Gamma_{\mu\nu}^z\dot{x}^{\mu}\dot{x}^{\nu}=0\]
    por tanto $\Gamma_{\mu\nu}^z=0$.\\
    Para el eje $r$:
    \[\frac{\partial\Tilde{L}}{\partial r}=2r\dot{\theta}^2;\hspace{3mm}\frac{\partial\Tilde{\mathcal{L}}}{\partial\dot{r}}=2\dot{r}\Rightarrow 2r\theta^2-\frac{d}{ds}(2\dot{r})=0\Rightarrow\ddot{r}=r\dot{\theta}\]
    luego, las ecuaciones de la geodésica para el eje $r$ quedan,
    \[\ddot{r}+\Gamma_{\mu\nu}^r\dot{x}^{\mu}\dot{x}^{\nu}=0\]
    por tanto,
    \[\Gamma
    _{\theta\theta}^r=-r;\hspace{3mm}\Gamma_{r\theta}^r=0=\Gamma_{rz}^r=\Gamma_{zr}^r=\Gamma_{\theta r}^r\]
    Para el eje $\theta$:
    \[\frac{\partial\Tilde{\mathcal{L}}}{\partial\theta}=0;\hspace{3mm}\frac{\partial\Tilde{\mathcal{L}}}{\partial\dot{\theta}}=2r^2\dot{\theta}\Rightarrow0-\frac{d}{ds}(2r^2\dot{\theta})=0\]
    luego, las ecuaciones de la geodésica para el eje $\theta$ quedan,
    \[\ddot{\theta}+2\frac{1}{r}\dot{r}\dot{\theta}=0\]
    por tanto,
    \[\Gamma_{r\theta}^{\theta}=\frac{1}{r}=\Gamma_{\theta r}^{\theta}\]
    donde no aparece el 2 porque es simétrico.
\end{example}
\subsection{Densidad tensorial}
Sea un tensor $T^{\mu\nu\dots}_{\rho\sigma\dots}$ multiplicando a $(\sqrt{g})^{\omega}$, con $\omega=\dots,-2,-1,0,1,2,\dots$, su derivada covariante se define como
\begin{equation}
    \nabla_{\mu}\left(\sqrt{g}^{\omega}T^{\mu_1\mu_2\dots}_{\nu_1\nu_2\dots}\right)=(\sqrt{-g})^{\omega}\nabla_{\mu}T^{\mu_1\mu_2\dots}_{\nu_1\nu_2\dots}-\frac{\omega}{2}\Gamma_{\nu\mu}^{\nu}(\sqrt{g})^{\omega}T^{\mu_1\mu_2\dots}_{\nu_1\nu_2\dots}
\end{equation}
de forma que $\nabla_{\mu}(\sqrt{g})=0$.
\section{Tensores de Curvatura}
Dada una conexión $\nabla_a$ y una uno-forma, el operador $(\nabla_a\nabla_b-\nabla_b\nabla_a)f_c$ es lineal, es decir, sea $h$ una función escalar, entonces
\[(\nabla_a\nabla_b-\nabla_b\nabla_a)(hf_c)=h(\nabla_a\nabla_b-\nabla_b\nabla_a)f_c\]
Esto implica que
\begin{equation}
    (\nabla_a\nabla_b-\nabla_b\nabla_a)f_c=\mathscr{R}_{abc}^df_d
\end{equation}
donde $\mathscr{R}_{abc}^d$ es el \textbf{tensor de Riemann}, se puede interpretar como que $\nabla_a\nabla_b\equiv\rightarrow\uparrow$ y $\nabla_b\nabla_a\equiv\uparrow\rightarrow$, sacando así la curvatura de la variedad.\\ \\
Para derivar $(\nabla_a\nabla_b-\nabla_b\nabla_a)v^c$, recordemos que la conexión es simétrica, por tanto $(\nabla_a\nabla_b-\nabla_b\nabla_a)(f_cv^c)=0$, luego
\begin{equation}
    (\nabla_a\nabla_b-\nabla_b\nabla_a)v^c=\mathscr{R}_{abd}^cv^d
\end{equation}
Las propiedades del tensor de Riemann son:
\begin{enumerate}
    \item Es antisimétrico, $\mathscr{R}_{abc}^d=-\mathscr{R}_{bac}^d$.
    \item $\mathscr{R}_{[abc]}^d=0$, puesto que $\nabla_{[a}\nabla_{b}f_{c]}=0$.
    \item Cumple la identidad de Bianchi,
    \[\nabla_{[a}\mathscr{R}_{bc]d}^e=0\]
    puesto que $\nabla_{[a}\nabla_{b}\nabla_{c]}f_d=0$.
    \item Si la conexión es de Levi-Civita, entonces tenemos que
    \[(\nabla_a\nabla_b-\nabla_b\nabla_a)T^{a_1a_2\dots}_{b_1b_2\dots}=-\mathscr{R}_{abc}^{a_1}T^{ca_2\dots}_{b_1b_2\dots}-\dots+\mathscr{R}_{abb_1}^cT^{a_1a_2\dots}_{cb_2\dots}+\dots\]
    Luego, en una base coordenada en la conexión de Levi-Civita, podemos definir el tensor de Riemann como,
    \begin{equation}
        \mathscr{R}_{\mu\nu\rho}^{\sigma}=\partial_{\nu}\Gamma_{\mu\rho}^{\sigma}-\partial_{\mu}\Gamma_{\nu\rho}^{\sigma}+\Gamma_{\mu\rho}^{\lambda}\Gamma_{\lambda\nu}^{\sigma}-\Gamma_{\nu\rho}^{\lambda}\Gamma_{\lambda\mu}^{\sigma}
    \end{equation}
\end{enumerate}
El \textbf{tensor de Ricci} se define como,
    \begin{equation}
        \mathscr{R}_{ab}=\mathscr{R}_{acb}^c
    \end{equation}
El \textbf{escalar de Ricci} se define como,
\begin{equation}
    \mathscr{R}=\mathscr{R}_{ab}g^{ab}
\end{equation}
Para la conexión de Levi-Civita, tenemos que el tensor de Ricci es simétrico, es decir, $\mathscr{R}_{ab}=\mathscr{R}_{ba}$.


    \chapter{Relatividad General} %
\label{Capitulo4} %
\lhead{\emph{Relatividad General}}
\textit{“La alegría de ver y entender es el más perfecto don de la naturaleza”.}\\
(A. Einstein)
\newpage
%-------------------------------------------------------------------------------
%SECCION 1
\section{Repaso histórico} % Main chapter title
\label{cap2-sec1} 
%------------------------------------------------------------------------------
	La física clásica, del siglo XIX, era una física bien asentada. La cuál explica la mecánica con el libro de Sir Isaac Newton titulado \textit{Philosophiae Naturalis Principia Mathematica} y el electromagnetismo se explica con el libro de Maxwell titulado \textit{Electricity and Magnetism}.\\
 En 1887, Michelson y Morley iniciaron una revolución en la física con un experimento para medir la velocidad de la luz. El experimento consistía en medir la velocidad de la luz de un rayo paralelo al eje de rotación de la Tierra y de otro rayo perpendicular a este, esperándose obtener resultados diferentes. En cambio, se observó que ambos rayos iban exactamente igual, cosa que no tenía sentido en la época., por tanto, determinaron que la velocidad de la luz no era instantánea, sino que debía ser finita, y llegaron a un resultado de ésta bastante próximo al valor actual de la velocidad de la luz.
 \subsection{Relatividad Galileana}
 El Principio de Relatividad de Galileo establece que,
 \begin{center}
 \textit{''Es imposible determinar a base de experimentos (mecánicos) si un sistema de referencia está en reposo o en movimiento uniforme y rectilíneo''.}
 \end{center}
 Esto se derivó de que en la Relatividad Galileana hay un espacio absoluto en el que las leyes de Newton son ciertas. Definiremos un \textit{sistema de referencia inercial} (SRI) como aquel sistema referencia que se mueve a velocidad constante respecto al espacio absoluto. Además, todos los sistemas de referencia inerciales comparten un tiempo absoluto. Pero con la definición de SRI, el Principio de Relatividad se debe reformular con este concepto, así tenemos el Principio de Relatividad en formulación de equivalencia, que dice que
 \begin{center}
 \textit{''Todos los sistemas inerciales son equivalentes, es decir, todos los observadores inerciales ven la misma física''.}
 \end{center}
 \textbf{Leyes de Newton}\\ \\
 La Ley de Newton por excelencia es $\vec{F}=m\vec{a}=-\nabla V(\vec{r}-\vec{r}_0)$, donde $V$ es la función potencial. Esta ley (y las demás) transforman bien bajo el grupo de transformaciones de Galileo, que son:
 \begin{enumerate}
     \item \textbf{Traslaciones temporales:}
     \[t\to t'=t+t_0\]
     \item \textbf{Traslaciones espaciales:}
     \[\vec{r}\to\vec{r}'=\vec{r}+\vec{r}_i+\vec{v}t\]
     donde $\vec{v}$ es la velocidad relativa de un SRI con respecto al otro, y $\vec{r}_i$ es el vector de posición entre los orígenes de ambos SRI al inicio.
     \item \textbf{Rotaciones espaciales:}
     \[\vec{a}'=R(\theta)\vec{a}\]
     donde $R(\theta)$ es la matriz de rotación.
\end{enumerate}
Se puede ver que las Leyes de Newton no son covariantes, pero sí transforman bien, pues la física se mantiene, esto quiere decir que \textit{las Leyes de Newton de la física transforman de forma covariante}.\\ \\
El grupo de transformaciones de Galileo son simetrías que dan lugar a cantidades conservadas. Por tanto, si tenemos un Lagrangiano que sea invariante bajo traslaciones temporales, tendremos que el sistema conserva energía; si es invariante bajo traslaciones espaciales, conserva momento lineal; y si es invariante bajo rotaciones espaciales; conserva momento angular.\\ \\
El grupo de transformaciones de Galileo NO deja invariante las ecuaciones de Maxwell, que son
\[(i)\hspace{2mm}\nabla\cdot\vec{E}=\rho/\epsilon_0;\hspace{5mm}(iii)\hspace{2mm}\nabla\cdot\vec{B}=0\]
\[(ii)\hspace{2mm}\nabla\times\vec{B}=\partial_t\vec{E}/c^2+\mu_0\vec{J};\hspace{5mm}(iv)\hspace{2mm}\nabla\times\vec{E}=-\partial_t\vec{B}\]
Si $\rho=0$ y $\vec{J}=0$, es decir, estamos en vacío, podemos combinar las ecuaciones de Maxwell en una sola ecuación de ondas que se propaga a velocidad $c=299792,458$ m/s, resultado muy próximo al valor obtenido por Michelson y Morley, que además es independiente del sistema de referencia.
\subsection{Transformaciones de Lorentz}

Las transformaciones de Lorentz hacen que las ecuaciones de Maxwell transformen bien (sean covariantes). Estas transformaciones son:
\[\begin{array}{rcrc}
    (i) & t'=\gamma\left(t-\frac{v}{c^2}x\right); & (iii) & y'=y \\
    (ii) & x'=\gamma\left(x-vt\right); & (iv) & z'=z
\end{array}\]
donde $v$ es la velocidad relativa entre SRI (que suponemos que se mueven en el eje $X$), y $\gamma=\frac{1}{\sqrt{1-\frac{v^2}{c^2}}}$.\\ \\
Como estas transformaciones hacen que las leyes de Maxwell sean covariantes, diremos que las transformaciones de Lorentz sean más fundamentales que las transformaciones de Galileo.\\ \\
Además, vemos que por la transformación $(i)$ el tiempo ya \textbf{no es absoluto}, sino que depende del SRI, por lo que diremos que el tiempo es \textbf{relativo}.
%SECCION 2
\section{Álgebra de Tensores} % Main chapter title
\label{cap1-sec2} 
Llegamos a lo groso del capítulo, el \textbf{Álgebra de Tensores}. En este apartado vamos a ver qué es un tensor de forma matemática y cómo trabajar con ellos. También se mencionará cómo trabajamos los físicos con los tensores.
%------------------------------------------------------------------------------

\subsection{Producto tensorial: caso de dos términos} % Main chapter title
\label{cap1-sec2-subsec1} 
Vamos a ver qué es el \textbf{producto tensorial} y cómo los tensores se definen a partir de este.
\begin{proposition}
    Sea $V$ un $\mathbb{K}$-espacio vectorial, $\scalar{\cdot}{\cdot}$ el producto escalar euclídeo y $B=\curlybraces{v_1,\dots,v_n}$ base de $V$, 
    \[\begin{array}{cccl}
        f_v: & V & \to & V^*\\
         & v & \mapsto & f_v(v)=\scalar{v}{\cdot}
    \end{array}\]
     $f_v$ es una aplicación lineal, concretamente es un isomorfismo.
\end{proposition}
\begin{proof}
    Vemos que $f_v$ es aplicación lineal,
    \[f_v(w_1+w_2)=\scalar{v}{w_1+w_2}=\scalar{v}{w_1}+\scalar{v}{w_2}=f_v(w_1)+f_v(w_2)\checkmark \]
    \[f_v(\lambda\cdot w)=\scalar{v}{\lambda\cdot w}=\lambda\scalar{v}{w}=\lambda f_v(w)\checkmark\]
    para $\forall\lambda\in\mathbb{K}$ y $\forall w_1,w_2,w\in V$. Luego, es aplicación lineal.\\ \\
    Veamos que es isomorfo demostrando que es biyectivo, pues ya hemos visto que es aplicación lineal.\\
    Sabemos que $ker\curlybraces{f_v}=\curlybraces{0}\Leftrightarrow f_v$ es inyectiva. Luego, vemos si $ker\curlybraces{f_v}=\curlybraces{0}$:
    \[ker\curlybraces{f_v}=\curlybraces{w\in V,f_v(w)=0}=\curlybraces{w\in V;\scalar{v}{w}=0\Leftrightarrow w=0}\]
    Por tanto, $ker\curlybraces{f_v}=\curlybraces{0}$ y así, $f_v$ es inyectiva. $\checkmark$\\ \\
    Usando el Primer Teorema de isomorfía, tenemos que $dim(V)=\cancelto{0}{dim(ker\curlybraces{f_v})}+dim(Im f_v)$, pero como la $dim B=dim B^*$, siendo $B$ base de $V$ y $B^*$ base de $V^*$, entonces $dimV=dimV^*$, y por tanto, $dimV=dimImf_v=dimV^*$, luego $Imf_v$ es $V^*$ y por tanto, $f_v$ es sobreyectiva. $\checkmark$\\
    Luego, $f_v$ es un isomorfismo.
\end{proof}
\noindent Veamos cómo se define el producto tensorial y sus propiedades.
\begin{definition}
    Sea $V$ un $\mathbb{K}$-espacio vectorial, $V^*$ el dual de $V$, y $g^1,g^2\in V^*$ aplicaciones lineales, tal que $g^1:V\to\mathbb{K}$ y $g^2:V\to\mathbb{K}$. Así, definimos el producto tensorial como,
    \begin{enumerate}[label=(\roman*)]
        \item Producto tensorial entre dos formas $g^1,g^2\in V^*$,
        \[\begin{array}{cccl}
            \ptensor{g^1}{g^2}: & V\times V & \to & \mathbb{K}\\
            & (v,w) & \mapsto & g^1(v)g^2(w)
        \end{array}\]
        \item Producto tensorial entre dos vectores $v_1,v_2\in V$,
        \[\begin{array}{cccl}
            \ptensor{v_1}{v_2}: & V^*\times V^* & \to & \mathbb{K}\\
             & (f,g) & \mapsto & f(v_1)g(v_2)
        \end{array}\]
        \item Producto tensorial de una forma y un vector $v_1\in V$, $f^1\in V^*$,
        \[\begin{array}{cccl}
            \ptensor{v_1}{f^1}: & V^*\times V & \to & \mathbb{K}\\
             & (g,w) & \mapsto & g(v_1)f^1(w)
        \end{array}\]
    \end{enumerate}
\end{definition}

\begin{proposition}
    Los productos tensoriales definidos anteriormente son formas bilineales.
\end{proposition}
\begin{proof} 
Usando $\forall v_1,v_2,u_1,u_2,v,w,u\in V$, $\forall f^1,f^2,g,p,q\in V^*$ y $\forall \lambda\in\mathbb{K}$,
    \begin{enumerate}[label=(\roman*)]
        \item \[\begin{array}{cccl}
            \ptensor{f^1}{f^2}: & V\times V & \to & \mathbb{K}\\
            & (v,w) & \mapsto & f^1(v)f^2(w)
        \end{array}\]
        siendo $f^1,f^2\in V^*$. Veamos que es forma bilineal,
        \[\begin{array}{lrl} \text{\textbullet)} &(\ptensor{f^1}{f^2})(u_1+u_2,v)=&f^1(u_1+u_2)f^2(v)=\brackets{f^1(u_1)+f^1(u_2)}f^2(v)\\ &=&f^1(u_1)f^2(v)+f^1(u_2)f^2(v)=(\ptensor{f^1}{f^2})(u_1,v)+(\ptensor{f^1}{f^2})(u_2,v),\checkmark\\  \text{\textbullet)} &(\ptensor{f^1}{f^2})(v,u_1+u_2)  =&f^1(v)f^2(u_1+u_2)=f^1(v)\brackets{f^2(u_1)+f^2(u_2)}\\ &=&f^1(v)f^2(u_1)+f^1(v)f^2(u_2)=(\ptensor{f^1}{f^2})(v,u_1)+(\ptensor{f^1}{f^2})(v,u_2)\checkmark\\
             \text{\textbullet)} & (\ptensor{f^1}{f^2})(\lambda v,u) =& f^1(\lambda v)f^2(u)=\lambda f^1(v)f^2(u)=\lambda(\ptensor{f^1}{f^2})(v,u)\checkmark\\
        \text{\textbullet)}&(\ptensor{f^1}{f^2})(u,\lambda v)=&f^1(u)f^2(\lambda v)=\lambda f^1(u)f^2(v)=\lambda(\ptensor{f^1}{f^2})(u,v)\checkmark
          \end{array}\]
        Luego, $\ptensor{f^1}{f^2}$ es una forma bilineal. $\qedh $
        \item \[\begin{array}{cccl}
            \ptensor{v_1}{v_2}: & V^*\times V^* & \to & \mathbb{K}\\
             & (f,g) & \mapsto & f(v_1)g(v_2)
        \end{array}\]
         \[\begin{array}{lrl}
         \text{\textbullet)}&(\ptensor{v_1}{v_2})(f^1+f^2,g)=&(f^1+f^2)(v_1)g(v_2)=\brackets{f^1(v_1)+f^2(v_1)}g(v_2)\\
         &=&f^1(v_1)g(v_2)+f^2(v_1)g(v_2)=(\ptensor{v_1}{v_2})(f^1,g)+(\ptensor{v_1}{v_2})(f^2,g)\checkmark\\
         \text{\textbullet)}&(\ptensor{v_1}{v_2})(g,f^1+f^2)=&g(v_1)(f^1+f^2)(v_2)g=g(v_1)\brackets{f^1(v_2)+f^2(v_2)}\\
         &=&g(v_1)f^1(v_2)+g(v_1)f^2(v_2)=(\ptensor{v_1}{v_2})(g,f^1)+(\ptensor{v_1}{v_2})(g,f^2)\checkmark\\
         \text{\textbullet)}&(\ptensor{v_1}{v_2})(\lambda f,g)=&(\lambda f)(v_1)g(v_2)=\lambda f(v_1)g(v_2)=\lambda(\ptensor{v_1}{v_2})(f,g)\checkmark\\
         \text{\textbullet)}&(\ptensor{v_1}{v_2})(g,\lambda f)=&g(v_1)(\lambda f)(v_2)=\lambda g(v_1)f(v_2)=\lambda(\ptensor{v_1}{v_2})(g,f)\checkmark
         \end{array}\]
        Luego, $\ptensor{v_1}{v_2}$ es una forma bilineal. $\qedh $
        \item \[\begin{array}{cccl}
            \ptensor{v_1}{f^1}: & V^*\times V & \to & \mathbb{K}\\
             & (g,w) & \mapsto & g(v_1)f(w)
        \end{array}\]
        \[\begin{array}{lrl}
        \text{\textbullet)}&(\ptensor{v_1}{f^1})(p+q,w)=&(p+q)(v_1)f^1(w)=\brackets{p(v_1)+q(v_1)}f^1(w)=\\
        &=&p(v_1)f^1(w)+q(v_1)f^1(w)=(\ptensor{v_1}{f^1})(p,w)+(\ptensor{v_1}{f^1})(q,w)\checkmark\\
        \text{\textbullet)}&(\ptensor{v_1}{f^1})(g,u+w)=&g(v_1)f^1(u+w)=g(v_1)\brackets{f^1(u)+f^1(w)}=\\
        &=&g(v_1)f^1(u)+g(v_1)f^1(w)=(\ptensor{v_1}{f^1})(g,u)+(\ptensor{v_1}{f^1})(g,w)\checkmark\\
        \text{\textbullet)}&(\ptensor{v_1}{f^1})(\lambda g,w)=&(\lambda g)(v_1)f^1(w)=\lambda g(v_1)f^1(w)=\lambda(\ptensor{v_1}{f^1})(g,w)\checkmark\\
        \text{\textbullet)}&(\ptensor{v_1}{f^1})(g,\lambda w)=&g(v_1)f^1(\lambda w)=\lambda g(v_1)f^1(w)=\lambda(\ptensor{v_1}{f^1})(g,w)\checkmark
        \end{array}\]
            Luego, $\ptensor{v_1}{f^1}$ es una forma bilineal. \qedhere
    \end{enumerate}
\end{proof}
\noindent El producto tensorial no se da solo entre elementos de los espacios vectoriales o duales, sino que también se puede dar entre espacios, siendo el nuevo espacio generado un \textbf{espacio vectorial}.
\begin{proposition}
    El espacio $\ptensor{V}{V}$ tiene estructura de espacio vectorial.
\end{proposition}
\begin{proof}
    \begin{enumerate}
        \item Vemos que $(\ptensor{V}{V},+)$ es grupo abeliano:
        \begin{enumerate}[label=(\roman*)]
            \item Vemos si la operación $+$ es cerrada:
            \\
            $\forall v,w,z\in V$ con $\ptensor{v}{w},\ptensor{v}{z},\ptensor{w}{z}\in\ptensor{V}{V}$, tenemos que ver si $\ptensor{(v+w)}{z}\in\ptensor{V}{V}$. Sabemos que,
            \[\begin{array}{cccl}
                \ptensor{v}{w}: & \pcart{V^*}{V^*} & \to &\mathbb{R}  \\
                 & (f,g) & \mapsto & f(v)g(w)
            \end{array}\]
            luego,
            \[\begin{array}{cccl}
                \ptensor{(v+w)}{z}: & \pcart{V^*}{V^*} & \to &\mathbb{R}  \\
                 & (f,p) & \mapsto & f(v+w)p(z)
            \end{array}\]
            Entonces,
            \[(\ptensor{(v+w)}{z})(g,p)=f(v+w)p(z)=\brackets{f(v)+f(w)}p(z)=\]\[=f(v)p(z)+f(w)p(z)=(\ptensor{v}{z})(f,p)+(\ptensor{w}{z})(f,p)\]
            Luego, $\ptensor{(v+w)}{z}\in\ptensor{V}{V}$ y así, la operación $+$ es cerrada. $\checkmark$
            \item Asociatividad:
            \\
            Sean $\ptensor{a}{b},\ptensor{c}{d},\ptensor{e}{f}\in\ptensor{V}{V}$, tenemos que ver si $\ptensor{a}{b}+\brackets{\ptensor{c}{d}+\ptensor{e}{f}}=\brackets{\ptensor{a}{b}+\ptensor{c}{d}}+\ptensor{e}{f}$, tal que
            \[(\ptensor{a}{b}+\brackets{\ptensor{c}{d}+\ptensor{e}{f}})(p,q)=p(a)q(b)+\brackets{p(c)q(d)+p(e)q(f)}=p(a)q(b)+p(c+e)q(d+f)=\]
            \[=p(a+c+e)q(b+d+f)=p(a+c)q(b+d)+p(e)q(f)=\brackets{p(a)q(b)+p(c)q(d)}+p(e)q(f)=\]\[=(\brackets{\ptensor{a}{b}+\ptensor{c}{d}}+\ptensor{e}{f})(p,q)\checkmark\]
            \item Elemento neutro:\\
            Sea $\ptensor{e_1}{e_2}\in\ptensor{V}{V}$ el elemento neutro de $\ptensor{V}{V}$, tal que
            \[\ptensor{e_1}{e_2}+\ptensor{v}{w}=\ptensor{v}{w}+\ptensor{e_1}{e_2}=\ptensor{v}{w}\]
            Vemos el valor de este elemento neutro,
            \[(\ptensor{e_1}{e_2}+\ptensor{v}{w})(f,g)=(\ptensor{v}{w})(f,g)\]
            \[f(e_1)g(e_2)+f(v)+g(w)=f(v)g(w)\]
            \[f(e_1+v)g(e_w+w)=f(v)g(w)\Leftrightarrow\left\lbrace\begin{matrix}
                e_1=0\\
                e_2=0
            \end{matrix}\right.\]
            luego, $\ptensor{e_1}{e_2}=0$. $\checkmark$
            \item Elemento simétrico:
            \\
            $\forall\ptensor{v}{u}\in\ptensor{V}{V}$, $\exists\ptensor{\Tilde{v}}{\Tilde{u}}\in\ptensor{V}{V}$, tal que
            \[\ptensor{v}{u}+\ptensor{\Tilde{v}}{\Tilde{u}}=\ptensor{\Tilde{v}}{\Tilde{u}}+\ptensor{v}{u}=\ptensor{e_1}{e_2}=0\]
         Veamos quién es $\ptensor{\Tilde{v}}{\Tilde{u}}$,
        \[(\ptensor{v}{u}+\ptensor{\Tilde{v}}{\Tilde{u}})(f,g)=f(v)g(u)+f(\Tilde{v})g(\Tilde{u})=(\ptensor{0}{0})(f,g)=f(0)g(0)\]
        luego,
        \[v+\Tilde{v}=0\Rightarrow\Tilde{v}=-v\]
        \[u+\Tilde{u}=0\Rightarrow\Tilde{u}=-u\]
        Por tanto, el elemento simétrico de $\ptensor{v}{u}$ es $\ptensor{(-v)}{(-u)}$. $\checkmark$
        \item Conmutabilidad:\\
        Sean $\ptensor{v}{w},\ptensor{u}{z}\in\ptensor{V}{V}$, entonces
        \[(\ptensor{v}{w}+\ptensor{u}{z})(f,g)=f(v)g(w)+f(u)g(z)=f(v+u)g(w+z)=\]\[=f(u+v)g(z+w)=f(u)g(z)+f(v)g(w)=(\ptensor{u}{z}+\ptensor{v}{w})(f,g)\checkmark\]
    Luego, es grupo abeliano. $\checkmark$
         \end{enumerate}
         \item Doble propiedad distributiva:
         \begin{enumerate}
             \item $\forall\lambda,\mu\in\mathbb{R}$, $\forall\ptensor{v}{w}\in\ptensor{V}{V}$,
             \[(\lambda+\mu)\cdot(\ptensor{v}{w})(f,g)=(\lambda+\mu)f(v)g(w)=\]\[=\lambda f(v)g(w)+\mu f(v)g(w)=\lambda(\ptensor{v}{w})(f,g)+\mu(\ptensor{v}{w})(f,g)\checkmark\]
             \item $\forall\lambda\in\mathbb{R}$, $\forall\ptensor{v}{w},\ptensor{u}{z}\in\ptensor{V}{V}$, tenemos que
             \[\lambda(\ptensor{v}{w})(f,g)+\lambda(\ptensor{u}{z})(f,g)=\lambda f(v)g(w)+\lambda f(u)g(z)=\]\[=\lambda\brackets{f(v)g(w)+f(u)g(z)}=\lambda(\ptensor{v}{w}+\ptensor{u}{z})(f,g)\checkmark\]
             \end{enumerate}
             \item Propiedad pseudo-asociativa:\\
             $\forall\lambda,\mu\in\mathbb{R}$; $\forall\ptensor{v}{w}\in\ptensor{V}{V}$, tenemos que
             \[\lambda\cdot\brackets{\mu\cdot(\ptensor{v}{w})(f,g)}=\lambda\brackets{\mu f(v)g(w)}=\lambda f(\mu v)g(\mu w)=\]\[=f(\lambda\mu v)g(\lambda\mu w)=f(\mu\lambda v)g(\mu\lambda w)=\mu\brackets{f(\lambda v)g(\lambda w)}=(\mu\cdot\lambda)f(v)g(w)=(\mu\cdot\lambda)(\ptensor{v}{w})(f,g)\checkmark\]
             \item Elemento unitario del cuerpo: $\forall\ptensor{v}{w}\in\ptensor{V}{V}$; $\Tilde{\mu}\in\mathbb{R}$, entonces $\Tilde{\mu}\cdot\ptensor{v}{w}=\ptensor{v}{w}\cdot\Tilde{\mu}=\ptensor{v}{w}$
             \[(\Tilde{\mu}\cdot\ptensor{v}{w})(f,g)=f(\Tilde{\mu}v)g(\Tilde{\mu}w)=(\ptensor{v}{w})(f,g)=f(v)g(w)\Rightarrow\begin{matrix}
                 \Tilde{\mu}\cdot v=v\\
                 \Tilde{\mu}\cdot w=w
             \end{matrix}\Leftrightarrow\Tilde{\mu}=1\checkmark\]
       \end{enumerate}
       Luego, $(\ptensor{V}{V}, +, \cdot)$ es un $\mathbb{R}$-espacio vectorial.
\end{proof}
\noindent Al igual que cualquier otro espacio vectorial, el espacio $V\otimes V$ deberá tener una \textbf{base}.
\begin{proposition}
    Si tenemos un $V$ espacio vectorial sobre $\mathbb{K}$ con base $B=\curlybraces{v_1,\dots,v_n}$, entonces todo $\ptensor{v}{w}$ será combinación lineal de los elementos de la base de $\ptensor{V}{V}$ dada por $\ptensor{B}{B}=\curlybraces{\ptensor{v_i}{v_j}}_{i,j=1}^{n}$
\end{proposition}
\begin{proof}
    Queremos ver que $\curlybraces{\ptensor{v_i}{}v_j}_{i,j=1}^n$ es base de $\ptensor{V}{V}$. Para ello, tendremos que ver que esta base $\ptensor{B}{B}$ complete el espacio $\ptensor{V}{V}$ y que los vectores de la misma sean linealmente independientes.\\
    Sabemos que $\ptensor{v}{w}\in\ptensor{V}{V}$ y que
    \[\begin{array}{cccl}
        \ptensor{v}{w}: & \pcart{V^*}{V^*} & \to & \mathbb{R}\\
         & (f,g) & \mapsto & f(v)g(w)
    \end{array}\]
    Luego, para que la base $\ptensor{B}{B}$ complete el espacio $\ptensor{V}{V}$, se deberá poder expresar cualquier vector $\ptensor{v}{w}\in\ptensor{V}{V}$ como combinación lineal de los vectores de $\ptensor{B}{B}$. Podemos usar $B=\curlybraces{v_i}_{i=1}^n$ base de $V$, tal que
    \[v=\sum\limits_{i=1}^n\lambda^iv_i=\lambda^iv_i,\hspace{4mm}w=\sum\limits_{j=1}^n\mu^jv_j=\mu^jv_j\]
    Por tanto, usando $f,g\in V^*$, tenemos que
    \[\ptensor{v}{w}(f,g)=f(v)g(w)=f(\lambda^iv_i)g(\mu^jv_j)=\lambda^if(v_i)\mu^jg(v_j)=\lambda^i\mu^jf(v_i)g(v_j)=\lambda^i\mu^j(\ptensor{v_i}{v_j})(f,g)\]
    Luego, hemos expresado un vector del espacio $\ptensor{V}{V}$ como combinación lineal de los vectores de la base $\ptensor{B}{B}$. $\checkmark$\\ \\
    Veamos que son linealmente independientes, para ello, se debe cumplir que,
    \[\sum\limits_{i,j=1}^n\lambda^{ij}(\ptensor{v_i}{v_j})=\lambda^{ij}(\ptensor{v_i}{v_j})=0\Leftrightarrow\lambda^{ij}=0\]
    Sabiendo que la base de $V^*$ es $B^*=\curlybraces{f^1,f^2,\dots,f^n}$, tal que
    \[f^i(v_i)=1\hspace{5mm}f^j(v_i)\overset{i\neq j}{=}0\Rightarrow f^i(v_j)=\delta_{ij}\]
    Podemos evaluar lo anterior en dos elementos arbitrarios de $B^*$, tal que
    \[0=\lambda^{ij}(\ptensor{v_i}{v_j})(f^n,f^m)= \lambda^{ij}f^n(v_i)f^m(v_j)=\lambda_{ij}\delta_{n}^i\delta_{m}^j=\lambda^{nm}\]
    luego, $\lambda^{nm}=0$ y por tanto, los vectores son linealmente independientes. $\checkmark$\\ \\
    Así, hemos demostrado que $\ptensor{B}{B}$ es base de $\ptensor{V}{V}$.
\end{proof}

\begin{note}
    Denotaremos $\ptensor{v}{w}\equiv h$, tal que
    \[
    \begin{array}{cccl}
        h: & \pcart{V^*}{V^*} & \to & \mathbb{R} \\
         & (f^i,f^j) & \mapsto & h(f^i,f^j)=h^{ij}
    \end{array}
    \]
    siendo $f^i,f^j\in B^*$. Por tanto, para dos $p,q\in V^*$ cualesquiera, escribiremos
    \[(\ptensor{v}{w})(p,q)=h(p,q)=h\left(\sum_{i=1}^np_if^i,\sum_{j=1}^nq_jf^j\right)=p_iq_j(f^i,f^j)=h^{ij}p_iq_j\]
\end{note}
\noindent Veamos algunas \textbf{propiedades} del producto tensorial.
\begin{proposition}
    Sea $V$ un $\mathbb{R}$-espacio vectorial,
    \begin{enumerate}[label=(\roman*)]
        \item $\ptensor{(v_1+v_2)}{w}=\ptensor{v_1}{w}+\ptensor{v_2}{w}$; $\forall v_1,v_2,w\in V$.
        \item $\ptensor{w}{(v_1+v_2)}=\ptensor{w}{v_1}+\ptensor{w}{v_2}$, $\forall v_1,v_2,w\in V$.
        \item $\ptensor{(\lambda v)}{w}=\lambda\ptensor{v}{w}$, $\forall v,w\in V$, $\forall\lambda\in\mathbb{R}$.
        \item $\ptensor{w}{(\lambda v)}=\lambda\ptensor{w}{v}$, $\forall v,w\in V$, $\forall \lambda\in\mathbb{R}$.
        \item $\ptensor{v}{w}\neq\ptensor{w}{v}$.
        \item $\ptensor{v}{w}\neq0$ si $v\neq0$ ó $w\neq 0$.
        \item Sea $\ptensor{a}{b}\neq0$, $\ptensor{a}{b}=\ptensor{a'}{b'}\Leftrightarrow a'=\lambda a$ y $b'=\lambda^{-1}b$.
        \item $\ptensor{V}{W}$ es isomorfo con $\ptensor{W}{V}$.
    \end{enumerate}
\end{proposition}
\begin{proof}
    \begin{enumerate}[label=(\roman*)]
        \item $\forall v_1,v_2,w\in V$,
        \[(\ptensor{(v_1+v_2)}{w})(f,g)=f(v_1+v_2)g(w)=\brackets{f(v_1+f(v_2)}g(w)=\]\[=f(v_1)g(w)+f(v_2)g(w)=(\ptensor{v_1}{w})(f,g)+(\ptensor{v_2}{w})(f,g)\qedh\]
        \item $\forall v_1,v_2,w\in V$,
        \[(\ptensor{w}{(v_1+v_2)})(f,g)=f(w)g(v_1+v_2)=f(w)\brackets{g(v_1)+g(v_2)}=\]\[=f(w)g(v_1)+f(w)g(v_2)=(\ptensor{w}{v_1})(f,g)+(\ptensor{w}{v_2})(f,g)\qedh\]
        \item $\forall v,w\in V$ y $\forall\lambda\in\mathbb{R}$,
        \[(\ptensor{(\lambda\cdot v)}{w})(f,g)=f(\lambda\cdot v)g(w)=\lambda\cdot f(v)g(w)=\lambda\cdot(\ptensor{v}{w})(f,g)\qedh\]
        \item $\forall v,w\in V$ y $\forall\mu\in\mathbb{R}$,
        \[(\ptensor{w}{(\lambda\cdot v)})(f,g)=f(w)g(\lambda\cdot v)=\lambda\cdot f(w)g(v)=\lambda\cdot(\ptensor{w}{v})(f,g)\qedh\]
        \item Vemos que, $(\ptensor{v}{w})(f,g)=f(v)g(w)$ y que $(\ptensor{w}{v})(f,g)=f(w)g(v)$, luego estos elementos serían iguales solo si $f\equiv g$. $\qedh$
        \item Sean $v,w\in V$ y $f,g\in V^*$, tales que $f\not\equiv0$ y $g\not\equiv0$, entonces
        \[(\ptensor{v}{w})(f,g)=f(v)g(w)=0\Leftrightarrow\begin{matrix}
            f(v)=0 & \Leftrightarrow v=0\\
            \text{ó} & \\
            g(w)=0 & \Leftrightarrow w=0
        \end{matrix}\qedh\]
        \item \begin{tabular}{c|}
             $\Rightarrow$ \\ \hline
        \end{tabular} 
        Sea $\ptensor{a}{b}=\ptensor{a'}{b'}$ entonces
        \[(\ptensor{a}{b})(f,g)=f(a)g(b)=(\ptensor{a'}{b'})(f,g)=f(a')g(b')\]
        luego,
        \[f(a)g(b)=f(a')g(b')\]
        pero como $a\neq a'$ y $b\neq b'$, debe haber una relación entre ambos, de tal forma que se cumpla la igualdad anterior. Supondremos que $a$ y $a'$ tienen una relación lineal (la más sencilla), tal que $a'=\lambda a+c$, luego 
        \[f(a)g(b)=f(a')g(b')=f(\lambda a+c)g(b')=f(\lambda a)g(b')+f(c)g(b')=\lambda f(a)g(b')+f(c)g(b')\]
        Agrupamos términos de la igualdad, tal que,
        \[0:\hspace{5mm}0=f(c)g(b')\]
        \[f(a):\hspace{5mm}g(b)=\lambda g(b')\]
        Por la propiedad \textit{(vi)}, como $b'\neq0$, entonces $c=0$. Además,
        \[g(b)=\lambda g(b')\Rightarrow g(b')=\lambda^{-1}g(b)\Rightarrow g(b')=g(\lambda^{-1}b)\Rightarrow b'=\lambda^{-1}b\]
        Luego,
        \[\begin{matrix}
            a'=\lambda a\\
            b'=\lambda^{-1}b
        \end{matrix}\hspace{4mm}\checkmark\]
        \begin{tabular}{c|}
             $\Leftarrow$ \\ \hline
        \end{tabular} 
        Sea $a'=\lambda a$ y $b'=\lambda^{-1}b$, entonces
        \[(\ptensor{a'}{b'})(f,g)=f(a')g(b')=f(\lambda a)g(\lambda^{-1}b)=\cancel{\lambda}\cancel{\lambda^{-1}}f(a)g(b)=(\ptensor{a}{b})(f,g)\checkmark\]
        \item Sean $V,W$ espacios vectoriales, tales que
        \[\begin{array}{ccl}
            \ptensor{V}{W} & \to & \ptensor{W}{V}  \\
            \ptensor{v}{w} & \mapsto & \ptensor{w}{v}
        \end{array}\]
        Si suponemos que $dimV=n$ y $dimW=m$, sabemos por tanto que $dim(V\otimes W)=n\cdot m$ y $dim(W\otimes V)=m\cdot n$, luego tienen la misma dimensión y por tanto, son isomorfos. $\checkmark$\\ \\
        También podemos hacerlo sin usar la proposición de que $dim(\ptensor{V}{W})=n\cdot m$. Es claro ver que la aplicación es inyectiva, pues no hay dos elementos con la misma imagen, ya que la imagen se forma al permutar los elementos. Luego, al ser inyectivo, tenemos que $dimKer=0$. Por el Primer Teorema de Isomorfía,
        \[dim(\ptensor{V}{W})=\cancelto{0}{dimKer}+dimIm=dimIm=dim(\ptensor{W}{V})\]
        Luego, como $\ptensor{V}{W}$ y $\ptensor{W}{V}$ tienen la misma dimensión, entonces son isomorfos.
    \end{enumerate}
\end{proof}

\subsection{Aplicaciones lineales} % Main chapter title
\label{cap1-sec1-subsec2} 

Veamos ahora las aplicaciones lineales.
\begin{definition}
    Sean $V$ y $V'$ dos espacios vectoriales sobre el mismo cuerpo $\mathbb{K}$.
    Se dice que en una aplicación $f:V\longrightarrow V'$ es una aplicación lineal, o también llamado homomorfismo de espacios vectoriales, si se verifica:
    \begin{enumerate}[label=(\roman*)]
        \item $f(x+y)=f(x)+f(y),\forall x,y\in V$
        \item $f(\lambda\cdot x)=\lambda\cdot f(x),\forall\lambda\in\mathbb{K},\forall x\in V$
    \end{enumerate}
    Diremos además que $f$ es un isomorfismo lineal si es biyectiva, que $f$ es un endomorfismo si $V=V'$ y que es un automorfismo si es un endomorfismo biyectivo. 
\end{definition}

\noindent Las aplicaciones lineales tienen asociados dos conjuntos cuyas características son de interés, a saber, el núcleo y la imagen.

\begin{definition}
    Sea $f:V\longrightarrow W$ definimos el núcleo o kernel de la aplicación $f$ como
    \[Kerf=\curlybraces{v\in V:f(v)=0}\]
    y la imagen como
    \[Imf=\curlybraces{w\in W:\exists v\in V/f(v)=w}.\]
\end{definition}

\noindent Veamos algunas propiedades básicas de ambos conjuntos.

\begin{proposition}
Sea $f:V\to V'$ una aplicación lineal, se tienen las siguientes propiedades:
\begin{enumerate}[label=(\roman*)]
    \item \label{prop1:item1} $\rm{Im}f$ es un subespacio de $V'$ y que $\rm{Ker}f$ es un subespacio de $V$.
    \item \label{prop1:item2}Si $W$ es un subespacio vectorial de $V$, entonces $f(W):=\curlybraces{f(w): w\in W}$ es un subespacio de $V'$.
    \item \label{prop1:item3}Si $W'$ es un subespacio de $V'$, entonces $f^{-1}(W'):=\curlybraces{v\in V: f(v)\in W'}$ es también un subespacio de $V$.
\end{enumerate}  
\end{proposition}
%\newpage
\begin{proof}
\begin{enumerate}[label=\ref{prop1:item1}]
    \item Por definición, como los elementos de la $\rm{Im}f$ son pertenecientes a $V'$, entonces la $\rm{Im}f$ es subespacio de $V'$. De igual forma ocurre con el $\rm{Ker}f$, pues sus elementos pertenecen a $V$ y por tanto, este es subespacio de $V$.
\end{enumerate}
\begin{enumerate}[label=\ref{prop1:item2}]
    \item Como $W$ es subespacio de $V$, tenemos que $w\in V$ también, por tanto, los $f(w)$ pertenecerán a $V'$, cosa que implica que $f(W)$ es subespacio de $V'$, pues los $f(w)$ de $f(W)$ pertenecen a $V'$.
\end{enumerate}
\begin{enumerate}[label=\ref{prop1:item3}]
    \item Por analogía a $\ref{prop1:item2}$ vemos que $f^{-1}(W)$ es subespacio de $V$.
\end{enumerate}
\end{proof}
\noindent Ahora veamos algunas propiedades esenciales de las aplicaciones lineaales.
\begin{proposition}
    Sea $f:V\longrightarrow V'$ una aplicación lineal,
    \begin{enumerate}[label=(\roman*)]
        \item \label{pro1:item1} entonces $f$ es inyectiva si y solo si $Kerf=\curlybraces{0}$.
        \item \label{pro1:item2} si $G$ es un conjunto generador de $V$, $<G>=V$, entonces $f(G)$ es conjunto generador
        de $Imf$, $<f(G)>=Imf$.
        \item \label{pro1:item3} si $S\subset V$ es un conjunto de vectores linealmente independientes, si $f$ es inyectiva, entonces $f(S)$ es linealmente independiente.
        \item \label{pro1:item4} $f$ es inyectiva $\Leftrightarrow$ conserva la independencia lineal.
   
        \item \label{pro1:item5} si $f$ es biyectiva y $B$ es una base de $V$, entonces $f(B)$ es base de $V'$.

        \item \label{pro1:item6} $f$ es sobreyectiva $\Leftrightarrow$ $Imf=V'$
    \end{enumerate}
\end{proposition}



\begin{proof}
\ref{pro1:item1} \begin{tabular}{c|}
                 $\Rightarrow$ \\ \hline
            \end{tabular}
            Suponiendo que $f$ es inyectiva, sabemos que su Kernel es,
            \[\rm{Ker}f=\curlybraces{v\in V:f(v)=0}\]
            pero como la inyectividad nos implica que la imagen debe provenir de un único vector de entrada, entonces este vector será $v=0$, y por tanto, $ker f=\curlybraces{0}$. $\checkmark$
\\     
            \begin{tabular}{c|}
                 $\Leftarrow$  \\ \hline
            \end{tabular}
            Suponiendo que $ker f=\curlybraces{0}$, esto nos quiere decir que únicamente el vector $v=0$ satisface $f(v)=0$, luego como un vector tiene una única imagen, decimos que $f$ es inyectiva. \qedh
\\ \\
\ref{pro1:item2}  Veamos que el conjunto $f(G)$ es sistema generador de la imagen, es decir,   \[<f(G)>=Imf\Leftrightarrow\forall y\in Imf,\exists\lambda^1,\dots,\lambda^n\in\mathbb{K},y_1,\dots,y_n\in f(G)\text{ tales que }y=\lambda^1y_1+\dots+\lambda^ny_n.\] Sabemos que $G$ es conjunto generador, luego sea $y\in Imf$. Entonces por definición se tiene que existe $x\in V$ tal que $f(x)=y$. Como $<G>=V$, existen $\lambda^1,\dots,\lambda^n\in\mathbb{K}$, $v_1,\dots,v_n\in G$ tales que \[ x= \sum_{i=1}^n \lambda^i v_i.\] Tenemos entonces que:  \[y=f(x)=f(\lambda^1v_1+\dots+\lambda^nv_n)=\lambda^1f(v_1)+\dots+\lambda^nf(v_n).\] Por lo tanto, $y$ es combinación lineal de elementos de $f(G)$, es decir, $<f(G)>=\mathrm{Im}f$. \qedh
\\ \\
\ref{pro1:item3}
            Sea $S$ un conjunto linealmente independiente en $V$. 
            Supongamos que $f$ es inyectiva, vamos a probar que $f(S)$ es linealmente independiente, es decir,
            \[\lambda^1y_1+\dots+\lambda^ny_n=0\Rightarrow\lambda^1=\lambda^2=\dots=\lambda^n=0\hspace{4mm} \forall y_1,\dots,y_n\in f(S),\hspace{2mm}
                \forall\lambda^1,\dots,\lambda^n\in\mathbb{K}\]
            Supongamos $\lambda^1y_1+\dots+\lambda^ny_n=0$.  Como $y_j\in f(S),\exists x_j\in S/f(x_j)=y_j$.
             \[\left.\begin{array}{r}
                \lambda_1f(x_1)+\dots+\lambda_nf(x_n)=0\\
                f(\lambda_1x_1+\dots+\lambda_nx_n)=0
            \end{array}\right\rbrace f(0)=0\Rightarrow\lambda_1x_1+\dots+\lambda_nx_n=0\Rightarrow f\text{ inyectiva}\Rightarrow\lambda_1=\dots=\lambda_n=0\]
                \\ \\
                \ref{pro1:item4} \begin{tabular}{c|}
                 $\Longrightarrow$ \\ \hline
            \end{tabular} 
            Trivial por (iii) $\checkmark$\\
            \begin{tabular}{c|}
                 $\Longleftarrow$ \\ \hline
            \end{tabular} 
            Por reducción al absurdo:\\
            Supongamos que existen $v_1,v_2\in V$ distintos, tales que $f(v_1)=f(v_2)\Leftrightarrow f(v_1)-f(v_2)=0\Leftrightarrow f(v_1-v_2)=0$. 
            Luego, $v=v_1-v_2\neq0$ verifica que $f(v)=0$, $\curlybraces{v}$ es un conjunto linealmente independiente, $f(\curlybraces{v})$ tendría que ser un conjunto l.i. por hipótesis, pero $f(\curlybraces{v})=\curlybraces{0}$ que no es un conjunto l.i. cosa absurda. \qedh
            \\ \\
            \ref{pro1:item5}  Sea una aplicación lineal biyectiva $f:V\to V'$
            y una base de $V$, $B=\curlybraces{v_1,\dots,v_n}$. Entones, si aplicamos
\[f(B)=\curlybraces{f(v_1),\dots,f(v_n)}=\curlybraces{v_1',\dots,v_n'}\]
            y entonces, estos $v_i'\in V'$ van a formar una base de $V'$, pues al ser $f$ biyectiva, los vectores serán linealmente independientes, pues los de $B$ lo son; y además, como tienen la misma dimensión que $V'$, pasan de ser conjunto generador a base. $\qedh$
            \\ \\
            \ref{pro1:item6} \begin{tabular}{c|}
                 $\Rightarrow$  \\ \hline
            \end{tabular}
            Suponiendo que $f$ es sobreyectiva, tendremos que para cada $y\in V'$, existe al menos un $x\in V$, tal que $f(x)=y$. Por consiguiente, cada elemento de $V'$ es la imagen de un elemento de $V$, es decir, $Imf=V'$. $\checkmark$\\
            \begin{tabular}{c|}
                 $\Leftarrow$  \\ \hline
            \end{tabular}
            Suponiendo que $imf=V'$, tenemos que todos los elementos de $V'$ son imagen de los elementos de $V$, siendo esta la propia definición de sobreyectividad, luego $f$ es sobreyectiva.
\end{proof}
\noindent Una vez visto estas propiedades, de $\ref{pro1:item5}$ podemos obtener un resultado interesante, que es la siguiente proposición.
\begin{proposition}
Sea $B=\curlybraces{v_1,\dots,v_n}$ una base de $V$, y sea $f:V\rightarrow V'$ una aplicación lineal. Se tiene entonces que $\curlybraces{f(v_1),\dots,f(v_n)}$ es un sistema generador de la imagen.
\end{proposition}
\begin{proof}
    Supongamos que $B$ es una base y que conocemos $f(v_j),\forall v_j\in B$. 
    Sea $v\in V$, escrito en coordenadas de la base como $v=\lambda^1v_1+\dots+\lambda^nv_n$, con $v_i\in B$, 
 y $\lambda^i\in\mathbb{K}$, entonces  $f(x)=f(\lambda^1v_1+\dots+\lambda^nv_n)=\lambda^1f(v_1)+\dots\lambda^nf(v_n)$, luego hemos puesto $f(x)$ en coordenadas de $\curlybraces{f(v_1,\dots,f(v_n)}$.
\end{proof}

\noindent Ahora vamos a ver un resultado bastante importante, el cuál nos permitirá representar aplicaciones lineales en matrices, denominadas \textbf{matrices asociadas a la aplicación $f$}. Además, este resultado es importante para Física, pues los físicos no solemos trabajar con aplicaciones, sino que trabajamos con sus matrices asociadas, pues se puede decir que "tienen" la misma información que las aplicaciones.

\begin{proposition}  
\label{prop1.4}
    Sean $(V,+,\cdot)$ y $(V',+,\cdot)$ $\mathbb{K}$-espacios vectoriales de dimensión finita con $dimV=n$ y $dimV'=m$. 
    Sea $f:V\longrightarrow V'$ una aplicación lineal, entonces dadas $\left\lbrace\begin{matrix}
        B=\curlybraces{v_1,\dots,v_n}\text{ base de }V\\
        B'=\curlybraces{v_1',\dots,v_n'}\text{ base de }V'
    \end{matrix}\right.$\\
    $f$ se representa en esas bases como una matriz en $\mathcal{M}_{m\times n}(\mathbb{K})$.
\end{proposition}

\begin{proof}
    Como $f$ es lineal, me basta con conocer $f(B)$, para ello, tenemos que conocer $f(v_1),f(v_2),\dots,f(v_n)$, teniendo:
    \[\begin{matrix}
        f(v_1) & = & a_{1}^1v_1'+a_{1}^2v_2'+\dots+a_{1}^mv_m', & a_{1}^i\in\mathbb{K}\\
        f(v_2) & = & a_{2}^1v_1'+a_{2}^2v_2'+\dots+a_{2}^mv_m', & a_{2}^i\in\mathbb{K}\\
        \vdots & & \vdots & \vdots\\
        f(v_n) & = & a_{n}^1v_1'+a_{n}^2v_2'+\dots+a_{n}^mv_m', & a_{n}^i\in\mathbb{K}
    \end{matrix}\]
    Sea $v\in V:v=\lambda^1v_1+\dots+\lambda^nv_n,\hspace{2mm}\lambda^i\in\mathbb{K}$, si le aplicamos $f$ tenemos,
    \[\begin{array}{rll}
        f(v) & = &\lambda^1f(v_1)+\dots+\lambda^nf(v_n) \\
         & = & \lambda^1(a_{1}^1v_1'+\dots+a_{1}^mv_m')+\lambda^2(a_{2}^1v_1'+\dots+a_{2}^mv_m')+\dots+\lambda^n(a_{n}^1v_1'+\dots+a_{n}^mv_m')\\
         & = & (a_{1}^1\lambda^1+a_{2}^1\lambda^2+\dots+a_{n}^1\lambda^n)v_1'+(a_{1}^2\lambda^1+\dots+a_{n}^2\lambda^n)v_2'+\dots+(a_{1}^m\lambda^1+\dots+a_{n}^m\lambda^n)v_m'
    \end{array}
        \]
   Luego,  $f(v)=\mu^1v_1'+\mu^2v_2'+\dots+\mu^mv_m'$, siendo $\mu^i=(a_{1}^i\lambda^1+\dots+a_{n}^i\lambda^n)$, luego, para construir la matriz $A$, ponemos las coordenadas de $v_1$ en la primera columna, las de $v_2$ en la segunda y así sucesivamente, tal que:
    \[\begin{pmatrix}
        \mu^1\\
        \mu^2\\
        \vdots\\
        \mu^m
    \end{pmatrix}=\begin{pmatrix}
        a_{1}^1 & a_{1}^1 & \dots & a_{1}^m\\
        a_{2}^1 & a_{2}^2 & \dots & a_{2}^m\\
        \vdots & \vdots & \ddots & \vdots\\
        a_{n}^1 & a_{n}^2 & \dots & a_{n}^m
    \end{pmatrix}\begin{pmatrix}
        \lambda^1\\
        \lambda^2\\
        \vdots\\
        \lambda^n
    \end{pmatrix}\Rightarrow \mu=A\cdot\lambda\]
\end{proof}

\noindent Vamos a introducir ahora el concepto de \textbf{rango} de una aplicación lineal, que puede extenderse al rango de su matriz asociada.

\begin{definition}
    Se llama rango de una aplicación lineal (matriz) a la dimensión de su imagen y se denota por $rg()$.
\end{definition}

\noindent Como un mismo espacio vectorial puede estar generado por varias bases, es lógico pensar que debe haber una relación entre estas bases o al menos una forma de cambiar de una base a otra, lo que se conoce como \textbf{cambio de base}. Esto es posible y una forma sencilla de hacerlo es mediante las matrices asociadas.

\begin{proposition}
    -Sean $V$ y $V'$ dos espacios vectoriales en $\mathbb{K}$, sea $f:V\longrightarrow V'$ lineal.\\
    -Sea $B_1=\curlybraces{v_1,\dots,v_n}$ base de $V$, $B_1'=\curlybraces{v_1',\dots,v_m'}$ base de $V'$.\\
    -Sea $A\in\mathcal{M}_{m\times n}(\mathbb{K})$ la matriz que representa a $f$ en $B_1,B_1'$.\\
    -Sea $B_2=\curlybraces{u_1,\dots,u_n}$ base de $V$, $B_2'=\curlybraces{u_1',\dots,u_m'}$ base de $V'$.\\
    -Sea $\tilde{A}\in\mathcal{M}_{m\times n}(\mathbb{K})$ la matriz que representa a $f$ en $B_2,B_2'$.\\
    -Sea $P$ la matriz de cambio de base de $B_1$ en $B_2$.\\
    -Sea $Q$ la matriz de cambio de base de $B_1'$ en $B_2'$.\\
    Entonces $\tilde{A}=Q^{-1}\cdot A\cdot P$.
\end{proposition}
\begin{proof}
    Sea $f:V\to V'$ una aplicación lineal con $n=dim V$ y $m=dim V'$. Si $A$ y $\tilde{A}$ son las matrices asociadas a $f$ respecto de distintas bases, entonces
    \[rg(A)=dim(Imf)=rg(\tilde{A})\]
    Luego $A$ y $\Tilde{A}$ tienen igual rango, y por tanto, son matrices equivalentes. Concretemos más esta situación:\\
    Sean $B_1$ y $B_2$ bases de $V$ con cambio de base de $B_1$ a $B_2$ dado por $X_1=PX_2$ y sean $B_1'$ y $B_2'$ bases de $V'$, con cambio de $B_1'$ a $B_2'$ dado por $Y_1=QY_2$.\\
    Consideremos la matriz asociada a $f$ respecto de $B_1$ y $B_1'$, $A\in\mathcal{M}_{m\times n}(\mathcal{K})$, tal que $A=\mathcal{M}_{B_1,B_1'}(f)$ y la ecuación matricial
    \[Y_1=AX_1\]
    De igual forma, sea $\tilde{A}\in\mathcal{M}_{m\times n}(\mathbb{K})$ la matriz asociada a $f$ respecto de $B_2$ y $B_2'$, tal que $\tilde{A}=\mathcal{M}_{B_2,B_2'}(\mathbb{K})$ y la ecuación matricial de $f$ respecto de estas bases,
    \[Y_2=\tilde{A}X_2\]
    Gráficamente,
    \[\begin{matrix}
    & V & \to & V' & \\
            & & A & &\\
            & B_1 & \longrightarrow & B_1' & \\
            P & \uparrow & & \uparrow & Q\\
             & & \tilde{A} & & \\
             & B_2 & \longrightarrow & B_2' &
        \end{matrix}\]
        Entonces,
        \[Y_2=\left\lbrace \begin{array}{l}
        \tilde{A}X_2\\    Q^{-1}Y_1=Q^{-1}AX_1=Q^{-1}APX_2\end{array}\right.\]
        y en consecuente,
        \[\tilde{A}=Q^{-1}AP\]
        O bien,
        \[X_2=\left\lbrace\begin{array}{l}
            \tilde{A}^{-1}Y_2\\
            P^{-1}X_1=P^{-1}A^{-1}Y_1=P^{-1}A^{-1}QY_2
        \end{array}\right.\]
        y en consecuente,
        \[\tilde{A}^{-1}=P^{-1}A^{-1}Q\]
\end{proof}

Ahora vamos a enunciar el \textbf{Primer Teorema de Isomorfía}, del que obtendremos un Corolario muy importante a la hora de trabajar con aplicaciones lineales. Este teorema no se va a demostrar (si se quiere ver la prueba consultar  \cite[Chapter 6, Theorem 6.5, Page 77]{IntroducciónTeoríaDeGrupos}).

\begin{theorem}[Primer teorema de isomorfismo de Noether]
    Sea $f:V\longrightarrow V'$ una aplicación lineal, entonces:
    \begin{enumerate}[label=(\roman*)]
        \item Existe una aplicación lineal sobreyectiva $\pi:V\longrightarrow V/Kerf$
        \item Existe un isomorfismo $\bar{f}:V/Kerf\longrightarrow Imf$
        \item Existe una aplicación lineal inyectiva $i:Imf\longrightarrow V'$, tales que $f=i\circ\bar{f}\circ\pi$, tal que
        \[\begin{matrix}
            & & f & \\
            & V & \longrightarrow & V' & \\
            \pi & \downarrow & & \uparrow & i\\
             & & \bar{f} & & \\
             & V/Kerf & \longrightarrow & Imf &
        \end{matrix}\]
    \end{enumerate}
   
\end{theorem}
\begin{corollary}
     Si además $V$ es finitamente generado,
    \[dimV=dim(Kerf)+dim(Imf)\]
\end{corollary}
\subsection{Contrtacción de tensores} % Main chapter title
\label{cap1-sec2-subsec3} 

 Una vez que hemos visto cómo subir y bajar índices, podemos definir una operación denominada \textbf{contracción} de tensores, la cuál encoge un tensor $(r,s)$ a uno $(r-1,s-1)$. La definición general se obtiene a partir del siguiente caso especial.

\begin{lemma}
    Hay una única aplicación lineal
    $C:\Omega_1^1\to\mathbb{R}$
    llamada \textit{contracción (1,1)}, tal que
    \[\begin{array}{rlll}
        C: & \Omega_1^1 (V)& \to & \mathbb{R} \\
         & \ptensor{v}{f} & \mapsto & C(\ptensor{v}{f})=f(v)
    \end{array}\]
    para todo $v\in V$ y $f\in V^*$.
\end{lemma}
\begin{proof} (Esta demostración usa el concepto de matrices de cambio de base, por lo que se recomienda ver la sección \ref{CambioBasesTensores(1,1)})\\ 
    Tomando $B=\curlybraces{v^1,v^2,\dots,v^n}$ base de $V$ y $B^*= \curlybraces{f_1,f_2,\dots,f_n}$ base de $V^*$, podemos escribir un tensor de tipo $(1,1)$ como
    \[A\equiv\sum A^i_j\ptensor{f_i}{v^j}\]
    Como $C(\ptensor{f_i}{v^j})=f_i(v^j)=\delta^j_i$, por la condición de base dual, no nos queda otra opción, más que definir,
    \[C(A)=\sum A_i^i=\sum A(f_i,v^i)\]
    Entonces, $C$ tiene las propiedades requeridas en las bases $B,B^*$. Luego, para obtener la función general requerida es suficiente con mostrar que esta definición es independiente de la elección del sistema de coordenadas. Así, tomando una nueva base de $V$, $B'=\curlybraces{w^1,w^2,\dots,w^n}$ y otra de $V^*$, $B^{*'}=\curlybraces{q_1,q_2,\dots,q_n}$, tenemos
    \[\begin{array}{rrl}
        C(A) & = & \sum\limits_mA(q_m,w^m)= \sum\limits_mA\left(\sum\limits_i a_i^mf_m,\sum_jb_m^jv^m\right)\\
         & = & \sum\limits_{i,j,m}a_i^mb_m^jA(f_i,v^j)=\sum\limits_{i,j}\delta^j_iA(f_i,v^j)\\
         & = & \sum\limits_iA(f_i,v^j)
    \end{array}\]
\end{proof}
\noindent Para extender las contracciones $(1,1)$, $C$, a un tensor de un tipo mayor, el esquema es especificar una componente covariante y otra contravariante y aplicar $C$ a estos.\\

\noindent Suponemos un tensor $A\in\Omega_r^s(V)$ y $1\leq r$ y $1\leq j \leq s$. Fijamos las formas $p_1,p_2,\dots,p_{r-1}$ y los vectores $u_1,u_2,\dots ,u_{s-1}$. Entonces la función
\[(p,u) \to A(p_1, \dots, \underbrace{p_{i}}_{\mathclap{i\text{-ésima componente contravariante}}}, \dots, p_{r-1}, u^{1}, \dots, \overbrace{u^{j}}^{\mathclap{j\text{-ésima componente covariante}}}, \ldots, u^{s-1})\]
es un tensor $(1,1)$ que puede escribirse como

\[A(p_1,\dots,\cdot,\dots,p_{r-1},u^1,\dots,\cdot,\dots,u^{s-1})\]
Aplicando la contracción $(1,1)$ a este tensor, produce una función de valor real denotada por

\[\left(C_j^iA\right)\left(p_1,\dots,p_{r-1},u^1,\dots,u^{s-1}\right)\]
Siendo $C_j^iA$ una función multilineal. Por tanto, esto es un tensor de tipo $(r-1,s-1)$ llamado \textit{la contracción de }$A$\textit{ sobre }$i,j$.

%\begin{definition}
 %   La contracción de un tensor $A$ de tipo $(r,s)$ con respecto al índice contravariante $p$ $(p\leq r)$ y al índice covariante $q$ $(q\leq s)$ es el tensor de tipo $(r-1,s-1)$, teniendo las componentes,
  %  \[B^{i_1\dots i_{r-1}}_{j_1\dots j_{s-1}}=A^{i_1\dots i_{p-1}ki_p\dots i_{r-1}}_{j_1\dots j_{q-1}kj_q\dots j_{s-1}}\]
%\end{definition}

\begin{note}
    Para poder contraer tensores, debemos tener superíndices y subíndices, así, podemos usar primero la métrica para subir o bajar índices y luego aplicar la contracción.
   \end{note} 
\begin{example}
        Si tenemos un tensor de tipo (0,2),
    $S\equiv S_{\alpha\beta}$, podemos hacer,
    \[\begin{array}{rllll}
        S_{\alpha\beta} & \to & g^{\gamma\alpha}S_{\alpha\beta}=S^{\alpha}_{\beta} & \to & C^1_1S^{\gamma}_{\beta}=S^{\beta}_{\beta} \\
        \text{Tensor (0,2)} & \to & \text{Tensor (1,1)} & \to &\text{Escalar}
    \end{array}\]
    cosa que se puede simplificar simplemente usando,
    \[S\equiv S_{\alpha\beta}\to g^{\beta\alpha}S_{\alpha\beta}=S^{\beta}_{\beta}\]
    es decir, podemos contraer tensores con la propia métrica.
\end{example}

\begin{example}
    Si
    \[U^j_i=T^{kj}_{ik}\]
    entonces
    \[U'^{j'}_{i'}=T'^{k'j'}_{i'k'}=S^i_{i'}S^l_{k'}R^{k'}_kR^{j'}_jT^{kj}_{il}=S^i_{i'}\delta^l_kR^{j'}_jT^{kj}_{il}=S^i_{i'}R^{j'}_jT^{kj}_{ik}=S^i_{i'}R^{j'}_jU^j_i\]
    donde hemos utilizado $S^l_{k'}R^{k'}_k=\delta^l_k$. Vemos que se transforma como un tensor (1,1).\\

\noindent    Así, dado un tensor $T^{ij}_{kl}$ de tipo (2,2), serán posible las 4 contracciones
    \[T^{kj}_{ki},\hspace{3mm}T^{jk}_{ik},\hspace{3mm}T^{kj}_{ik},\hspace{3mm}T^{jk}_{ki}\]
    que originan 4 tensores de tipo (1,1). Por otro lado, las dos posibles contracciones dobles que dan lugar a un escalar (tensor de tipo (0,0)) son
    \[T^{kj}_{kj},\hspace{3mm}T^{jk}_{kj}\]
\end{example}
\begin{note}
    El producto escalar $(\mathbb{R}^n,g_{ij})$ también se puede contraer. Pues $g_{ij}$ es un tensor de tipo (0,2), al cual le podemos aplicar una contracción 1,1, pero primero lo pasamos a un tensor de tipo (1,1), variando sus índices, tal que
    \[C^1_1\left(g^{ki}g_{ij}\right)=C^1_1(g^k_j)=g^j_j=n\]
    donde sabemos que vale $n$, pues al ser un espacio de dimensión $n$, la matriz asociada a $g$ será $G\in\mathcal{M}_{n\times n}$ y por tanto, la traza será la suma de $n$-elementos. Sabemos que estos elementos son el 1, porque la traza es invariante frente a los cambios de base (cosa que veremos más adelante), por tanto, si cogemos el producto escalar usual en la base usual, la matriz asociada es la matriz de Gram, cuyos elementos son todos nulos, salvo la diagonal que está formada por 1.
\end{note}
\subsection{Notación de Einstein} % Main chapter title
\label{cap1-sec1-subsec4} 

La notación de Einstein va a servir para facilitarnos la escritura, pues cada vez que tengamos un vector o una forma escrita como combinación lineal, vamos a poder redefinirlos como
\[w=\sum\limits_{i=1}^n\lambda^iv_i\equiv\lambda^iv_i\]
esto para un vector. Para una forma, tendremos
\[p=\sum\limits_{i=1}^n\mu_if^i\equiv\mu_i f^i\]
Además, para simplificar aún más la notación y dejarnos de tantas letras, vamos a identificar los escalares de $w$ como 
\[\lambda^i\equiv w^i\]
Así, los vectores como combinación lineal de otros vectores, los escribiremos como
\[w=w^iv_i\]
Y para las formas, haremos la identificación
\[\mu_i\equiv p_i\]
Así, las formas como combinación lineal de otras formas se escribirán como
\[p=p_if^i\]
\begin{example}
Un ejemplo de ello, será a la hora de identificar un vector en los términos de su base, pues suponiendo un $V$ espacio vectorial sobre el cuerpo $\mathbb{K}$ y cuya base sea $B=\curlybraces{v_1,v_2,\dots,v_n}$, tomando un $u\in V$, lo denotaremos como,
\[u=u^iv_i\]
\end{example}
\begin{example}
    Otro ejemplo será a la hora de identificar una forma en términos de la base dual, pues suponiendo un $V^*$ espacio dual de $V$, cuya base dual es $B^*=\curlybraces{f^1,f^2,\dots,f^n}$, tomando un $q\in V^*$, lo denotaremos como,
    \[q=q_if^i\]
\end{example}
\begin{note}
    En un artículo físico, se identifica directamente el escalar con el vector, es decir,
    \[w^i\equiv w\]
    pues se presupone que existe una base donde $w$ está bien definido. Así, los físicos usaremos de forma indistinguible los vectores y sus componentes respecto de una base fijada.
\end{note}
\subsection{Invariantes} % Main chapter title
\label{cap1-sec2-subsec5} 

Dado que los tensores suelen describirse en términos respecto de ciertas bases, cuando estos términos no dependen de la base empleada, los tensores se llamarán \textbf{invariantes}. O en otras palabras, los tensores que no se transforman frente a un cambio de base, serán los que llamaremos \textbf{invariantes}.\\ \\
Vamos a intentar ilustrar este concepto definiendo un tensor invariante de tipo (1,1), denominado \textit{traza}, que es un invariante conocido de las matrices. Si tenemos un tensor $A=A_j^i\ptensor{e_i}{f^j}$ que definimos como
\[\text{traza de }A=\rm{tr}A=A^i_i\]
siendo la suma de los elementos de la diagonal principal de la matriz $(A^i_j)$. No es a priori evidente que hayamos definido algo que depende únicamente de $A$, ya que los $A_j^i$ dependen no solo de $A$ sino también de la base $\curlybraces{e_i}$. Para mostrar que $\rm{tr} A$ es un número determinado enteramente por $A$ mismo y no por los $e_i$ también, debemos demostrar la invariancia; es decir, si $A$ se expresa en términos de otra base ${\tilde{e}_i}$, entonces la fórmula correspondiente en los nuevos componentes da el mismo número que antes. Así, escribimos $A=\tilde{A}^i_j\ptensor{\tilde{e}_i}{\tilde{f}^j}=A^i_j\ptensor{e_i}{f^j}$ y veremos que $A^i_j=\tilde{A}^i_j$. Usando la misma notación de cambios de base que hemos visto en el apartado anterior, tenemos la ley de transformación siguiente,
\[\tilde{A}^n_m=A^i_ja_m^jb_i^n\]
de lo cual se obtiene
\[\tilde{A}^i_i=A^p_ja^j_ib^i_p=A^p_j\delta^j_p=A^i_i\]
Queda demostrado. Luego, tenemos la proposición,
\begin{proposition}
    La traza de un tensor de tipo (1,1) es un invariante.
\end{proposition}
Para ver que no todas las expresiones en términos de las componentes de un tensor necesariamente serán un invariante, veamos el siguiente ejemplo. 
\begin{example}
    Supongamos $d=2$ y $A=\ptensor{e_1}{e_1}+\ptensor{e_1}{e_2}$, un tensor de tipo (0,2). La expresión de $A_{ii}$ en este caso será $A_{11}+A_{22}$=1+0=1. Ahora consideramos una nueva base dada por $e_1=\tilde{e}_1+\tilde{e}_2$ y $e_2=\tilde{e}_2$, entonces
    \[\begin{array}{rrl}
        A & = & (\tilde{e}_1+\tilde{e}_2)\otimes(\tilde{e}_1+\tilde{e}_2)+(\tilde{e}_1+\tilde{e}_2)\otimes\tilde{e}_2 \\
         & = & \tilde{e}_1\otimes\tilde{e}_1+2\tilde{e}_1\otimes\tilde{e}_2+\tilde{e}_2\otimes\tilde{e}_1+2\tilde{e}_2\otimes\tilde{e}_2
    \end{array}\]
    de la cuál se obtiene que $\tilde{A}_{ii}=\tilde{A}_{11}+\tilde{A}_{22}=1+2=3$. Por tanto es diferente a la base primera, luego no es un invariante.
\end{example}
\subsubsection*{Nota Final}
    Finalmente diremos que un tensor es todo aquel objeto matemático que satisfaga los cambios de base, o en otras palabras: \textit{Un tensor es todo objeto matemático que transforma como un tensor}.
%SECCION 3
\section{Dilatación temporal} % Main chapter title
\label{cap2-sec3} 
%------------------------------------------------------------------------------
La dilatación temporal es una causa directa de los postulados de Einstein. Veámoslo con un esquema,
\begin{multicols}{2}
    \begin{Figura}
        \centering
        \includegraphics[width=0.8\textwidth]{Capitulos/Capitulo2/Seccion3/Lrep.png}
        \captionof{figure}{Espejos en reposo.}
        \label{fig2.1}
    \end{Figura}
    \begin{Figura}
        \centering
        \includegraphics[width=0.8\textwidth]{Capitulos/Capitulo2/Seccion3/Lmov.png}
        \captionof{figure}{Espejos en movimiento.}
        \label{fig2.2}
    \end{Figura}
\end{multicols}
Si nos fijamos en la Figura \ref{fig2.1}, al estar los espejos en reposo, el rayo de luz que sale de la linterna vuelve en un tiempo $\Delta t=\frac{2l_0}{c}$. En cambio, suponiendo que los espejos se mueven a velocidad $\vec{v}$, y que la distancia de los brazos del rayo es $D$, entonces ahora el tiempo que tarda el rayo en ir y volver es $\Delta t'=\frac{2D}{c}$. Usando el Teorema de Pitágoras podemos calcular $D$, tal que
\[D^2=l_0^2+\left(\frac{\Delta t'v}{2}\right)^2\]
Sustituyendo $D$ y $l_0$ de las ecuaciones de $\Delta t$ y $\Delta t'$, tenemos
\[\left(\frac{\Delta t'c}{2}\right)^2=\left(\frac{\delta t'v}{2}\right)^2+\left(\frac{\Delta tc}{2}\right)^2\]
Por tanto, tenemos que el tiempo se dilata de la forma,
\begin{equation}
    \Delta t'=\gamma\Delta t
\end{equation}
y como $\gamma>1$ siempre, entonces $\Delta t'>\Delta t$, por eso se dilata el tiempo.\\ \\
Vemos que en el SRI $S'$ los relojes van más lento que en el SRI $S$, pues si consideramos como reloj el rebote de los fotones en los espejos, entonces en $S$ los fotones van más rápido que los fotones en $S'$.

%SECCION 4
\section{Contracción de longitudes} % Main chapter title
\label{cap2-sec4} 
%------------------------------------------------------------------------------
Tomamos dos eventos del espacio-tiempo, tal que
\[\Delta t'=\gamma\left(\Delta t-\frac{v}{c^2}\Delta x\right)\]
\[\Delta x'=\gamma\left(\Delta x-v\Delta t\right)\]
Asumimos que tomamos eventos que no están separados temporalmente, es decir, como si en $S'$ tomásemos una foto, así, $\Delta t'=0$. Por tanto, tendremos que $\Delta x'=L'$ y $\Delta x=L$. Luego, sustituyendo tenemos que
\[\Delta x'=\frac{\Delta x}{\gamma}\Longrightarrow L'=\frac{L}{\gamma}\]
Además, como $\gamma>1$, tendremos que $L>L'$, por tanto, se habla de contracción de longitudes; donde $L$ se conoce como \textbf{longitud propia}, que es la longitud del objeto respecto a un SRI en reposo respecto al objeto, es decir, el SRI centro de masas del objeto.
\begin{note}
    Las transformaciones de Lorentz dejan invariante las distancias espacio-temporales, pues dados dos eventos $(t_1,x_1)$ y $(t_2,x_2)$ en $S$, y los eventos correspondientes $(t_1',x_1')$ y $(t_2',x_2')$ en $S'$, entonces
    \[-c^2(t_2-t_1)^2+(x_2-x_1)^2=-c^2(t_2'-t_1')+(x_2'-x_1')^2\]
    por tanto, tenemos una cantidad que es invariante al SRI.
\end{note}
%SECCION 3
\section{Derivada de Lie} % Main chapter title
\label{cap3-sec5} 
%------------------------------------------------------------------------------
La derivada de Lie nos dice cómo derivar funciones escalares, campos vectoriales y campos tensoriales de forma general. Además, nos dice cómo conectar espacios tangentes $T_p$ y $T_q$ de dos puntos $p,q\in\mathscr{M}$ con $p\neq q$.\\ \\
Para ello, necesitamos un campo vectorial $\vec{\xi}$, que define un conjunto de difeomorfismos. En concreto, por cada punto $p\in\mathscr{M}$ pasa una curva $\gamma_p(t)$ tal que
\[\dot{\gamma}_p(t)=\xi^{\mu}(p)\]
Esta curva define el difeomorfismo siguiente,
\[\varphi_t(p)=\gamma_p(t);\hspace{4mm}\varphi_t:U\to\varphi_t(U)\]
La derivada de Lie de una función escalar $f$ a lo largo de $\vec{\xi}$ se define como,
\begin{equation}
    \mathscr{L}_{\vec{\xi}}f|_p=\vec{\xi}(f)\equiv\xi^{\mu}\partial_{\mu}f
\end{equation}
La derivada de Lie de un campo vectorial se define como,
\begin{equation}
    \mathscr{L}_{\vec{\xi}}\vec{v}|_p=\lim_{t\to0}\frac{\varphi^*_t(\vec{v}(\varphi_t(p)))-\vec{v}|_p}{t}
\end{equation}
Se puede demostrar que
\[\mathscr{L}_{\vec{\xi}}\vec{v}|_p=\brackets{\vec{\xi},\vec{v}}=-\mathscr{L}_{\vec{v}}\vec{\xi}|_p\]
En coordenadas se escribe,
\[(\mathscr{L}_{\vec{\xi}\vec{v}})^{\mu}=\xi^{\nu}\partial_{\nu}v^{\mu}-v^{\nu}\partial_{\nu}\xi^{\mu}\]
Sus propiedades son:
\begin{enumerate}
    \item La derivada de Lie preserva el tipo tensorial, las simetrías y las operaciones.
    \item Es una aplicación lineal.
    \item Satisface la regla de Leibtniz,
    \[\mathscr{L}_{\vec{\xi}}(\vec{u}\otimes\vec{v})=(\mathscr{L}_{\vec{\xi}}\vec{u})\otimes\vec{v}+\vec{u}\otimes(\mathscr{L}_{\vec{\xi}}\vec{v})\]
    \item Cumple que
    \[\left<\vec{f},\mathscr{L}_{\vec{\xi}}\vec{v}\right>=\mathscr{L}_{\vec{\xi}}(<\vec{f},\vec{v}>)-\left<\mathscr{L}_{\vec{\xi}}\vec{f},\vec{v}\right>\]
\end{enumerate}
La derivada de un tensor se define como
\begin{equation}
    \begin{array}{rl}
    \mathscr{L}_{\vec{\xi}}\left(T^{\mu_1\mu_2\dots\mu_r}_{\nu_1\nu_2\dots\nu_s}\right)&=\xi^{\mu}\partial_{\mu}T^{\mu_1\mu_2\dots\mu_r}_{\nu_1\nu_2\dots\nu_s}-T^{\nu\mu_2\dots\mu_r}_{\nu_1\nu_2\dots\nu_s}\partial_{\nu}\xi^{\mu_1}-T^{\mu_1\nu\dots\mu_r}_{\nu_1\nu_2\dots\nu_s}\partial_{\nu}\xi^{\mu_2}-\dots-T^{\mu_1\mu_2\dots\mu_{r-1}\nu}_{\nu_1\nu_2\dots\nu_s}\partial_{\nu}\xi^{\mu_r}+\\
    &+T^{\mu_1\mu_2\dots\mu_r}_{\nu\nu_2\dots\nu_s}\partial_{\nu_1}\xi^{\nu}+\dots+T^{\mu_1\mu_2\dots\mu_r}_{\nu_1\nu_2\dots\nu_{s-1}\nu}\partial_{\nu_s}\xi^{\nu}
    \end{array}
\end{equation}
La derivada de Lie nos permite saber cuándo hay una simetría, pues si $\mathscr{L}_{\vec{\xi}}\vec{v}=0$, entonces $\vec{v}$ es simétrico bajo $\varphi_t(p)$.\\ \\
La derivada de Lie necesita conocer los campos vectoriales y sus derivadas en las direcciones no tangenciales a las curvas que unen $p$ y $q$. Como la derivada de Lie necesita mucha información de primeras, puesto que hay infinitos campos fuera de la curva, introduciremos la noción de \textbf{derivada covariante}. Esta derivada, por otro lado, solo necesitará información sobre la curva y las direcciones tangentes a ella.
\subsection{Derivada covariante}
La derivada covariante es una derivada que está enteramente definida en $T_p$ y al actuar sobre un tensor, $T^{\mu_1\mu_2\dots\mu_r}_{\nu_1\nu_2\dots\nu_s}$, nos devolverá otro tensor de tipo $(r,s+1)$, tal que
\begin{equation}
    \nabla_{\nu_{s+1}}T^{\mu_1\mu_2\dots\mu_r}_{\nu_1\nu_2\dots\nu_s}\equiv T^{\mu_1\mu_2\dots\mu_r}_{\nu_1\nu_2\dots\nu_s;\nu_{s+1}}
\end{equation}
Sus propiedades son las siguientes:
\begin{enumerate}
    \item Es lineal,
    \[\nabla_{\nu}\left(\alpha T^{\mu_1\mu_2\dots\mu_r}_{\nu_1\nu_2\dots\nu_s}+\beta S^{\mu_1\mu_2\dots\mu_r}_{\nu_1\nu_2\dots\nu_s}\right)=\alpha\nabla_{\nu}T^{\mu_1\mu_2\dots\mu_r}_{\nu_1\nu_2\dots\nu_s}+\beta\nabla_{\nu}S^{\mu_1\mu_2\dots\mu_r}_{\nu_1\nu_2\dots\nu_s}\]
    con $\alpha,\beta\in\mathbb{R}$.
    \item Satisface la regla de Leibtniz (regla de la cadena),
    \[\nabla_{\nu}\left(T^{\mu_1\mu_2\dots\mu_r}_{\nu_1\nu_2\dots\nu_s}S^{\rho_1\rho_2\dots\rho_{r'}}_{\sigma_1\sigma_2\dots\sigma_{s'}}\right)=S^{\rho_1\rho_2\dots\rho_{r'}}_{\sigma_1\sigma_2\dots\sigma_{s'}}\nabla_{\nu}T^{\mu_1\mu_2\dots\mu_r}_{\nu_1\nu_2\dots\nu_s}+T^{\mu_1\mu_2\dots\mu_r}_{\nu_1\nu_2\dots\nu_s}\nabla_{\nu}S^{\rho_1\rho_2\dots\rho_{r'}}_{\sigma_1\sigma_2\dots\sigma_{s'}}\]
    \item Conmuta con la contracción,
    \[\nabla_{\nu}\left(T^{\mu\mu_2\dots\mu_r}_{\mu\nu_2\dots\nu_s}\right)=\nabla_{\nu}T^{\mu\mu_2\dots\mu_r}_{\mu\nu_2\dots\nu_s}\]
    \item Sobre funciones actúa como,
    \[\nabla_{\mu}f=(df)_{\mu}\]
    En coordenadas,
    \[(df)_{\mu}=\partial_{\mu}f\]
\end{enumerate}
De la propiedad 4. tenemos que 
\[\nabla_{\mu}f=(d\vec{f})_{\mu}\partial_{\mu}f\]
Sobre $<\vec{f},\vec{v}>=f_{\mu}v^{\mu}$ actúa $\nabla_{\mu}(f_{\nu}v^{\mu})=f_{\nu}\partial_{\mu}v^{\nu}$.
\begin{remark}
$\hspace{5mm}$
    \begin{itemize}
        \item $\partial_{\mu}v^{\nu}$ no es un tensor, pues
        \[\partial'_{\mu}v^{'\nu}=\frac{\partial x^{\rho}}{\partial x^{'\mu}}\partial_{\rho}\left(\frac{\partial x^{'\nu}}{\partial x^{\sigma}}v^{\sigma}\right)=\frac{\partial x^{\rho}}{\partial x^{'\mu}}\frac{\partial x^{'\nu}}{\partial x^{\sigma}}\partial_{\rho}v^{\sigma}+\underbrace{\frac{\partial x^{\rho}}{\partial x^{'\mu}}\partial_{\rho}\left(\frac{\partial x^{'\nu}}{\partial x^{\sigma}}\right)v^{\sigma}}\]
        donde el término señalado hace que no sea un tensor, pues hace que no transforme como un tensor.
        \item $\nabla_{\mu}v^{\nu}$ es un tensor, pues
        \[\begin{array}{cl}
            \nabla_{\mu}(f_{\nu}v^{\nu}) & =\partial_{\mu}(f_{\nu}v^{\nu})=f_{\nu}\partial_{\mu}v^{\nu}+v^{\nu}\partial_{\mu}f^{\nu} \\
            || & \\
             f_{\nu}\nabla_{\mu}v^{\nu}+v^{\nu}\nabla_{\mu}f_{\nu}&=f_{\nu}\left(\partial_{\mu}v^{\nu}+\Gamma_{\mu\rho}^{\rho}v^{\rho}\right)+v^{\nu}\left(\partial_{\mu}f_{\nu}+\overline{\Gamma}_{\mu\nu}^{\rho}f_{\rho}\right)=\\
             &=f_{\nu}\partial_{\mu}v^{\nu}+v^{\nu}\partial_{\mu}f_{\nu}+\underbrace{\Gamma_{\mu\sigma}^{\rho}v^{\sigma}f_{\rho}+\overline{\Gamma}_{\mu\rho}^{\sigma}v^{\rho}f_{\sigma}}_{\begin{matrix}
                 ||\\
                 0
             \end{matrix}}
        \end{array}\]
        donde el último término se anula porque $\overline{\Gamma}_{\mu\rho}^{\sigma}=-\Gamma_{\mu\rho}^{\sigma}$.
    \end{itemize}
\end{remark}
En resumen, tenemos que
\begin{equation}
    \nabla_{\mu}v^{\nu}=\partial_{\mu}v^{\nu}+\Gamma_{\mu\rho}^{\nu}v^{\rho}
\end{equation}
es la derivada covariante de vectores.
\begin{equation}
    \nabla_{\mu}f_{\nu}=\partial_{\mu}f_{\nu}-\Gamma_{\mu\nu}^{\rho}f_{\rho}
\end{equation}
es la derivada covariante de formas.\\ \\
Notemos que $\nabla_{\mu}v^{\nu}$ es un tensor, pero $\partial_{\mu}v^{\nu}$ no es un tensor, por tanto $\Gamma_{\mu\nu}^{\rho}$ no transforma como un tensor, pues
\[\Gamma_{\mu\rho}^{'\nu}=\frac{\partial x^{'\nu}}{\partial x^{\gamma}}\frac{\partial x^{\sigma}}{\partial x^{'\mu}}\frac{\partial x^{\delta}}{\partial x^{'\rho}}\Gamma_{\sigma\delta}^{\gamma}-\frac{\partial x^{\gamma}}{\partial x^{'\nu}}\frac{\partial x^{\sigma}}{\partial x^{'\rho}}\partial_{\gamma}\left(\frac{\partial x^{'\nu}}{\partial x^{\sigma}}\right)\]
En cambio, la combinación $\partial_{\mu}v^{\nu}+\Gamma_{\mu\rho}^{\nu}v^{\rho}$ sí es un tensor.\\ \\
La derivada covariante de un tensor tipo $(r,s)$ viene dada por,
\begin{equation}
    \begin{array}{rl}
        \nabla_{\mu}T^{\mu_1\mu_2\dots\mu_r}_{\nu_1\nu_2\dots\nu_s} & =\partial_{\mu}T^{\mu_1\mu_2\dots\mu_r}_{\nu_1\nu_2\dots\nu_s}+\Gamma_{\mu\rho}^{\mu_1}T^{\rho\mu_2\dots\mu_r}_{\nu_1\nu_2\dots\nu_s}+\dots+\Gamma_{\mu\rho}^{\mu_r}T^{\mu_1\mu_2\dots\mu_{r-1}\rho}_{\nu_1\nu_2\dots\nu_s}- \\
         & -\Gamma_{\mu\nu_1}^{\rho}T^{\mu_1\mu_2\dots\mu_r}_{\rho\nu_2\dots\nu_s}-\dots-\Gamma_{\mu\nu_s}^{\rho}T^{\mu_1\mu_2\dots\mu_r}_{\nu_1\nu_2\dots\nu_{s-1}\rho}
    \end{array}
\end{equation}
En una base no coordenada cualquiera, tenemos que la diferencia de dos conexiones en un tensor de tipo $(1,2)$ se cumple que
\[\left\lbrace\begin{array}{l}
     (\nabla_a-\overline{\nabla}_a)f_b=-C_{ab}^{c}f_c \\
     (\nabla_a-\overline{\nabla}_a)v^b=C_{ac}^bv^c 
\end{array}\right.\hspace{4mm}\text{con}\hspace{3mm}C_{bc}^a=\Gamma_{bc}^a-\overline{\Gamma}_{bc}^a\text{, que es un tensor.}\]
De entre todas las conexiones posibles, vamos a tomar aquellas conexiones que son simétricas, es decir, que $\nabla_{\mu}\nabla_{\nu}f=\nabla_{\nu}\nabla_{\mu}f$. Por tanto, tendremos que los símbolos de Christoffel, en una base coordenada, son simétricos,
\[\Gamma_{\mu\nu}^{\rho}=\Gamma_{\nu\mu}^{\rho}=\Gamma_{(\mu\nu)}^{\rho}\]
Como consecuencia, tenemos que
\[\left(\mathscr{L}_{\vec{\xi}}\vec{v}\right)^{\mu}=\xi^{\nu}\partial_{\nu}v^{\mu}-v^{\nu}\partial_{\nu}\xi^{\mu}=\xi^{\nu}\nabla_{\nu}v^{\mu}-v^{\nu}\nabla_{\nu}\xi^{\mu}\]
Por tanto, podemos sustituir $\partial_{\mu}\to\nabla_{\mu}$ en la derivada de Lie.
\section{Conexión de Levi-Civita}
Es la única conexión simétrica y compatible con la métrica, es decir, 
\[\nabla_{\mu}g_{\nu\rho}=\partial_{\mu}g_{\nu\rho}-\Gamma_{\mu\nu}^{\sigma}g_{\sigma\rho}-\Gamma_{\mu\rho}^{\sigma}g_{\nu\sigma}=0\]
Por tanto, podemos llegar a una definición de los símbolos de Christoffel en la conexión de Levi-Civita que solo depende de la métrica, tal que
\begin{equation}
    \Gamma_{\nu\rho}^{\mu}=\frac{1}{2}g^{\mu\sigma}\left(g_{\sigma\nu,\rho}+g_{\sigma\rho,\nu}-g_{\nu\rho,\sigma}\right)
\end{equation}
La conexión de Levi-Civita preserva las normas y los ángulos bajo el transporte paralelo (que se explicará en la siguiente sección).\\ \\
Además, para la conexión de Levi-Civita se cumple que
\begin{equation}
    \left(\mathscr{L}_{\vec{\xi}}g_{\mu\nu}\right)=\nabla_{\mu}\xi_{\nu}+\nabla_{\nu}\xi_{\mu}
\end{equation}
Podemos calcular las simetrías de nuestra variedad resolviendo $\left(\mathscr{L}_{\vec{\xi}}g_{\mu\nu}\right)=\nabla_{\mu}\xi_{\nu}+\nabla_{\nu}\xi_{\mu}=0$, obteniendo así las cantidades simétricas de nuestra variedad con la métrica escogida.
\section{Transporte paralelo}
Es una forma 'barata' de movernos de un punto $p$ a un punto $q$ de la variedad. Decimos que es 'barata' porque solo necesitamos una curva $\gamma_p(t)$ que pase por $p$ y $q$, y una conexión $\nabla_a$.\\ \\
Sea $\vec{v}\in T_p$ y sea $t^a$ el vector tangente a $\gamma_p(t)$. Dada una conexión $\nabla_a$, se define el transporte paralelo de $\vec{v}$ sobre $\gamma_p(t)$ como
\begin{equation}
    t^a\nabla_av^b=0
\end{equation}
teniendo una solución única. En coordenadas tenemos que  $t^{\mu}=\dot{x}^{\mu}=\frac{dx^{\mu}}{dt}$, tal que
\begin{equation}
    t^{\mu}\partial_{\mu}v^{\nu}+\Gamma_{\mu\rho}^{\nu}t^{\mu}v^{\rho}=0
\end{equation}
que equivale a
\begin{equation}
    \dot{x}^{\mu}\partial_{\mu}v^{\nu}+\Gamma_{\mu\rho}^{\nu}\frac{dx^{\mu}}{dt}v^{\rho}=\frac{dv^{\nu}}{dt}+\Gamma_{\mu\rho}^{\nu}\frac{dx^{\mu}}{dt}v^{\rho}=0
\end{equation}
Tenemos un conjunto de $n-$ecuaciones diferenciales ordinarias de primer orden y lineales en $v^{\mu}$. Existe solución y es única. Además, cualquier combinación lineal de vectores $v^{\mu}$ también es un vector de transporte paralelamente.\\ \\
El transporte paralelo induce un isomorfismo entre los espacios tangentes $T_p$ y $T_q$. Este isomorfismo depende de $\nabla_{\mu}$ y de $\gamma_p(t)$.
\begin{note}
    Si la curvatura asociada a $\nabla_a$ es cero, entonces el transporte paralelo no dependerá de $\gamma_p(t)$.
\end{note}
Para la conexión de Levi-Civita, dados $v^{\mu}$ y $u^{\mu}$, que satisfacen $t^{\mu}\nabla_{\mu}v^{\nu}=0$ y $t^{\mu}\nabla_{\mu}u^{\nu}=0$, con $t^{\mu}$ tangente a $\gamma_p(t)$, entonces
\begin{equation}
    t^{\mu}\nabla_{\mu}\left(v^{\nu}u^{\rho}g_{\nu\rho}\right)=u^{\rho}g_{\nu\rho}\cancelto{0}{t^{\mu}\nabla_{\mu}v^{\nu}}+v^{\nu}g_{\nu\rho}\cancelto{0}{t^{\mu}\nabla_{\mu}u^{\rho}}+v^{\nu}u^{\rho}t^{\mu}\cancelto{0}{\nabla_{\mu}g_{\nu\rho}}=0
\end{equation}
Por tanto, los ángulos y las normas de los vectores se preservan al ser transportados paralelamente con la conexión de Levi-Civita, independientemente de $\gamma_p(t)$.
\section{Geodésicas}
Una curva $\gamma_p(s)$, donde usamos el parámetro afín $s$, se dice que es geodésica si su vector tangente cumple que,
\begin{equation}
    \frac{dv^{\mu}}{ds}=0\Longleftrightarrow v^{\mu}\nabla_{\mu}v^{\nu}=0
\end{equation}
Este parámetro afín es único salvo multiplicación y adición de una constante. En coordenadas tenemos,
\[\begin{array}{c}
     v^{\mu}\partial_{\mu}v^{\nu}+\Gamma_{\mu\rho}^{\nu}v^{\mu}v^{\rho}=0  \\
      ||| \\
      \frac{dx^{\mu}}{ds}\partial_{\mu}\left(\frac{dx^{\nu}}{ds}\right)+\Gamma_{\mu\rho}^{\nu}\frac{dx^{\mu}}{ds}\frac{dx^{\rho}}{ds}=0
\end{array}\]
Por tanto, la ecuación de la geodésica queda
\begin{equation}
    \frac{d^2x^{\nu}}{ds^2}+\Gamma_{\mu\rho}^{\nu}\frac{dx^{\mu}}{ds}\frac{dx^{\rho}}{ds}=0
\end{equation}
teniendo así un sistema de $n-$ecuaciones diferenciales ordinarias de segundo orden no lineales. Estas ecuaciones diferenciales tienen solución y es única, pues satisfacen teoremas de existencia y unicidad.\\ \\
Localmente, podemos encontrar coordenadas normales $x=x(x')$, donde la ecuación de las geodésicas queda como,
\begin{equation}
    \frac{d^2x^{'\mu}}{ds^2}=0
\end{equation}
\subsection{Geodésicas como Principio Variacional}
Postulamos una acción,
\begin{equation}
    S=\int\sqrt{ds^2}=\int\sqrt{g_{\mu\nu}(x)\frac{dx^{\mu}}{d\lambda}\frac{dx^{\nu}}{d\lambda}}d\lambda=\int\mathcal{L}d\lambda
\end{equation}
donde si $\mathcal{L}^2>0$, entonces tenemos vectores espaciales.\\ \\
Si interpretamos $\mathcal{L}$ como un Lagrangiano, podemos obtener las ecuaciones de Euler-Lagrange, tal que
\[\frac{\partial\mathcal{L}}{\partial x^{\mu}}-\frac{d}{d\lambda}\left(\frac{\partial\mathcal{L}}{\partial\dot{x}^{\mu}}\right)=0\]
o bien,
\[\frac{\partial(\mathcal{L}^2)}{dx^{\mu}}-\frac{d}{d\lambda}\left(\frac{\partial(\mathcal{L}^2)}{\partial \dot{x}^{\mu}}\right)=-2\frac{\partial\mathcal{L}}{\partial\dot{x}^{\mu}}\dot{\mathcal{L}}\]
donde $\dot{x}^{\mu}=\frac{\partial x^{\mu}}{\partial\lambda}$. Lo calculamos,
\[\begin{array}{l}
     \frac{\partial(\mathcal{L}^2)}{\partial x^{\mu}}=(\partial_{\mu}g_{\rho\nu})\frac{dx^{\rho}}{d\lambda}\frac{dx^{\nu}}{d\lambda}  \\ \\
     \frac{\partial(\mathcal{L}^2)}{\partial\dot{x}^{\mu}}=2g_{\mu\nu}\frac{dx^{\nu}}{d\lambda} \\ \\
     \frac{d}{d\lambda}\left(\frac{\partial (\mathcal{L}^2)}{\partial\dot{x}^{\mu}}\right)=2g_{\mu\nu}\frac{d^2x^{\nu}}{d\lambda^2}+2(\partial_{\sigma}g_{\mu\nu})\frac{dx^{\sigma}}{d\lambda}\frac{dx^{\nu}}{d\lambda}
\end{array}\]
Por tanto,
\[\frac{\partial(\mathcal{L}^2)}{\partial x^{\mu}}-\frac{d}{d\lambda}\left(\frac{\partial(\mathcal{L}^2)}{\partial\dot{x}^{\mu}}\right)=-2g_{\mu\nu}\left(\frac{d^2x^{\nu}}{d\lambda^2}+\Gamma_{\rho\sigma}^{\nu}\dot{x}^{\rho}\dot{x}^{\sigma}\right)=-2\frac{\partial\mathcal{L}}{\partial\dot{x}^{\mu}}\dot{\mathcal{L}}\]
Entonces, la ecuación de la geodésica generalizada queda,
\begin{equation}
    \frac{d^2x^{\mu}}{d\lambda^2}+\Gamma_{\rho\sigma}^{\mu}\dot{x}^{\rho}\dot{x}^{\sigma}=\dot{x}^{\mu}\frac{\dot{\mathcal{L}}}{\mathcal{L}}
\end{equation}
Para llegar a la ecuación de la geodésica debemos usar el parámetro afín, pues el término de la derecha de la igualdad no se anula debido a que $\lambda$ no es el parámetro afín. Luego, redefinimos $\mathcal{L}d\lambda\equiv ds$, teniendo así la ecuación de la geodésica,
\begin{equation}
    \frac{d^2x^{\mu}}{ds^2}+\Gamma_{\rho\sigma}^{\mu}\frac{dx^{\rho}}{ds}\frac{dx^{\sigma}}{ds}=0
\end{equation}
\subsection{Derivación de los términos de Christoffel mediante las geodésicas}
Tomamos como Lagrangiano $\Tilde{\mathcal{L}}=g_{\mu\nu}\dot{x}^{\mu}\dot{x}^{\nu}$. Veamos un ejemplo de cómo derivar los símbolos de Christoffel usando la ecuación de la geodésica.
\begin{example}
    Tomamos la métrica en coordenadas cilíndricas,
    \[ds^2=dr^2+r^2d\theta^2+dz^2\]
    por tanto, el Lagrangiano queda
    \[\Tilde{\mathcal{L}}=\dot{r}^2+r^2\dot{\theta}^2+\dot{z}^2\]
    resolvemos las ecuaciones de Euler-Lagrange, \\
    para el eje $z$:
    \[\frac{\partial\Tilde{\mathcal{L}}}{\partial z}=0;\hspace{3mm}\frac{\partial\mathcal{L}}{\partial\dot{z}}=2\dot{z}\Rightarrow 0-\frac{d}{ds}(2\dot{z})=0\Rightarrow\ddot{z}=0\]
    luego, las ecuaciones de la geodésica para el eje $z$ quedan,
    \[\ddot{z}+\Gamma_{\mu\nu}^z\dot{x}^{\mu}\dot{x}^{\nu}=0\]
    por tanto $\Gamma_{\mu\nu}^z=0$.\\
    Para el eje $r$:
    \[\frac{\partial\Tilde{L}}{\partial r}=2r\dot{\theta}^2;\hspace{3mm}\frac{\partial\Tilde{\mathcal{L}}}{\partial\dot{r}}=2\dot{r}\Rightarrow 2r\theta^2-\frac{d}{ds}(2\dot{r})=0\Rightarrow\ddot{r}=r\dot{\theta}\]
    luego, las ecuaciones de la geodésica para el eje $r$ quedan,
    \[\ddot{r}+\Gamma_{\mu\nu}^r\dot{x}^{\mu}\dot{x}^{\nu}=0\]
    por tanto,
    \[\Gamma
    _{\theta\theta}^r=-r;\hspace{3mm}\Gamma_{r\theta}^r=0=\Gamma_{rz}^r=\Gamma_{zr}^r=\Gamma_{\theta r}^r\]
    Para el eje $\theta$:
    \[\frac{\partial\Tilde{\mathcal{L}}}{\partial\theta}=0;\hspace{3mm}\frac{\partial\Tilde{\mathcal{L}}}{\partial\dot{\theta}}=2r^2\dot{\theta}\Rightarrow0-\frac{d}{ds}(2r^2\dot{\theta})=0\]
    luego, las ecuaciones de la geodésica para el eje $\theta$ quedan,
    \[\ddot{\theta}+2\frac{1}{r}\dot{r}\dot{\theta}=0\]
    por tanto,
    \[\Gamma_{r\theta}^{\theta}=\frac{1}{r}=\Gamma_{\theta r}^{\theta}\]
    donde no aparece el 2 porque es simétrico.
\end{example}
\subsection{Densidad tensorial}
Sea un tensor $T^{\mu\nu\dots}_{\rho\sigma\dots}$ multiplicando a $(\sqrt{g})^{\omega}$, con $\omega=\dots,-2,-1,0,1,2,\dots$, su derivada covariante se define como
\begin{equation}
    \nabla_{\mu}\left(\sqrt{g}^{\omega}T^{\mu_1\mu_2\dots}_{\nu_1\nu_2\dots}\right)=(\sqrt{-g})^{\omega}\nabla_{\mu}T^{\mu_1\mu_2\dots}_{\nu_1\nu_2\dots}-\frac{\omega}{2}\Gamma_{\nu\mu}^{\nu}(\sqrt{g})^{\omega}T^{\mu_1\mu_2\dots}_{\nu_1\nu_2\dots}
\end{equation}
de forma que $\nabla_{\mu}(\sqrt{g})=0$.
\section{Tensores de Curvatura}
Dada una conexión $\nabla_a$ y una uno-forma, el operador $(\nabla_a\nabla_b-\nabla_b\nabla_a)f_c$ es lineal, es decir, sea $h$ una función escalar, entonces
\[(\nabla_a\nabla_b-\nabla_b\nabla_a)(hf_c)=h(\nabla_a\nabla_b-\nabla_b\nabla_a)f_c\]
Esto implica que
\begin{equation}
    (\nabla_a\nabla_b-\nabla_b\nabla_a)f_c=\mathscr{R}_{abc}^df_d
\end{equation}
donde $\mathscr{R}_{abc}^d$ es el \textbf{tensor de Riemann}, se puede interpretar como que $\nabla_a\nabla_b\equiv\rightarrow\uparrow$ y $\nabla_b\nabla_a\equiv\uparrow\rightarrow$, sacando así la curvatura de la variedad.\\ \\
Para derivar $(\nabla_a\nabla_b-\nabla_b\nabla_a)v^c$, recordemos que la conexión es simétrica, por tanto $(\nabla_a\nabla_b-\nabla_b\nabla_a)(f_cv^c)=0$, luego
\begin{equation}
    (\nabla_a\nabla_b-\nabla_b\nabla_a)v^c=\mathscr{R}_{abd}^cv^d
\end{equation}
Las propiedades del tensor de Riemann son:
\begin{enumerate}
    \item Es antisimétrico, $\mathscr{R}_{abc}^d=-\mathscr{R}_{bac}^d$.
    \item $\mathscr{R}_{[abc]}^d=0$, puesto que $\nabla_{[a}\nabla_{b}f_{c]}=0$.
    \item Cumple la identidad de Bianchi,
    \[\nabla_{[a}\mathscr{R}_{bc]d}^e=0\]
    puesto que $\nabla_{[a}\nabla_{b}\nabla_{c]}f_d=0$.
    \item Si la conexión es de Levi-Civita, entonces tenemos que
    \[(\nabla_a\nabla_b-\nabla_b\nabla_a)T^{a_1a_2\dots}_{b_1b_2\dots}=-\mathscr{R}_{abc}^{a_1}T^{ca_2\dots}_{b_1b_2\dots}-\dots+\mathscr{R}_{abb_1}^cT^{a_1a_2\dots}_{cb_2\dots}+\dots\]
    Luego, en una base coordenada en la conexión de Levi-Civita, podemos definir el tensor de Riemann como,
    \begin{equation}
        \mathscr{R}_{\mu\nu\rho}^{\sigma}=\partial_{\nu}\Gamma_{\mu\rho}^{\sigma}-\partial_{\mu}\Gamma_{\nu\rho}^{\sigma}+\Gamma_{\mu\rho}^{\lambda}\Gamma_{\lambda\nu}^{\sigma}-\Gamma_{\nu\rho}^{\lambda}\Gamma_{\lambda\mu}^{\sigma}
    \end{equation}
\end{enumerate}
El \textbf{tensor de Ricci} se define como,
    \begin{equation}
        \mathscr{R}_{ab}=\mathscr{R}_{acb}^c
    \end{equation}
El \textbf{escalar de Ricci} se define como,
\begin{equation}
    \mathscr{R}=\mathscr{R}_{ab}g^{ab}
\end{equation}
Para la conexión de Levi-Civita, tenemos que el tensor de Ricci es simétrico, es decir, $\mathscr{R}_{ab}=\mathscr{R}_{ba}$.
%SECCION 1
\section{Ecuaciones de Einstein y Principio de Mínima Acción} % Main chapter title
\label{cap4-sec6} 

La acción de la formulación Lagrangiana para la relatividad general es formulada por Hilbert, pero éste dejó que Einstein lo publicara, y así, la acción de la relatividad general se denomina \textit{acción de ''Hilbert-Einstein''}. Se construye con cantidades invariantes bajo transformaciones generales de coordenadas (difeomorfismos). Se define como,
\begin{equation}
    S_{HE}=\int d^4x\sqrt{|g|}R
\end{equation}
donde $\sqrt{|g|}$ es el Jacobiano y $R$ el escalar de Ricci. que involucra derivadas segundas de $g_{\mu\nu}$, pero se pueden eliminar integrando por partes.\\ \\
Consideramos variaciones $\delta g_{\mu\nu}$ que se anulan en la frontera, es decir, $\left.\delta g_{\mu\nu}\right|_{\partial\mathscr{M}}=0$. Así, podemos variar la acción, tal que
\begin{equation}
    \delta S_{HE}=\int d^4x\brackets{\sqrt{|g|}g^{\mu\nu}\delta R_{\mu\nu}+\sqrt{|g|}(\delta g^{\mu\nu})R_{\mu\nu}+(\delta\sqrt{|g|})g^{\mu\nu}R_{\mu\nu}} 
\end{equation}
donde, asumiendo que la conexión es de Levi-Civita, tenemos
\[\delta\sqrt{|g|}=-\frac{1}{2}\sqrt{|g|}g_{\mu\nu}\delta g^{\mu\nu};\hspace{5mm}\delta R_{\mu\nu\rho}^{\sigma}=\nabla_{\nu}(\delta\Gamma_{\mu\rho}^{\sigma})-\nabla_{\mu}(\delta\Gamma_{\nu\rho}^{\sigma})\]
recordando que $\delta\partial()=\partial\delta()$ y que $\Gamma$ no es un tensor, pero $\delta\Gamma$ sí lo es.
\begin{proof}
    (...)
\end{proof}
Por tanto,
\begin{equation}
    g^{\mu\nu}\delta R_{\mu\nu}=\nabla_{\nu}\left(g^{\mu\rho}\delta\Gamma_{\mu\rho}^{\nu}-g^{\nu\rho}\delta\Gamma_{\sigma\rho}^{\sigma}\right)=\nabla_{\nu}v^{\nu}
\end{equation}
es decir, es igual a la divergencia de un vector. Por tanto,
\begin{equation}
    \int_{\mathscr{M}}d^4x\sqrt{|g|}g^{\mu\nu}\delta R_{\mu\nu}=\int_{\mathscr{M}}d^4x\sqrt{|g|}\nabla_{\nu}v^{\nu}=\int_{\partial\mathscr{M}}d\sigma n_{\nu}v^{\nu}=0
\end{equation}
donde hemos usado el Teorema de Stokes.\\ \\
Si acoplamos materia,
\[S=\frac{1}{\kappa}S_{HE}+S_{\mathscr{M}}\]
y tomamos variaciones con respecto a $g_{\mu\nu}$,
\[\begin{array}{rl}
     \delta_gS&=\frac{1}{\kappa}\delta_gS_{HE}+\delta_gS_{\mathscr{M}}=0  \\
     & =\frac{1}{\kappa}\int d^4x\sqrt{|g|}(R_{\mu\nu}-\frac{1}{2}gR-\kappa T_{\mu\nu})\delta g^{\mu\nu}=0
\end{array}\]
donde hemos definido el tensor de energía-impulso como $T_{\mu\nu}=-\frac{1}{\sqrt{g}}\frac{\delta S_{\mathscr{M}}}{\delta g^{\mu\nu}}$ y $T^{\mu\nu}=\frac{1}{\sqrt{|g|}}\frac{\delta S_{\mathscr{M}}}{\delta g_{\mu\nu}}$.\\ \\
En términos de la densidad lagrangiana (o Lagrangiano), tenemos
\[S_{\mathscr{M}}=\int d^4x\sqrt{|g|}\mathscr{L}_{\mathscr{M}}=\int d^4x\mathcal{L}_{\mathscr{M}}\]
donde $\mathscr{L}_{\mathscr{M}}$ es el Lagrangiano y $\mathcal{L}_{\mathscr{M}}$ es la densidad lagrangiana. Así, el tensor de energía-impulso se reescribe como 
\[T^{\mu\nu}=\frac{2}{\sqrt{|g|}}\frac{\partial\mathcal{L}_{\mathscr{M}}}{\partial g_{\mu\nu}}=2\frac{\partial\mathscr{L}_{\mathscr{M}}}{\partial g_{\mu\nu}}+g^{\mu\nu}\mathscr{L}_{\mathscr{M}}\]
\[T_{\mu\nu}=-\frac{2}{\sqrt{|g|}}\frac{\partial\mathcal{L}_{\mathscr{M}}}{\partial g^{\mu\nu}}=-2\frac{\partial\mathscr{L}_{\mathscr{M}}}{\partial g^{\mu\nu}}+g_{\mu\nu}\mathscr{L}_{\mathscr{M}}\]
El tensor $T_{\mu\nu}$ es conservado como consecuencia de la invariancia bajo transformaciones generales de coordenadas (difeomorfismos) de la acción,
\begin{equation}
    S_{\mathscr{M}}=\int_{\mathcal{M}}d^4x\mathscr{L}(\phi,\nabla\phi,g)
\end{equation}
Dado una densidad lagrangiana, aplicamos un difeomorfismo y vemos como transforman los campos.
Consideramos un difeomorfismo $\xi^{\mu}$ que se anula en frontera $\partial\mathscr{M}$. Tal que
\[\delta_{\xi}\phi=-\mathcal{L}_{\xi}\phi;\hspace{5mm}\delta_{\xi}g_{\mu\nu}=-\mathcal{L}_{\xi}g_{\mu\nu}=-2\nabla_{(\mu}\xi_{\nu)}\]
teniendo así una variación de la acción tal que
\[\delta_{\xi}S_{\mathscr{M}}=-\int_{\mathscr{M}}d^4x(\delta_{\phi}\mathcal{L}_{\mathscr{M}}\mathcal{L}_{\xi}\phi+\frac{\delta\mathcal{L}_{\mathscr{M}}}{\delta g_{\mu\nu}}\Delta_{\mu}\xi_{\nu})\]
donde $\delta_{\phi}\mathscr{L}_{\mathscr{M}}\mathscr{L}_{\xi}\phi=0$, pues se cumplen las ecuaciones del movimiento. Así,
\[\delta_{\xi}S_{\mathscr{M}}=-\int d^4x\sqrt{|g|}(T^{\mu\nu}\nabla_{\mu}\xi_{\nu})=-\int d^4x\sqrt{|g|}\brackets{\underbrace{\nabla_{\mu}(T^{\mu\nu}\xi_{\nu})}_{\int_{\partial\mathscr{M}}(...)=0\text{( }\xi^{\mu}\text{ es cero en }\partial\mathscr{M})}-\xi_{\nu}\nabla_{\mu}T^{\mu\nu}}=0\]
Derivando $T_{\mu\nu}$ tenemos las acciones,
\[\text{Campo escalar}:\hspace{4mm}S_{\phi}=-\frac{1}{2}\int d^4x\sqrt{|g|}(\nabla_{\mu}\phi\nabla_{\nu}\phi g^{\mu\nu}+m^2\phi^2)\]
\[\text{Campo electromagnético}:\hspace{4mm}S_A=-\frac{1}{16\pi}\int d^4x\sqrt{|g|}F_{\mu\nu}F^{\mu\nu};\hspace{3mm}F_{\mu\nu}=2\nabla_{[\mu,}A_{\nu]}\]






































%SECCION 5
\section{Repaso de álgebra} % Main chapter title
\label{cap2-sec7} 
Un vector $\vec{v}\in V$, siendo $V$ un espacio vectorial, podemos escribir $\vec{v} $ en función de una base de $V$, de la forma,
\[\vec{v}=v^{\mu}\hat{e}_{\mu}=v^0\hat{e}_0+v^1\hat{e}_1+\dots+v^s\hat{e}_s\]
donde $v^{\mu}$ son las componentes de $\vec{v}$ en la base $\curlybraces{\hat{e}_{\mu}}$, siendo un vector columna.\\ \\
Los vectores duales son aplicaciones lineales, tal que $\vec{f}:V\to\mathbb{R}$, formando un espacio vectorial dual $V^*$. Además, $\vec{f}=f_{\mu}\hat{e}^{\mu}$, siendo $f_{\mu}$ un vector fila. Es decir, podemos representar $\vec{f}$ en las componentes de la base dual $\curlybraces{\hat{e}^{\mu}}$, que es la única base que cumple que $<\hat{e}^{\mu},\hat{e}_{\nu}>=\delta^{\mu}_{\nu}$.\\ \\
Dada una base dual, podemos extraer las componentes de $\vec{v}\in V$ usando esta base, es decir,
\[<\hat{e}^{\mu},\vec{v}>=<\hat{e}^{\mu},v^{\nu}\hat{e}_{\nu}>=v^{\nu}<\hat{e}^{\mu},\hat{e}_{\nu}>=v^{\nu}\delta_{\mu}^{\nu}=v^{\mu}\]
Podemos hacer lo mismo con los vectores duales dada una base vectorial, tal que
\[<\hat{e}_{\mu},\vec{f}>=<\hat{e}_{\mu},f_{\nu}\hat{e}^{\nu}>=f_{\nu}<\hat{e}_{\mu},\hat{e}^{\nu}>=f_{\nu}\delta^{\mu}_{\nu}=f_{\mu}\]
También podemos hacer el producto escalar entre vectores y vectores duales, tal que
\[<\vec{f},\vec{v}>=<f_{\mu}\hat{e}^{\mu},v^{\nu}\hat{e}_{\nu}>=f_{\mu}v^{\nu}<\hat{e}^{\mu},\hat{e}_{\nu}>=f_{\mu}v^{\nu}\delta_{\nu}^{\mu}=f_{\mu}v^{\mu}\]
Si aplicamos una transformación pasiva $M_{\mu}^{\nu}$, es decir, dejamos el vector fijo y movemos el sistema de referencia, entonces vamos a decir que $v^{'\mu}=M_{\nu}^{\mu}v^{\nu}$ y que $\hat{e}_{\mu}'=(M^{-1})_{\mu}^{\nu}\hat{e}_{\nu}$, pues
\[\vec{v}'=v^{'\mu}\hat{e}'_{\mu}=M_{\nu}^{\mu}v^{\nu}(M^{-1})^{\rho}_{\mu}\hat{e}_{\rho}=\delta_{\nu}^{\rho} v^{\nu}\hat{e}_{\rho}=v^{\nu}\hat{e}_{\nu}=\vec{v}\]
es decir, vemos que el vector no se ve afectado por las transformaciones pasivas, pero sus componentes sí se alteran.\\ \\
Diremos que la métrica de Minkowski nos permite mapear vectores de $V$ a vectores duales de $V^*$, pues dado un vector $v^{\mu}\hat{e}_{\mu}\in V$, vemos que
\[\begin{array}{rlcl}
    \eta_{\mu\nu}: & \hat{e}_{\rho} & \to & \hat{e}^{\rho}=\eta^{\rho\sigma}\hat{e}_{\sigma} \\
     & v^{\rho} & \mapsto & v_{\rho}=\eta_{\rho\sigma}v^{\sigma}
\end{array}\]
es decir, la métrica $\eta_{\mu\nu}:\vec{v}\to\vec{v}^*$ transforma los vectores tal que,
\[\begin{array}{ccl}
    v_0=-v^0; & v_i=v^i; & i=1,2,3 \\
    \hat{e}^0=-\hat{e}_0; & \hat{e}^i=\hat{e}_i &
\end{array}\]
También podemos usar $\eta_{\mu\nu}$ para mapear vectores duales a vectores.\\ \\
La métrica nos da un producto escalar, tal que
\[<\vec{v}^*,\vec{v}>=<v_{\mu}\hat{e}^{\mu},v^{\nu}\hat{e}_{\nu}>=v_{\mu}v^{\nu}<\hat{e}^{\mu},\hat{e}_{\nu}>=v_{\mu}v^{\nu}\delta_{\nu}^{\mu}=v_{\mu}v^{\mu}=\eta_{\mu\nu}v^{\mu}v^{\nu}\]
La métrica $\eta_{\mu\nu}$ nos permite subir y bajar índices, es decir, contraer índices. \\ \\
La métrica transforma como un tensor de rango $(0,2)$, es decir,
\begin{equation}
    \eta_{\mu\nu}^{(M)}=(M^{-1})^{\rho}_{\mu}(M^{-1})_{\nu}^{\sigma}\eta_{\rho\sigma}
\end{equation}
Denotaremos a $\vec{v}$ y $v^{\mu}$ indistintamente, y diremos que son \textbf{vectores contravariantes}. A los vectores del espacio dual los denotaremos por $\vec{f}$ y $f_{\mu}$ indistintamente, denominados \textbf{vectores covariantes}.\\ \\
Un tensor de rango $(r,s)$ es una aplicación multilineal, tal que
\[T_{\nu_1\nu_2\dots\nu_s}^{\mu_1\mu_2\dots\mu_r}:V_1^*\otimes V_2^*\otimes\dots\otimes V_r^*\otimes V_1\otimes V_2\otimes\dots\otimes V_s\to\mathbb{R}\]
Los tensores transforman como,
\begin{equation}
    T_{\nu_1\nu_2\dots\nu_s}^{'\mu_1\mu_2\dots\mu_r}=M^{\mu_1}_{\rho1}M^{\mu_2}_{\rho_2}\dots M^{\mu_r}_{\rho_r}(M^{-1})^{\sigma_1}_{\nu_1}(M^{-1})_{\nu_2}^{\sigma_2}\dots(M^{-1})^{\sigma_s}_{\nu_s}T_{\sigma_1\sigma_2\dots\sigma_s}^{\rho_1\rho_2\dots\rho_r}
\end{equation}
Con la métrica $\eta_{\mu\nu}$ podemos calcular la traza de un tensor, tal que para un tensor $T_{\mu\nu}$, su traza será $T=T_{\mu\nu}\eta^{\mu\nu}$, es decir, hemos contraído todos los índices. Pero si tenemos $T_{\mu}^{\nu}$, su traza será $T=T_{\mu}^{\nu}\delta_{\nu}^{\mu}$ y si tenemos $T^{\mu\nu}$, su traza será $T=T^{\mu\nu}\eta_{\mu\nu}$. O bien, podemos transformar los tensores y aplicar la primera definición, tal que $T_{\mu}^{\nu}=T_{\mu\nu}\eta^{\sigma\nu}$ y $T^{\mu\nu}=T_{\rho\sigma}\eta^{\rho\mu}\eta^{\sigma\nu}$.\\ \\
La métrica está relacionada con el elemento de línea, llegando a tener también la misma clasificación, tal que
\begin{itemize}
    \item Si $\eta_{\mu\nu}v^{\mu}v^{\nu}>0$, entonces diremos que $v^{\mu}$ es un vector espacial.
    \item Si $\eta_{\mu\nu}v^{\mu}v^{\nu}<0$, entonces diremos que $v^{\mu}$ es un vector temporal.
    \item Si $\eta_{\mu\nu}v^{\mu}v^{\nu}=0$, entonces diremos que $v^{\mu}$ es un vector nulo.
\end{itemize}
Además, se cumple que un vector ortogonal a uno de tipo tiempo es espacial y un vector ortogonal a uno de tipo espacio o nulo, no tiene por qué ser temporal, es decir, es de cualquier género.
\subsection{Tensor simétrico y antisimétrico}
Un tensor simétrico será aquel que si se le permutan dos índices, permanece invariante, es decir, $T_{\mu\nu}=T_{\nu\mu}$. La parte simétrica de un tensor es
\begin{equation}
    T_{(\mu\nu)}=\frac{1}{2}(T_{\mu\nu}+T_{\nu\mu})
\end{equation}
Un tensor antisimétrico es aquel que si se le permutan dos índices, cambia de signo, es decir, $T_{\mu\nu}=-T_{\nu\mu}$. La parte antisimétrica de un tensor es
\[T_{[\mu\nu]}=\frac{1}{2}(T_{\mu\nu}-T_{\nu\mu})\]
Un tensor cualquiera siempre se puede descomponer en la suma de su parte simétrica y su parte antisimétrica, es decir,
\[R_{\mu\nu}=R_{(\mu\nu)}+R_{[\mu\nu]}\]
\subsection{Transformaciones de Lorentz. Versión covariante}
Recordemos que las transformaciones de Lorentz son,
\[\begin{array}{rcrc}
    (i) & dt'=\gamma\left(dt-\frac{v}{c^2}dx\right); & (iii) & dy'=dy \\
    (ii) & dx'=\gamma\left(dx-vdt\right); & (iv) & dz'=dz
\end{array}\]
Por tanto, usando la notación covariante, podemos agruparlas todas en una sola ecuación, tal que
\begin{equation}
    dx^{'\mu}=\Lambda_{\nu}^{\mu}dx^{\nu}
\end{equation}
donde 
\[\Lambda_{\nu}^{\mu}=\begin{pmatrix}
    \gamma & -\gamma\frac{v}{c^2} & 0 & 0\\
    -\gamma v & \gamma & 0 & 0 \\
    0 & 0 & 1 & 0\\
    0 & 0 & 0 & 1
\end{pmatrix}\]
es la matriz de Lorentz.\\ \\
Recordando que $ds^2=(ds')^2$, vemos que la métrica de Minkowski y la matriz de Lorentz se pueden relacionar, tal que
\[\begin{array}{cllc}
    (ds')^2 & = & \eta_{\mu\nu}dx^{'\mu}dx^{'\nu}&=\eta_{\mu\nu}\Lambda_{\rho}^{\mu}dx^{\rho}\Lambda_{\sigma}^{\nu}dx^{\sigma} \\
    || & & & || \\
    ds^2 & = & \eta_{\mu\nu}dx^{\mu}dx^{\nu}&=\eta_{\mu\nu}\Lambda_{\rho}^{\mu}\Lambda_{\sigma}^{\nu}dx^{\rho}dx^{\sigma}
\end{array}\]
Por tanto,
\begin{equation}
    \eta_{\rho\sigma}=\eta_{\mu\nu}\Lambda_{\rho}^{\mu}\Lambda_{\sigma}^{\nu}
\end{equation}
Es decir, la métrica de Minkowski permanece invariante bajo transformaciones de Lorentz.\\ \\
Además, la inversa de la matriz de Lorentz también está definida, $(\Lambda^{-1})_{\nu}^{\mu}$, que también deja invariante la inversa de la métrica de Minkowski, y cumple que
\begin{equation}
    \Lambda_{\nu}^{\mu}(\Lambda^{-1})_{\rho}^{\nu}=\delta_{\rho}^{\mu}
\end{equation}
Para pasar de la matriz de Lorentz $\Lambda_{\mu}^{\nu}$ a su inversa $(\Lambda^{-1})_{\nu}^{\mu}$, teniendo un movimiento con velocidad $v$ a lo largo del eje $X$, solo debemos cambiar $v$ por $-v$; igual para rotaciones. Las propiedades de esta matriz son:
\begin{enumerate}
    \item La traspuesta de la matriz de Lorentz es $(\Lambda^{-1})_{\nu}^{\mu}=\Lambda_{\mu}^{\nu}$.
    \item Las componentes de la matriz de Lorentz para boosts de forma general son,
    \[\Lambda_0^0=\gamma;\hspace{3mm}\Lambda_i^0=-\gamma\frac{v^i}{c};\hspace{3mm}\Lambda_0^i=-\gamma\frac{v^i}{c};\hspace{3mm}\Lambda_j^i=\delta_j^i+(\gamma-1)\frac{v^iv^j}{\vec{v}^2}\]
    \item Para las rotaciones de ángulo $\theta$ alrededor de un eje $\hat{\omega}^i$, con $\hat{\omega}^{\mu}=(0,\hat{\omega}^i)$ y $\hat{\omega}^{\mu}\hat{\omega}_{\mu}=1$, las componentes de la matriz de Lorentz serán,
    \[\Lambda_0^0=0;\hspace{3mm}\Lambda_0^i=0=\Lambda_i^0;\hspace{3mm}\Lambda_j^i=\cos\theta\delta_j^i+(1-\cos\theta)\hat{\omega}^i\hat{\omega}_j-\sin\theta\mathscr{E}_{jk}^i\hat{\omega}^k\]
    donde $\mathscr{E}_{jk}^i$ es el tensor de Levi-Civita espacial, que actúa parecido a la delta de Kronceker, tal que
    \[\mathscr{E}_{ijk}=\left\lbrace
    \begin{array}{lcl}
        +1 & \text{si} & i\neq j\neq k\text{ y la permutación es par} \\
        -1 & \text{si} & i\neq j\neq k\text{ y la permutación es impar}\\
        0 & \text{en}&\text{otro caso}
    \end{array}\right.\]
    Para la métrica de Minkowski, $\eta_{\mu\nu}=diag[-1,1,1,1]$, tenemos que el tensor de Levi-Civita cumple que
    \[\mathscr{E}_{ijk}=\mathscr{E}^{ijk}=\mathscr{E}_{jk}^i\]
\end{enumerate}

%SECCION 5
\section{Cinemática relativista} % Main chapter title
\label{cap2-sec8} 
Para tratar la cinemática de forma relativista, debemos formalizar los conceptos de la cinemática clásica en formulación covariante.
\subsection{Vector cuadrivelocidad}
Representa la velocidad espacio-temporal de una partícula puntual y se obtiene derivando respecto al tiempo propio de la partícula su cuadrivector, tal que
\begin{equation}
    u^{\mu}=\frac{dx^{\mu}}{d\tau}=\dot{x}^{\mu}
\end{equation}
Si estamos en el sistema de referencia comóvil de la partícula, entonces la cuadrivelocidad será $u^{\mu}=(c,0,0,0)$, pues en este sistema de referencia, la partícula se encuentra en reposo espacial relativo; no existe el reposo relativo temporal.\\ \\
Si estamos en un SRI cualquiera donde la partícula se mueve a velocidad $\vec{v}=\frac{d\vec{x}}{dt}$, entonces la cuadrivelocidad será $u^{\mu}=\gamma(c,\vec{v})$, pues debemos reemplazar el $dt$ por $d\tau$, recordando que $dt=\gamma d\tau$.\\ \\
Además, sabemos que el producto escalar es un invariante, por lo que usando el sistema de referencia comóvil de la partícula, obtenemos que $u^{\mu}u_{\mu}=-c^2$.
\subsection{Vector cuadrimomento}
Representa el momento espacio-temporal de una partícula, y se obtiene multiplicando la masa en reposo de la partícula por su cuadrivelocidad. Así,
\[p^{\mu}=m_0u^{\mu}\Rightarrow\left\lbrace\begin{array}{l}
      p^0=m_0\gamma c\Rightarrow E=cp^0=m_0\gamma c^2 \\
     p^i=m_0\gamma v^i\Rightarrow \vec{p}=m_0\gamma\vec{v}
\end{array}\right.\]
Por tanto, el cuadrimomento es
\begin{equation}
    p^{\mu}=(E/c,p^1,p^2,p^3)
\end{equation}
Podemos calcular también el producto escalar de cuadrimomentos, tal que $p^{\mu}p_{\mu}=-m^2c^2$.
\subsection{Vector cuadriaceleración}
Representa la aceleración espacio-temporal de una partícula puntual. Se calcula derivando respecto al tiempo propio el vector cuadrivelocidad de la partícula, o derivando dos veces el cuadrivector de la partícula respecto al tiempo propio, tal que
\[b^{\mu}=\frac{d^2x^{\mu}}{d\tau^2}=\frac{du^{\mu}}{d\tau}=\dot{u}^{\mu}=\Ddot{x}^{\mu}\]
La cuadriaceleración de una partícula puntual que se mueve a velocidad $v$ y aceleración $a$ en un SRI cualquiera es,
\begin{equation}
    b^{\mu}=\left(\gamma^4\frac{\vec{v}\cdot\vec{a}}{d\tau^2},\gamma^4\frac{(\vec{v}\cdot\vec{a})\vec{v}}{c^2}+\gamma^2\vec{a}\right)
\end{equation}
donde $\vec{a}=\frac{d\vec{v}}{dt}$ y $\frac{d\gamma}{dt}=\gamma^3\frac{\vec{v}\cdot\vec{a}}{c^2}$.\\ \\
Las propiedades de la cuadriaceleración son:
\begin{enumerate}
    \item $u^{\mu}b_{\mu}=0$
    \item $b^{\mu}b_{\mu}=\gamma^4\left(\gamma^2\frac{\vec{a}\cdot\vec{v}}{c^2}+\vec{a}\cdot\vec{a}\right)\geq0$. Por tanto, esto implica que $b^{\mu}$ es de género espacio.
\end{enumerate}
\subsection{Derivación}
Al igual que tenemos un gradiente tridimensional, podemos construir un gradiente cuadridimensional, tal que
\begin{equation}
    \partial_{\mu}f=f_{,\mu}=\frac{\partial f}{\partial x^{\mu}}=\left(\frac{1}{c}\frac{\partial f}{\partial t},\nabla f\right)
\end{equation}
siendo un vector covariante, por lo que podemos construir su versión contravariante, tal que
\begin{equation}
    \partial^{\mu}f=\eta^{\mu\nu}\partial_{\nu}f=\left(-1\frac{1}{c}\frac{\partial f}{\partial t},\nabla f\right)
\end{equation}
\subsection{Operador D'Alembertiano}
Se define como el Laplaciano cuadridimensional, tal que
\begin{equation}
    \square f=\eta^{\mu\nu}\partial_{\mu}\partial_{\nu}f=-c^2\partial_t^2f+\nabla^2f
\end{equation}
Es invariante bajo transformaciones de Lorentz, pues si tenemos dos SRI $S'$ y $S$, los operadores D'Alembertianos de cada SRI cumplirán que $\square'=\square$.
\subsection{Tensor de Levi-Civita}
Es un tensor totalmente antisimétrico. se define como,
\begin{equation}
    \mathscr{E}^{\mu\nu\rho\sigma}=\left\lbrace\begin{array}{ll}
        +1 &\text{si }\mu\neq\nu\neq\rho\neq\sigma \text{y la perturbación es par} \\
        -1 & \text{si }\mu\neq\nu\neq\rho\neq\sigma \text{y la perturbación es imppar}\\
        0 & \text{en otro caso}
    \end{array}\right.
\end{equation}
Este tensor cumple que,
\begin{enumerate}
    \item $\mathscr{E}_{0123}=-\mathscr{E}^{0123}$.
    \item $\mathscr{E}^{\mu\nu\rho\sigma}\mathscr{E}_{\mu\nu\rho\sigma}=-4!$.
    \item $\mathscr{E}^{\mu\nu\rho\sigma}\mathscr{E}_{\mu\nu\rho\gamma}=-3!\delta_{\gamma}^{\sigma}$.
    \item $\mathscr{E}^{\mu\nu\rho\sigma}\mathscr{E}_{\alpha\beta\rho\sigma}=-2!\delta_{\mu}^{[\alpha}\delta_{\beta]}^{\nu}$.
    \item $\mathscr{E}^{\mu\nu\sigma\rho}\mathscr{E}_{\alpha\beta\gamma\rho}=-1!\delta_{[\alpha}^{\mu}\delta_{\beta}^{\nu}\delta_{\gamma]}^{\sigma}$.
    \item $\mathscr{E}^{\mu\nu\sigma\rho}\mathscr{E}_{\alpha\beta\gamma\delta}=\delta_{[\alpha}^{\mu}\delta_{\beta}^{\nu}\delta_{\gamma}^{\sigma}\delta_{\delta]}^{\rho}$.
\end{enumerate}
%SECCION 5
\section{Grupo de Poincaré} % Main chapter title
\label{cap2-sec9} 
Este grupo es el grupo de transformaciones que deja invariante el elemento de línea. Tenemos,
\begin{itemize}
    \item \textbf{Traslaciones espaciotemporales:} son del tipo $x^{'\mu}=x^{\mu}+\alpha^{\mu}$, donde $\alpha^{\mu}=cte$.\\
    El operador momento en mecánica cuántica se define como $\hat{P}_{\mu}=-i\partial_{\mu}=-i\frac{\partial}{\partial x^{\mu}}$, pero si definimos el operador $\hat{U}_{\alpha^{\mu}}=\exp[i\alpha^{\mu}\hat{P}_{\mu}]$, que podemos expandirlo en serie de potencias, tal que
    \[\hat{U}_{\alpha^{\mu}}f(x^{\mu})=\left[1+\alpha^{\mu}(\partial_{\mu})+\frac{1}{2|}\left(\alpha^{\mu}(\partial_{\mu})^2\right)+\dots\right]f(x^{\mu})=f(x^{\mu}+\alpha^{\mu})=f(x^{'\mu})\]
    por tanto, este operador nos permite hacer que $\hat{U}_{\alpha^{\mu}}x^{\mu}=x^{\mu}+\alpha^{\mu}=x^{'\mu}$.\\ \\
    Al operador $\hat{P}^{\mu}$ se le denomina \textit{generador de traslaciones}. Podemos calcular el conmutador, tal que $\brackets{\hat{P}_{\mu},\hat{P}_{\nu}}=0$, si y solo si el grupo de traslaciones es un grupo abeliano, por lo que las traslaciones conmutan.
    \item \textbf{Transformaciones de Lorentz:} están formadas por las rotaciones espaciales y las transformaciones de Lorentz puras, los boosts.\\ \\
    Sabemos que $\det(\Lambda_{nu}^{\mu})=\pm1$, pero nos vamos a quedar con el subgrupo propio de las transformaciones con $\det(\Lambda_{\nu}^{\mu})=+1$ y dentro de este subgrupo, nos quedamos con el subgrupo ortocrono, es decir, el subgrupo donde se mantiene la dirección temporal, $\Lambda_{0}^{0}\geq1$. Si consideramos una transformación de Lorentz infinitesimal, tendremos que
    \begin{equation}
    \Lambda_{\nu}^{\mu}=\delta_{\nu}^{\mu}+\delta\omega^{\mu}_{\nu}
    \end{equation}
    Recordamos que $\eta_{\mu\nu}=\eta^{\rho}_{\mu}\Lambda_{\nu}^{\sigma}\eta_{\rho\sigma}$, por lo que sustituyendo tenemos que,
    \begin{equation}
        \delta\omega_{\mu\nu}+\delta\omega_{\nu\mu}=0
    \end{equation}
    a primer orden. Es decir, $\delta\omega_{\mu\nu}$ es un tensor antisimétrico.\\ \\
    Al ser un tensor antisimétrico, podemos hacer la siguiente clasificación:
    \[\delta\omega_{0i}=-\delta\omega_{i0}=\frac{\delta v_i}{c}=\delta\xi_i\]
    que se hace cargo de los boosts.
    \[\delta\omega_{ij}=-\delta\omega_{ji}=-\mathscr{E}_{ijk}\delta\theta^k\]
    que se hace cargo de las rotaciones.\\ \\
    Decimos que los $\xi_i$ y $\theta^k$ son los vectores de Killing de las traslaciones espaciales y rotaciones, respectivamente.
\end{itemize}
Como el grupo de Lorentz tiene 4 traslaciones espaciotemporales, 3 boosts y 3 rotaciones; decimos que este grupo tiene 10 grados de libertad.
\\ \\
Veamos cómo cambian los vectores,
\[x^{'\mu}=x^{\mu}+\delta\ x^{\mu}\]
con
\[\delta x^{\mu}=\delta\omega^{\mu}_{\nu}x^{\nu}=-\frac{i}{2}\delta\omega^{\rho\sigma}\hat{L}_{\rho\sigma}x^{\mu}\]
donde 
\[\hat{L}_{\mu\nu}=\hat{X}_{\mu}(-i\partial_{\nu})-\hat{X}_{\nu}(-i\partial_{\mu})=\hat{X}_{\mu}\hat{P}_{\nu}-\hat{X}_{\nu}\hat{P}_{\mu}\]
Definimos el momento angular espacial tal que
\begin{equation}
    \hat{L}^i=\frac{1}{2}\mathscr{E}^{ijk}\hat{L}_{jk}
\end{equation}
Definimos $\hat{K}_i=\hat{L}_{0i}$ como el generador de los boosts, tal que
\begin{equation}
    \delta x^{\mu}=i\left(\delta\vec{\theta}\cdot\vec{L}+\delta\vec{\xi}\cdot\hat{\vec{K}}\right)x^{\mu}
\end{equation}
Vemos que
\[[\hat{L}_i,\hat{L}_j]f(x)=i\mathscr{E}_{ij}^k\hat{L}_k(f(x))\]
\[[\hat{L}_i,\hat{K}_j]=i\mathscr{E}_{ij}^k\hat{K}_k\]
Luego, no conmutan, pues no es lo mismo rotar y hacer un boost, que hacer un boost y luego rotar.\\ \\
Además, si hacemos dos transformaciones de Lorentz en dos direcciones diferentes, tampoco conmutan, pues $[\hat{K}_i,\hat{K}_j]=-i\mathscr{E}_{ij}^k\hat{L}_k$. Esto es lo que se conoce como \textit{Rotación de Weigner}, que se traduce en: 'boost + boost = boost + rotación'.\\ \\
Para boosts finitos, es decir, velocidad finita, podemos identificar el vector $\vec{\xi}$ con la velocidad, tal que
\begin{equation}
    \vec{\xi}=\vec{v}\cdot\arctan\left(\frac{|\vec{v}|}{c}\right)
\end{equation}
La versión matricial de $\hat{L}_i$ y $\hat{K}_i$ son,
\[\begin{array}{ccc}
    (L_1)^{\mu}_{\nu}=\begin{pmatrix}
        0 & 0 & 0 & 0\\
        0 & 0 & 0 & 0\\
        0 & 0 & 0 & -i\\
        0 & 0 & i & 0
    \end{pmatrix}; & (L_2)_{\nu}^{\mu}=\begin{pmatrix}
        0 & 0 & 0 & 0 \\
        0 & 0 & 0 & i \\
        0 & 0 & 0 & 0 \\
        0 & -i & 0 & 0
    \end{pmatrix}; & (L_3)_{\nu}^{\mu}=\begin{pmatrix}
        0 & 0 & 0 & 0 \\
        0 & 0 & -i & 0 \\
        0 & i & 0 & 0 \\
        0 & 0 & 0 & 0
    \end{pmatrix} \\ \\
    (K_1)_{\nu}^{\mu}=\begin{pmatrix}
        0 & i & 0 & 0 \\
        i & 0 & 0 & 0 \\
        0 & 0 & 0 & 0 \\
        0 & 0 & 0 & 0
    \end{pmatrix}; & (K_2)_{\nu}^{\mu}=\begin{pmatrix}
        0 & 0 & i & 0 \\
        0 & 0 & 0 & 0 \\
        i & 0 & 0 & 0 \\
        0 & 0 & 0 & 0 
    \end{pmatrix}; & (K_3)_{\nu}^{\mu}=\begin{pmatrix}
        0 & 0 & 0 & i \\
        0 & 0 & 0 & 0 \\
        0 & 0 & 0 & 0 \\
        i & 0 & 0 & 0
    \end{pmatrix}
\end{array}\]
    \chapter{Ondas gravitacionales} %
\label{Capitulo5} %
\lhead{\emph{Ondas gravitacionales}}
\textit{“La creatividad es la inteligencia divirtiéndose”.}\\
(A. Einstein)
\newpage
%-------------------------------------------------------------------------------
%SECCION 1
\section{Repaso histórico} % Main chapter title
\label{cap2-sec1} 
%------------------------------------------------------------------------------
	La física clásica, del siglo XIX, era una física bien asentada. La cuál explica la mecánica con el libro de Sir Isaac Newton titulado \textit{Philosophiae Naturalis Principia Mathematica} y el electromagnetismo se explica con el libro de Maxwell titulado \textit{Electricity and Magnetism}.\\
 En 1887, Michelson y Morley iniciaron una revolución en la física con un experimento para medir la velocidad de la luz. El experimento consistía en medir la velocidad de la luz de un rayo paralelo al eje de rotación de la Tierra y de otro rayo perpendicular a este, esperándose obtener resultados diferentes. En cambio, se observó que ambos rayos iban exactamente igual, cosa que no tenía sentido en la época., por tanto, determinaron que la velocidad de la luz no era instantánea, sino que debía ser finita, y llegaron a un resultado de ésta bastante próximo al valor actual de la velocidad de la luz.
 \subsection{Relatividad Galileana}
 El Principio de Relatividad de Galileo establece que,
 \begin{center}
 \textit{''Es imposible determinar a base de experimentos (mecánicos) si un sistema de referencia está en reposo o en movimiento uniforme y rectilíneo''.}
 \end{center}
 Esto se derivó de que en la Relatividad Galileana hay un espacio absoluto en el que las leyes de Newton son ciertas. Definiremos un \textit{sistema de referencia inercial} (SRI) como aquel sistema referencia que se mueve a velocidad constante respecto al espacio absoluto. Además, todos los sistemas de referencia inerciales comparten un tiempo absoluto. Pero con la definición de SRI, el Principio de Relatividad se debe reformular con este concepto, así tenemos el Principio de Relatividad en formulación de equivalencia, que dice que
 \begin{center}
 \textit{''Todos los sistemas inerciales son equivalentes, es decir, todos los observadores inerciales ven la misma física''.}
 \end{center}
 \textbf{Leyes de Newton}\\ \\
 La Ley de Newton por excelencia es $\vec{F}=m\vec{a}=-\nabla V(\vec{r}-\vec{r}_0)$, donde $V$ es la función potencial. Esta ley (y las demás) transforman bien bajo el grupo de transformaciones de Galileo, que son:
 \begin{enumerate}
     \item \textbf{Traslaciones temporales:}
     \[t\to t'=t+t_0\]
     \item \textbf{Traslaciones espaciales:}
     \[\vec{r}\to\vec{r}'=\vec{r}+\vec{r}_i+\vec{v}t\]
     donde $\vec{v}$ es la velocidad relativa de un SRI con respecto al otro, y $\vec{r}_i$ es el vector de posición entre los orígenes de ambos SRI al inicio.
     \item \textbf{Rotaciones espaciales:}
     \[\vec{a}'=R(\theta)\vec{a}\]
     donde $R(\theta)$ es la matriz de rotación.
\end{enumerate}
Se puede ver que las Leyes de Newton no son covariantes, pero sí transforman bien, pues la física se mantiene, esto quiere decir que \textit{las Leyes de Newton de la física transforman de forma covariante}.\\ \\
El grupo de transformaciones de Galileo son simetrías que dan lugar a cantidades conservadas. Por tanto, si tenemos un Lagrangiano que sea invariante bajo traslaciones temporales, tendremos que el sistema conserva energía; si es invariante bajo traslaciones espaciales, conserva momento lineal; y si es invariante bajo rotaciones espaciales; conserva momento angular.\\ \\
El grupo de transformaciones de Galileo NO deja invariante las ecuaciones de Maxwell, que son
\[(i)\hspace{2mm}\nabla\cdot\vec{E}=\rho/\epsilon_0;\hspace{5mm}(iii)\hspace{2mm}\nabla\cdot\vec{B}=0\]
\[(ii)\hspace{2mm}\nabla\times\vec{B}=\partial_t\vec{E}/c^2+\mu_0\vec{J};\hspace{5mm}(iv)\hspace{2mm}\nabla\times\vec{E}=-\partial_t\vec{B}\]
Si $\rho=0$ y $\vec{J}=0$, es decir, estamos en vacío, podemos combinar las ecuaciones de Maxwell en una sola ecuación de ondas que se propaga a velocidad $c=299792,458$ m/s, resultado muy próximo al valor obtenido por Michelson y Morley, que además es independiente del sistema de referencia.
\subsection{Transformaciones de Lorentz}

Las transformaciones de Lorentz hacen que las ecuaciones de Maxwell transformen bien (sean covariantes). Estas transformaciones son:
\[\begin{array}{rcrc}
    (i) & t'=\gamma\left(t-\frac{v}{c^2}x\right); & (iii) & y'=y \\
    (ii) & x'=\gamma\left(x-vt\right); & (iv) & z'=z
\end{array}\]
donde $v$ es la velocidad relativa entre SRI (que suponemos que se mueven en el eje $X$), y $\gamma=\frac{1}{\sqrt{1-\frac{v^2}{c^2}}}$.\\ \\
Como estas transformaciones hacen que las leyes de Maxwell sean covariantes, diremos que las transformaciones de Lorentz sean más fundamentales que las transformaciones de Galileo.\\ \\
Además, vemos que por la transformación $(i)$ el tiempo ya \textbf{no es absoluto}, sino que depende del SRI, por lo que diremos que el tiempo es \textbf{relativo}.
%SECCION 2
\section{Álgebra de Tensores} % Main chapter title
\label{cap1-sec2} 
Llegamos a lo groso del capítulo, el \textbf{Álgebra de Tensores}. En este apartado vamos a ver qué es un tensor de forma matemática y cómo trabajar con ellos. También se mencionará cómo trabajamos los físicos con los tensores.
%------------------------------------------------------------------------------

\subsection{Producto tensorial: caso de dos términos} % Main chapter title
\label{cap1-sec2-subsec1} 
Vamos a ver qué es el \textbf{producto tensorial} y cómo los tensores se definen a partir de este.
\begin{proposition}
    Sea $V$ un $\mathbb{K}$-espacio vectorial, $\scalar{\cdot}{\cdot}$ el producto escalar euclídeo y $B=\curlybraces{v_1,\dots,v_n}$ base de $V$, 
    \[\begin{array}{cccl}
        f_v: & V & \to & V^*\\
         & v & \mapsto & f_v(v)=\scalar{v}{\cdot}
    \end{array}\]
     $f_v$ es una aplicación lineal, concretamente es un isomorfismo.
\end{proposition}
\begin{proof}
    Vemos que $f_v$ es aplicación lineal,
    \[f_v(w_1+w_2)=\scalar{v}{w_1+w_2}=\scalar{v}{w_1}+\scalar{v}{w_2}=f_v(w_1)+f_v(w_2)\checkmark \]
    \[f_v(\lambda\cdot w)=\scalar{v}{\lambda\cdot w}=\lambda\scalar{v}{w}=\lambda f_v(w)\checkmark\]
    para $\forall\lambda\in\mathbb{K}$ y $\forall w_1,w_2,w\in V$. Luego, es aplicación lineal.\\ \\
    Veamos que es isomorfo demostrando que es biyectivo, pues ya hemos visto que es aplicación lineal.\\
    Sabemos que $ker\curlybraces{f_v}=\curlybraces{0}\Leftrightarrow f_v$ es inyectiva. Luego, vemos si $ker\curlybraces{f_v}=\curlybraces{0}$:
    \[ker\curlybraces{f_v}=\curlybraces{w\in V,f_v(w)=0}=\curlybraces{w\in V;\scalar{v}{w}=0\Leftrightarrow w=0}\]
    Por tanto, $ker\curlybraces{f_v}=\curlybraces{0}$ y así, $f_v$ es inyectiva. $\checkmark$\\ \\
    Usando el Primer Teorema de isomorfía, tenemos que $dim(V)=\cancelto{0}{dim(ker\curlybraces{f_v})}+dim(Im f_v)$, pero como la $dim B=dim B^*$, siendo $B$ base de $V$ y $B^*$ base de $V^*$, entonces $dimV=dimV^*$, y por tanto, $dimV=dimImf_v=dimV^*$, luego $Imf_v$ es $V^*$ y por tanto, $f_v$ es sobreyectiva. $\checkmark$\\
    Luego, $f_v$ es un isomorfismo.
\end{proof}
\noindent Veamos cómo se define el producto tensorial y sus propiedades.
\begin{definition}
    Sea $V$ un $\mathbb{K}$-espacio vectorial, $V^*$ el dual de $V$, y $g^1,g^2\in V^*$ aplicaciones lineales, tal que $g^1:V\to\mathbb{K}$ y $g^2:V\to\mathbb{K}$. Así, definimos el producto tensorial como,
    \begin{enumerate}[label=(\roman*)]
        \item Producto tensorial entre dos formas $g^1,g^2\in V^*$,
        \[\begin{array}{cccl}
            \ptensor{g^1}{g^2}: & V\times V & \to & \mathbb{K}\\
            & (v,w) & \mapsto & g^1(v)g^2(w)
        \end{array}\]
        \item Producto tensorial entre dos vectores $v_1,v_2\in V$,
        \[\begin{array}{cccl}
            \ptensor{v_1}{v_2}: & V^*\times V^* & \to & \mathbb{K}\\
             & (f,g) & \mapsto & f(v_1)g(v_2)
        \end{array}\]
        \item Producto tensorial de una forma y un vector $v_1\in V$, $f^1\in V^*$,
        \[\begin{array}{cccl}
            \ptensor{v_1}{f^1}: & V^*\times V & \to & \mathbb{K}\\
             & (g,w) & \mapsto & g(v_1)f^1(w)
        \end{array}\]
    \end{enumerate}
\end{definition}

\begin{proposition}
    Los productos tensoriales definidos anteriormente son formas bilineales.
\end{proposition}
\begin{proof} 
Usando $\forall v_1,v_2,u_1,u_2,v,w,u\in V$, $\forall f^1,f^2,g,p,q\in V^*$ y $\forall \lambda\in\mathbb{K}$,
    \begin{enumerate}[label=(\roman*)]
        \item \[\begin{array}{cccl}
            \ptensor{f^1}{f^2}: & V\times V & \to & \mathbb{K}\\
            & (v,w) & \mapsto & f^1(v)f^2(w)
        \end{array}\]
        siendo $f^1,f^2\in V^*$. Veamos que es forma bilineal,
        \[\begin{array}{lrl} \text{\textbullet)} &(\ptensor{f^1}{f^2})(u_1+u_2,v)=&f^1(u_1+u_2)f^2(v)=\brackets{f^1(u_1)+f^1(u_2)}f^2(v)\\ &=&f^1(u_1)f^2(v)+f^1(u_2)f^2(v)=(\ptensor{f^1}{f^2})(u_1,v)+(\ptensor{f^1}{f^2})(u_2,v),\checkmark\\  \text{\textbullet)} &(\ptensor{f^1}{f^2})(v,u_1+u_2)  =&f^1(v)f^2(u_1+u_2)=f^1(v)\brackets{f^2(u_1)+f^2(u_2)}\\ &=&f^1(v)f^2(u_1)+f^1(v)f^2(u_2)=(\ptensor{f^1}{f^2})(v,u_1)+(\ptensor{f^1}{f^2})(v,u_2)\checkmark\\
             \text{\textbullet)} & (\ptensor{f^1}{f^2})(\lambda v,u) =& f^1(\lambda v)f^2(u)=\lambda f^1(v)f^2(u)=\lambda(\ptensor{f^1}{f^2})(v,u)\checkmark\\
        \text{\textbullet)}&(\ptensor{f^1}{f^2})(u,\lambda v)=&f^1(u)f^2(\lambda v)=\lambda f^1(u)f^2(v)=\lambda(\ptensor{f^1}{f^2})(u,v)\checkmark
          \end{array}\]
        Luego, $\ptensor{f^1}{f^2}$ es una forma bilineal. $\qedh $
        \item \[\begin{array}{cccl}
            \ptensor{v_1}{v_2}: & V^*\times V^* & \to & \mathbb{K}\\
             & (f,g) & \mapsto & f(v_1)g(v_2)
        \end{array}\]
         \[\begin{array}{lrl}
         \text{\textbullet)}&(\ptensor{v_1}{v_2})(f^1+f^2,g)=&(f^1+f^2)(v_1)g(v_2)=\brackets{f^1(v_1)+f^2(v_1)}g(v_2)\\
         &=&f^1(v_1)g(v_2)+f^2(v_1)g(v_2)=(\ptensor{v_1}{v_2})(f^1,g)+(\ptensor{v_1}{v_2})(f^2,g)\checkmark\\
         \text{\textbullet)}&(\ptensor{v_1}{v_2})(g,f^1+f^2)=&g(v_1)(f^1+f^2)(v_2)g=g(v_1)\brackets{f^1(v_2)+f^2(v_2)}\\
         &=&g(v_1)f^1(v_2)+g(v_1)f^2(v_2)=(\ptensor{v_1}{v_2})(g,f^1)+(\ptensor{v_1}{v_2})(g,f^2)\checkmark\\
         \text{\textbullet)}&(\ptensor{v_1}{v_2})(\lambda f,g)=&(\lambda f)(v_1)g(v_2)=\lambda f(v_1)g(v_2)=\lambda(\ptensor{v_1}{v_2})(f,g)\checkmark\\
         \text{\textbullet)}&(\ptensor{v_1}{v_2})(g,\lambda f)=&g(v_1)(\lambda f)(v_2)=\lambda g(v_1)f(v_2)=\lambda(\ptensor{v_1}{v_2})(g,f)\checkmark
         \end{array}\]
        Luego, $\ptensor{v_1}{v_2}$ es una forma bilineal. $\qedh $
        \item \[\begin{array}{cccl}
            \ptensor{v_1}{f^1}: & V^*\times V & \to & \mathbb{K}\\
             & (g,w) & \mapsto & g(v_1)f(w)
        \end{array}\]
        \[\begin{array}{lrl}
        \text{\textbullet)}&(\ptensor{v_1}{f^1})(p+q,w)=&(p+q)(v_1)f^1(w)=\brackets{p(v_1)+q(v_1)}f^1(w)=\\
        &=&p(v_1)f^1(w)+q(v_1)f^1(w)=(\ptensor{v_1}{f^1})(p,w)+(\ptensor{v_1}{f^1})(q,w)\checkmark\\
        \text{\textbullet)}&(\ptensor{v_1}{f^1})(g,u+w)=&g(v_1)f^1(u+w)=g(v_1)\brackets{f^1(u)+f^1(w)}=\\
        &=&g(v_1)f^1(u)+g(v_1)f^1(w)=(\ptensor{v_1}{f^1})(g,u)+(\ptensor{v_1}{f^1})(g,w)\checkmark\\
        \text{\textbullet)}&(\ptensor{v_1}{f^1})(\lambda g,w)=&(\lambda g)(v_1)f^1(w)=\lambda g(v_1)f^1(w)=\lambda(\ptensor{v_1}{f^1})(g,w)\checkmark\\
        \text{\textbullet)}&(\ptensor{v_1}{f^1})(g,\lambda w)=&g(v_1)f^1(\lambda w)=\lambda g(v_1)f^1(w)=\lambda(\ptensor{v_1}{f^1})(g,w)\checkmark
        \end{array}\]
            Luego, $\ptensor{v_1}{f^1}$ es una forma bilineal. \qedhere
    \end{enumerate}
\end{proof}
\noindent El producto tensorial no se da solo entre elementos de los espacios vectoriales o duales, sino que también se puede dar entre espacios, siendo el nuevo espacio generado un \textbf{espacio vectorial}.
\begin{proposition}
    El espacio $\ptensor{V}{V}$ tiene estructura de espacio vectorial.
\end{proposition}
\begin{proof}
    \begin{enumerate}
        \item Vemos que $(\ptensor{V}{V},+)$ es grupo abeliano:
        \begin{enumerate}[label=(\roman*)]
            \item Vemos si la operación $+$ es cerrada:
            \\
            $\forall v,w,z\in V$ con $\ptensor{v}{w},\ptensor{v}{z},\ptensor{w}{z}\in\ptensor{V}{V}$, tenemos que ver si $\ptensor{(v+w)}{z}\in\ptensor{V}{V}$. Sabemos que,
            \[\begin{array}{cccl}
                \ptensor{v}{w}: & \pcart{V^*}{V^*} & \to &\mathbb{R}  \\
                 & (f,g) & \mapsto & f(v)g(w)
            \end{array}\]
            luego,
            \[\begin{array}{cccl}
                \ptensor{(v+w)}{z}: & \pcart{V^*}{V^*} & \to &\mathbb{R}  \\
                 & (f,p) & \mapsto & f(v+w)p(z)
            \end{array}\]
            Entonces,
            \[(\ptensor{(v+w)}{z})(g,p)=f(v+w)p(z)=\brackets{f(v)+f(w)}p(z)=\]\[=f(v)p(z)+f(w)p(z)=(\ptensor{v}{z})(f,p)+(\ptensor{w}{z})(f,p)\]
            Luego, $\ptensor{(v+w)}{z}\in\ptensor{V}{V}$ y así, la operación $+$ es cerrada. $\checkmark$
            \item Asociatividad:
            \\
            Sean $\ptensor{a}{b},\ptensor{c}{d},\ptensor{e}{f}\in\ptensor{V}{V}$, tenemos que ver si $\ptensor{a}{b}+\brackets{\ptensor{c}{d}+\ptensor{e}{f}}=\brackets{\ptensor{a}{b}+\ptensor{c}{d}}+\ptensor{e}{f}$, tal que
            \[(\ptensor{a}{b}+\brackets{\ptensor{c}{d}+\ptensor{e}{f}})(p,q)=p(a)q(b)+\brackets{p(c)q(d)+p(e)q(f)}=p(a)q(b)+p(c+e)q(d+f)=\]
            \[=p(a+c+e)q(b+d+f)=p(a+c)q(b+d)+p(e)q(f)=\brackets{p(a)q(b)+p(c)q(d)}+p(e)q(f)=\]\[=(\brackets{\ptensor{a}{b}+\ptensor{c}{d}}+\ptensor{e}{f})(p,q)\checkmark\]
            \item Elemento neutro:\\
            Sea $\ptensor{e_1}{e_2}\in\ptensor{V}{V}$ el elemento neutro de $\ptensor{V}{V}$, tal que
            \[\ptensor{e_1}{e_2}+\ptensor{v}{w}=\ptensor{v}{w}+\ptensor{e_1}{e_2}=\ptensor{v}{w}\]
            Vemos el valor de este elemento neutro,
            \[(\ptensor{e_1}{e_2}+\ptensor{v}{w})(f,g)=(\ptensor{v}{w})(f,g)\]
            \[f(e_1)g(e_2)+f(v)+g(w)=f(v)g(w)\]
            \[f(e_1+v)g(e_w+w)=f(v)g(w)\Leftrightarrow\left\lbrace\begin{matrix}
                e_1=0\\
                e_2=0
            \end{matrix}\right.\]
            luego, $\ptensor{e_1}{e_2}=0$. $\checkmark$
            \item Elemento simétrico:
            \\
            $\forall\ptensor{v}{u}\in\ptensor{V}{V}$, $\exists\ptensor{\Tilde{v}}{\Tilde{u}}\in\ptensor{V}{V}$, tal que
            \[\ptensor{v}{u}+\ptensor{\Tilde{v}}{\Tilde{u}}=\ptensor{\Tilde{v}}{\Tilde{u}}+\ptensor{v}{u}=\ptensor{e_1}{e_2}=0\]
         Veamos quién es $\ptensor{\Tilde{v}}{\Tilde{u}}$,
        \[(\ptensor{v}{u}+\ptensor{\Tilde{v}}{\Tilde{u}})(f,g)=f(v)g(u)+f(\Tilde{v})g(\Tilde{u})=(\ptensor{0}{0})(f,g)=f(0)g(0)\]
        luego,
        \[v+\Tilde{v}=0\Rightarrow\Tilde{v}=-v\]
        \[u+\Tilde{u}=0\Rightarrow\Tilde{u}=-u\]
        Por tanto, el elemento simétrico de $\ptensor{v}{u}$ es $\ptensor{(-v)}{(-u)}$. $\checkmark$
        \item Conmutabilidad:\\
        Sean $\ptensor{v}{w},\ptensor{u}{z}\in\ptensor{V}{V}$, entonces
        \[(\ptensor{v}{w}+\ptensor{u}{z})(f,g)=f(v)g(w)+f(u)g(z)=f(v+u)g(w+z)=\]\[=f(u+v)g(z+w)=f(u)g(z)+f(v)g(w)=(\ptensor{u}{z}+\ptensor{v}{w})(f,g)\checkmark\]
    Luego, es grupo abeliano. $\checkmark$
         \end{enumerate}
         \item Doble propiedad distributiva:
         \begin{enumerate}
             \item $\forall\lambda,\mu\in\mathbb{R}$, $\forall\ptensor{v}{w}\in\ptensor{V}{V}$,
             \[(\lambda+\mu)\cdot(\ptensor{v}{w})(f,g)=(\lambda+\mu)f(v)g(w)=\]\[=\lambda f(v)g(w)+\mu f(v)g(w)=\lambda(\ptensor{v}{w})(f,g)+\mu(\ptensor{v}{w})(f,g)\checkmark\]
             \item $\forall\lambda\in\mathbb{R}$, $\forall\ptensor{v}{w},\ptensor{u}{z}\in\ptensor{V}{V}$, tenemos que
             \[\lambda(\ptensor{v}{w})(f,g)+\lambda(\ptensor{u}{z})(f,g)=\lambda f(v)g(w)+\lambda f(u)g(z)=\]\[=\lambda\brackets{f(v)g(w)+f(u)g(z)}=\lambda(\ptensor{v}{w}+\ptensor{u}{z})(f,g)\checkmark\]
             \end{enumerate}
             \item Propiedad pseudo-asociativa:\\
             $\forall\lambda,\mu\in\mathbb{R}$; $\forall\ptensor{v}{w}\in\ptensor{V}{V}$, tenemos que
             \[\lambda\cdot\brackets{\mu\cdot(\ptensor{v}{w})(f,g)}=\lambda\brackets{\mu f(v)g(w)}=\lambda f(\mu v)g(\mu w)=\]\[=f(\lambda\mu v)g(\lambda\mu w)=f(\mu\lambda v)g(\mu\lambda w)=\mu\brackets{f(\lambda v)g(\lambda w)}=(\mu\cdot\lambda)f(v)g(w)=(\mu\cdot\lambda)(\ptensor{v}{w})(f,g)\checkmark\]
             \item Elemento unitario del cuerpo: $\forall\ptensor{v}{w}\in\ptensor{V}{V}$; $\Tilde{\mu}\in\mathbb{R}$, entonces $\Tilde{\mu}\cdot\ptensor{v}{w}=\ptensor{v}{w}\cdot\Tilde{\mu}=\ptensor{v}{w}$
             \[(\Tilde{\mu}\cdot\ptensor{v}{w})(f,g)=f(\Tilde{\mu}v)g(\Tilde{\mu}w)=(\ptensor{v}{w})(f,g)=f(v)g(w)\Rightarrow\begin{matrix}
                 \Tilde{\mu}\cdot v=v\\
                 \Tilde{\mu}\cdot w=w
             \end{matrix}\Leftrightarrow\Tilde{\mu}=1\checkmark\]
       \end{enumerate}
       Luego, $(\ptensor{V}{V}, +, \cdot)$ es un $\mathbb{R}$-espacio vectorial.
\end{proof}
\noindent Al igual que cualquier otro espacio vectorial, el espacio $V\otimes V$ deberá tener una \textbf{base}.
\begin{proposition}
    Si tenemos un $V$ espacio vectorial sobre $\mathbb{K}$ con base $B=\curlybraces{v_1,\dots,v_n}$, entonces todo $\ptensor{v}{w}$ será combinación lineal de los elementos de la base de $\ptensor{V}{V}$ dada por $\ptensor{B}{B}=\curlybraces{\ptensor{v_i}{v_j}}_{i,j=1}^{n}$
\end{proposition}
\begin{proof}
    Queremos ver que $\curlybraces{\ptensor{v_i}{}v_j}_{i,j=1}^n$ es base de $\ptensor{V}{V}$. Para ello, tendremos que ver que esta base $\ptensor{B}{B}$ complete el espacio $\ptensor{V}{V}$ y que los vectores de la misma sean linealmente independientes.\\
    Sabemos que $\ptensor{v}{w}\in\ptensor{V}{V}$ y que
    \[\begin{array}{cccl}
        \ptensor{v}{w}: & \pcart{V^*}{V^*} & \to & \mathbb{R}\\
         & (f,g) & \mapsto & f(v)g(w)
    \end{array}\]
    Luego, para que la base $\ptensor{B}{B}$ complete el espacio $\ptensor{V}{V}$, se deberá poder expresar cualquier vector $\ptensor{v}{w}\in\ptensor{V}{V}$ como combinación lineal de los vectores de $\ptensor{B}{B}$. Podemos usar $B=\curlybraces{v_i}_{i=1}^n$ base de $V$, tal que
    \[v=\sum\limits_{i=1}^n\lambda^iv_i=\lambda^iv_i,\hspace{4mm}w=\sum\limits_{j=1}^n\mu^jv_j=\mu^jv_j\]
    Por tanto, usando $f,g\in V^*$, tenemos que
    \[\ptensor{v}{w}(f,g)=f(v)g(w)=f(\lambda^iv_i)g(\mu^jv_j)=\lambda^if(v_i)\mu^jg(v_j)=\lambda^i\mu^jf(v_i)g(v_j)=\lambda^i\mu^j(\ptensor{v_i}{v_j})(f,g)\]
    Luego, hemos expresado un vector del espacio $\ptensor{V}{V}$ como combinación lineal de los vectores de la base $\ptensor{B}{B}$. $\checkmark$\\ \\
    Veamos que son linealmente independientes, para ello, se debe cumplir que,
    \[\sum\limits_{i,j=1}^n\lambda^{ij}(\ptensor{v_i}{v_j})=\lambda^{ij}(\ptensor{v_i}{v_j})=0\Leftrightarrow\lambda^{ij}=0\]
    Sabiendo que la base de $V^*$ es $B^*=\curlybraces{f^1,f^2,\dots,f^n}$, tal que
    \[f^i(v_i)=1\hspace{5mm}f^j(v_i)\overset{i\neq j}{=}0\Rightarrow f^i(v_j)=\delta_{ij}\]
    Podemos evaluar lo anterior en dos elementos arbitrarios de $B^*$, tal que
    \[0=\lambda^{ij}(\ptensor{v_i}{v_j})(f^n,f^m)= \lambda^{ij}f^n(v_i)f^m(v_j)=\lambda_{ij}\delta_{n}^i\delta_{m}^j=\lambda^{nm}\]
    luego, $\lambda^{nm}=0$ y por tanto, los vectores son linealmente independientes. $\checkmark$\\ \\
    Así, hemos demostrado que $\ptensor{B}{B}$ es base de $\ptensor{V}{V}$.
\end{proof}

\begin{note}
    Denotaremos $\ptensor{v}{w}\equiv h$, tal que
    \[
    \begin{array}{cccl}
        h: & \pcart{V^*}{V^*} & \to & \mathbb{R} \\
         & (f^i,f^j) & \mapsto & h(f^i,f^j)=h^{ij}
    \end{array}
    \]
    siendo $f^i,f^j\in B^*$. Por tanto, para dos $p,q\in V^*$ cualesquiera, escribiremos
    \[(\ptensor{v}{w})(p,q)=h(p,q)=h\left(\sum_{i=1}^np_if^i,\sum_{j=1}^nq_jf^j\right)=p_iq_j(f^i,f^j)=h^{ij}p_iq_j\]
\end{note}
\noindent Veamos algunas \textbf{propiedades} del producto tensorial.
\begin{proposition}
    Sea $V$ un $\mathbb{R}$-espacio vectorial,
    \begin{enumerate}[label=(\roman*)]
        \item $\ptensor{(v_1+v_2)}{w}=\ptensor{v_1}{w}+\ptensor{v_2}{w}$; $\forall v_1,v_2,w\in V$.
        \item $\ptensor{w}{(v_1+v_2)}=\ptensor{w}{v_1}+\ptensor{w}{v_2}$, $\forall v_1,v_2,w\in V$.
        \item $\ptensor{(\lambda v)}{w}=\lambda\ptensor{v}{w}$, $\forall v,w\in V$, $\forall\lambda\in\mathbb{R}$.
        \item $\ptensor{w}{(\lambda v)}=\lambda\ptensor{w}{v}$, $\forall v,w\in V$, $\forall \lambda\in\mathbb{R}$.
        \item $\ptensor{v}{w}\neq\ptensor{w}{v}$.
        \item $\ptensor{v}{w}\neq0$ si $v\neq0$ ó $w\neq 0$.
        \item Sea $\ptensor{a}{b}\neq0$, $\ptensor{a}{b}=\ptensor{a'}{b'}\Leftrightarrow a'=\lambda a$ y $b'=\lambda^{-1}b$.
        \item $\ptensor{V}{W}$ es isomorfo con $\ptensor{W}{V}$.
    \end{enumerate}
\end{proposition}
\begin{proof}
    \begin{enumerate}[label=(\roman*)]
        \item $\forall v_1,v_2,w\in V$,
        \[(\ptensor{(v_1+v_2)}{w})(f,g)=f(v_1+v_2)g(w)=\brackets{f(v_1+f(v_2)}g(w)=\]\[=f(v_1)g(w)+f(v_2)g(w)=(\ptensor{v_1}{w})(f,g)+(\ptensor{v_2}{w})(f,g)\qedh\]
        \item $\forall v_1,v_2,w\in V$,
        \[(\ptensor{w}{(v_1+v_2)})(f,g)=f(w)g(v_1+v_2)=f(w)\brackets{g(v_1)+g(v_2)}=\]\[=f(w)g(v_1)+f(w)g(v_2)=(\ptensor{w}{v_1})(f,g)+(\ptensor{w}{v_2})(f,g)\qedh\]
        \item $\forall v,w\in V$ y $\forall\lambda\in\mathbb{R}$,
        \[(\ptensor{(\lambda\cdot v)}{w})(f,g)=f(\lambda\cdot v)g(w)=\lambda\cdot f(v)g(w)=\lambda\cdot(\ptensor{v}{w})(f,g)\qedh\]
        \item $\forall v,w\in V$ y $\forall\mu\in\mathbb{R}$,
        \[(\ptensor{w}{(\lambda\cdot v)})(f,g)=f(w)g(\lambda\cdot v)=\lambda\cdot f(w)g(v)=\lambda\cdot(\ptensor{w}{v})(f,g)\qedh\]
        \item Vemos que, $(\ptensor{v}{w})(f,g)=f(v)g(w)$ y que $(\ptensor{w}{v})(f,g)=f(w)g(v)$, luego estos elementos serían iguales solo si $f\equiv g$. $\qedh$
        \item Sean $v,w\in V$ y $f,g\in V^*$, tales que $f\not\equiv0$ y $g\not\equiv0$, entonces
        \[(\ptensor{v}{w})(f,g)=f(v)g(w)=0\Leftrightarrow\begin{matrix}
            f(v)=0 & \Leftrightarrow v=0\\
            \text{ó} & \\
            g(w)=0 & \Leftrightarrow w=0
        \end{matrix}\qedh\]
        \item \begin{tabular}{c|}
             $\Rightarrow$ \\ \hline
        \end{tabular} 
        Sea $\ptensor{a}{b}=\ptensor{a'}{b'}$ entonces
        \[(\ptensor{a}{b})(f,g)=f(a)g(b)=(\ptensor{a'}{b'})(f,g)=f(a')g(b')\]
        luego,
        \[f(a)g(b)=f(a')g(b')\]
        pero como $a\neq a'$ y $b\neq b'$, debe haber una relación entre ambos, de tal forma que se cumpla la igualdad anterior. Supondremos que $a$ y $a'$ tienen una relación lineal (la más sencilla), tal que $a'=\lambda a+c$, luego 
        \[f(a)g(b)=f(a')g(b')=f(\lambda a+c)g(b')=f(\lambda a)g(b')+f(c)g(b')=\lambda f(a)g(b')+f(c)g(b')\]
        Agrupamos términos de la igualdad, tal que,
        \[0:\hspace{5mm}0=f(c)g(b')\]
        \[f(a):\hspace{5mm}g(b)=\lambda g(b')\]
        Por la propiedad \textit{(vi)}, como $b'\neq0$, entonces $c=0$. Además,
        \[g(b)=\lambda g(b')\Rightarrow g(b')=\lambda^{-1}g(b)\Rightarrow g(b')=g(\lambda^{-1}b)\Rightarrow b'=\lambda^{-1}b\]
        Luego,
        \[\begin{matrix}
            a'=\lambda a\\
            b'=\lambda^{-1}b
        \end{matrix}\hspace{4mm}\checkmark\]
        \begin{tabular}{c|}
             $\Leftarrow$ \\ \hline
        \end{tabular} 
        Sea $a'=\lambda a$ y $b'=\lambda^{-1}b$, entonces
        \[(\ptensor{a'}{b'})(f,g)=f(a')g(b')=f(\lambda a)g(\lambda^{-1}b)=\cancel{\lambda}\cancel{\lambda^{-1}}f(a)g(b)=(\ptensor{a}{b})(f,g)\checkmark\]
        \item Sean $V,W$ espacios vectoriales, tales que
        \[\begin{array}{ccl}
            \ptensor{V}{W} & \to & \ptensor{W}{V}  \\
            \ptensor{v}{w} & \mapsto & \ptensor{w}{v}
        \end{array}\]
        Si suponemos que $dimV=n$ y $dimW=m$, sabemos por tanto que $dim(V\otimes W)=n\cdot m$ y $dim(W\otimes V)=m\cdot n$, luego tienen la misma dimensión y por tanto, son isomorfos. $\checkmark$\\ \\
        También podemos hacerlo sin usar la proposición de que $dim(\ptensor{V}{W})=n\cdot m$. Es claro ver que la aplicación es inyectiva, pues no hay dos elementos con la misma imagen, ya que la imagen se forma al permutar los elementos. Luego, al ser inyectivo, tenemos que $dimKer=0$. Por el Primer Teorema de Isomorfía,
        \[dim(\ptensor{V}{W})=\cancelto{0}{dimKer}+dimIm=dimIm=dim(\ptensor{W}{V})\]
        Luego, como $\ptensor{V}{W}$ y $\ptensor{W}{V}$ tienen la misma dimensión, entonces son isomorfos.
    \end{enumerate}
\end{proof}

\subsection{Aplicaciones lineales} % Main chapter title
\label{cap1-sec1-subsec2} 

Veamos ahora las aplicaciones lineales.
\begin{definition}
    Sean $V$ y $V'$ dos espacios vectoriales sobre el mismo cuerpo $\mathbb{K}$.
    Se dice que en una aplicación $f:V\longrightarrow V'$ es una aplicación lineal, o también llamado homomorfismo de espacios vectoriales, si se verifica:
    \begin{enumerate}[label=(\roman*)]
        \item $f(x+y)=f(x)+f(y),\forall x,y\in V$
        \item $f(\lambda\cdot x)=\lambda\cdot f(x),\forall\lambda\in\mathbb{K},\forall x\in V$
    \end{enumerate}
    Diremos además que $f$ es un isomorfismo lineal si es biyectiva, que $f$ es un endomorfismo si $V=V'$ y que es un automorfismo si es un endomorfismo biyectivo. 
\end{definition}

\noindent Las aplicaciones lineales tienen asociados dos conjuntos cuyas características son de interés, a saber, el núcleo y la imagen.

\begin{definition}
    Sea $f:V\longrightarrow W$ definimos el núcleo o kernel de la aplicación $f$ como
    \[Kerf=\curlybraces{v\in V:f(v)=0}\]
    y la imagen como
    \[Imf=\curlybraces{w\in W:\exists v\in V/f(v)=w}.\]
\end{definition}

\noindent Veamos algunas propiedades básicas de ambos conjuntos.

\begin{proposition}
Sea $f:V\to V'$ una aplicación lineal, se tienen las siguientes propiedades:
\begin{enumerate}[label=(\roman*)]
    \item \label{prop1:item1} $\rm{Im}f$ es un subespacio de $V'$ y que $\rm{Ker}f$ es un subespacio de $V$.
    \item \label{prop1:item2}Si $W$ es un subespacio vectorial de $V$, entonces $f(W):=\curlybraces{f(w): w\in W}$ es un subespacio de $V'$.
    \item \label{prop1:item3}Si $W'$ es un subespacio de $V'$, entonces $f^{-1}(W'):=\curlybraces{v\in V: f(v)\in W'}$ es también un subespacio de $V$.
\end{enumerate}  
\end{proposition}
%\newpage
\begin{proof}
\begin{enumerate}[label=\ref{prop1:item1}]
    \item Por definición, como los elementos de la $\rm{Im}f$ son pertenecientes a $V'$, entonces la $\rm{Im}f$ es subespacio de $V'$. De igual forma ocurre con el $\rm{Ker}f$, pues sus elementos pertenecen a $V$ y por tanto, este es subespacio de $V$.
\end{enumerate}
\begin{enumerate}[label=\ref{prop1:item2}]
    \item Como $W$ es subespacio de $V$, tenemos que $w\in V$ también, por tanto, los $f(w)$ pertenecerán a $V'$, cosa que implica que $f(W)$ es subespacio de $V'$, pues los $f(w)$ de $f(W)$ pertenecen a $V'$.
\end{enumerate}
\begin{enumerate}[label=\ref{prop1:item3}]
    \item Por analogía a $\ref{prop1:item2}$ vemos que $f^{-1}(W)$ es subespacio de $V$.
\end{enumerate}
\end{proof}
\noindent Ahora veamos algunas propiedades esenciales de las aplicaciones lineaales.
\begin{proposition}
    Sea $f:V\longrightarrow V'$ una aplicación lineal,
    \begin{enumerate}[label=(\roman*)]
        \item \label{pro1:item1} entonces $f$ es inyectiva si y solo si $Kerf=\curlybraces{0}$.
        \item \label{pro1:item2} si $G$ es un conjunto generador de $V$, $<G>=V$, entonces $f(G)$ es conjunto generador
        de $Imf$, $<f(G)>=Imf$.
        \item \label{pro1:item3} si $S\subset V$ es un conjunto de vectores linealmente independientes, si $f$ es inyectiva, entonces $f(S)$ es linealmente independiente.
        \item \label{pro1:item4} $f$ es inyectiva $\Leftrightarrow$ conserva la independencia lineal.
   
        \item \label{pro1:item5} si $f$ es biyectiva y $B$ es una base de $V$, entonces $f(B)$ es base de $V'$.

        \item \label{pro1:item6} $f$ es sobreyectiva $\Leftrightarrow$ $Imf=V'$
    \end{enumerate}
\end{proposition}



\begin{proof}
\ref{pro1:item1} \begin{tabular}{c|}
                 $\Rightarrow$ \\ \hline
            \end{tabular}
            Suponiendo que $f$ es inyectiva, sabemos que su Kernel es,
            \[\rm{Ker}f=\curlybraces{v\in V:f(v)=0}\]
            pero como la inyectividad nos implica que la imagen debe provenir de un único vector de entrada, entonces este vector será $v=0$, y por tanto, $ker f=\curlybraces{0}$. $\checkmark$
\\     
            \begin{tabular}{c|}
                 $\Leftarrow$  \\ \hline
            \end{tabular}
            Suponiendo que $ker f=\curlybraces{0}$, esto nos quiere decir que únicamente el vector $v=0$ satisface $f(v)=0$, luego como un vector tiene una única imagen, decimos que $f$ es inyectiva. \qedh
\\ \\
\ref{pro1:item2}  Veamos que el conjunto $f(G)$ es sistema generador de la imagen, es decir,   \[<f(G)>=Imf\Leftrightarrow\forall y\in Imf,\exists\lambda^1,\dots,\lambda^n\in\mathbb{K},y_1,\dots,y_n\in f(G)\text{ tales que }y=\lambda^1y_1+\dots+\lambda^ny_n.\] Sabemos que $G$ es conjunto generador, luego sea $y\in Imf$. Entonces por definición se tiene que existe $x\in V$ tal que $f(x)=y$. Como $<G>=V$, existen $\lambda^1,\dots,\lambda^n\in\mathbb{K}$, $v_1,\dots,v_n\in G$ tales que \[ x= \sum_{i=1}^n \lambda^i v_i.\] Tenemos entonces que:  \[y=f(x)=f(\lambda^1v_1+\dots+\lambda^nv_n)=\lambda^1f(v_1)+\dots+\lambda^nf(v_n).\] Por lo tanto, $y$ es combinación lineal de elementos de $f(G)$, es decir, $<f(G)>=\mathrm{Im}f$. \qedh
\\ \\
\ref{pro1:item3}
            Sea $S$ un conjunto linealmente independiente en $V$. 
            Supongamos que $f$ es inyectiva, vamos a probar que $f(S)$ es linealmente independiente, es decir,
            \[\lambda^1y_1+\dots+\lambda^ny_n=0\Rightarrow\lambda^1=\lambda^2=\dots=\lambda^n=0\hspace{4mm} \forall y_1,\dots,y_n\in f(S),\hspace{2mm}
                \forall\lambda^1,\dots,\lambda^n\in\mathbb{K}\]
            Supongamos $\lambda^1y_1+\dots+\lambda^ny_n=0$.  Como $y_j\in f(S),\exists x_j\in S/f(x_j)=y_j$.
             \[\left.\begin{array}{r}
                \lambda_1f(x_1)+\dots+\lambda_nf(x_n)=0\\
                f(\lambda_1x_1+\dots+\lambda_nx_n)=0
            \end{array}\right\rbrace f(0)=0\Rightarrow\lambda_1x_1+\dots+\lambda_nx_n=0\Rightarrow f\text{ inyectiva}\Rightarrow\lambda_1=\dots=\lambda_n=0\]
                \\ \\
                \ref{pro1:item4} \begin{tabular}{c|}
                 $\Longrightarrow$ \\ \hline
            \end{tabular} 
            Trivial por (iii) $\checkmark$\\
            \begin{tabular}{c|}
                 $\Longleftarrow$ \\ \hline
            \end{tabular} 
            Por reducción al absurdo:\\
            Supongamos que existen $v_1,v_2\in V$ distintos, tales que $f(v_1)=f(v_2)\Leftrightarrow f(v_1)-f(v_2)=0\Leftrightarrow f(v_1-v_2)=0$. 
            Luego, $v=v_1-v_2\neq0$ verifica que $f(v)=0$, $\curlybraces{v}$ es un conjunto linealmente independiente, $f(\curlybraces{v})$ tendría que ser un conjunto l.i. por hipótesis, pero $f(\curlybraces{v})=\curlybraces{0}$ que no es un conjunto l.i. cosa absurda. \qedh
            \\ \\
            \ref{pro1:item5}  Sea una aplicación lineal biyectiva $f:V\to V'$
            y una base de $V$, $B=\curlybraces{v_1,\dots,v_n}$. Entones, si aplicamos
\[f(B)=\curlybraces{f(v_1),\dots,f(v_n)}=\curlybraces{v_1',\dots,v_n'}\]
            y entonces, estos $v_i'\in V'$ van a formar una base de $V'$, pues al ser $f$ biyectiva, los vectores serán linealmente independientes, pues los de $B$ lo son; y además, como tienen la misma dimensión que $V'$, pasan de ser conjunto generador a base. $\qedh$
            \\ \\
            \ref{pro1:item6} \begin{tabular}{c|}
                 $\Rightarrow$  \\ \hline
            \end{tabular}
            Suponiendo que $f$ es sobreyectiva, tendremos que para cada $y\in V'$, existe al menos un $x\in V$, tal que $f(x)=y$. Por consiguiente, cada elemento de $V'$ es la imagen de un elemento de $V$, es decir, $Imf=V'$. $\checkmark$\\
            \begin{tabular}{c|}
                 $\Leftarrow$  \\ \hline
            \end{tabular}
            Suponiendo que $imf=V'$, tenemos que todos los elementos de $V'$ son imagen de los elementos de $V$, siendo esta la propia definición de sobreyectividad, luego $f$ es sobreyectiva.
\end{proof}
\noindent Una vez visto estas propiedades, de $\ref{pro1:item5}$ podemos obtener un resultado interesante, que es la siguiente proposición.
\begin{proposition}
Sea $B=\curlybraces{v_1,\dots,v_n}$ una base de $V$, y sea $f:V\rightarrow V'$ una aplicación lineal. Se tiene entonces que $\curlybraces{f(v_1),\dots,f(v_n)}$ es un sistema generador de la imagen.
\end{proposition}
\begin{proof}
    Supongamos que $B$ es una base y que conocemos $f(v_j),\forall v_j\in B$. 
    Sea $v\in V$, escrito en coordenadas de la base como $v=\lambda^1v_1+\dots+\lambda^nv_n$, con $v_i\in B$, 
 y $\lambda^i\in\mathbb{K}$, entonces  $f(x)=f(\lambda^1v_1+\dots+\lambda^nv_n)=\lambda^1f(v_1)+\dots\lambda^nf(v_n)$, luego hemos puesto $f(x)$ en coordenadas de $\curlybraces{f(v_1,\dots,f(v_n)}$.
\end{proof}

\noindent Ahora vamos a ver un resultado bastante importante, el cuál nos permitirá representar aplicaciones lineales en matrices, denominadas \textbf{matrices asociadas a la aplicación $f$}. Además, este resultado es importante para Física, pues los físicos no solemos trabajar con aplicaciones, sino que trabajamos con sus matrices asociadas, pues se puede decir que "tienen" la misma información que las aplicaciones.

\begin{proposition}  
\label{prop1.4}
    Sean $(V,+,\cdot)$ y $(V',+,\cdot)$ $\mathbb{K}$-espacios vectoriales de dimensión finita con $dimV=n$ y $dimV'=m$. 
    Sea $f:V\longrightarrow V'$ una aplicación lineal, entonces dadas $\left\lbrace\begin{matrix}
        B=\curlybraces{v_1,\dots,v_n}\text{ base de }V\\
        B'=\curlybraces{v_1',\dots,v_n'}\text{ base de }V'
    \end{matrix}\right.$\\
    $f$ se representa en esas bases como una matriz en $\mathcal{M}_{m\times n}(\mathbb{K})$.
\end{proposition}

\begin{proof}
    Como $f$ es lineal, me basta con conocer $f(B)$, para ello, tenemos que conocer $f(v_1),f(v_2),\dots,f(v_n)$, teniendo:
    \[\begin{matrix}
        f(v_1) & = & a_{1}^1v_1'+a_{1}^2v_2'+\dots+a_{1}^mv_m', & a_{1}^i\in\mathbb{K}\\
        f(v_2) & = & a_{2}^1v_1'+a_{2}^2v_2'+\dots+a_{2}^mv_m', & a_{2}^i\in\mathbb{K}\\
        \vdots & & \vdots & \vdots\\
        f(v_n) & = & a_{n}^1v_1'+a_{n}^2v_2'+\dots+a_{n}^mv_m', & a_{n}^i\in\mathbb{K}
    \end{matrix}\]
    Sea $v\in V:v=\lambda^1v_1+\dots+\lambda^nv_n,\hspace{2mm}\lambda^i\in\mathbb{K}$, si le aplicamos $f$ tenemos,
    \[\begin{array}{rll}
        f(v) & = &\lambda^1f(v_1)+\dots+\lambda^nf(v_n) \\
         & = & \lambda^1(a_{1}^1v_1'+\dots+a_{1}^mv_m')+\lambda^2(a_{2}^1v_1'+\dots+a_{2}^mv_m')+\dots+\lambda^n(a_{n}^1v_1'+\dots+a_{n}^mv_m')\\
         & = & (a_{1}^1\lambda^1+a_{2}^1\lambda^2+\dots+a_{n}^1\lambda^n)v_1'+(a_{1}^2\lambda^1+\dots+a_{n}^2\lambda^n)v_2'+\dots+(a_{1}^m\lambda^1+\dots+a_{n}^m\lambda^n)v_m'
    \end{array}
        \]
   Luego,  $f(v)=\mu^1v_1'+\mu^2v_2'+\dots+\mu^mv_m'$, siendo $\mu^i=(a_{1}^i\lambda^1+\dots+a_{n}^i\lambda^n)$, luego, para construir la matriz $A$, ponemos las coordenadas de $v_1$ en la primera columna, las de $v_2$ en la segunda y así sucesivamente, tal que:
    \[\begin{pmatrix}
        \mu^1\\
        \mu^2\\
        \vdots\\
        \mu^m
    \end{pmatrix}=\begin{pmatrix}
        a_{1}^1 & a_{1}^1 & \dots & a_{1}^m\\
        a_{2}^1 & a_{2}^2 & \dots & a_{2}^m\\
        \vdots & \vdots & \ddots & \vdots\\
        a_{n}^1 & a_{n}^2 & \dots & a_{n}^m
    \end{pmatrix}\begin{pmatrix}
        \lambda^1\\
        \lambda^2\\
        \vdots\\
        \lambda^n
    \end{pmatrix}\Rightarrow \mu=A\cdot\lambda\]
\end{proof}

\noindent Vamos a introducir ahora el concepto de \textbf{rango} de una aplicación lineal, que puede extenderse al rango de su matriz asociada.

\begin{definition}
    Se llama rango de una aplicación lineal (matriz) a la dimensión de su imagen y se denota por $rg()$.
\end{definition}

\noindent Como un mismo espacio vectorial puede estar generado por varias bases, es lógico pensar que debe haber una relación entre estas bases o al menos una forma de cambiar de una base a otra, lo que se conoce como \textbf{cambio de base}. Esto es posible y una forma sencilla de hacerlo es mediante las matrices asociadas.

\begin{proposition}
    -Sean $V$ y $V'$ dos espacios vectoriales en $\mathbb{K}$, sea $f:V\longrightarrow V'$ lineal.\\
    -Sea $B_1=\curlybraces{v_1,\dots,v_n}$ base de $V$, $B_1'=\curlybraces{v_1',\dots,v_m'}$ base de $V'$.\\
    -Sea $A\in\mathcal{M}_{m\times n}(\mathbb{K})$ la matriz que representa a $f$ en $B_1,B_1'$.\\
    -Sea $B_2=\curlybraces{u_1,\dots,u_n}$ base de $V$, $B_2'=\curlybraces{u_1',\dots,u_m'}$ base de $V'$.\\
    -Sea $\tilde{A}\in\mathcal{M}_{m\times n}(\mathbb{K})$ la matriz que representa a $f$ en $B_2,B_2'$.\\
    -Sea $P$ la matriz de cambio de base de $B_1$ en $B_2$.\\
    -Sea $Q$ la matriz de cambio de base de $B_1'$ en $B_2'$.\\
    Entonces $\tilde{A}=Q^{-1}\cdot A\cdot P$.
\end{proposition}
\begin{proof}
    Sea $f:V\to V'$ una aplicación lineal con $n=dim V$ y $m=dim V'$. Si $A$ y $\tilde{A}$ son las matrices asociadas a $f$ respecto de distintas bases, entonces
    \[rg(A)=dim(Imf)=rg(\tilde{A})\]
    Luego $A$ y $\Tilde{A}$ tienen igual rango, y por tanto, son matrices equivalentes. Concretemos más esta situación:\\
    Sean $B_1$ y $B_2$ bases de $V$ con cambio de base de $B_1$ a $B_2$ dado por $X_1=PX_2$ y sean $B_1'$ y $B_2'$ bases de $V'$, con cambio de $B_1'$ a $B_2'$ dado por $Y_1=QY_2$.\\
    Consideremos la matriz asociada a $f$ respecto de $B_1$ y $B_1'$, $A\in\mathcal{M}_{m\times n}(\mathcal{K})$, tal que $A=\mathcal{M}_{B_1,B_1'}(f)$ y la ecuación matricial
    \[Y_1=AX_1\]
    De igual forma, sea $\tilde{A}\in\mathcal{M}_{m\times n}(\mathbb{K})$ la matriz asociada a $f$ respecto de $B_2$ y $B_2'$, tal que $\tilde{A}=\mathcal{M}_{B_2,B_2'}(\mathbb{K})$ y la ecuación matricial de $f$ respecto de estas bases,
    \[Y_2=\tilde{A}X_2\]
    Gráficamente,
    \[\begin{matrix}
    & V & \to & V' & \\
            & & A & &\\
            & B_1 & \longrightarrow & B_1' & \\
            P & \uparrow & & \uparrow & Q\\
             & & \tilde{A} & & \\
             & B_2 & \longrightarrow & B_2' &
        \end{matrix}\]
        Entonces,
        \[Y_2=\left\lbrace \begin{array}{l}
        \tilde{A}X_2\\    Q^{-1}Y_1=Q^{-1}AX_1=Q^{-1}APX_2\end{array}\right.\]
        y en consecuente,
        \[\tilde{A}=Q^{-1}AP\]
        O bien,
        \[X_2=\left\lbrace\begin{array}{l}
            \tilde{A}^{-1}Y_2\\
            P^{-1}X_1=P^{-1}A^{-1}Y_1=P^{-1}A^{-1}QY_2
        \end{array}\right.\]
        y en consecuente,
        \[\tilde{A}^{-1}=P^{-1}A^{-1}Q\]
\end{proof}

Ahora vamos a enunciar el \textbf{Primer Teorema de Isomorfía}, del que obtendremos un Corolario muy importante a la hora de trabajar con aplicaciones lineales. Este teorema no se va a demostrar (si se quiere ver la prueba consultar  \cite[Chapter 6, Theorem 6.5, Page 77]{IntroducciónTeoríaDeGrupos}).

\begin{theorem}[Primer teorema de isomorfismo de Noether]
    Sea $f:V\longrightarrow V'$ una aplicación lineal, entonces:
    \begin{enumerate}[label=(\roman*)]
        \item Existe una aplicación lineal sobreyectiva $\pi:V\longrightarrow V/Kerf$
        \item Existe un isomorfismo $\bar{f}:V/Kerf\longrightarrow Imf$
        \item Existe una aplicación lineal inyectiva $i:Imf\longrightarrow V'$, tales que $f=i\circ\bar{f}\circ\pi$, tal que
        \[\begin{matrix}
            & & f & \\
            & V & \longrightarrow & V' & \\
            \pi & \downarrow & & \uparrow & i\\
             & & \bar{f} & & \\
             & V/Kerf & \longrightarrow & Imf &
        \end{matrix}\]
    \end{enumerate}
   
\end{theorem}
\begin{corollary}
     Si además $V$ es finitamente generado,
    \[dimV=dim(Kerf)+dim(Imf)\]
\end{corollary}
\subsection{Contrtacción de tensores} % Main chapter title
\label{cap1-sec2-subsec3} 

 Una vez que hemos visto cómo subir y bajar índices, podemos definir una operación denominada \textbf{contracción} de tensores, la cuál encoge un tensor $(r,s)$ a uno $(r-1,s-1)$. La definición general se obtiene a partir del siguiente caso especial.

\begin{lemma}
    Hay una única aplicación lineal
    $C:\Omega_1^1\to\mathbb{R}$
    llamada \textit{contracción (1,1)}, tal que
    \[\begin{array}{rlll}
        C: & \Omega_1^1 (V)& \to & \mathbb{R} \\
         & \ptensor{v}{f} & \mapsto & C(\ptensor{v}{f})=f(v)
    \end{array}\]
    para todo $v\in V$ y $f\in V^*$.
\end{lemma}
\begin{proof} (Esta demostración usa el concepto de matrices de cambio de base, por lo que se recomienda ver la sección \ref{CambioBasesTensores(1,1)})\\ 
    Tomando $B=\curlybraces{v^1,v^2,\dots,v^n}$ base de $V$ y $B^*= \curlybraces{f_1,f_2,\dots,f_n}$ base de $V^*$, podemos escribir un tensor de tipo $(1,1)$ como
    \[A\equiv\sum A^i_j\ptensor{f_i}{v^j}\]
    Como $C(\ptensor{f_i}{v^j})=f_i(v^j)=\delta^j_i$, por la condición de base dual, no nos queda otra opción, más que definir,
    \[C(A)=\sum A_i^i=\sum A(f_i,v^i)\]
    Entonces, $C$ tiene las propiedades requeridas en las bases $B,B^*$. Luego, para obtener la función general requerida es suficiente con mostrar que esta definición es independiente de la elección del sistema de coordenadas. Así, tomando una nueva base de $V$, $B'=\curlybraces{w^1,w^2,\dots,w^n}$ y otra de $V^*$, $B^{*'}=\curlybraces{q_1,q_2,\dots,q_n}$, tenemos
    \[\begin{array}{rrl}
        C(A) & = & \sum\limits_mA(q_m,w^m)= \sum\limits_mA\left(\sum\limits_i a_i^mf_m,\sum_jb_m^jv^m\right)\\
         & = & \sum\limits_{i,j,m}a_i^mb_m^jA(f_i,v^j)=\sum\limits_{i,j}\delta^j_iA(f_i,v^j)\\
         & = & \sum\limits_iA(f_i,v^j)
    \end{array}\]
\end{proof}
\noindent Para extender las contracciones $(1,1)$, $C$, a un tensor de un tipo mayor, el esquema es especificar una componente covariante y otra contravariante y aplicar $C$ a estos.\\

\noindent Suponemos un tensor $A\in\Omega_r^s(V)$ y $1\leq r$ y $1\leq j \leq s$. Fijamos las formas $p_1,p_2,\dots,p_{r-1}$ y los vectores $u_1,u_2,\dots ,u_{s-1}$. Entonces la función
\[(p,u) \to A(p_1, \dots, \underbrace{p_{i}}_{\mathclap{i\text{-ésima componente contravariante}}}, \dots, p_{r-1}, u^{1}, \dots, \overbrace{u^{j}}^{\mathclap{j\text{-ésima componente covariante}}}, \ldots, u^{s-1})\]
es un tensor $(1,1)$ que puede escribirse como

\[A(p_1,\dots,\cdot,\dots,p_{r-1},u^1,\dots,\cdot,\dots,u^{s-1})\]
Aplicando la contracción $(1,1)$ a este tensor, produce una función de valor real denotada por

\[\left(C_j^iA\right)\left(p_1,\dots,p_{r-1},u^1,\dots,u^{s-1}\right)\]
Siendo $C_j^iA$ una función multilineal. Por tanto, esto es un tensor de tipo $(r-1,s-1)$ llamado \textit{la contracción de }$A$\textit{ sobre }$i,j$.

%\begin{definition}
 %   La contracción de un tensor $A$ de tipo $(r,s)$ con respecto al índice contravariante $p$ $(p\leq r)$ y al índice covariante $q$ $(q\leq s)$ es el tensor de tipo $(r-1,s-1)$, teniendo las componentes,
  %  \[B^{i_1\dots i_{r-1}}_{j_1\dots j_{s-1}}=A^{i_1\dots i_{p-1}ki_p\dots i_{r-1}}_{j_1\dots j_{q-1}kj_q\dots j_{s-1}}\]
%\end{definition}

\begin{note}
    Para poder contraer tensores, debemos tener superíndices y subíndices, así, podemos usar primero la métrica para subir o bajar índices y luego aplicar la contracción.
   \end{note} 
\begin{example}
        Si tenemos un tensor de tipo (0,2),
    $S\equiv S_{\alpha\beta}$, podemos hacer,
    \[\begin{array}{rllll}
        S_{\alpha\beta} & \to & g^{\gamma\alpha}S_{\alpha\beta}=S^{\alpha}_{\beta} & \to & C^1_1S^{\gamma}_{\beta}=S^{\beta}_{\beta} \\
        \text{Tensor (0,2)} & \to & \text{Tensor (1,1)} & \to &\text{Escalar}
    \end{array}\]
    cosa que se puede simplificar simplemente usando,
    \[S\equiv S_{\alpha\beta}\to g^{\beta\alpha}S_{\alpha\beta}=S^{\beta}_{\beta}\]
    es decir, podemos contraer tensores con la propia métrica.
\end{example}

\begin{example}
    Si
    \[U^j_i=T^{kj}_{ik}\]
    entonces
    \[U'^{j'}_{i'}=T'^{k'j'}_{i'k'}=S^i_{i'}S^l_{k'}R^{k'}_kR^{j'}_jT^{kj}_{il}=S^i_{i'}\delta^l_kR^{j'}_jT^{kj}_{il}=S^i_{i'}R^{j'}_jT^{kj}_{ik}=S^i_{i'}R^{j'}_jU^j_i\]
    donde hemos utilizado $S^l_{k'}R^{k'}_k=\delta^l_k$. Vemos que se transforma como un tensor (1,1).\\

\noindent    Así, dado un tensor $T^{ij}_{kl}$ de tipo (2,2), serán posible las 4 contracciones
    \[T^{kj}_{ki},\hspace{3mm}T^{jk}_{ik},\hspace{3mm}T^{kj}_{ik},\hspace{3mm}T^{jk}_{ki}\]
    que originan 4 tensores de tipo (1,1). Por otro lado, las dos posibles contracciones dobles que dan lugar a un escalar (tensor de tipo (0,0)) son
    \[T^{kj}_{kj},\hspace{3mm}T^{jk}_{kj}\]
\end{example}
\begin{note}
    El producto escalar $(\mathbb{R}^n,g_{ij})$ también se puede contraer. Pues $g_{ij}$ es un tensor de tipo (0,2), al cual le podemos aplicar una contracción 1,1, pero primero lo pasamos a un tensor de tipo (1,1), variando sus índices, tal que
    \[C^1_1\left(g^{ki}g_{ij}\right)=C^1_1(g^k_j)=g^j_j=n\]
    donde sabemos que vale $n$, pues al ser un espacio de dimensión $n$, la matriz asociada a $g$ será $G\in\mathcal{M}_{n\times n}$ y por tanto, la traza será la suma de $n$-elementos. Sabemos que estos elementos son el 1, porque la traza es invariante frente a los cambios de base (cosa que veremos más adelante), por tanto, si cogemos el producto escalar usual en la base usual, la matriz asociada es la matriz de Gram, cuyos elementos son todos nulos, salvo la diagonal que está formada por 1.
\end{note}
\subsection{Notación de Einstein} % Main chapter title
\label{cap1-sec1-subsec4} 

La notación de Einstein va a servir para facilitarnos la escritura, pues cada vez que tengamos un vector o una forma escrita como combinación lineal, vamos a poder redefinirlos como
\[w=\sum\limits_{i=1}^n\lambda^iv_i\equiv\lambda^iv_i\]
esto para un vector. Para una forma, tendremos
\[p=\sum\limits_{i=1}^n\mu_if^i\equiv\mu_i f^i\]
Además, para simplificar aún más la notación y dejarnos de tantas letras, vamos a identificar los escalares de $w$ como 
\[\lambda^i\equiv w^i\]
Así, los vectores como combinación lineal de otros vectores, los escribiremos como
\[w=w^iv_i\]
Y para las formas, haremos la identificación
\[\mu_i\equiv p_i\]
Así, las formas como combinación lineal de otras formas se escribirán como
\[p=p_if^i\]
\begin{example}
Un ejemplo de ello, será a la hora de identificar un vector en los términos de su base, pues suponiendo un $V$ espacio vectorial sobre el cuerpo $\mathbb{K}$ y cuya base sea $B=\curlybraces{v_1,v_2,\dots,v_n}$, tomando un $u\in V$, lo denotaremos como,
\[u=u^iv_i\]
\end{example}
\begin{example}
    Otro ejemplo será a la hora de identificar una forma en términos de la base dual, pues suponiendo un $V^*$ espacio dual de $V$, cuya base dual es $B^*=\curlybraces{f^1,f^2,\dots,f^n}$, tomando un $q\in V^*$, lo denotaremos como,
    \[q=q_if^i\]
\end{example}
\begin{note}
    En un artículo físico, se identifica directamente el escalar con el vector, es decir,
    \[w^i\equiv w\]
    pues se presupone que existe una base donde $w$ está bien definido. Así, los físicos usaremos de forma indistinguible los vectores y sus componentes respecto de una base fijada.
\end{note}
\subsection{Invariantes} % Main chapter title
\label{cap1-sec2-subsec5} 

Dado que los tensores suelen describirse en términos respecto de ciertas bases, cuando estos términos no dependen de la base empleada, los tensores se llamarán \textbf{invariantes}. O en otras palabras, los tensores que no se transforman frente a un cambio de base, serán los que llamaremos \textbf{invariantes}.\\ \\
Vamos a intentar ilustrar este concepto definiendo un tensor invariante de tipo (1,1), denominado \textit{traza}, que es un invariante conocido de las matrices. Si tenemos un tensor $A=A_j^i\ptensor{e_i}{f^j}$ que definimos como
\[\text{traza de }A=\rm{tr}A=A^i_i\]
siendo la suma de los elementos de la diagonal principal de la matriz $(A^i_j)$. No es a priori evidente que hayamos definido algo que depende únicamente de $A$, ya que los $A_j^i$ dependen no solo de $A$ sino también de la base $\curlybraces{e_i}$. Para mostrar que $\rm{tr} A$ es un número determinado enteramente por $A$ mismo y no por los $e_i$ también, debemos demostrar la invariancia; es decir, si $A$ se expresa en términos de otra base ${\tilde{e}_i}$, entonces la fórmula correspondiente en los nuevos componentes da el mismo número que antes. Así, escribimos $A=\tilde{A}^i_j\ptensor{\tilde{e}_i}{\tilde{f}^j}=A^i_j\ptensor{e_i}{f^j}$ y veremos que $A^i_j=\tilde{A}^i_j$. Usando la misma notación de cambios de base que hemos visto en el apartado anterior, tenemos la ley de transformación siguiente,
\[\tilde{A}^n_m=A^i_ja_m^jb_i^n\]
de lo cual se obtiene
\[\tilde{A}^i_i=A^p_ja^j_ib^i_p=A^p_j\delta^j_p=A^i_i\]
Queda demostrado. Luego, tenemos la proposición,
\begin{proposition}
    La traza de un tensor de tipo (1,1) es un invariante.
\end{proposition}
Para ver que no todas las expresiones en términos de las componentes de un tensor necesariamente serán un invariante, veamos el siguiente ejemplo. 
\begin{example}
    Supongamos $d=2$ y $A=\ptensor{e_1}{e_1}+\ptensor{e_1}{e_2}$, un tensor de tipo (0,2). La expresión de $A_{ii}$ en este caso será $A_{11}+A_{22}$=1+0=1. Ahora consideramos una nueva base dada por $e_1=\tilde{e}_1+\tilde{e}_2$ y $e_2=\tilde{e}_2$, entonces
    \[\begin{array}{rrl}
        A & = & (\tilde{e}_1+\tilde{e}_2)\otimes(\tilde{e}_1+\tilde{e}_2)+(\tilde{e}_1+\tilde{e}_2)\otimes\tilde{e}_2 \\
         & = & \tilde{e}_1\otimes\tilde{e}_1+2\tilde{e}_1\otimes\tilde{e}_2+\tilde{e}_2\otimes\tilde{e}_1+2\tilde{e}_2\otimes\tilde{e}_2
    \end{array}\]
    de la cuál se obtiene que $\tilde{A}_{ii}=\tilde{A}_{11}+\tilde{A}_{22}=1+2=3$. Por tanto es diferente a la base primera, luego no es un invariante.
\end{example}
\subsubsection*{Nota Final}
    Finalmente diremos que un tensor es todo aquel objeto matemático que satisfaga los cambios de base, o en otras palabras: \textit{Un tensor es todo objeto matemático que transforma como un tensor}.
%SECCION 3
\section{Dilatación temporal} % Main chapter title
\label{cap2-sec3} 
%------------------------------------------------------------------------------
La dilatación temporal es una causa directa de los postulados de Einstein. Veámoslo con un esquema,
\begin{multicols}{2}
    \begin{Figura}
        \centering
        \includegraphics[width=0.8\textwidth]{Capitulos/Capitulo2/Seccion3/Lrep.png}
        \captionof{figure}{Espejos en reposo.}
        \label{fig2.1}
    \end{Figura}
    \begin{Figura}
        \centering
        \includegraphics[width=0.8\textwidth]{Capitulos/Capitulo2/Seccion3/Lmov.png}
        \captionof{figure}{Espejos en movimiento.}
        \label{fig2.2}
    \end{Figura}
\end{multicols}
Si nos fijamos en la Figura \ref{fig2.1}, al estar los espejos en reposo, el rayo de luz que sale de la linterna vuelve en un tiempo $\Delta t=\frac{2l_0}{c}$. En cambio, suponiendo que los espejos se mueven a velocidad $\vec{v}$, y que la distancia de los brazos del rayo es $D$, entonces ahora el tiempo que tarda el rayo en ir y volver es $\Delta t'=\frac{2D}{c}$. Usando el Teorema de Pitágoras podemos calcular $D$, tal que
\[D^2=l_0^2+\left(\frac{\Delta t'v}{2}\right)^2\]
Sustituyendo $D$ y $l_0$ de las ecuaciones de $\Delta t$ y $\Delta t'$, tenemos
\[\left(\frac{\Delta t'c}{2}\right)^2=\left(\frac{\delta t'v}{2}\right)^2+\left(\frac{\Delta tc}{2}\right)^2\]
Por tanto, tenemos que el tiempo se dilata de la forma,
\begin{equation}
    \Delta t'=\gamma\Delta t
\end{equation}
y como $\gamma>1$ siempre, entonces $\Delta t'>\Delta t$, por eso se dilata el tiempo.\\ \\
Vemos que en el SRI $S'$ los relojes van más lento que en el SRI $S$, pues si consideramos como reloj el rebote de los fotones en los espejos, entonces en $S$ los fotones van más rápido que los fotones en $S'$.

%SECCION 4
\section{Contracción de longitudes} % Main chapter title
\label{cap2-sec4} 
%------------------------------------------------------------------------------
Tomamos dos eventos del espacio-tiempo, tal que
\[\Delta t'=\gamma\left(\Delta t-\frac{v}{c^2}\Delta x\right)\]
\[\Delta x'=\gamma\left(\Delta x-v\Delta t\right)\]
Asumimos que tomamos eventos que no están separados temporalmente, es decir, como si en $S'$ tomásemos una foto, así, $\Delta t'=0$. Por tanto, tendremos que $\Delta x'=L'$ y $\Delta x=L$. Luego, sustituyendo tenemos que
\[\Delta x'=\frac{\Delta x}{\gamma}\Longrightarrow L'=\frac{L}{\gamma}\]
Además, como $\gamma>1$, tendremos que $L>L'$, por tanto, se habla de contracción de longitudes; donde $L$ se conoce como \textbf{longitud propia}, que es la longitud del objeto respecto a un SRI en reposo respecto al objeto, es decir, el SRI centro de masas del objeto.
\begin{note}
    Las transformaciones de Lorentz dejan invariante las distancias espacio-temporales, pues dados dos eventos $(t_1,x_1)$ y $(t_2,x_2)$ en $S$, y los eventos correspondientes $(t_1',x_1')$ y $(t_2',x_2')$ en $S'$, entonces
    \[-c^2(t_2-t_1)^2+(x_2-x_1)^2=-c^2(t_2'-t_1')+(x_2'-x_1')^2\]
    por tanto, tenemos una cantidad que es invariante al SRI.
\end{note}


    \chapter{Cosmología} %
\label{Capitulo6} %
\lhead{\emph{Cosmología}}
\textit{“Dos cosas son infinitas: la estupidez humana y el universo; y no estoy seguro de lo segundo”.}\\
(A. Einstein)
\newpage
%-------------------------------------------------------------------------------
%SECCION 1
\section{Repaso histórico} % Main chapter title
\label{cap2-sec1} 
%------------------------------------------------------------------------------
	La física clásica, del siglo XIX, era una física bien asentada. La cuál explica la mecánica con el libro de Sir Isaac Newton titulado \textit{Philosophiae Naturalis Principia Mathematica} y el electromagnetismo se explica con el libro de Maxwell titulado \textit{Electricity and Magnetism}.\\
 En 1887, Michelson y Morley iniciaron una revolución en la física con un experimento para medir la velocidad de la luz. El experimento consistía en medir la velocidad de la luz de un rayo paralelo al eje de rotación de la Tierra y de otro rayo perpendicular a este, esperándose obtener resultados diferentes. En cambio, se observó que ambos rayos iban exactamente igual, cosa que no tenía sentido en la época., por tanto, determinaron que la velocidad de la luz no era instantánea, sino que debía ser finita, y llegaron a un resultado de ésta bastante próximo al valor actual de la velocidad de la luz.
 \subsection{Relatividad Galileana}
 El Principio de Relatividad de Galileo establece que,
 \begin{center}
 \textit{''Es imposible determinar a base de experimentos (mecánicos) si un sistema de referencia está en reposo o en movimiento uniforme y rectilíneo''.}
 \end{center}
 Esto se derivó de que en la Relatividad Galileana hay un espacio absoluto en el que las leyes de Newton son ciertas. Definiremos un \textit{sistema de referencia inercial} (SRI) como aquel sistema referencia que se mueve a velocidad constante respecto al espacio absoluto. Además, todos los sistemas de referencia inerciales comparten un tiempo absoluto. Pero con la definición de SRI, el Principio de Relatividad se debe reformular con este concepto, así tenemos el Principio de Relatividad en formulación de equivalencia, que dice que
 \begin{center}
 \textit{''Todos los sistemas inerciales son equivalentes, es decir, todos los observadores inerciales ven la misma física''.}
 \end{center}
 \textbf{Leyes de Newton}\\ \\
 La Ley de Newton por excelencia es $\vec{F}=m\vec{a}=-\nabla V(\vec{r}-\vec{r}_0)$, donde $V$ es la función potencial. Esta ley (y las demás) transforman bien bajo el grupo de transformaciones de Galileo, que son:
 \begin{enumerate}
     \item \textbf{Traslaciones temporales:}
     \[t\to t'=t+t_0\]
     \item \textbf{Traslaciones espaciales:}
     \[\vec{r}\to\vec{r}'=\vec{r}+\vec{r}_i+\vec{v}t\]
     donde $\vec{v}$ es la velocidad relativa de un SRI con respecto al otro, y $\vec{r}_i$ es el vector de posición entre los orígenes de ambos SRI al inicio.
     \item \textbf{Rotaciones espaciales:}
     \[\vec{a}'=R(\theta)\vec{a}\]
     donde $R(\theta)$ es la matriz de rotación.
\end{enumerate}
Se puede ver que las Leyes de Newton no son covariantes, pero sí transforman bien, pues la física se mantiene, esto quiere decir que \textit{las Leyes de Newton de la física transforman de forma covariante}.\\ \\
El grupo de transformaciones de Galileo son simetrías que dan lugar a cantidades conservadas. Por tanto, si tenemos un Lagrangiano que sea invariante bajo traslaciones temporales, tendremos que el sistema conserva energía; si es invariante bajo traslaciones espaciales, conserva momento lineal; y si es invariante bajo rotaciones espaciales; conserva momento angular.\\ \\
El grupo de transformaciones de Galileo NO deja invariante las ecuaciones de Maxwell, que son
\[(i)\hspace{2mm}\nabla\cdot\vec{E}=\rho/\epsilon_0;\hspace{5mm}(iii)\hspace{2mm}\nabla\cdot\vec{B}=0\]
\[(ii)\hspace{2mm}\nabla\times\vec{B}=\partial_t\vec{E}/c^2+\mu_0\vec{J};\hspace{5mm}(iv)\hspace{2mm}\nabla\times\vec{E}=-\partial_t\vec{B}\]
Si $\rho=0$ y $\vec{J}=0$, es decir, estamos en vacío, podemos combinar las ecuaciones de Maxwell en una sola ecuación de ondas que se propaga a velocidad $c=299792,458$ m/s, resultado muy próximo al valor obtenido por Michelson y Morley, que además es independiente del sistema de referencia.
\subsection{Transformaciones de Lorentz}

Las transformaciones de Lorentz hacen que las ecuaciones de Maxwell transformen bien (sean covariantes). Estas transformaciones son:
\[\begin{array}{rcrc}
    (i) & t'=\gamma\left(t-\frac{v}{c^2}x\right); & (iii) & y'=y \\
    (ii) & x'=\gamma\left(x-vt\right); & (iv) & z'=z
\end{array}\]
donde $v$ es la velocidad relativa entre SRI (que suponemos que se mueven en el eje $X$), y $\gamma=\frac{1}{\sqrt{1-\frac{v^2}{c^2}}}$.\\ \\
Como estas transformaciones hacen que las leyes de Maxwell sean covariantes, diremos que las transformaciones de Lorentz sean más fundamentales que las transformaciones de Galileo.\\ \\
Además, vemos que por la transformación $(i)$ el tiempo ya \textbf{no es absoluto}, sino que depende del SRI, por lo que diremos que el tiempo es \textbf{relativo}.
        
	\begin{appendices}
    	%\chapter{Álgebra: Lógica, Conjuntos y Grupos} % Main appendix title

\label{AppendixX} % Change X to a consecutive letter; for referencing this appendix elsewhere, use \ref{AppendixX}

\lhead{Ap\'endice A. \emph{Álgebra: Lógica, Conjuntos y Grupos}} % Change X to a consecutive letter; this is for the header on each page - perhaps a shortened title

\label{ApendiceA} 
\lhead{\emph{Álgebra: Lógica, Conjuntos y Grupos}} 
%-------------------------------------------------------------------------------
%------------------------------------------------------------------------------
%-------------------------------------------------------------------------------
%SECCION 1
\section{Repaso histórico} % Main chapter title
\label{cap2-sec1} 
%------------------------------------------------------------------------------
	La física clásica, del siglo XIX, era una física bien asentada. La cuál explica la mecánica con el libro de Sir Isaac Newton titulado \textit{Philosophiae Naturalis Principia Mathematica} y el electromagnetismo se explica con el libro de Maxwell titulado \textit{Electricity and Magnetism}.\\
 En 1887, Michelson y Morley iniciaron una revolución en la física con un experimento para medir la velocidad de la luz. El experimento consistía en medir la velocidad de la luz de un rayo paralelo al eje de rotación de la Tierra y de otro rayo perpendicular a este, esperándose obtener resultados diferentes. En cambio, se observó que ambos rayos iban exactamente igual, cosa que no tenía sentido en la época., por tanto, determinaron que la velocidad de la luz no era instantánea, sino que debía ser finita, y llegaron a un resultado de ésta bastante próximo al valor actual de la velocidad de la luz.
 \subsection{Relatividad Galileana}
 El Principio de Relatividad de Galileo establece que,
 \begin{center}
 \textit{''Es imposible determinar a base de experimentos (mecánicos) si un sistema de referencia está en reposo o en movimiento uniforme y rectilíneo''.}
 \end{center}
 Esto se derivó de que en la Relatividad Galileana hay un espacio absoluto en el que las leyes de Newton son ciertas. Definiremos un \textit{sistema de referencia inercial} (SRI) como aquel sistema referencia que se mueve a velocidad constante respecto al espacio absoluto. Además, todos los sistemas de referencia inerciales comparten un tiempo absoluto. Pero con la definición de SRI, el Principio de Relatividad se debe reformular con este concepto, así tenemos el Principio de Relatividad en formulación de equivalencia, que dice que
 \begin{center}
 \textit{''Todos los sistemas inerciales son equivalentes, es decir, todos los observadores inerciales ven la misma física''.}
 \end{center}
 \textbf{Leyes de Newton}\\ \\
 La Ley de Newton por excelencia es $\vec{F}=m\vec{a}=-\nabla V(\vec{r}-\vec{r}_0)$, donde $V$ es la función potencial. Esta ley (y las demás) transforman bien bajo el grupo de transformaciones de Galileo, que son:
 \begin{enumerate}
     \item \textbf{Traslaciones temporales:}
     \[t\to t'=t+t_0\]
     \item \textbf{Traslaciones espaciales:}
     \[\vec{r}\to\vec{r}'=\vec{r}+\vec{r}_i+\vec{v}t\]
     donde $\vec{v}$ es la velocidad relativa de un SRI con respecto al otro, y $\vec{r}_i$ es el vector de posición entre los orígenes de ambos SRI al inicio.
     \item \textbf{Rotaciones espaciales:}
     \[\vec{a}'=R(\theta)\vec{a}\]
     donde $R(\theta)$ es la matriz de rotación.
\end{enumerate}
Se puede ver que las Leyes de Newton no son covariantes, pero sí transforman bien, pues la física se mantiene, esto quiere decir que \textit{las Leyes de Newton de la física transforman de forma covariante}.\\ \\
El grupo de transformaciones de Galileo son simetrías que dan lugar a cantidades conservadas. Por tanto, si tenemos un Lagrangiano que sea invariante bajo traslaciones temporales, tendremos que el sistema conserva energía; si es invariante bajo traslaciones espaciales, conserva momento lineal; y si es invariante bajo rotaciones espaciales; conserva momento angular.\\ \\
El grupo de transformaciones de Galileo NO deja invariante las ecuaciones de Maxwell, que son
\[(i)\hspace{2mm}\nabla\cdot\vec{E}=\rho/\epsilon_0;\hspace{5mm}(iii)\hspace{2mm}\nabla\cdot\vec{B}=0\]
\[(ii)\hspace{2mm}\nabla\times\vec{B}=\partial_t\vec{E}/c^2+\mu_0\vec{J};\hspace{5mm}(iv)\hspace{2mm}\nabla\times\vec{E}=-\partial_t\vec{B}\]
Si $\rho=0$ y $\vec{J}=0$, es decir, estamos en vacío, podemos combinar las ecuaciones de Maxwell en una sola ecuación de ondas que se propaga a velocidad $c=299792,458$ m/s, resultado muy próximo al valor obtenido por Michelson y Morley, que además es independiente del sistema de referencia.
\subsection{Transformaciones de Lorentz}

Las transformaciones de Lorentz hacen que las ecuaciones de Maxwell transformen bien (sean covariantes). Estas transformaciones son:
\[\begin{array}{rcrc}
    (i) & t'=\gamma\left(t-\frac{v}{c^2}x\right); & (iii) & y'=y \\
    (ii) & x'=\gamma\left(x-vt\right); & (iv) & z'=z
\end{array}\]
donde $v$ es la velocidad relativa entre SRI (que suponemos que se mueven en el eje $X$), y $\gamma=\frac{1}{\sqrt{1-\frac{v^2}{c^2}}}$.\\ \\
Como estas transformaciones hacen que las leyes de Maxwell sean covariantes, diremos que las transformaciones de Lorentz sean más fundamentales que las transformaciones de Galileo.\\ \\
Además, vemos que por la transformación $(i)$ el tiempo ya \textbf{no es absoluto}, sino que depende del SRI, por lo que diremos que el tiempo es \textbf{relativo}.
%SECCION 2
\section{Álgebra de Tensores} % Main chapter title
\label{cap1-sec2} 
Llegamos a lo groso del capítulo, el \textbf{Álgebra de Tensores}. En este apartado vamos a ver qué es un tensor de forma matemática y cómo trabajar con ellos. También se mencionará cómo trabajamos los físicos con los tensores.
%------------------------------------------------------------------------------

\subsection{Producto tensorial: caso de dos términos} % Main chapter title
\label{cap1-sec2-subsec1} 
Vamos a ver qué es el \textbf{producto tensorial} y cómo los tensores se definen a partir de este.
\begin{proposition}
    Sea $V$ un $\mathbb{K}$-espacio vectorial, $\scalar{\cdot}{\cdot}$ el producto escalar euclídeo y $B=\curlybraces{v_1,\dots,v_n}$ base de $V$, 
    \[\begin{array}{cccl}
        f_v: & V & \to & V^*\\
         & v & \mapsto & f_v(v)=\scalar{v}{\cdot}
    \end{array}\]
     $f_v$ es una aplicación lineal, concretamente es un isomorfismo.
\end{proposition}
\begin{proof}
    Vemos que $f_v$ es aplicación lineal,
    \[f_v(w_1+w_2)=\scalar{v}{w_1+w_2}=\scalar{v}{w_1}+\scalar{v}{w_2}=f_v(w_1)+f_v(w_2)\checkmark \]
    \[f_v(\lambda\cdot w)=\scalar{v}{\lambda\cdot w}=\lambda\scalar{v}{w}=\lambda f_v(w)\checkmark\]
    para $\forall\lambda\in\mathbb{K}$ y $\forall w_1,w_2,w\in V$. Luego, es aplicación lineal.\\ \\
    Veamos que es isomorfo demostrando que es biyectivo, pues ya hemos visto que es aplicación lineal.\\
    Sabemos que $ker\curlybraces{f_v}=\curlybraces{0}\Leftrightarrow f_v$ es inyectiva. Luego, vemos si $ker\curlybraces{f_v}=\curlybraces{0}$:
    \[ker\curlybraces{f_v}=\curlybraces{w\in V,f_v(w)=0}=\curlybraces{w\in V;\scalar{v}{w}=0\Leftrightarrow w=0}\]
    Por tanto, $ker\curlybraces{f_v}=\curlybraces{0}$ y así, $f_v$ es inyectiva. $\checkmark$\\ \\
    Usando el Primer Teorema de isomorfía, tenemos que $dim(V)=\cancelto{0}{dim(ker\curlybraces{f_v})}+dim(Im f_v)$, pero como la $dim B=dim B^*$, siendo $B$ base de $V$ y $B^*$ base de $V^*$, entonces $dimV=dimV^*$, y por tanto, $dimV=dimImf_v=dimV^*$, luego $Imf_v$ es $V^*$ y por tanto, $f_v$ es sobreyectiva. $\checkmark$\\
    Luego, $f_v$ es un isomorfismo.
\end{proof}
\noindent Veamos cómo se define el producto tensorial y sus propiedades.
\begin{definition}
    Sea $V$ un $\mathbb{K}$-espacio vectorial, $V^*$ el dual de $V$, y $g^1,g^2\in V^*$ aplicaciones lineales, tal que $g^1:V\to\mathbb{K}$ y $g^2:V\to\mathbb{K}$. Así, definimos el producto tensorial como,
    \begin{enumerate}[label=(\roman*)]
        \item Producto tensorial entre dos formas $g^1,g^2\in V^*$,
        \[\begin{array}{cccl}
            \ptensor{g^1}{g^2}: & V\times V & \to & \mathbb{K}\\
            & (v,w) & \mapsto & g^1(v)g^2(w)
        \end{array}\]
        \item Producto tensorial entre dos vectores $v_1,v_2\in V$,
        \[\begin{array}{cccl}
            \ptensor{v_1}{v_2}: & V^*\times V^* & \to & \mathbb{K}\\
             & (f,g) & \mapsto & f(v_1)g(v_2)
        \end{array}\]
        \item Producto tensorial de una forma y un vector $v_1\in V$, $f^1\in V^*$,
        \[\begin{array}{cccl}
            \ptensor{v_1}{f^1}: & V^*\times V & \to & \mathbb{K}\\
             & (g,w) & \mapsto & g(v_1)f^1(w)
        \end{array}\]
    \end{enumerate}
\end{definition}

\begin{proposition}
    Los productos tensoriales definidos anteriormente son formas bilineales.
\end{proposition}
\begin{proof} 
Usando $\forall v_1,v_2,u_1,u_2,v,w,u\in V$, $\forall f^1,f^2,g,p,q\in V^*$ y $\forall \lambda\in\mathbb{K}$,
    \begin{enumerate}[label=(\roman*)]
        \item \[\begin{array}{cccl}
            \ptensor{f^1}{f^2}: & V\times V & \to & \mathbb{K}\\
            & (v,w) & \mapsto & f^1(v)f^2(w)
        \end{array}\]
        siendo $f^1,f^2\in V^*$. Veamos que es forma bilineal,
        \[\begin{array}{lrl} \text{\textbullet)} &(\ptensor{f^1}{f^2})(u_1+u_2,v)=&f^1(u_1+u_2)f^2(v)=\brackets{f^1(u_1)+f^1(u_2)}f^2(v)\\ &=&f^1(u_1)f^2(v)+f^1(u_2)f^2(v)=(\ptensor{f^1}{f^2})(u_1,v)+(\ptensor{f^1}{f^2})(u_2,v),\checkmark\\  \text{\textbullet)} &(\ptensor{f^1}{f^2})(v,u_1+u_2)  =&f^1(v)f^2(u_1+u_2)=f^1(v)\brackets{f^2(u_1)+f^2(u_2)}\\ &=&f^1(v)f^2(u_1)+f^1(v)f^2(u_2)=(\ptensor{f^1}{f^2})(v,u_1)+(\ptensor{f^1}{f^2})(v,u_2)\checkmark\\
             \text{\textbullet)} & (\ptensor{f^1}{f^2})(\lambda v,u) =& f^1(\lambda v)f^2(u)=\lambda f^1(v)f^2(u)=\lambda(\ptensor{f^1}{f^2})(v,u)\checkmark\\
        \text{\textbullet)}&(\ptensor{f^1}{f^2})(u,\lambda v)=&f^1(u)f^2(\lambda v)=\lambda f^1(u)f^2(v)=\lambda(\ptensor{f^1}{f^2})(u,v)\checkmark
          \end{array}\]
        Luego, $\ptensor{f^1}{f^2}$ es una forma bilineal. $\qedh $
        \item \[\begin{array}{cccl}
            \ptensor{v_1}{v_2}: & V^*\times V^* & \to & \mathbb{K}\\
             & (f,g) & \mapsto & f(v_1)g(v_2)
        \end{array}\]
         \[\begin{array}{lrl}
         \text{\textbullet)}&(\ptensor{v_1}{v_2})(f^1+f^2,g)=&(f^1+f^2)(v_1)g(v_2)=\brackets{f^1(v_1)+f^2(v_1)}g(v_2)\\
         &=&f^1(v_1)g(v_2)+f^2(v_1)g(v_2)=(\ptensor{v_1}{v_2})(f^1,g)+(\ptensor{v_1}{v_2})(f^2,g)\checkmark\\
         \text{\textbullet)}&(\ptensor{v_1}{v_2})(g,f^1+f^2)=&g(v_1)(f^1+f^2)(v_2)g=g(v_1)\brackets{f^1(v_2)+f^2(v_2)}\\
         &=&g(v_1)f^1(v_2)+g(v_1)f^2(v_2)=(\ptensor{v_1}{v_2})(g,f^1)+(\ptensor{v_1}{v_2})(g,f^2)\checkmark\\
         \text{\textbullet)}&(\ptensor{v_1}{v_2})(\lambda f,g)=&(\lambda f)(v_1)g(v_2)=\lambda f(v_1)g(v_2)=\lambda(\ptensor{v_1}{v_2})(f,g)\checkmark\\
         \text{\textbullet)}&(\ptensor{v_1}{v_2})(g,\lambda f)=&g(v_1)(\lambda f)(v_2)=\lambda g(v_1)f(v_2)=\lambda(\ptensor{v_1}{v_2})(g,f)\checkmark
         \end{array}\]
        Luego, $\ptensor{v_1}{v_2}$ es una forma bilineal. $\qedh $
        \item \[\begin{array}{cccl}
            \ptensor{v_1}{f^1}: & V^*\times V & \to & \mathbb{K}\\
             & (g,w) & \mapsto & g(v_1)f(w)
        \end{array}\]
        \[\begin{array}{lrl}
        \text{\textbullet)}&(\ptensor{v_1}{f^1})(p+q,w)=&(p+q)(v_1)f^1(w)=\brackets{p(v_1)+q(v_1)}f^1(w)=\\
        &=&p(v_1)f^1(w)+q(v_1)f^1(w)=(\ptensor{v_1}{f^1})(p,w)+(\ptensor{v_1}{f^1})(q,w)\checkmark\\
        \text{\textbullet)}&(\ptensor{v_1}{f^1})(g,u+w)=&g(v_1)f^1(u+w)=g(v_1)\brackets{f^1(u)+f^1(w)}=\\
        &=&g(v_1)f^1(u)+g(v_1)f^1(w)=(\ptensor{v_1}{f^1})(g,u)+(\ptensor{v_1}{f^1})(g,w)\checkmark\\
        \text{\textbullet)}&(\ptensor{v_1}{f^1})(\lambda g,w)=&(\lambda g)(v_1)f^1(w)=\lambda g(v_1)f^1(w)=\lambda(\ptensor{v_1}{f^1})(g,w)\checkmark\\
        \text{\textbullet)}&(\ptensor{v_1}{f^1})(g,\lambda w)=&g(v_1)f^1(\lambda w)=\lambda g(v_1)f^1(w)=\lambda(\ptensor{v_1}{f^1})(g,w)\checkmark
        \end{array}\]
            Luego, $\ptensor{v_1}{f^1}$ es una forma bilineal. \qedhere
    \end{enumerate}
\end{proof}
\noindent El producto tensorial no se da solo entre elementos de los espacios vectoriales o duales, sino que también se puede dar entre espacios, siendo el nuevo espacio generado un \textbf{espacio vectorial}.
\begin{proposition}
    El espacio $\ptensor{V}{V}$ tiene estructura de espacio vectorial.
\end{proposition}
\begin{proof}
    \begin{enumerate}
        \item Vemos que $(\ptensor{V}{V},+)$ es grupo abeliano:
        \begin{enumerate}[label=(\roman*)]
            \item Vemos si la operación $+$ es cerrada:
            \\
            $\forall v,w,z\in V$ con $\ptensor{v}{w},\ptensor{v}{z},\ptensor{w}{z}\in\ptensor{V}{V}$, tenemos que ver si $\ptensor{(v+w)}{z}\in\ptensor{V}{V}$. Sabemos que,
            \[\begin{array}{cccl}
                \ptensor{v}{w}: & \pcart{V^*}{V^*} & \to &\mathbb{R}  \\
                 & (f,g) & \mapsto & f(v)g(w)
            \end{array}\]
            luego,
            \[\begin{array}{cccl}
                \ptensor{(v+w)}{z}: & \pcart{V^*}{V^*} & \to &\mathbb{R}  \\
                 & (f,p) & \mapsto & f(v+w)p(z)
            \end{array}\]
            Entonces,
            \[(\ptensor{(v+w)}{z})(g,p)=f(v+w)p(z)=\brackets{f(v)+f(w)}p(z)=\]\[=f(v)p(z)+f(w)p(z)=(\ptensor{v}{z})(f,p)+(\ptensor{w}{z})(f,p)\]
            Luego, $\ptensor{(v+w)}{z}\in\ptensor{V}{V}$ y así, la operación $+$ es cerrada. $\checkmark$
            \item Asociatividad:
            \\
            Sean $\ptensor{a}{b},\ptensor{c}{d},\ptensor{e}{f}\in\ptensor{V}{V}$, tenemos que ver si $\ptensor{a}{b}+\brackets{\ptensor{c}{d}+\ptensor{e}{f}}=\brackets{\ptensor{a}{b}+\ptensor{c}{d}}+\ptensor{e}{f}$, tal que
            \[(\ptensor{a}{b}+\brackets{\ptensor{c}{d}+\ptensor{e}{f}})(p,q)=p(a)q(b)+\brackets{p(c)q(d)+p(e)q(f)}=p(a)q(b)+p(c+e)q(d+f)=\]
            \[=p(a+c+e)q(b+d+f)=p(a+c)q(b+d)+p(e)q(f)=\brackets{p(a)q(b)+p(c)q(d)}+p(e)q(f)=\]\[=(\brackets{\ptensor{a}{b}+\ptensor{c}{d}}+\ptensor{e}{f})(p,q)\checkmark\]
            \item Elemento neutro:\\
            Sea $\ptensor{e_1}{e_2}\in\ptensor{V}{V}$ el elemento neutro de $\ptensor{V}{V}$, tal que
            \[\ptensor{e_1}{e_2}+\ptensor{v}{w}=\ptensor{v}{w}+\ptensor{e_1}{e_2}=\ptensor{v}{w}\]
            Vemos el valor de este elemento neutro,
            \[(\ptensor{e_1}{e_2}+\ptensor{v}{w})(f,g)=(\ptensor{v}{w})(f,g)\]
            \[f(e_1)g(e_2)+f(v)+g(w)=f(v)g(w)\]
            \[f(e_1+v)g(e_w+w)=f(v)g(w)\Leftrightarrow\left\lbrace\begin{matrix}
                e_1=0\\
                e_2=0
            \end{matrix}\right.\]
            luego, $\ptensor{e_1}{e_2}=0$. $\checkmark$
            \item Elemento simétrico:
            \\
            $\forall\ptensor{v}{u}\in\ptensor{V}{V}$, $\exists\ptensor{\Tilde{v}}{\Tilde{u}}\in\ptensor{V}{V}$, tal que
            \[\ptensor{v}{u}+\ptensor{\Tilde{v}}{\Tilde{u}}=\ptensor{\Tilde{v}}{\Tilde{u}}+\ptensor{v}{u}=\ptensor{e_1}{e_2}=0\]
         Veamos quién es $\ptensor{\Tilde{v}}{\Tilde{u}}$,
        \[(\ptensor{v}{u}+\ptensor{\Tilde{v}}{\Tilde{u}})(f,g)=f(v)g(u)+f(\Tilde{v})g(\Tilde{u})=(\ptensor{0}{0})(f,g)=f(0)g(0)\]
        luego,
        \[v+\Tilde{v}=0\Rightarrow\Tilde{v}=-v\]
        \[u+\Tilde{u}=0\Rightarrow\Tilde{u}=-u\]
        Por tanto, el elemento simétrico de $\ptensor{v}{u}$ es $\ptensor{(-v)}{(-u)}$. $\checkmark$
        \item Conmutabilidad:\\
        Sean $\ptensor{v}{w},\ptensor{u}{z}\in\ptensor{V}{V}$, entonces
        \[(\ptensor{v}{w}+\ptensor{u}{z})(f,g)=f(v)g(w)+f(u)g(z)=f(v+u)g(w+z)=\]\[=f(u+v)g(z+w)=f(u)g(z)+f(v)g(w)=(\ptensor{u}{z}+\ptensor{v}{w})(f,g)\checkmark\]
    Luego, es grupo abeliano. $\checkmark$
         \end{enumerate}
         \item Doble propiedad distributiva:
         \begin{enumerate}
             \item $\forall\lambda,\mu\in\mathbb{R}$, $\forall\ptensor{v}{w}\in\ptensor{V}{V}$,
             \[(\lambda+\mu)\cdot(\ptensor{v}{w})(f,g)=(\lambda+\mu)f(v)g(w)=\]\[=\lambda f(v)g(w)+\mu f(v)g(w)=\lambda(\ptensor{v}{w})(f,g)+\mu(\ptensor{v}{w})(f,g)\checkmark\]
             \item $\forall\lambda\in\mathbb{R}$, $\forall\ptensor{v}{w},\ptensor{u}{z}\in\ptensor{V}{V}$, tenemos que
             \[\lambda(\ptensor{v}{w})(f,g)+\lambda(\ptensor{u}{z})(f,g)=\lambda f(v)g(w)+\lambda f(u)g(z)=\]\[=\lambda\brackets{f(v)g(w)+f(u)g(z)}=\lambda(\ptensor{v}{w}+\ptensor{u}{z})(f,g)\checkmark\]
             \end{enumerate}
             \item Propiedad pseudo-asociativa:\\
             $\forall\lambda,\mu\in\mathbb{R}$; $\forall\ptensor{v}{w}\in\ptensor{V}{V}$, tenemos que
             \[\lambda\cdot\brackets{\mu\cdot(\ptensor{v}{w})(f,g)}=\lambda\brackets{\mu f(v)g(w)}=\lambda f(\mu v)g(\mu w)=\]\[=f(\lambda\mu v)g(\lambda\mu w)=f(\mu\lambda v)g(\mu\lambda w)=\mu\brackets{f(\lambda v)g(\lambda w)}=(\mu\cdot\lambda)f(v)g(w)=(\mu\cdot\lambda)(\ptensor{v}{w})(f,g)\checkmark\]
             \item Elemento unitario del cuerpo: $\forall\ptensor{v}{w}\in\ptensor{V}{V}$; $\Tilde{\mu}\in\mathbb{R}$, entonces $\Tilde{\mu}\cdot\ptensor{v}{w}=\ptensor{v}{w}\cdot\Tilde{\mu}=\ptensor{v}{w}$
             \[(\Tilde{\mu}\cdot\ptensor{v}{w})(f,g)=f(\Tilde{\mu}v)g(\Tilde{\mu}w)=(\ptensor{v}{w})(f,g)=f(v)g(w)\Rightarrow\begin{matrix}
                 \Tilde{\mu}\cdot v=v\\
                 \Tilde{\mu}\cdot w=w
             \end{matrix}\Leftrightarrow\Tilde{\mu}=1\checkmark\]
       \end{enumerate}
       Luego, $(\ptensor{V}{V}, +, \cdot)$ es un $\mathbb{R}$-espacio vectorial.
\end{proof}
\noindent Al igual que cualquier otro espacio vectorial, el espacio $V\otimes V$ deberá tener una \textbf{base}.
\begin{proposition}
    Si tenemos un $V$ espacio vectorial sobre $\mathbb{K}$ con base $B=\curlybraces{v_1,\dots,v_n}$, entonces todo $\ptensor{v}{w}$ será combinación lineal de los elementos de la base de $\ptensor{V}{V}$ dada por $\ptensor{B}{B}=\curlybraces{\ptensor{v_i}{v_j}}_{i,j=1}^{n}$
\end{proposition}
\begin{proof}
    Queremos ver que $\curlybraces{\ptensor{v_i}{}v_j}_{i,j=1}^n$ es base de $\ptensor{V}{V}$. Para ello, tendremos que ver que esta base $\ptensor{B}{B}$ complete el espacio $\ptensor{V}{V}$ y que los vectores de la misma sean linealmente independientes.\\
    Sabemos que $\ptensor{v}{w}\in\ptensor{V}{V}$ y que
    \[\begin{array}{cccl}
        \ptensor{v}{w}: & \pcart{V^*}{V^*} & \to & \mathbb{R}\\
         & (f,g) & \mapsto & f(v)g(w)
    \end{array}\]
    Luego, para que la base $\ptensor{B}{B}$ complete el espacio $\ptensor{V}{V}$, se deberá poder expresar cualquier vector $\ptensor{v}{w}\in\ptensor{V}{V}$ como combinación lineal de los vectores de $\ptensor{B}{B}$. Podemos usar $B=\curlybraces{v_i}_{i=1}^n$ base de $V$, tal que
    \[v=\sum\limits_{i=1}^n\lambda^iv_i=\lambda^iv_i,\hspace{4mm}w=\sum\limits_{j=1}^n\mu^jv_j=\mu^jv_j\]
    Por tanto, usando $f,g\in V^*$, tenemos que
    \[\ptensor{v}{w}(f,g)=f(v)g(w)=f(\lambda^iv_i)g(\mu^jv_j)=\lambda^if(v_i)\mu^jg(v_j)=\lambda^i\mu^jf(v_i)g(v_j)=\lambda^i\mu^j(\ptensor{v_i}{v_j})(f,g)\]
    Luego, hemos expresado un vector del espacio $\ptensor{V}{V}$ como combinación lineal de los vectores de la base $\ptensor{B}{B}$. $\checkmark$\\ \\
    Veamos que son linealmente independientes, para ello, se debe cumplir que,
    \[\sum\limits_{i,j=1}^n\lambda^{ij}(\ptensor{v_i}{v_j})=\lambda^{ij}(\ptensor{v_i}{v_j})=0\Leftrightarrow\lambda^{ij}=0\]
    Sabiendo que la base de $V^*$ es $B^*=\curlybraces{f^1,f^2,\dots,f^n}$, tal que
    \[f^i(v_i)=1\hspace{5mm}f^j(v_i)\overset{i\neq j}{=}0\Rightarrow f^i(v_j)=\delta_{ij}\]
    Podemos evaluar lo anterior en dos elementos arbitrarios de $B^*$, tal que
    \[0=\lambda^{ij}(\ptensor{v_i}{v_j})(f^n,f^m)= \lambda^{ij}f^n(v_i)f^m(v_j)=\lambda_{ij}\delta_{n}^i\delta_{m}^j=\lambda^{nm}\]
    luego, $\lambda^{nm}=0$ y por tanto, los vectores son linealmente independientes. $\checkmark$\\ \\
    Así, hemos demostrado que $\ptensor{B}{B}$ es base de $\ptensor{V}{V}$.
\end{proof}

\begin{note}
    Denotaremos $\ptensor{v}{w}\equiv h$, tal que
    \[
    \begin{array}{cccl}
        h: & \pcart{V^*}{V^*} & \to & \mathbb{R} \\
         & (f^i,f^j) & \mapsto & h(f^i,f^j)=h^{ij}
    \end{array}
    \]
    siendo $f^i,f^j\in B^*$. Por tanto, para dos $p,q\in V^*$ cualesquiera, escribiremos
    \[(\ptensor{v}{w})(p,q)=h(p,q)=h\left(\sum_{i=1}^np_if^i,\sum_{j=1}^nq_jf^j\right)=p_iq_j(f^i,f^j)=h^{ij}p_iq_j\]
\end{note}
\noindent Veamos algunas \textbf{propiedades} del producto tensorial.
\begin{proposition}
    Sea $V$ un $\mathbb{R}$-espacio vectorial,
    \begin{enumerate}[label=(\roman*)]
        \item $\ptensor{(v_1+v_2)}{w}=\ptensor{v_1}{w}+\ptensor{v_2}{w}$; $\forall v_1,v_2,w\in V$.
        \item $\ptensor{w}{(v_1+v_2)}=\ptensor{w}{v_1}+\ptensor{w}{v_2}$, $\forall v_1,v_2,w\in V$.
        \item $\ptensor{(\lambda v)}{w}=\lambda\ptensor{v}{w}$, $\forall v,w\in V$, $\forall\lambda\in\mathbb{R}$.
        \item $\ptensor{w}{(\lambda v)}=\lambda\ptensor{w}{v}$, $\forall v,w\in V$, $\forall \lambda\in\mathbb{R}$.
        \item $\ptensor{v}{w}\neq\ptensor{w}{v}$.
        \item $\ptensor{v}{w}\neq0$ si $v\neq0$ ó $w\neq 0$.
        \item Sea $\ptensor{a}{b}\neq0$, $\ptensor{a}{b}=\ptensor{a'}{b'}\Leftrightarrow a'=\lambda a$ y $b'=\lambda^{-1}b$.
        \item $\ptensor{V}{W}$ es isomorfo con $\ptensor{W}{V}$.
    \end{enumerate}
\end{proposition}
\begin{proof}
    \begin{enumerate}[label=(\roman*)]
        \item $\forall v_1,v_2,w\in V$,
        \[(\ptensor{(v_1+v_2)}{w})(f,g)=f(v_1+v_2)g(w)=\brackets{f(v_1+f(v_2)}g(w)=\]\[=f(v_1)g(w)+f(v_2)g(w)=(\ptensor{v_1}{w})(f,g)+(\ptensor{v_2}{w})(f,g)\qedh\]
        \item $\forall v_1,v_2,w\in V$,
        \[(\ptensor{w}{(v_1+v_2)})(f,g)=f(w)g(v_1+v_2)=f(w)\brackets{g(v_1)+g(v_2)}=\]\[=f(w)g(v_1)+f(w)g(v_2)=(\ptensor{w}{v_1})(f,g)+(\ptensor{w}{v_2})(f,g)\qedh\]
        \item $\forall v,w\in V$ y $\forall\lambda\in\mathbb{R}$,
        \[(\ptensor{(\lambda\cdot v)}{w})(f,g)=f(\lambda\cdot v)g(w)=\lambda\cdot f(v)g(w)=\lambda\cdot(\ptensor{v}{w})(f,g)\qedh\]
        \item $\forall v,w\in V$ y $\forall\mu\in\mathbb{R}$,
        \[(\ptensor{w}{(\lambda\cdot v)})(f,g)=f(w)g(\lambda\cdot v)=\lambda\cdot f(w)g(v)=\lambda\cdot(\ptensor{w}{v})(f,g)\qedh\]
        \item Vemos que, $(\ptensor{v}{w})(f,g)=f(v)g(w)$ y que $(\ptensor{w}{v})(f,g)=f(w)g(v)$, luego estos elementos serían iguales solo si $f\equiv g$. $\qedh$
        \item Sean $v,w\in V$ y $f,g\in V^*$, tales que $f\not\equiv0$ y $g\not\equiv0$, entonces
        \[(\ptensor{v}{w})(f,g)=f(v)g(w)=0\Leftrightarrow\begin{matrix}
            f(v)=0 & \Leftrightarrow v=0\\
            \text{ó} & \\
            g(w)=0 & \Leftrightarrow w=0
        \end{matrix}\qedh\]
        \item \begin{tabular}{c|}
             $\Rightarrow$ \\ \hline
        \end{tabular} 
        Sea $\ptensor{a}{b}=\ptensor{a'}{b'}$ entonces
        \[(\ptensor{a}{b})(f,g)=f(a)g(b)=(\ptensor{a'}{b'})(f,g)=f(a')g(b')\]
        luego,
        \[f(a)g(b)=f(a')g(b')\]
        pero como $a\neq a'$ y $b\neq b'$, debe haber una relación entre ambos, de tal forma que se cumpla la igualdad anterior. Supondremos que $a$ y $a'$ tienen una relación lineal (la más sencilla), tal que $a'=\lambda a+c$, luego 
        \[f(a)g(b)=f(a')g(b')=f(\lambda a+c)g(b')=f(\lambda a)g(b')+f(c)g(b')=\lambda f(a)g(b')+f(c)g(b')\]
        Agrupamos términos de la igualdad, tal que,
        \[0:\hspace{5mm}0=f(c)g(b')\]
        \[f(a):\hspace{5mm}g(b)=\lambda g(b')\]
        Por la propiedad \textit{(vi)}, como $b'\neq0$, entonces $c=0$. Además,
        \[g(b)=\lambda g(b')\Rightarrow g(b')=\lambda^{-1}g(b)\Rightarrow g(b')=g(\lambda^{-1}b)\Rightarrow b'=\lambda^{-1}b\]
        Luego,
        \[\begin{matrix}
            a'=\lambda a\\
            b'=\lambda^{-1}b
        \end{matrix}\hspace{4mm}\checkmark\]
        \begin{tabular}{c|}
             $\Leftarrow$ \\ \hline
        \end{tabular} 
        Sea $a'=\lambda a$ y $b'=\lambda^{-1}b$, entonces
        \[(\ptensor{a'}{b'})(f,g)=f(a')g(b')=f(\lambda a)g(\lambda^{-1}b)=\cancel{\lambda}\cancel{\lambda^{-1}}f(a)g(b)=(\ptensor{a}{b})(f,g)\checkmark\]
        \item Sean $V,W$ espacios vectoriales, tales que
        \[\begin{array}{ccl}
            \ptensor{V}{W} & \to & \ptensor{W}{V}  \\
            \ptensor{v}{w} & \mapsto & \ptensor{w}{v}
        \end{array}\]
        Si suponemos que $dimV=n$ y $dimW=m$, sabemos por tanto que $dim(V\otimes W)=n\cdot m$ y $dim(W\otimes V)=m\cdot n$, luego tienen la misma dimensión y por tanto, son isomorfos. $\checkmark$\\ \\
        También podemos hacerlo sin usar la proposición de que $dim(\ptensor{V}{W})=n\cdot m$. Es claro ver que la aplicación es inyectiva, pues no hay dos elementos con la misma imagen, ya que la imagen se forma al permutar los elementos. Luego, al ser inyectivo, tenemos que $dimKer=0$. Por el Primer Teorema de Isomorfía,
        \[dim(\ptensor{V}{W})=\cancelto{0}{dimKer}+dimIm=dimIm=dim(\ptensor{W}{V})\]
        Luego, como $\ptensor{V}{W}$ y $\ptensor{W}{V}$ tienen la misma dimensión, entonces son isomorfos.
    \end{enumerate}
\end{proof}

\subsection{Aplicaciones lineales} % Main chapter title
\label{cap1-sec1-subsec2} 

Veamos ahora las aplicaciones lineales.
\begin{definition}
    Sean $V$ y $V'$ dos espacios vectoriales sobre el mismo cuerpo $\mathbb{K}$.
    Se dice que en una aplicación $f:V\longrightarrow V'$ es una aplicación lineal, o también llamado homomorfismo de espacios vectoriales, si se verifica:
    \begin{enumerate}[label=(\roman*)]
        \item $f(x+y)=f(x)+f(y),\forall x,y\in V$
        \item $f(\lambda\cdot x)=\lambda\cdot f(x),\forall\lambda\in\mathbb{K},\forall x\in V$
    \end{enumerate}
    Diremos además que $f$ es un isomorfismo lineal si es biyectiva, que $f$ es un endomorfismo si $V=V'$ y que es un automorfismo si es un endomorfismo biyectivo. 
\end{definition}

\noindent Las aplicaciones lineales tienen asociados dos conjuntos cuyas características son de interés, a saber, el núcleo y la imagen.

\begin{definition}
    Sea $f:V\longrightarrow W$ definimos el núcleo o kernel de la aplicación $f$ como
    \[Kerf=\curlybraces{v\in V:f(v)=0}\]
    y la imagen como
    \[Imf=\curlybraces{w\in W:\exists v\in V/f(v)=w}.\]
\end{definition}

\noindent Veamos algunas propiedades básicas de ambos conjuntos.

\begin{proposition}
Sea $f:V\to V'$ una aplicación lineal, se tienen las siguientes propiedades:
\begin{enumerate}[label=(\roman*)]
    \item \label{prop1:item1} $\rm{Im}f$ es un subespacio de $V'$ y que $\rm{Ker}f$ es un subespacio de $V$.
    \item \label{prop1:item2}Si $W$ es un subespacio vectorial de $V$, entonces $f(W):=\curlybraces{f(w): w\in W}$ es un subespacio de $V'$.
    \item \label{prop1:item3}Si $W'$ es un subespacio de $V'$, entonces $f^{-1}(W'):=\curlybraces{v\in V: f(v)\in W'}$ es también un subespacio de $V$.
\end{enumerate}  
\end{proposition}
%\newpage
\begin{proof}
\begin{enumerate}[label=\ref{prop1:item1}]
    \item Por definición, como los elementos de la $\rm{Im}f$ son pertenecientes a $V'$, entonces la $\rm{Im}f$ es subespacio de $V'$. De igual forma ocurre con el $\rm{Ker}f$, pues sus elementos pertenecen a $V$ y por tanto, este es subespacio de $V$.
\end{enumerate}
\begin{enumerate}[label=\ref{prop1:item2}]
    \item Como $W$ es subespacio de $V$, tenemos que $w\in V$ también, por tanto, los $f(w)$ pertenecerán a $V'$, cosa que implica que $f(W)$ es subespacio de $V'$, pues los $f(w)$ de $f(W)$ pertenecen a $V'$.
\end{enumerate}
\begin{enumerate}[label=\ref{prop1:item3}]
    \item Por analogía a $\ref{prop1:item2}$ vemos que $f^{-1}(W)$ es subespacio de $V$.
\end{enumerate}
\end{proof}
\noindent Ahora veamos algunas propiedades esenciales de las aplicaciones lineaales.
\begin{proposition}
    Sea $f:V\longrightarrow V'$ una aplicación lineal,
    \begin{enumerate}[label=(\roman*)]
        \item \label{pro1:item1} entonces $f$ es inyectiva si y solo si $Kerf=\curlybraces{0}$.
        \item \label{pro1:item2} si $G$ es un conjunto generador de $V$, $<G>=V$, entonces $f(G)$ es conjunto generador
        de $Imf$, $<f(G)>=Imf$.
        \item \label{pro1:item3} si $S\subset V$ es un conjunto de vectores linealmente independientes, si $f$ es inyectiva, entonces $f(S)$ es linealmente independiente.
        \item \label{pro1:item4} $f$ es inyectiva $\Leftrightarrow$ conserva la independencia lineal.
   
        \item \label{pro1:item5} si $f$ es biyectiva y $B$ es una base de $V$, entonces $f(B)$ es base de $V'$.

        \item \label{pro1:item6} $f$ es sobreyectiva $\Leftrightarrow$ $Imf=V'$
    \end{enumerate}
\end{proposition}



\begin{proof}
\ref{pro1:item1} \begin{tabular}{c|}
                 $\Rightarrow$ \\ \hline
            \end{tabular}
            Suponiendo que $f$ es inyectiva, sabemos que su Kernel es,
            \[\rm{Ker}f=\curlybraces{v\in V:f(v)=0}\]
            pero como la inyectividad nos implica que la imagen debe provenir de un único vector de entrada, entonces este vector será $v=0$, y por tanto, $ker f=\curlybraces{0}$. $\checkmark$
\\     
            \begin{tabular}{c|}
                 $\Leftarrow$  \\ \hline
            \end{tabular}
            Suponiendo que $ker f=\curlybraces{0}$, esto nos quiere decir que únicamente el vector $v=0$ satisface $f(v)=0$, luego como un vector tiene una única imagen, decimos que $f$ es inyectiva. \qedh
\\ \\
\ref{pro1:item2}  Veamos que el conjunto $f(G)$ es sistema generador de la imagen, es decir,   \[<f(G)>=Imf\Leftrightarrow\forall y\in Imf,\exists\lambda^1,\dots,\lambda^n\in\mathbb{K},y_1,\dots,y_n\in f(G)\text{ tales que }y=\lambda^1y_1+\dots+\lambda^ny_n.\] Sabemos que $G$ es conjunto generador, luego sea $y\in Imf$. Entonces por definición se tiene que existe $x\in V$ tal que $f(x)=y$. Como $<G>=V$, existen $\lambda^1,\dots,\lambda^n\in\mathbb{K}$, $v_1,\dots,v_n\in G$ tales que \[ x= \sum_{i=1}^n \lambda^i v_i.\] Tenemos entonces que:  \[y=f(x)=f(\lambda^1v_1+\dots+\lambda^nv_n)=\lambda^1f(v_1)+\dots+\lambda^nf(v_n).\] Por lo tanto, $y$ es combinación lineal de elementos de $f(G)$, es decir, $<f(G)>=\mathrm{Im}f$. \qedh
\\ \\
\ref{pro1:item3}
            Sea $S$ un conjunto linealmente independiente en $V$. 
            Supongamos que $f$ es inyectiva, vamos a probar que $f(S)$ es linealmente independiente, es decir,
            \[\lambda^1y_1+\dots+\lambda^ny_n=0\Rightarrow\lambda^1=\lambda^2=\dots=\lambda^n=0\hspace{4mm} \forall y_1,\dots,y_n\in f(S),\hspace{2mm}
                \forall\lambda^1,\dots,\lambda^n\in\mathbb{K}\]
            Supongamos $\lambda^1y_1+\dots+\lambda^ny_n=0$.  Como $y_j\in f(S),\exists x_j\in S/f(x_j)=y_j$.
             \[\left.\begin{array}{r}
                \lambda_1f(x_1)+\dots+\lambda_nf(x_n)=0\\
                f(\lambda_1x_1+\dots+\lambda_nx_n)=0
            \end{array}\right\rbrace f(0)=0\Rightarrow\lambda_1x_1+\dots+\lambda_nx_n=0\Rightarrow f\text{ inyectiva}\Rightarrow\lambda_1=\dots=\lambda_n=0\]
                \\ \\
                \ref{pro1:item4} \begin{tabular}{c|}
                 $\Longrightarrow$ \\ \hline
            \end{tabular} 
            Trivial por (iii) $\checkmark$\\
            \begin{tabular}{c|}
                 $\Longleftarrow$ \\ \hline
            \end{tabular} 
            Por reducción al absurdo:\\
            Supongamos que existen $v_1,v_2\in V$ distintos, tales que $f(v_1)=f(v_2)\Leftrightarrow f(v_1)-f(v_2)=0\Leftrightarrow f(v_1-v_2)=0$. 
            Luego, $v=v_1-v_2\neq0$ verifica que $f(v)=0$, $\curlybraces{v}$ es un conjunto linealmente independiente, $f(\curlybraces{v})$ tendría que ser un conjunto l.i. por hipótesis, pero $f(\curlybraces{v})=\curlybraces{0}$ que no es un conjunto l.i. cosa absurda. \qedh
            \\ \\
            \ref{pro1:item5}  Sea una aplicación lineal biyectiva $f:V\to V'$
            y una base de $V$, $B=\curlybraces{v_1,\dots,v_n}$. Entones, si aplicamos
\[f(B)=\curlybraces{f(v_1),\dots,f(v_n)}=\curlybraces{v_1',\dots,v_n'}\]
            y entonces, estos $v_i'\in V'$ van a formar una base de $V'$, pues al ser $f$ biyectiva, los vectores serán linealmente independientes, pues los de $B$ lo son; y además, como tienen la misma dimensión que $V'$, pasan de ser conjunto generador a base. $\qedh$
            \\ \\
            \ref{pro1:item6} \begin{tabular}{c|}
                 $\Rightarrow$  \\ \hline
            \end{tabular}
            Suponiendo que $f$ es sobreyectiva, tendremos que para cada $y\in V'$, existe al menos un $x\in V$, tal que $f(x)=y$. Por consiguiente, cada elemento de $V'$ es la imagen de un elemento de $V$, es decir, $Imf=V'$. $\checkmark$\\
            \begin{tabular}{c|}
                 $\Leftarrow$  \\ \hline
            \end{tabular}
            Suponiendo que $imf=V'$, tenemos que todos los elementos de $V'$ son imagen de los elementos de $V$, siendo esta la propia definición de sobreyectividad, luego $f$ es sobreyectiva.
\end{proof}
\noindent Una vez visto estas propiedades, de $\ref{pro1:item5}$ podemos obtener un resultado interesante, que es la siguiente proposición.
\begin{proposition}
Sea $B=\curlybraces{v_1,\dots,v_n}$ una base de $V$, y sea $f:V\rightarrow V'$ una aplicación lineal. Se tiene entonces que $\curlybraces{f(v_1),\dots,f(v_n)}$ es un sistema generador de la imagen.
\end{proposition}
\begin{proof}
    Supongamos que $B$ es una base y que conocemos $f(v_j),\forall v_j\in B$. 
    Sea $v\in V$, escrito en coordenadas de la base como $v=\lambda^1v_1+\dots+\lambda^nv_n$, con $v_i\in B$, 
 y $\lambda^i\in\mathbb{K}$, entonces  $f(x)=f(\lambda^1v_1+\dots+\lambda^nv_n)=\lambda^1f(v_1)+\dots\lambda^nf(v_n)$, luego hemos puesto $f(x)$ en coordenadas de $\curlybraces{f(v_1,\dots,f(v_n)}$.
\end{proof}

\noindent Ahora vamos a ver un resultado bastante importante, el cuál nos permitirá representar aplicaciones lineales en matrices, denominadas \textbf{matrices asociadas a la aplicación $f$}. Además, este resultado es importante para Física, pues los físicos no solemos trabajar con aplicaciones, sino que trabajamos con sus matrices asociadas, pues se puede decir que "tienen" la misma información que las aplicaciones.

\begin{proposition}  
\label{prop1.4}
    Sean $(V,+,\cdot)$ y $(V',+,\cdot)$ $\mathbb{K}$-espacios vectoriales de dimensión finita con $dimV=n$ y $dimV'=m$. 
    Sea $f:V\longrightarrow V'$ una aplicación lineal, entonces dadas $\left\lbrace\begin{matrix}
        B=\curlybraces{v_1,\dots,v_n}\text{ base de }V\\
        B'=\curlybraces{v_1',\dots,v_n'}\text{ base de }V'
    \end{matrix}\right.$\\
    $f$ se representa en esas bases como una matriz en $\mathcal{M}_{m\times n}(\mathbb{K})$.
\end{proposition}

\begin{proof}
    Como $f$ es lineal, me basta con conocer $f(B)$, para ello, tenemos que conocer $f(v_1),f(v_2),\dots,f(v_n)$, teniendo:
    \[\begin{matrix}
        f(v_1) & = & a_{1}^1v_1'+a_{1}^2v_2'+\dots+a_{1}^mv_m', & a_{1}^i\in\mathbb{K}\\
        f(v_2) & = & a_{2}^1v_1'+a_{2}^2v_2'+\dots+a_{2}^mv_m', & a_{2}^i\in\mathbb{K}\\
        \vdots & & \vdots & \vdots\\
        f(v_n) & = & a_{n}^1v_1'+a_{n}^2v_2'+\dots+a_{n}^mv_m', & a_{n}^i\in\mathbb{K}
    \end{matrix}\]
    Sea $v\in V:v=\lambda^1v_1+\dots+\lambda^nv_n,\hspace{2mm}\lambda^i\in\mathbb{K}$, si le aplicamos $f$ tenemos,
    \[\begin{array}{rll}
        f(v) & = &\lambda^1f(v_1)+\dots+\lambda^nf(v_n) \\
         & = & \lambda^1(a_{1}^1v_1'+\dots+a_{1}^mv_m')+\lambda^2(a_{2}^1v_1'+\dots+a_{2}^mv_m')+\dots+\lambda^n(a_{n}^1v_1'+\dots+a_{n}^mv_m')\\
         & = & (a_{1}^1\lambda^1+a_{2}^1\lambda^2+\dots+a_{n}^1\lambda^n)v_1'+(a_{1}^2\lambda^1+\dots+a_{n}^2\lambda^n)v_2'+\dots+(a_{1}^m\lambda^1+\dots+a_{n}^m\lambda^n)v_m'
    \end{array}
        \]
   Luego,  $f(v)=\mu^1v_1'+\mu^2v_2'+\dots+\mu^mv_m'$, siendo $\mu^i=(a_{1}^i\lambda^1+\dots+a_{n}^i\lambda^n)$, luego, para construir la matriz $A$, ponemos las coordenadas de $v_1$ en la primera columna, las de $v_2$ en la segunda y así sucesivamente, tal que:
    \[\begin{pmatrix}
        \mu^1\\
        \mu^2\\
        \vdots\\
        \mu^m
    \end{pmatrix}=\begin{pmatrix}
        a_{1}^1 & a_{1}^1 & \dots & a_{1}^m\\
        a_{2}^1 & a_{2}^2 & \dots & a_{2}^m\\
        \vdots & \vdots & \ddots & \vdots\\
        a_{n}^1 & a_{n}^2 & \dots & a_{n}^m
    \end{pmatrix}\begin{pmatrix}
        \lambda^1\\
        \lambda^2\\
        \vdots\\
        \lambda^n
    \end{pmatrix}\Rightarrow \mu=A\cdot\lambda\]
\end{proof}

\noindent Vamos a introducir ahora el concepto de \textbf{rango} de una aplicación lineal, que puede extenderse al rango de su matriz asociada.

\begin{definition}
    Se llama rango de una aplicación lineal (matriz) a la dimensión de su imagen y se denota por $rg()$.
\end{definition}

\noindent Como un mismo espacio vectorial puede estar generado por varias bases, es lógico pensar que debe haber una relación entre estas bases o al menos una forma de cambiar de una base a otra, lo que se conoce como \textbf{cambio de base}. Esto es posible y una forma sencilla de hacerlo es mediante las matrices asociadas.

\begin{proposition}
    -Sean $V$ y $V'$ dos espacios vectoriales en $\mathbb{K}$, sea $f:V\longrightarrow V'$ lineal.\\
    -Sea $B_1=\curlybraces{v_1,\dots,v_n}$ base de $V$, $B_1'=\curlybraces{v_1',\dots,v_m'}$ base de $V'$.\\
    -Sea $A\in\mathcal{M}_{m\times n}(\mathbb{K})$ la matriz que representa a $f$ en $B_1,B_1'$.\\
    -Sea $B_2=\curlybraces{u_1,\dots,u_n}$ base de $V$, $B_2'=\curlybraces{u_1',\dots,u_m'}$ base de $V'$.\\
    -Sea $\tilde{A}\in\mathcal{M}_{m\times n}(\mathbb{K})$ la matriz que representa a $f$ en $B_2,B_2'$.\\
    -Sea $P$ la matriz de cambio de base de $B_1$ en $B_2$.\\
    -Sea $Q$ la matriz de cambio de base de $B_1'$ en $B_2'$.\\
    Entonces $\tilde{A}=Q^{-1}\cdot A\cdot P$.
\end{proposition}
\begin{proof}
    Sea $f:V\to V'$ una aplicación lineal con $n=dim V$ y $m=dim V'$. Si $A$ y $\tilde{A}$ son las matrices asociadas a $f$ respecto de distintas bases, entonces
    \[rg(A)=dim(Imf)=rg(\tilde{A})\]
    Luego $A$ y $\Tilde{A}$ tienen igual rango, y por tanto, son matrices equivalentes. Concretemos más esta situación:\\
    Sean $B_1$ y $B_2$ bases de $V$ con cambio de base de $B_1$ a $B_2$ dado por $X_1=PX_2$ y sean $B_1'$ y $B_2'$ bases de $V'$, con cambio de $B_1'$ a $B_2'$ dado por $Y_1=QY_2$.\\
    Consideremos la matriz asociada a $f$ respecto de $B_1$ y $B_1'$, $A\in\mathcal{M}_{m\times n}(\mathcal{K})$, tal que $A=\mathcal{M}_{B_1,B_1'}(f)$ y la ecuación matricial
    \[Y_1=AX_1\]
    De igual forma, sea $\tilde{A}\in\mathcal{M}_{m\times n}(\mathbb{K})$ la matriz asociada a $f$ respecto de $B_2$ y $B_2'$, tal que $\tilde{A}=\mathcal{M}_{B_2,B_2'}(\mathbb{K})$ y la ecuación matricial de $f$ respecto de estas bases,
    \[Y_2=\tilde{A}X_2\]
    Gráficamente,
    \[\begin{matrix}
    & V & \to & V' & \\
            & & A & &\\
            & B_1 & \longrightarrow & B_1' & \\
            P & \uparrow & & \uparrow & Q\\
             & & \tilde{A} & & \\
             & B_2 & \longrightarrow & B_2' &
        \end{matrix}\]
        Entonces,
        \[Y_2=\left\lbrace \begin{array}{l}
        \tilde{A}X_2\\    Q^{-1}Y_1=Q^{-1}AX_1=Q^{-1}APX_2\end{array}\right.\]
        y en consecuente,
        \[\tilde{A}=Q^{-1}AP\]
        O bien,
        \[X_2=\left\lbrace\begin{array}{l}
            \tilde{A}^{-1}Y_2\\
            P^{-1}X_1=P^{-1}A^{-1}Y_1=P^{-1}A^{-1}QY_2
        \end{array}\right.\]
        y en consecuente,
        \[\tilde{A}^{-1}=P^{-1}A^{-1}Q\]
\end{proof}

Ahora vamos a enunciar el \textbf{Primer Teorema de Isomorfía}, del que obtendremos un Corolario muy importante a la hora de trabajar con aplicaciones lineales. Este teorema no se va a demostrar (si se quiere ver la prueba consultar  \cite[Chapter 6, Theorem 6.5, Page 77]{IntroducciónTeoríaDeGrupos}).

\begin{theorem}[Primer teorema de isomorfismo de Noether]
    Sea $f:V\longrightarrow V'$ una aplicación lineal, entonces:
    \begin{enumerate}[label=(\roman*)]
        \item Existe una aplicación lineal sobreyectiva $\pi:V\longrightarrow V/Kerf$
        \item Existe un isomorfismo $\bar{f}:V/Kerf\longrightarrow Imf$
        \item Existe una aplicación lineal inyectiva $i:Imf\longrightarrow V'$, tales que $f=i\circ\bar{f}\circ\pi$, tal que
        \[\begin{matrix}
            & & f & \\
            & V & \longrightarrow & V' & \\
            \pi & \downarrow & & \uparrow & i\\
             & & \bar{f} & & \\
             & V/Kerf & \longrightarrow & Imf &
        \end{matrix}\]
    \end{enumerate}
   
\end{theorem}
\begin{corollary}
     Si además $V$ es finitamente generado,
    \[dimV=dim(Kerf)+dim(Imf)\]
\end{corollary}
\subsection{Contrtacción de tensores} % Main chapter title
\label{cap1-sec2-subsec3} 

 Una vez que hemos visto cómo subir y bajar índices, podemos definir una operación denominada \textbf{contracción} de tensores, la cuál encoge un tensor $(r,s)$ a uno $(r-1,s-1)$. La definición general se obtiene a partir del siguiente caso especial.

\begin{lemma}
    Hay una única aplicación lineal
    $C:\Omega_1^1\to\mathbb{R}$
    llamada \textit{contracción (1,1)}, tal que
    \[\begin{array}{rlll}
        C: & \Omega_1^1 (V)& \to & \mathbb{R} \\
         & \ptensor{v}{f} & \mapsto & C(\ptensor{v}{f})=f(v)
    \end{array}\]
    para todo $v\in V$ y $f\in V^*$.
\end{lemma}
\begin{proof} (Esta demostración usa el concepto de matrices de cambio de base, por lo que se recomienda ver la sección \ref{CambioBasesTensores(1,1)})\\ 
    Tomando $B=\curlybraces{v^1,v^2,\dots,v^n}$ base de $V$ y $B^*= \curlybraces{f_1,f_2,\dots,f_n}$ base de $V^*$, podemos escribir un tensor de tipo $(1,1)$ como
    \[A\equiv\sum A^i_j\ptensor{f_i}{v^j}\]
    Como $C(\ptensor{f_i}{v^j})=f_i(v^j)=\delta^j_i$, por la condición de base dual, no nos queda otra opción, más que definir,
    \[C(A)=\sum A_i^i=\sum A(f_i,v^i)\]
    Entonces, $C$ tiene las propiedades requeridas en las bases $B,B^*$. Luego, para obtener la función general requerida es suficiente con mostrar que esta definición es independiente de la elección del sistema de coordenadas. Así, tomando una nueva base de $V$, $B'=\curlybraces{w^1,w^2,\dots,w^n}$ y otra de $V^*$, $B^{*'}=\curlybraces{q_1,q_2,\dots,q_n}$, tenemos
    \[\begin{array}{rrl}
        C(A) & = & \sum\limits_mA(q_m,w^m)= \sum\limits_mA\left(\sum\limits_i a_i^mf_m,\sum_jb_m^jv^m\right)\\
         & = & \sum\limits_{i,j,m}a_i^mb_m^jA(f_i,v^j)=\sum\limits_{i,j}\delta^j_iA(f_i,v^j)\\
         & = & \sum\limits_iA(f_i,v^j)
    \end{array}\]
\end{proof}
\noindent Para extender las contracciones $(1,1)$, $C$, a un tensor de un tipo mayor, el esquema es especificar una componente covariante y otra contravariante y aplicar $C$ a estos.\\

\noindent Suponemos un tensor $A\in\Omega_r^s(V)$ y $1\leq r$ y $1\leq j \leq s$. Fijamos las formas $p_1,p_2,\dots,p_{r-1}$ y los vectores $u_1,u_2,\dots ,u_{s-1}$. Entonces la función
\[(p,u) \to A(p_1, \dots, \underbrace{p_{i}}_{\mathclap{i\text{-ésima componente contravariante}}}, \dots, p_{r-1}, u^{1}, \dots, \overbrace{u^{j}}^{\mathclap{j\text{-ésima componente covariante}}}, \ldots, u^{s-1})\]
es un tensor $(1,1)$ que puede escribirse como

\[A(p_1,\dots,\cdot,\dots,p_{r-1},u^1,\dots,\cdot,\dots,u^{s-1})\]
Aplicando la contracción $(1,1)$ a este tensor, produce una función de valor real denotada por

\[\left(C_j^iA\right)\left(p_1,\dots,p_{r-1},u^1,\dots,u^{s-1}\right)\]
Siendo $C_j^iA$ una función multilineal. Por tanto, esto es un tensor de tipo $(r-1,s-1)$ llamado \textit{la contracción de }$A$\textit{ sobre }$i,j$.

%\begin{definition}
 %   La contracción de un tensor $A$ de tipo $(r,s)$ con respecto al índice contravariante $p$ $(p\leq r)$ y al índice covariante $q$ $(q\leq s)$ es el tensor de tipo $(r-1,s-1)$, teniendo las componentes,
  %  \[B^{i_1\dots i_{r-1}}_{j_1\dots j_{s-1}}=A^{i_1\dots i_{p-1}ki_p\dots i_{r-1}}_{j_1\dots j_{q-1}kj_q\dots j_{s-1}}\]
%\end{definition}

\begin{note}
    Para poder contraer tensores, debemos tener superíndices y subíndices, así, podemos usar primero la métrica para subir o bajar índices y luego aplicar la contracción.
   \end{note} 
\begin{example}
        Si tenemos un tensor de tipo (0,2),
    $S\equiv S_{\alpha\beta}$, podemos hacer,
    \[\begin{array}{rllll}
        S_{\alpha\beta} & \to & g^{\gamma\alpha}S_{\alpha\beta}=S^{\alpha}_{\beta} & \to & C^1_1S^{\gamma}_{\beta}=S^{\beta}_{\beta} \\
        \text{Tensor (0,2)} & \to & \text{Tensor (1,1)} & \to &\text{Escalar}
    \end{array}\]
    cosa que se puede simplificar simplemente usando,
    \[S\equiv S_{\alpha\beta}\to g^{\beta\alpha}S_{\alpha\beta}=S^{\beta}_{\beta}\]
    es decir, podemos contraer tensores con la propia métrica.
\end{example}

\begin{example}
    Si
    \[U^j_i=T^{kj}_{ik}\]
    entonces
    \[U'^{j'}_{i'}=T'^{k'j'}_{i'k'}=S^i_{i'}S^l_{k'}R^{k'}_kR^{j'}_jT^{kj}_{il}=S^i_{i'}\delta^l_kR^{j'}_jT^{kj}_{il}=S^i_{i'}R^{j'}_jT^{kj}_{ik}=S^i_{i'}R^{j'}_jU^j_i\]
    donde hemos utilizado $S^l_{k'}R^{k'}_k=\delta^l_k$. Vemos que se transforma como un tensor (1,1).\\

\noindent    Así, dado un tensor $T^{ij}_{kl}$ de tipo (2,2), serán posible las 4 contracciones
    \[T^{kj}_{ki},\hspace{3mm}T^{jk}_{ik},\hspace{3mm}T^{kj}_{ik},\hspace{3mm}T^{jk}_{ki}\]
    que originan 4 tensores de tipo (1,1). Por otro lado, las dos posibles contracciones dobles que dan lugar a un escalar (tensor de tipo (0,0)) son
    \[T^{kj}_{kj},\hspace{3mm}T^{jk}_{kj}\]
\end{example}
\begin{note}
    El producto escalar $(\mathbb{R}^n,g_{ij})$ también se puede contraer. Pues $g_{ij}$ es un tensor de tipo (0,2), al cual le podemos aplicar una contracción 1,1, pero primero lo pasamos a un tensor de tipo (1,1), variando sus índices, tal que
    \[C^1_1\left(g^{ki}g_{ij}\right)=C^1_1(g^k_j)=g^j_j=n\]
    donde sabemos que vale $n$, pues al ser un espacio de dimensión $n$, la matriz asociada a $g$ será $G\in\mathcal{M}_{n\times n}$ y por tanto, la traza será la suma de $n$-elementos. Sabemos que estos elementos son el 1, porque la traza es invariante frente a los cambios de base (cosa que veremos más adelante), por tanto, si cogemos el producto escalar usual en la base usual, la matriz asociada es la matriz de Gram, cuyos elementos son todos nulos, salvo la diagonal que está formada por 1.
\end{note}
\subsection{Notación de Einstein} % Main chapter title
\label{cap1-sec1-subsec4} 

La notación de Einstein va a servir para facilitarnos la escritura, pues cada vez que tengamos un vector o una forma escrita como combinación lineal, vamos a poder redefinirlos como
\[w=\sum\limits_{i=1}^n\lambda^iv_i\equiv\lambda^iv_i\]
esto para un vector. Para una forma, tendremos
\[p=\sum\limits_{i=1}^n\mu_if^i\equiv\mu_i f^i\]
Además, para simplificar aún más la notación y dejarnos de tantas letras, vamos a identificar los escalares de $w$ como 
\[\lambda^i\equiv w^i\]
Así, los vectores como combinación lineal de otros vectores, los escribiremos como
\[w=w^iv_i\]
Y para las formas, haremos la identificación
\[\mu_i\equiv p_i\]
Así, las formas como combinación lineal de otras formas se escribirán como
\[p=p_if^i\]
\begin{example}
Un ejemplo de ello, será a la hora de identificar un vector en los términos de su base, pues suponiendo un $V$ espacio vectorial sobre el cuerpo $\mathbb{K}$ y cuya base sea $B=\curlybraces{v_1,v_2,\dots,v_n}$, tomando un $u\in V$, lo denotaremos como,
\[u=u^iv_i\]
\end{example}
\begin{example}
    Otro ejemplo será a la hora de identificar una forma en términos de la base dual, pues suponiendo un $V^*$ espacio dual de $V$, cuya base dual es $B^*=\curlybraces{f^1,f^2,\dots,f^n}$, tomando un $q\in V^*$, lo denotaremos como,
    \[q=q_if^i\]
\end{example}
\begin{note}
    En un artículo físico, se identifica directamente el escalar con el vector, es decir,
    \[w^i\equiv w\]
    pues se presupone que existe una base donde $w$ está bien definido. Así, los físicos usaremos de forma indistinguible los vectores y sus componentes respecto de una base fijada.
\end{note}
\subsection{Invariantes} % Main chapter title
\label{cap1-sec2-subsec5} 

Dado que los tensores suelen describirse en términos respecto de ciertas bases, cuando estos términos no dependen de la base empleada, los tensores se llamarán \textbf{invariantes}. O en otras palabras, los tensores que no se transforman frente a un cambio de base, serán los que llamaremos \textbf{invariantes}.\\ \\
Vamos a intentar ilustrar este concepto definiendo un tensor invariante de tipo (1,1), denominado \textit{traza}, que es un invariante conocido de las matrices. Si tenemos un tensor $A=A_j^i\ptensor{e_i}{f^j}$ que definimos como
\[\text{traza de }A=\rm{tr}A=A^i_i\]
siendo la suma de los elementos de la diagonal principal de la matriz $(A^i_j)$. No es a priori evidente que hayamos definido algo que depende únicamente de $A$, ya que los $A_j^i$ dependen no solo de $A$ sino también de la base $\curlybraces{e_i}$. Para mostrar que $\rm{tr} A$ es un número determinado enteramente por $A$ mismo y no por los $e_i$ también, debemos demostrar la invariancia; es decir, si $A$ se expresa en términos de otra base ${\tilde{e}_i}$, entonces la fórmula correspondiente en los nuevos componentes da el mismo número que antes. Así, escribimos $A=\tilde{A}^i_j\ptensor{\tilde{e}_i}{\tilde{f}^j}=A^i_j\ptensor{e_i}{f^j}$ y veremos que $A^i_j=\tilde{A}^i_j$. Usando la misma notación de cambios de base que hemos visto en el apartado anterior, tenemos la ley de transformación siguiente,
\[\tilde{A}^n_m=A^i_ja_m^jb_i^n\]
de lo cual se obtiene
\[\tilde{A}^i_i=A^p_ja^j_ib^i_p=A^p_j\delta^j_p=A^i_i\]
Queda demostrado. Luego, tenemos la proposición,
\begin{proposition}
    La traza de un tensor de tipo (1,1) es un invariante.
\end{proposition}
Para ver que no todas las expresiones en términos de las componentes de un tensor necesariamente serán un invariante, veamos el siguiente ejemplo. 
\begin{example}
    Supongamos $d=2$ y $A=\ptensor{e_1}{e_1}+\ptensor{e_1}{e_2}$, un tensor de tipo (0,2). La expresión de $A_{ii}$ en este caso será $A_{11}+A_{22}$=1+0=1. Ahora consideramos una nueva base dada por $e_1=\tilde{e}_1+\tilde{e}_2$ y $e_2=\tilde{e}_2$, entonces
    \[\begin{array}{rrl}
        A & = & (\tilde{e}_1+\tilde{e}_2)\otimes(\tilde{e}_1+\tilde{e}_2)+(\tilde{e}_1+\tilde{e}_2)\otimes\tilde{e}_2 \\
         & = & \tilde{e}_1\otimes\tilde{e}_1+2\tilde{e}_1\otimes\tilde{e}_2+\tilde{e}_2\otimes\tilde{e}_1+2\tilde{e}_2\otimes\tilde{e}_2
    \end{array}\]
    de la cuál se obtiene que $\tilde{A}_{ii}=\tilde{A}_{11}+\tilde{A}_{22}=1+2=3$. Por tanto es diferente a la base primera, luego no es un invariante.
\end{example}
\subsubsection*{Nota Final}
    Finalmente diremos que un tensor es todo aquel objeto matemático que satisfaga los cambios de base, o en otras palabras: \textit{Un tensor es todo objeto matemático que transforma como un tensor}.
%SECCION 3
\section{Dilatación temporal} % Main chapter title
\label{cap2-sec3} 
%------------------------------------------------------------------------------
La dilatación temporal es una causa directa de los postulados de Einstein. Veámoslo con un esquema,
\begin{multicols}{2}
    \begin{Figura}
        \centering
        \includegraphics[width=0.8\textwidth]{Capitulos/Capitulo2/Seccion3/Lrep.png}
        \captionof{figure}{Espejos en reposo.}
        \label{fig2.1}
    \end{Figura}
    \begin{Figura}
        \centering
        \includegraphics[width=0.8\textwidth]{Capitulos/Capitulo2/Seccion3/Lmov.png}
        \captionof{figure}{Espejos en movimiento.}
        \label{fig2.2}
    \end{Figura}
\end{multicols}
Si nos fijamos en la Figura \ref{fig2.1}, al estar los espejos en reposo, el rayo de luz que sale de la linterna vuelve en un tiempo $\Delta t=\frac{2l_0}{c}$. En cambio, suponiendo que los espejos se mueven a velocidad $\vec{v}$, y que la distancia de los brazos del rayo es $D$, entonces ahora el tiempo que tarda el rayo en ir y volver es $\Delta t'=\frac{2D}{c}$. Usando el Teorema de Pitágoras podemos calcular $D$, tal que
\[D^2=l_0^2+\left(\frac{\Delta t'v}{2}\right)^2\]
Sustituyendo $D$ y $l_0$ de las ecuaciones de $\Delta t$ y $\Delta t'$, tenemos
\[\left(\frac{\Delta t'c}{2}\right)^2=\left(\frac{\delta t'v}{2}\right)^2+\left(\frac{\Delta tc}{2}\right)^2\]
Por tanto, tenemos que el tiempo se dilata de la forma,
\begin{equation}
    \Delta t'=\gamma\Delta t
\end{equation}
y como $\gamma>1$ siempre, entonces $\Delta t'>\Delta t$, por eso se dilata el tiempo.\\ \\
Vemos que en el SRI $S'$ los relojes van más lento que en el SRI $S$, pues si consideramos como reloj el rebote de los fotones en los espejos, entonces en $S$ los fotones van más rápido que los fotones en $S'$.


           \chapter{Espacios topológicos}
\label{ApendiceB}
\lhead{Ap\'endice B. \emph{Espacios topológicos}}

El principal interés de explicar los Espacios Topológicos surge del hecho de que el espacio-tiempo en Relatividad General tiene la estructura de un espacio topológico. En este apéndice recogemos varias definiciones y teoremas clave relativos a los espacios topológicos. 
\begin{definition}
    Un \textit{espacio topológico} $(X, \mathcal{T})$ se trata de un conjunto $X$ con una colección $\mathcal{T}$ de subconjuntos de $X$ que satisface las siguientes propiedades:
    \begin{enumerate}
        \item La unión de un colección arbitraria finita o infinita de subespacios pertenecientes a $\mathcal{T}$, también pertenece a $\mathcal{T}$, es decir, si $O_{\alpha}\in\mathcal{T}$ para todo $\alpha$, entonces $\bigcup\limits_{\alpha}O_{\alpha}\in\mathcal{T}$.
        \item La intersección de un número finito de subespacios de $\mathcal{T}$ pertenece también a $\mathcal{T}$, es decir, si $O_1,\dots O_n\in\mathcal{T}$, entonces $\bigcap\limits_{i=1}^{n}O_i\in\mathcal{T}$.
        \item El conjunto completo $X$ y el conjunto vacío $\emptyset$, pertenecen a $\mathcal{T}$.
    \end{enumerate}
    \label{def-A-0-1}
\end{definition}
\begin{note}
    $\mathcal{T}$ es referido a una \textit{topología} sobre $X$, y los subconjuntos de $X$, que se enumeran en la colección $\mathcal{T}$, se denominan \textit{conjuntos abiertos}.
\end{note}
Cualquier conjunto $X$ se puede convertir fácilmente en un espacio topológico tomando $\mathcal{T}=\curlybraces{\text{todos los subconjuntos de }X}$, denominado \textit{topología discreta}, o tomando $\mathcal{T}=\curlybraces{X,\emptyset}$, denominado \textit{topología indiscreta}.\\
Un ejemplo mucho más interesante es el espacio topológico que se obtiene tomando $\mathcal{T}=\mathbb{R}$, el conjunto de los números reales, y definiendo $\mathcal{T}$ para que esté formado por todos los subconjuntos de $\mathbb{R}$, que puede ser expresado como la unión de intervalos abiertos $(a,b)$. Así, tomando $\mathcal{T}$ de esta forma sobre $\mathbb{R}$, un intervalo abierto es un conjunto abierto; históricamente, este ejemplo es la razón por la que la terminología de 'conjunto abierto' se usa en la discusión de un espacio topológico abstracto.\\ \\
También podemos definir topologías inducidas, pues la definición de espacio topológico nos permite jugar con los subconjuntos de subconjuntos.
\begin{definition}
    Si $(X, \mathcal{T})$ es un espacio topológico y $A$ es un subconjunto cualquiera de $X$, podemos convertir $A$ en un espacio topológico definiendo la topología $\mathcal{S}$ en $A$, que consiste en todos los subconjuntos de $A$ que pueden expresarse como intersecciones de elementos de $\mathcal{T}$ con $A$, es decir, $\mathcal{S} = \{ U \mid U = A \cap O, \, O \in \mathcal{T} \}$. $\mathcal{S}$ se llama la '\textbf{topología inducida}' (o '\textbf{topología relativa}').
\end{definition}
En estos espacios topológicos también podemos definir un producto cartesiano, pues al trabajar con conjuntos está bien definido. De hecho, si $(X_1,\mathcal{T}_1)$ y $(X_2,\mathcal{T}_2)$ son espacios topológicos, entonces podemos introducir el producto cartesiano, \[X_1\times X_2=\curlybraces{(x_1,x_2)\mid x_1\in X_1,x_2\in X_2}\] dentro de un espacio topológico $(X_1\times X_2,\mathcal{T})$, definiendo $\mathcal{T}$ para que esté compuesto por todos los subconjuntos de $X_1\times X_2$, que pueden ser expresados como uniones de la forma $O_1\times O_2$ con $O_1\in\mathcal{T}_1$ y $O_2\in\mathcal{T}_2$. $\mathcal{T}$ se denomina \textbf{\textit{producto topológico}}, y usando esta definición de topología en $\mathbb{R}$, por construcción de topologías producto, podemos definir una topología en $\mathbb{R}^n$ La topología que obtenemos es la misma que se obtendría directamente definiendo $\mathcal{T}$ para que esté formado por todos los subconjuntos de $\mathbb{R}^n$, que pueden ser expresados por uniones de bolas abiertas.\\ \\
También podemos definir la continuidad de las funciones de los conjuntos de los espacios topológicos,
\begin{definition}
    Sean \( (X, \mathcal{T}) \) un espacio topológico con topología \( \mathcal{T} \), y \( (Y, \mathcal{S}) \) un espacio topológico con topología \( \mathcal{S} \). Decimos que una función \( f : X \to Y \) es continua si para todo conjunto abierto \( O \in \mathcal{S} \) (es decir, cualquier conjunto abierto en \( Y \)), la imagen inversa de \( O \) bajo \( f \), denotada como 
\[
f^{-1}[O] = \{ x \in X \mid f(x) \in O \},
\]
es un conjunto abierto en \( X \) (es decir, \( f^{-1}[O] \in \mathcal{T} \)).
\end{definition}
Para funciones de $\mathbb{R}$ en $\mathbb{R}$, es fácil verificarlo usando la definición de topología sobre $\mathbb{R}$, esta definición de continuidad es equivalente a la definición usual $\epsilon-\delta$.
\begin{definition}
    Si $f$ es continua, inyectiva, sobreyectiva y su inversa es continua, entonces $f$ se denomina \textbf{\textit{homeomorfismo}}, y $(X,\mathcal{T})$ y $(Y,\mathcal{S})$ se dice que son homemorfos. Los espacios topológicos homeomorfos tienen las mismas propiedades que los espacios topológicos.
\end{definition}
Antes hemos usado el concepto de 'conjunto abierto', pero también podemos definir los 'conjuntos cerrados', de forma que si $(X,\mathcal{T})$ es un espacio topológico, un subconjunto $C$ de $X$ se dice que es \textit{cerrado} si su complemento $X-C\equiv\curlybraces{x\in X\mid x\notin C}$ es abierto. Así, por ejemplo, un intervalo cerrado $\brackets{a,b}$ de $\mathbb{R}$ (con la topología estándar sobre $\mathbb{R}$) es un conjunto cerrado. A partir de los axiomas de los espacios topológicos, es inmediato ver que la intersección de cualquier colección arbitraria de conjuntos cerrados es cerrada y que la unión de un número finito de conjuntos cerrados es también cerrada. Cabe resaltar que un conjunto puede ser ni abierto y ni cerrado, por ejemplo, el intervalo medio-abierto $[a,b)$ en $\mathbb{R}$; o puede ser abierto y cerrado al mismo tiempo, como son todos los subconjuntos de la topología discreta. De hecho, la posibilidad de tener subconjuntos abiertos y cerrados a la vez da origen a la definición de \textbf{conectividad},
\begin{definition}
    Un espacio topológico $(X,\mathcal{T})$ se dice que es \textit{conexo} (o conectado) si el único subconjunto que es abierto y cerrado al mismo tiempo es el conjunto completo $X$ y el conjunto vacío $\emptyset$.
\end{definition}
Tenemos que $\mathbb{R}^n$, con la topología estándar definida, es conexo.\\ \\
En topología, uno de los conceptos fundamentales asociados a un subconjunto de un espacio es el de su adhesión. Este concepto permite formalizar la idea de los puntos donde el conjunto 'se acumula' dentro del espacio, incluyendo tanto los puntos que pertenecen al conjunto como aquellos que se encuentran arbitrariamente cerca de él. Formalmente, podemos definir la adhesión de la siguiente manera,
\begin{definition}
    Sea $(X,\mathcal{T})$ es un espacio topológico y $A$ es un subconjunto arbitrario de $X$, la \textit{adhesión} (o \textit{cierre}), $\overline{A}$, de $A$ se define como la intersección de todos los conjuntos cerrados que contienen a $A$.
\end{definition}
Claramente, $\overline{A}$ es cerrado, contiene a $A$, y es igual a $A$ si y solo si $A$ es cerrado.
\begin{definition}
    Sea $(X,\mathcal{T})$ es un espacio topológico y $A$ es un subconjunto arbitrario de $X$, el \textit{interior} de $A$ se define como la unión todos los conjuntos abiertos contenidos dentro de $A$.
\end{definition}
Claramente, el interior de $A$ es abierto, está contenido en $A$, y es igual a $A$ si y solo si $A$ es abierto. 
\begin{definition}
    $(X,\mathcal{T})$ es un espacio topológico y $A$ es un subconjunto arbitrario de $X$, la \textit{frontera} de $A$, denotada como $\mathring{A}$, se define como todos los puntos que se encuentran en $\overline{A}$, pero no están en el interior de $A$.
\end{definition}

\begin{definition}
    Sea $(X,\mathcal{T})$ un espacio topológico, un \textit{entorno} de $x\in X$ es cualquier $A\subset X$ tal que $x$ pertenece al interior de $A$. 
\end{definition}
En particular, cualquier conjunto abierto conteniendo $x$ es un entorno de $x$.
\begin{definition}
    Una \textit{base de entornos} en $x$ es una colección de entornos de $x$ tal que todo entorno de \( x \) contiene a algún elemento de esta colección.
\end{definition}
En particular, la colección de todos los conjuntos abiertos conteniendo $x$ es una base de entornos sobre $x$, aunque generalmente hay muchas otras posibilidades para bases de entornos. 
\begin{definition}
Una \textit{base de entornos} de $X$ es una especificación de una base de entornos para cada $x\in X$
\end{definition}
Las topologías suelen definirse especificando una base de entornos. El procedimiento es el siguiente:
\begin{enumerate}
    \item Un entorno de $x$ es cualquier conjunto que contiene a algún entorno base de $x$.

    \item Un conjunto es abierto si es un entorno de cada uno de sus puntos.
\end{enumerate}
Este proceso asegura que todos los conjuntos abiertos se generan a partir de la base de entornos, cumpliendo con las propiedades necesarias para definir una topología en el espacio. También es interesante ver que los conjuntos cerrados, los puntos de adherencia y de frontera pueden definirse directamente en términos de la base de entornos,
\begin{definition}
        Un conjunto \( G \) es \textit{cerrado} si y solo si, para cada punto \( x \) que no pertenece a \( G \), existe una base de entornos de \( x \) que no interseca a \( G \). La \textit{adherencia} de un conjunto \( A \) consiste en aquellos puntos \( x \) tales que cada base de entornos de \( x \) interseca a \( A \). La \textit{frontera} de \( A \) consiste en aquellos puntos \( x \) tales que cada base de entornos de \( x \) interseca tanto a \( A \) como a \( X - A \).
\end{definition}
\subsubsection*{Espacios métricos}
Las bases de entornos, y por tanto las topologías, frecuentemente se definen en términos de una \textit{métrica} o \textit{función distancia}, que es la función $d:\hspace{2mm}X\times X\to\mathbb{R}$, que verifica:
\begin{enumerate}
    \item Para todo $x,y\in X$, $d(x,y)\geq 0$ (positividad).
    \item Si $d(x,y)=0$, entonces $x=y$ (no degeneración).
    \item Para todo $x,y\in X$, $d(x,y)=d(y,x)$ (simetría).
    \item Para todo $x,y,z\in X$, $d(x,y)+d(y,z)\geq d(x,z)$ (desigualdad triangular).
\end{enumerate}
No hay ningún cambio esencial si también añadimos $+\infty$ como un valor de $d$. Un conjunto con función métrica se denomina \textbf{\textit{espacio métrico}}.
\begin{definition}
    La \textit{bola abierta} con centro en $x$ y radio $r>0$ con respecto a $d$ se define como 
    \[
    B(x,r)=\curlybraces{y\mid d(x,y)<r}
    \]
\end{definition}
Entonces, puede demostrarse que cada bola abierta serviría como base de entornos para una topología $X$, la \textit{topología métrica} de $d$.
\\ \\
De forma más general: \textbf{\textit{Para cualquier espacio métrico, la colección de todos los subconjuntos que pueden expresarse como uniones de bolas abiertas define una topología}}.\\ \\
\subsubsection*{Espacio Hausdorff}
Vamos a introducir los espacios Hausdorff, pues nos permiten definir las variedades y son muy útiles en Geometría Diferencial. 
\begin{definition}
    Un espacio topológico $(X,\mathcal{T})$ se dice que es \textit{Hausdorff} si para cada par de puntos distintos $p,q\in X$, $p\neq q$, existen conjuntos abiertos $O_p$, $O_q\in\mathcal{T}$ tal que $p\in O_p$, $q\in O_q$, y $O_p\cap O_q=\emptyset$.
\end{definition}
Es fácil comprobar que $\mathbb{R}^n$, con la topología estándar, es Hausdorff. También tenemos que una topología métrica siempre es Hausdorff.
\subsubsection*{Compacidad}
Una de las nociones más importantes en topología es es la de la \textit{compacidad}, que se define como,
\begin{definition}
    Sea $(X,\mathcal{T})$ un espacio topológico y $A$ un subconjunto de $X$, una colección $\curlybraces{O_{\alpha}}$ de conjuntos abiertos se denomina \textit{recubrimiento abierto} de $A$ si la unión de estos conjuntos contiene a $A$, i.e. $A\subseteq\bigcup\limits_{\alpha}O_{\alpha}$. Una subcolección de estos conjuntos $\curlybraces{O_{\alpha}}$ que también recubren $A$ es referida como un \textit{sub-recubrimiento}. El conjunto $A$ se denomina \textit{compacto} si cada recubrimiento abierto de $A$ tiene un sub-recubrimiento finito (i.e., un sub-recubrimiento que consiste solo en un número finito de conjuntos).
\end{definition}
Así, por ejemplo, en cualquier espacio topológico, un conjunto formado por un solo punto, es compacto. Por otro lado, el intervalo abierto $(0,1)$ en $\mathbb{R}$ (con la topología estándar) no es compacto ya que los conjuntos $O_n=(1/n,1)$ para $n=2,3,\dots$, ceden un recubrimiento abierto de $(0,1)$ el cuál admite sub-recubrimientos no finitos.\\ \\
Los siguientes teoremas describen las implicaciones de la compacidad y muestran la utilidad de esta noción. Las demostraciones pueden encontrarse en cualquier texto de topología (e.g., Hocking and Young 1961; Kelley 1955).\\ \\
Quizás, el teorema más importante relativo a subconjuntos compactos de $\mathbb{R}$ es el Teorema de Heine-Borel,
\begin{theorem}[\textbf{Heine-Borel}]
    Un intervalo cerrado $[a,b]$ de números reales es compacto (con la topología estándar sobre $\mathbb{R}$).
\end{theorem}
La relación general entre conjuntos compactos y cerrados se describe con los siguientes dos teoremas, las demostraciones son directas,
\begin{theorem}
    Sea $(X,\mathcal{T})$ un espacio topológico de Hausdorff y sea $A\subset X$ compacto. Entonces $A$ es cerrado.
\end{theorem}
\begin{theorem}
    Sea $(X,\mathcal{T})$ compacto y $A\subset X$ cerrado. Entonces $A$ es compacto.
\end{theorem}
Combinando los tres teoremas anteriores, llegamos al siguiente enunciado importante sobre la compacidad de subconjuntos de $\mathbb{R}$,
\begin{theorem}
    Un subconjuntos $A$ de los números reales es compacto si y solo si es cerrado y acotado.
\end{theorem}
Se demuestra fácilmente que la propiedad de compacidad se conserva en mapas continuos. Tenemos,
\begin{theorem}
    Una función continua de un espacio topológico compacto en $\mathbb{R}$ es acotada y alcanza sus valores máximo y mínimo.
\end{theorem}
El siguiente teorema proporciona una extensión inmediata de los resultados de compacidad de $\mathbb{R}$ para $\mathbb{R}^n$.
\begin{theorem}[\textbf{Teorema de Tychonoff}]
    Sea $(X_1,\mathcal{T}_1)$ y $(X_2,\mathcal{T}_2)$ espacios topológicos compactos. Entonces, el producto cartesiano $X_1\times X_2$ es compacto en el producto topológico.
\end{theorem}
\begin{theorem}
    El teorema anterior puede generalizarse para aplicar el producto de infinitos productos topológicos, pero el axioma de elección es necesario para esta generalización.
\end{theorem}
Un corolario de este teorema y el anterior es
\begin{corollary}
    Un subconjunto, $A$, de $\mathbb{R}^n$ es compacto si y solo si es cerrado y acotado.
\end{corollary}
Así, por ejemplo, la esfera $n$-dimensional $S^n$ (definida como el conjunto de puntos en $\mathbb{R}^{n+1}$ satisfaciendo $x_1^2+\dots+x_n{n+1}^2=1$) en la topología inducida es compacta, así que es fácil ver que es un conjunto cerrado y acotado de $\mathbb{R}^{n+1}$.
\subsubsection*{Convergencia de sucesiones}
Otra noción que necesitaremos es la de convergencia de sucesiones. 
\begin{definition}
Una sucesión $\curlybraces{x_n}$ de puntos en un espacio topológico $(X,\mathcal{T})$ se dice que \textit{converge} a un punto $x$ si en cualquier entorno abierto $O$ de $x$ (i.e., un conjunto abierto $O$ que contiene $x$), hay un $N$ tal que $x_n\in O$, para todo $n>N$.
\end{definition}
El punto $x$ se dice que es el \textit{límite} de la sucesión. Es fácil comprobar que para $\mathbb{R}$ (con la topología estándar) esto lleva a la definición usual de convergencia.
\begin{definition}
    Un punto $y\in X$ se dice que es un \textit{punto de acumulación} (o \textit{punto límite}) de $\curlybraces{x_n}$ si cada entorno abierto de $y$ contiene infinitos números de la sucesión.
\end{definition}
Sin embargo, en un espacio topológico general, si $y$ es un punto de acumulación de $\curlybraces{x_n}$, podría no ser posible encontrar una sucesión $\curlybraces{y_n}$ de puntos de la sucesión $\curlybraces{x_n}$ tal que $\curlybraces{y_n}$ converge hacia $y$. No obstante, el sentido de la convergencia de sucesiones hacia $y$ siempre será posible si $(X,\mathcal{T})$ es \textit{primero numerable}, esto es, si para cada $p\in X$ hay una colección numerable $\curlybraces{O_n}$ de conjuntos abiertos tal que cada entorno abierto, $O$, de $p$ contiene al menos un miembro de esta colección. Para $\mathbb{R}^n$, las bolas abiertas con radio racional centradas en puntos con coordenadas racionales, componen una colección numerable de conjuntos abiertos.\\ \\
Una relación importante entre compacidad y convergencia de sucesiones está expresada en el Teorema de Bolzano-Weiestrass,
\begin{theorem}[\textbf{Teorema de Bolzano-Weiestrass}]
    Sea $(X,\mathcal{T})$ un espacio topológico y sea $A\subset X$. Si $A$ es compacto, entonces cada sucesión $\curlybraces{x_n}$ de puntos de $A$ tiene un punto de acumulación en $A$. Inversamente, si $(X,\mathbb{T})$ es segundo numerable y cada sucesión en $A$ tiene un punto de acumulación en $A$, entonces $A$ es compacto. Así, en particular, si $(X,\mathcal{T})$ es segundo numerable, $A$ es compacto si y solo si cada sucesión en $A$ tiene una convergencia de sucesiones cuyo límite está en $A$.
\end{theorem}
\subsubsection*{Para-compacidad}
Finalmente, definimos la noción de \textit{para-compacidad}, una propiedad que las variedades deben satisfacer para evitar que sean "demasiado grandes".
\begin{definition}
    -Sea $(X,\mathcal{T})$ un espacio topológico y sea $\curlybraces{O_{\alpha}}$ un recubrimiento abierto de $X$. Un recubrimiento abierto $\curlybraces{V_{\beta}}$ se dice que es un \textit{refinamiento} de $\curlybraces{O_{\alpha}}$ si para cada $V_{\beta}$ existe un $O_{\alpha}$ tal que $V_{\beta}\subset O_{\alpha}$.\\
    -El recubrimiento $\curlybraces{V_{\beta}}$ se dice que es \textit{localmente finito} si cada $x\in X$ tiene un entorno abierto $W$ tal que solo un número finito de $V_{\beta}$ satisfacen que $W\cap V_{\beta}\neq\emptyset$.\\
    -El espacio topológico $(X,\mathcal{T})$ se dice que es \textit{para-compacto} si cada recubrimiento abierto $\curlybraces{O_{\alpha}}$ de $X$ tiene un recubrimiento localmente finito $\curlybraces{V_{\beta}}$.
\end{definition}
%\subsubsection*{Variedades}
Los conceptos de variedades se explican en el Apéndice B.\\ \\
No es difícil ver que (verlo en e.g., Hocking and Young 1961) que cualquier espacio topológico de Hausdorff que sea localmente compacto (i.e., tal que cada punto tiene un entorno abierto con adherencia compacta) y que pueda ser expresado como una unión numerable de subconjuntos, es para-compacto. Así, $\mathbb{R}^n$, $S^m$ y sus productos verifican fácilmente ser para-compactos. En efecto, no es fácil construir ejemplos de espacios topológicos que satisfagan todos los requisitos de una variedad pero que no sean para-compactos; la 'recta larga' (ver Hocking and Young 1961) es quizás el ejemplo más simple, aunque para definirla se requiere el axioma de elección.\\ \\
           \chapter{Variedades}
\label{ApendiceC}
\lhead{Ap\'endice C. \emph{Variedades}}
\section{Definición de Variedad}
Una variedad, aproximadamente, es un espacio topológico en el cuál algunos entornos de cada punto admiten un sistema de coordenadas, siendo en funciones coordenadas reales en los puntos del entorno, que determina la posición de los puntos y la topología de ese entorno; esto es, el espacio es localmente cartesiano. Además, el paso de un sistema de coordenadas a otro es suave en la región superpuesta, por tanto, el significado de curva, función o mapa 'diferenciable' es consistente cuando se refiere a cualquier sistema. Una definición detallada se dará más adelante.\\ \\
Los modelos matemáticos que se usan para describir sistemas físicos, usan las variedades como objeto básico de estudio, sobre el cual se puede definir una \textit{estructura} adicional para obtener cualquier sistema en cuestión. El concepto generaliza e incluye los casos especiales de línea cartesiana, plano, espacio, y las superficies que se estudian en cálculo avanzado. La teoría de estos espacios se generaliza con las variedades, incluyendo las ideas de funciones diferenciables, curvas suaves, vectores tangentes, y campos vectoriales. Sin embargo, las nociones de distancia entre puntos y líneas rectas (o caminos más cortos) no son parte de la idea de una variedad, pero surgen como consecuencias de estructuras adicionales, las cuales pueden ser o no ser asumidas y en cualquier caso, no es único.\\ \\
Una variedad tiene dimensión tal que, al igual que en los modelos físicos, su valor será igual al número de grados de libertad. Nos limitaremos a estudiar las variedades de dimensión finita.\\ \\
Algunas definiciones preliminares podrán facilitar la definición de variedad:
\begin{definition}[\textbf{Carta}]
    Sea $(X,\mathcal{T})$ un espacio topológico, siendo $\mathcal{T}$ una topología sobre el conjunto $X$, decimos que una \textbf{\textit{carta}} en $p\in X$ es una función $\mu:\hspace{1mm}O\to\mathbb{R}^d$, donde $O$ es un conjunto abierto que contiene $p$ y $\mu$ es un homeomorfismo sobre un conjunto abierto de $\mathbb{R}^d$.
\end{definition}
-La \textit{dimensión} de una carta $\mu:\hspace{1mm}O\to\mathbb{R}^d$ es $d$.\\
-Las \textit{funciones de coordenadas} de una carta son las funciones de valor real sobre $O$, dadas por las entradas de valores de $\mu$; esto es, son las funciones $x^i=u^i\circ \mu:\hspace{1mm}O\to\mathbb{R}$, donde $u^i:\hspace{1mm}\mathbb{R}^d\to\mathbb{R}$ son las coordenadas estándar sobre $\mathbb{R}^d$.\\
-Los $u^i$ están definidos por $u^i(a^1,\dots,a^d)=a^i$. Los superíndices no son potencias, claro está, pero son simplemente la indexación tensorial habitual de coordenadas. Si se necesitan potencias, usaremos paréntesis extra, $(x)^2$ en lugar de $x^2$ para el cuadrado de $x$, pero normalmente el contexto tendrá información suficiente para saber la distinción y quitar el uso de estos paréntesis. Por tanto, para cada $q\in O$, tenemos que $\mu q=(x^1q,\dots,x^dq)$, y entonces, podremos escribir también $\mu=(x^1,\dots,x^d)$. En otra terminología, podemos denominar $\mu$ como un \textit{mapa coordenado}, $O$ como un \textit{entorno coordenado}, y la colección $(x^1,\dots,x^d)$ como \textit{coordenadas} o \textit{sistema coordenado sobre }$p$.\\ \\
Restringiremos los símbolos '$u^i$'a este uso como coordenadas estándar de $\mathbb{R}^d$. Para $\mathbb{R}^2$ y $\mathbb{R}^3$ también usaremos $x$, $y$, $z$ como coordenadas, como es habitual, exceptuando que las usaremos como funciones.
\begin{definition}
    -Una función de valor real $\funct{f}{V}{\mathbb{R}}$ es $\mathscr{C}^{\infty}$ (continua a orden $\infty$) si $V$ es un conjunto abierto en $\mathbb{R}^d$ y $f$ tiene derivadas parciales continuas para todos los órdenes y tipos (cruzadas o no).\\
    -Una función $\funct{\varphi}{V}{\mathbb{R}^e}$ es un \textbf{\textit{mapa}} $\mathscr{C}^{\infty}$ si las componentes de $\funct{u^i\circ\varphi}{V}{\mathbb{R}}$ son $\mathscr{C}^{\infty}$, $i=1,\dots e$.
\end{definition}
De forma más general, $\varphi$ es $\mathscr{C}^k$, siendo $k$ un entero no negativo, si todas las derivadas parciales, hasta e incluyendo estas de orden $k$, existen y son continuas. $\mathscr{C}^0$ implica simplemente continuidad. Un mapa $\varphi$ es \textit{analítico} si $u^i\circ\varphi$ es real-analítico, esto es, puede ser expresado en un entorno para cada punto por medio de una serie de potencias convergente en coordenadas cartesianas, teniendo su origen en el punto. Los mapas analíticos son $\mathscr{C}^{\infty}$, pero no al revés.
\begin{example}
    Sea $z=x+iv$, una variable compleja, definimos $u(x,y)$ como $u+iv=e^{-1/z^4}$, $u(0,0)=0$. Entonces, vemos que $u$ no es $\mathscr{C}^{\infty}$, y de hecho, no es continuo en el $(0,0)$, pero las derivadas parciales de $u$ existen en todos los órdenes y en todos los puntos, incluyendo el $(0,0)$. De este modo, los requisitos para continuidad en la definición de $\mathscr{C}^{\infty}$ no son triviales. Para funciones de una variable, es cierto que las funciones diferenciables son continuas.
\end{example}
Dos cartas $\funct{\mu}{U}{\mathbb{R}^d}$ y $\funct{\tau}{V}{\mathbb{R}^e}$ en un espacio topológico $(X,\mathcal{T})$ son $\mathscr{C}^{\infty}$-\textit{relacionadas} si $d=e$ y cualquier $U\cap V=\emptyset$ o $\mu\circ\tau^{-1}$ y $\tau\circ\mu^{-1}$ son mapas $\mathscr{C}^{\infty}$. El dominio de $\mu\circ\tau^{-1}$ es $\tau(U\cap V)$, un conjunto abierto es $\mathbb{R}^d$. Ver Figura \ref{Fig1-ApB}
\begin{Figura}
    \centering
    \includegraphics[width=0.8\textwidth]{Apendices/Imagenes/Fig1-ApB.png}
    \label{Fig1-ApB}
    \captionof{figure}{Interpretación de ambas cartas. (donde $R^d\equiv\mathbb{R}^d$) [\ref{BFig1-ApB}]}
\end{Figura}
Otros grados de relación se definen reemplazando '$\mathscr{C}^{\infty}$' por '$\mathscr{C}^k$' o por 'analítico'. Dos cartas de la misma dimensión siempre son $\mathscr{C}^0$-relacionadas porque los mapas de coordenadas son continuos.
\begin{definition}
    Una \textit{\textbf{variedad topológica}} $\mathscr{C}^0$ es un espacio separable de Hausdorff tal que haya una carta de dimensión $d$ en cada punto.     
\end{definition}
La \textit{dimensión} de la variedad es la misma que la de las cartas. De esta forma, existe una colección de cartas $\curlybraces{\funct{\mu_{\alpha}}{U_{\alpha}}{\mathbb{R}^{\alpha}}\mid\alpha\in I}$, tal que $\curlybraces{U_{\alpha}\mid \alpha\in I}$ es una cubierta del espacio. Dicha colección se denomina un \textit{atlas}. Un atlas $\mathscr{C}^{\infty}$ es aquel para el cual cada par de cartas es $\mathscr{C}^{\infty}$-relacionada. Una carta es \textit{admisible} para un atlas $\mathscr{C}^{\infty}$ si es $\mathscr{C}^{\infty}$-relacionado para cada carta en el atlas. En particular, los miembros de un atlas $\mathscr{C}^{\infty}$ son admisibles.
\begin{definition}
    Una \textbf{\textit{variedad}} $\mathscr{C}^{\infty}$ es una variedad topológica junto con todas las cartas admisibles con algún atlas $\mathscr{C}^{\infty}$.
\end{definition}
Nos referiremos a las 'variedades $\mathscr{C}^{\infty}$' como 'variedades'. La razón de incluir todas las cartas admisibles en vez de simplemente aquellas que están en algún atlas dado es para transmitir la idea que ningún sistema de coordenadas particular es preferible sobre otro y también para resolver el problema lógico de decir qué \textit{es} una variedad. El origen de este problema lógico es el hecho de que dos atlas diferentes pueden tener la misma colección de cartas admisibles, en cuyo caso, nos gustaría decir que tenemos una sola variedad, no dos variedades diferentes para cada atlas. Por otro lado, casi invariablemente ocurre que una variedad es especificada dando solo un atlas, no toda la colección de cartas admisibles.
\begin{definition}
    Las variedades $\mathscr{C}^{\infty}$ y las variedades analíticas reales se definen reemplazando '$\mathbb{C}^{\infty}$' por '$\mathscr{C}^k$' y 'analítica', respectivamente a lo largo de la cadena de definiciones anterior.
\end{definition}
Debe quedar claro que una variedad $\mathscr{C}^{\infty}$ se convierte en una variedad $\mathscr{C}^k$ simplemente ampliando la colección de cartas admisibles para incluir todas las relacionadas con $\mathscr{C}^k$, y, de forma similar, una variedad analítica real se convierte en una variedad $\mathscr{C}^{\infty}$. En cambio, una variedad $\mathscr{C}^1$ se convierte en una variedad analítica real (y por lo tanto $\mathscr{C}^{\infty}$), de muchas formas, descartando una colección adecuada de cartas admisibles de $\mathscr{C}^1$ para dejar solo las cartas que están mutuamente relacionados analíticamente, pero este resultado no es del todo obvio, siendo un teorema muy difícil de Whitney. Se sabe que una variedad $\mathscr{C}^0$ puede no llegar a convertirse en una variedad $\mathscr{C}^1$, y es aún más difícil de demostrar.
\begin{remark}
    En la definición de sistema de coordenadas requerimos que el entorno coordenado y el rango en $\mathbb{R}^d$ sean conjuntos abiertos. Esto es contrario al uso popular, o al menos más específico que el uso de coordenadas curvilíneas en cálculo avanzado.
\end{remark}
 Por ejemplo, las coordenadas esféricas se utilizan incluso a lo largo de puntos del eje $z$ donde ni siquiera son 1 a 1. Las razones de la restricción a conjuntos abiertos son que fuerzan uniformidad en la estructura local que simplifica el análisis en variedades (no hay 'puntos de frontera') e, incluso si la uniformidad local fuera forzada en otros aspectos, esto evita el problema de explicar qué entendemos por diferenciabilidad en puntos de frontera del entorno coordenado; esto es, las derivadas laterales no necesitan ser mencionadas. Por otro lado, en aplicaciones, con frecuencia surgen problemas de valores de frontera, cuya configuración es una \textit{variedad con borde}. Estos espacios son más generales que las variedades y la generalidad extra surge de permitir una \textit{variedad de borde} en una dimensión inferior. Los puntos de la variedad de borde tienen un entorno coordenado en la variedad de borde que está unido a un entorno coordenado del interior de manera muy similar a como una cara de un cubo está unida al interior.
 \section{Ejemplos de Variedades}
 \subsection{Espacio Cartesiano}
 Definimos una estructura de variedad en $\mathbb{R}^d$, de la manera más obvia, tomando como un atlas la única carta $\mathbb{I}:\mathbb{R}^d\to\mathbb{R}^d$, el mapa identidad. Las funciones coordenadas de esta carta son por lo tanto las coordenadas estándar (cartesianas) $u^i$. Cuando hablamos de $\mathbb{R}^d$ como una variedad, nuestra intención es esta estructura estándar, a menos que se indique lo contrario.\\ \\
 Un mapa coordenado admisible $\mathscr{C}^{\infty}$ sobre $\mathbb{R}^4$ es un mapa $\mathscr{C}^{\infty}$ 1 a 1 $\mu: U\to\mathbb{R}^d$, donde $U$ es un conjunto abierto y el determinante del jacobiano es $|\partial x^i/\partial u^i|\neq0$, donde $x^i=u^i\circ\mu$ son las funciones coordenadas. Que el determinante del jacobiano no sea nulo solo es otra forma de requerir que el mapa $\mu^{-1}$ sea $\mathscr{C}^{\infty}$.\\ \\
 Si $f^i$, $i=1,2,\dots,d$, son funciones reales $\mathscr{C}^{\infty}$ en algún conjunto abierto de $\mathbb{R}^d$ y para algún $p\in\mathbb{R}^d$ tenemos $|\partial f^i/\partial u^i|\neq0$, entonces el teorema de la \textit{función inversa} establece que hay un entorno $U$ de $p$ y un entorno $V$ de $(f^1p,f^2p,\dots,f^dp)$ tal que el mapa $\mu=(f^1,f^2,\dots,f^d)$ lleva $U$ a $V$, es 1 a 1, y tiene una inversa $\mathscr{C}^{\infty}$. Esto proporciona un medio eficaz de obtener coordenadas admisibles. En particular, coordenadas polares, coordenadas cilíndricas, coordenadas esféricas, y cualquier otras coordenadas curvilíneas personalizadas son coordenadas admisibles para $\mathbb{R}^2 $ y $\mathbb{R}^3$ siempre que estén adecuadamente restringidas para ser 1 a 1 y tener un determinante jacobiano distinto de cero.
 \subsection{Subvariedad Abierta}
 Si $M$ es una variedad y $N$ es cualquier conjunto abierto de $M$, entonces $N$ hereda una estructura de variedad restringiendo la topología y mapas coordenados de $M$ a $N$. Llamamos a $N$ una \textit{subvariedad abierta} de $M$. En particular, cualquier subconjunto de $\mathbb{R}^d$ es una variedad de dimensión $d$.
 \subsection{Producto de Variedades}
Si $M$ y $N$ son variedades de dimensión $d$ y $e$ respectivamente, entonces a $M\times N$ se le da una estructura de variedad tomando el producto topológico como su topología (básicamente, los entornos son productos de los entornos de $M$ y $N$) y como atlas se toma el producto de cartas de los atlas de $M$ y $N$. Si $\mu: U\to\mathbb{R}^d$ es una carta sobre $M$, y $\varphi: V\to\mathbb{R}^e$ es una carta sobre $N$, su producto es $(\mu,\varphi): U\times V\to\mathbb{R}^{d+e}$, que se define por $(\mu,\varphi)(m,n)=(\mu m,\varphi,n)$. Si $x^i$ son las funciones coordenadas de $\mu$ e $y^i$ son las funciones coordenadas de $\varphi$, entonces las coordenadas de $(m,n)$ en el producto de cartas son $(x^1m,\dots,x^dm,y^1n,\dots,y^en)$. Así, si $p: M\times N\to M$ y $q: M\times N\to N$ son las proyecciones, $p(m,n)=m$, $q(m,n)=n$, las funciones coordenadas sobre $U\times V$ son $z^1=x^1\circ p,\dots,z^d=x^d\circ p,z^{d+1}=y^1\circ q,\dots,z^{d+e}=y^e\circ q$.\\ \\
Esta operación de producto puede ser iterada evidentemente, y podemos tomar diferentes copias de la misma variedad como elementos. Así, incluso podemos descomponer una variedad $\mathbb{R}^d=\mathbb{R}\overset{d-veces}{\times}\mathbb{R}$ ($d$-elementos). Es fácil ver que un círculo $S^1$ (la curva) es una variedad unidimensional. Dibujando $S^1$ como una parte de $\mathbb{R}^2$, vemos que un cilindro (la superficie) es la variedad $S^1\times\mathbb{R}$ y puede dibujarse en $\mathbb{R}^3=\mathbb{R}^2\times\mathbb{R}$.\\ \\
Podemos considerar $S^1\times S^1$ como una unión, $\curlybraces{\curlybraces{p}\times S^1|p\in S^1}$, de círculos $\curlybraces{p}\times S^1$, para cada $p\in S^1$. Ahora, si dibujamos el primer elemento como si estuviera en el plano $XY$ de $\mathbb{R}^3$, satisfaciendo las ecuaciones $x^2+y^2=1$, $z=0$, y para cada $p$ en el primer elemento, representamos $\curlybraces{p}\times S^1$ como un círculo más pequeño con centro en $p$ y diámetros perpendiculares al primer círculo en $p$, entonces la unión $S^1\times S^1$ es la superficie de revolución del círculo pequeño sobre el eje $Z$, un toroide (Ver Figura \ref{Fig2-ApB}).
\begin{multicols}{2}
\begin{Figura}
    \centering
    \includegraphics[width=1\textwidth]{toroide.png}
    \captionof{figure}{Representación de un toroide. [\ref{BFig2-ApB}]}
    \label{Fig2-ApB}
\end{Figura}
\begin{Figura}
    \centering
    \includegraphics[width=0.7\linewidth]{toroide2.png}
    \captionof{figure}{Portada de \cite{TensorAnalysisOnManifolds}}
    \label{fig:enter-label}
\end{Figura}
\end{multicols}
No es difícil ver que la topología inducida de $\mathbb{R}^3$ en el toroide es el producto topológico.\\ \\
El toroide es la variedad subyacente que modela el conjunto de posiciones (el \textit{espacio de configuraciones}) de un péndulo doble. Estamos pensando en un sistema mecánico que consiste en dos varillas, la primera que es libre de rotar en un plano sobre un eje fijo, y la segunda que rota sobre un eje en el plano que se fija relativo a la primera varilla, usualmente, pero no necesariamente es el plano de la primera varilla. Los ángulos que estas varillas hacen con un eje de coordenadas en sus planos pueden ser identificados con los ángulos $u,v$ que se encuentran en la parametrización del toroide dado a continuación, dando una correspondencia 1 a 1 entre las posiciones del péndulo doble y del toroide. La articulación debe estar fija de modo que cada varilla es libre de hacer un giro completo sobre su eje, o solo una parte del toroide en este modelo. De hecho, si se bloquea la segunda varilla sobre el eje de la primera, de forma que $v$ se restringe a $0<\epsilon<v<2\pi$, entonces el modelo es de un cilindro en vez de un toroide.\\ \\
Añadiendo más varillas obtenemos sistemas físicos para los cuáles, el modelo es el producto de las copias de $S^1$. Si la articulación está fija de forma que la varilla es libre de moverse en el espacio en vez de en un plano, entonces serán necesarios algunos elementos de $S^2$. Finalmente, si un extremo de la primera varilla no está fijo del todo, pero permite moverse libremente en el espacio (o en un plano), entonces se necesitarán un elemento de $\mathbb{R}^3$ (o de $\mathbb{R}^2$).\\ \\
De forma más general, si un sistema físico está compuesto por dos sistemas, para los cuales se pueden asumir todas sus posiciones independientemente del otro, entonces el sistema compuesto tiene como sus variedades de posición el producto de las variedades de posición de las dos componentes del sistema. Esto es así a pesar de que hay alguna ligadura dinámica (por ejemplo, gravitacional o elástica) entre las componentes.
\\ \\
Para ver más ejemplos ver la referencia \cite[Chapter 1, pages 26-35]{TensorAnalysisOnManifolds}
\section{Mapas diferenciables}
Si $F:M\to N$, donde $M$ y $N$ son variedades $\mathscr{C}^{\infty}$, entonces llamamos a $F$ un mapa $\mathscr{C}^{\infty}$ si la expresión de coordenadas para $F$ consiste en mapas $\mathscr{C}^{\infty}$ en espacios cartesianos. Ahora elaboramos esta afirmación en una definición completa, en particular dejando claro qué significa 'expresión de coordenadas'.
\begin{definition}
    Sea $\mu_1:U\to\mathbb{R}^d$ y $\mu_2:V\to\mathbb{R}^e$ cartas $\mathscr{C}^{\infty}$ sobre $M$ y $N$ variedades $\mathscr{C}^{\infty}$, de forma que $U$ y $V$ son subconjuntos abiertos de $M$ y $N$ respectivamente. Asumimos que $F:M\to N$ es un mapa continuo, tal que $W=F^{-1}V$ es un subconjunto abierto de $M$, ver Figura \ref{Fig3-ApB}.
    \begin{Figura}
        \centering
        \includegraphics[width=0.8\textwidth]{coordinateexpression.png}
        \captionof{figure}{Representación de los subconjuntos. [\ref{BFig3-ApB}]}
        \label{Fig3-ApB}
    \end{Figura}
Sea $W_1=\mu_1 W$, tal que $W_1$ es un conjunto abierto en $\mathbb{R}^d$. La \textit{expresión de coordenadas }$\mu_1-\mu_2$ \textit{para }$F$ es el mapa $\mu_2\circ F\circ\mu_1^{-1}:W_1\to\mathbb{R}^e$. El mapa $F$ es $\mathscr{C}^{\infty}$ si todas las expresiones de coordenadas, para todas las cartas admisibles $\mu_1,\mu_2$, son mapas cartesianos $\mathscr{C}^{\infty}$.
\end{definition}
\begin{proposition}
    Un mapa $F:M\to N$ es $\mathscr{C}^{\infty}$ si las expresiones de coordenadas $\mu_{\alpha}-\mu_{\beta}$ de $F$ son $\mathscr{C}^{\infty}$ para aquellos $\mu_{\alpha}$ en algún atlas de $M$ y aquellos $\mu_{\beta}$ en algún atlas de $N$.
\end{proposition}
\begin{proof}
    Sean $\curlybraces{\mu_{\alpha}:U_{\alpha}\to\mathbb{R}^d|\alpha\in I}$ y $\curlybraces{\mu_{\beta}:V_{\beta}\to\mathbb{R}^e|\beta\in J}$ los atlas de $M$ y $N$, respectivamente, tal que para cada $\alpha\in I$, $\beta\in J$, $\mu_{\beta}\circ F\circ\mu_{\alpha}^{-1}$ es un mapa $\mathscr{C}^{\infty}$. Supongamos que $\mu_1$, $\mu_2$ son cualquier otras cartas como en la definición, así $\mu_2\circ F\circ\mu_1^{-1}:W_1\to\mathbb{R}^e$. Debemos demostrar que esto es $\mathscr{C}^{\infty}$, pero como ser $\mathscr{C}^{\infty}$ es una propiedad local, basta con demostrarlo en un entorno de cada punto de $W_1$. Si $m_1\in W_1$, entonces hay un $\alpha\in I$ y un $\beta\in J$ tal que $\mu_1^{-1}m_1=m\in U_{\alpha}$ y $n=Fm\in V_{\beta}$. Por hipótesis, $\mu_{\beta}\circ F\circ\mu_{\alpha}^{-1}$ es un mapa cartesiano $\mathscr{C}^{\infty}$. Pero $\mu_1$ y $\mu_2$ están relacionadas de forma $\mathscr{C}^{\infty}$ con $\mu_{\alpha}$ y $\mu_{\beta}$, respectivamente. Por lo tanto, $\mu_{\beta}\circ\mu_{\alpha}^{-1}$ está definida y es $\mathscr{C}^{\infty}$ en algún entorno de $m_1$, y $\mu_2\circ\mu_1^{-1}$ está definida y es $\mathscr{C}^{\infty}$ en algún entorno de $n_{\beta}=\mu_{\beta} n$.\\
    La composición de mapas cartesianos $\mathscr{C}^{\infty}$ es $\mathscr{C}^{\infty}$, por tanto $\mu_2\circ \mu_{\beta}^{-1}\circ\mu_{\beta}\circ F\circ\mu_{\alpha}^{-1}\circ\mu_{\alpha}\circ\mu_1^{-1}$ es un mapa  $\mathscr{C}^{\infty}$. Sin embargo, está definido en algún entorno de $m_1$ y coincide con la restricción de $\mu_2\circ F\circ\mu_1^{-1}$ en ese entorno, por tanto $\mu_2\circ F\circ\mu_1^{-1}$ es $\mathscr{C}^{\infty}$ en un entorno de $m_1$.
\end{proof}
En la práctica, para verificar que los mapas son $\mathscr{C}^{\infty}$ se debe hacer mostrando que las componentes individuales de las expresiones de coordenadas tienen derivadas parciales continuas en todos los órdenes. Estas componentes son funciones $u^i\circ\mu_2\circ F\circ\mu_1^{-1}=f^i$, $i=1,2,\dots,e$, que son funciones reales con $d$ variables reales definidas en un subconjunto abierto de $W_1$ o $\mathbb{R}^d$.\\ \\
Si tomamos $y^i=u^i\circ\mu_2$ las funciones de coordenadas de $\mu_2$ y $x^j=u^j\circ\mu_1$, las de $\mu_1$, entonces tenemos que $y^i\circ F\circ\mu_1^{-1}=f^i$, o $y^i\circ F=f^i\circ \mu_1$. Aplicando esto a $m\in W$, tenemos
\[y^iFm=f^i\mu_1m=f^i(x^1m,x^2m,\dots,x^dm)\]
Suele escribirse esto como una ecuación entre funciones de la forma
\begin{equation}
    y^i=f^i(x^1,x^2,\dots,x^d)
\end{equation}
pero como esto no nos indica el papel del mapa $F$ por sí mismo, preferimos usar la versión más precisa, que es
\begin{equation}
    y^i\circ F=f^i(x^1,x^2,\dots,x^d)
\end{equation}
Estas ecuaciones también se llaman \textit{expresión de coordenadas de $F$}.
\\ \\
En particular, podemos considerar el caso $N=\mathbb{R}$ de funciones reales evaluadas sobre $M$. Es interesante que las funciones $\mathscr{C}^{\infty}$ necesitan ser definidas directamente solo en este caso, y la definición general de un mapa $\mathscr{C}^{\infty}$ se sigue con la siguiente proposición.
\begin{proposition}
    Si $F:M\to N$, entonces $F$ es $\mathscr{C}^{\infty}$ si para cada función real $\mathscr{C}^{\infty}$ $y:V\to\mathbb{R}$, donde $V$ es una subvariedad abierta de $N$, $y\circ F$ es una función real $\mathscr{C}^{\infty}$ sobre la subvariedad abierta $F^{-1}V$ de $M$.
\end{proposition}
\begin{proof}
    Esto se deduce de manera trivial al tomar, sucesivamente, como $y$ las funciones coordenadas $y^i$ definidas en $V\subset N$.
\end{proof}
Un \textbf{difeomorfismo} de $M$ sobre $N$ es una función $\mathscr{C}^{\infty}$ biyectiva, $F:M\to N$, tal que el mapa inverso $F^{-1}:N\to M$ también es $\mathscr{C}^{\infty}$. Dos variedades son \textbf{difeomorfas} si existe un difeomorfismo entre ellas. Esta es la noción natural de isomorfismo, o equivalencia, para variedades. Es una relación de equivalencia. Dos variedades difeomorfas son iguales en todas las propiedades que conciernen únicamente a su estructura como variedades. En particular, son topológicamente iguales, es decir, \textbf{homeomorfas}.\\ \\
Una variedad topológica puede tener dos atlas $\mathscr{C}^{\infty}$ diferentes que estén $\mathscr{C}^{\infty}$-relacionados, pero las dos variedades $\mathscr{C}^{\infty}$ determinadas por estos atlas pueden seguir siendo difeomorfas. El problema es que el mapa identidad no es un difeomorfismo. De hecho, dos estructuras de variedad $\mathscr{C}^{\infty}$ en una variedad de dimensión $\leq4$ son invariablemente difeomorfas. Por otro lado, cualquier variedad compacta de dimensión $\geq7$ admite varias estructuras de variedad $\mathscr{C}^{\infty}$ no difeomorfas; esto es, puede haber un homeomorfismo entre dos variedades, pero no un difeomorfismo.\\ \\
Para un simple ejemplo de diferentes estructuras $\mathscr{C}^{\infty}$ que son difeomorfas, consideramos $\mathbb{R}$ con la estructura estándar $\curlybraces{u:\mathbb{R}\to\mathbb{R}}$ como atlas (con una carta), y $M=\mathbb{R}$ con la estructura $\curlybraces{u^3:\mathbb{R}\to\mathbb{R}}$ como atlas (de nuevo, una carta). Ya que una carta admisible siempre es un difeomorfismo en su dominio de subvariedad abierta, el difeomorfismo de $M$ sobre $\mathbb{R}$ es un mapa coordenado de $M$, $u^3:M\to\mathbb{R}$. El difeomorfismo inverso es $u^{1/3}:\mathbb{R}\to M$. El mapa identidad $u:\mathbb{R}\to M$ es $\mathscr{C}^{\infty}$, ya que la expresión de coordenadas es $u^3\circ u\circ u=u^3:\mathbb{R}\to\mathbb{R}$. El mapa identidad $u:M\to\mathbb{R}$ no es $\mathscr{C}^{\infty}$, ya que la expresión de coordenadas es $u\circ u\circ u^{1/3}=u^{1/3}:\mathbb{R}\to\mathbb{R}$, que no es $\mathscr{C}^{\infty}$. Por tanto, el mapa identidad no es un difeomorfismo.\\ \\
Existen ejemplos de estructuras de variedad $\mathscr{C}^{\infty}$ no difeomorfas de dimensión $\geq7$. No son fácil de describir.
\section{Subvariedades}
Una variedad $M$ está \textbf{incrustada} en una variedad $N$ si existe un mapa $\mathscr{C}^{\infty}$ biyectivo, $F:M\to N$, tal que para cada punto $m\in M$, existe un entorno $U$ de $m$ y una carta de $N$ en $Fm$, $\mu:V\to\mathbb{R}^e$, $\mu=(y^1,\dots,y^e)$, tal que $x^i=y^i\circ F|_U$, $i=1,2,\dots,d$, son coordenadas en $U$ para $M$. El mapa $F$ se denomina entonces una \textbf{incrustación} de $M$ en $N$.\\ \\
Si el requerimiento de que $F$ sea biyectiva se omite, pero el requerimiento de obtener coordenadas para $M$ de las de $N$ por composición con $F$ todavía se sostiene, entonces $M$ se denomina \textbf{inmersa} en $N$ y $F$ se denomina una \textbf{inmersión}. Otra forma de indicar esto es requerir que cada punto $m$ de $M$ esté contenido en una subvariedad abierta $U$ de $M$, la cuál es incrustada en $N$ usando $F$. Por tanto, una inmersión es una incrustación local. Ver Figura \ref{Fig4-ApB}
\begin{Figura}
    \centering
    \includegraphics[width=0.8\textwidth]{Apendices/incursion.png}
    \captionof{figure}{Esquemas de una incrustación de $\mathbb{R}^2$ en $\mathbb{R}^3$ (A) y una inmersión (B). [\ref{BFig4-ApB}]}
    \label{Fig4-ApB}
\end{Figura}
\begin{definition}
    Una \textbf{subvariedad} de $N$ es un subconjunto $FM$, donde $F:M\to N$ es una incrustación, dotada de estructura de variedad para la cuál $F:M\to FM$ es un difeomorfismo.
\end{definition}
La dimensión de una subvariedad obviamente no es mayor que la dimensión de la variedad que la contiene. Si fuera igual a la dimensión de la variedad que la contiene, entonces la subvariedad no es más que una subvariedad abierta, que definimos previamente.\\ \\
La topología de una subvariedad no tiene por qué ser la topología inducida de la variedad más grande. Por supuesto, el mapa de inclusión es $\mathscr{C}^{\infty}$, en particular continuo, de modo que los conjuntos abiertos de la topología inducida son conjuntos abiertos en la topología de la subvariedad. Sin embargo, la topología de la subvariedad puede tener muchos más conjuntos abiertos.\\ \\
Una subvariedad debe estar metida en la variedad que la contiene de una forma especial. Por ejemplo, cosas como cúspides y esquinas están descartados, incluso si pueden aparecer en la imagen de un mapa inyectivo $\mathscr{C}^{\infty}$, que no es una incrustación. Para describir cuidadosamente la naturaleza especial de una subvariedad, definimos una \textit{porción de coordenadas} de dimensión $d$ en una variedad $N$ de dimensión $e$, siendo un conjunto de puntos  en un entorno coordenado $U$ con coordenadas $y^1,\dots,y^e$ de la forma $\curlybraces{m|m\in U,y^{d+1}m=c^{d+1},\dots y^em=c^e}$, donde $c^i$ son constantes que determinan la porción. En otras palabras, una porción de coordenadas es la imagen bajo la inversa de un mapa coordenado de la parte $d$-dimensional del plano $\mathbb{R}^e$ que yace dentro del rango de coordenadas.
\begin{proposition}
    Si $M$ es una subvariedad de $N$, entonces para cada $m\in M$ existen coordenadas $y^1,\dots,y^e$ para $N$ en un entorno de $m$ en $N$, tal que la porción de coordenadas correspondientes a las constantes $c^{d+1}=y^{d+1}m,c^{d+2}=y^{d+2}m,\dots,c^e=y^em$ es un entorno de $m$ en $M$, y la restricción de $y^1,\dots,y^d$ a esa porción son coordenadas para $M$.\label{prop1ApC}
\end{proposition}
\begin{proof}
    Sea $F:P\to N$ una incrustación tal que $FP=M$. Tomamos las coordenadas $z^1,z^2,\dots,z^e$ para $N$ en un entorno de $m$ en $N$ tal que $x^1=z^1\circ F|_{U},\dots,x^d=z^d\circ F|_U$ son coordenadas en $p=F^{-1}m$ en un entorno coordenado $U\subset P$. Ya que $F$ es $\mathscr{C}^{\infty}$, podemos escribir
    \[z^i\circ F=f^i(x^1,\dots,x^d);\hspace{3mm}i=1,\dots,e\]
    donde las $f^i$ son funciones $\mathscr{C}^{\infty}$ sobre un conjunto abierto en $\mathbb{R}^d$. Es claro que $f^i(x^1,\dots,x^d)=x^i$ para $i=1,\dots,d$, pero los demás $f^i$, $i>d$, no es tan simple.\\
    Definimos
    \[\begin{array}{cc}
        y^i=z^i, & i\leq d, \\
        y^i=z^i-f^i(z^1,\dots,z^d), & i>d
    \end{array}\]
    Por tanto, es claro que
    \[\begin{array}{cc}
        z^i=y^i, & i\leq d, \\
        z^i=y^i+f^i(y^1,\dots,y^d), & i>d
    \end{array}\]
    así que los mapas $\mu_1=(z^1,\dots,z^e)$ y $\mu_2=(y^1,\dots,y^e)$ están $\mathscr{C}^{\infty}$ relacionados en ambos sentidos. El dominio de los $y^i$ se incluye en el de los $z^i$, por lo que podemos afirmar que $\mu_2$ es una carta admisible sin comprobar relaciones adicionales con otras coordenadas. Además, $FU$ es la porción de coordenadas $y^{d+1}=0,\dots,y^e=0$, y las restricciones de $y^1,\dots,y^d$ para $FU$ corresponden a $x^i$ bajo $F$, por lo que hay coordenadas para $M$ sobre $F$.
\end{proof}
\begin{remark}
    No se afirma que obtengamos \textit{todos} los puntos de $M$ que yacen en $N$-entorno de $m$ como miembros de una sola porción de coordenadas. De hecho, esto no es posible en algunos casos.
\end{remark}
Lo contrario a la Proposición \ref{prop1ApC} es obvio usando la definición; esto es, si un subconjunto tiene estructura de variedad que está localmente determinada por una porción de coordenadas, donde coordenadas no constantes que proporcionan las coordenadas en la porción para la estructura de variedad del subconjunto, entonces el subconjunto es una subvariedad.\\ \\
Whitney demostró que cualquier variedad es difeomorfa a una subvariedad de $\mathbb{R}^e$; si $d$ es la dimensión de la variedad, entonces necesitamos tomar un $e$ no mayor a $2d+1$.
\begin{theorem}[Teorema de Whitney]
    Cualquier variedad suave de dimensión $d$ puede ser inmersa en $\mathbb{R}^{2d}$ e incorporada en $\mathbb{R}^{2d+1}$
\end{theorem}
\begin{proof}
    Ver la demostración en \cite[Chapter 6, page 134]{IntroductionToSmoothManifolds}
\end{proof}
Por tanto, la teoría de variedades puede considerarse el estudio de subconjuntos especiales de espacios cartesianos, si se desea.
\section{Curvas diferenciables}
En algunos contextos una curva es casi lo mismo que una subvariedad unidimensional, pero vamos a preferir tratar solo las curvas que tengan una parametrización específica. Técnicamente, entonces, cambiando la parametrización de una curva tendremos una curva diferente, pero normalmente ignoraremos la distinción y hablaremos de una curva como si fuera un conjunto de puntos. Generalmente nuestras curvas tendrán un punto inicial y final pero también consideraremos curvas con finales abiertos.\\ \\

	 \end{appendices}
%---------------------------------------------------------------------------
%	BIBLIOGRAPHY
%---------------------------------------------------------------------------

\newpage
\lhead{\emph{Bibliografía}}
	\label{Bibliography}
	\lhead{\emph{Bibliograf\'ia}}
        \nocite{*}
	\bibliographystyle{plainnat} 
	\bibliography{librarys/library}

\addcontentsline{toc}{section}{\textbf{Bibliografía}}

% \renewcommand{\refname}{Bibliografía de Figuras}
%%%%%%%%%%%%%%%%%%%%%%%%%%%%%%%%%%%%%%%%%%%%%%%%%%%%%%%%%%
%---------------BIBLIOGRAFÍA DE FIGURAS------------------%
%%%%%%%%%%%%%%%%%%%%%%%%%%%%%%%%%%%%%%%%%%%%%%%%%%%%%%%%%%
\renewcommand\bibname{Bibliografía de Figuras}
\begin{thebibliography}{99}
    \bibitem{BFig1-ApB}
        Richard L. Bishop and Samuel I. Goldberg, \textit{Tensor Analysis on Manyfolds}. Garden City, New York, Dover edition, 1968-1980-2023. ISBN 978-0-486-64039-6. Pag. 21 \label{BFig1-ApB}
\bibitem{BFig2-ApB}
        Richard L. Bishop and Samuel I. Goldberg, \textit{Tensor Analysis on Manyfolds}. Garden City, New York, Dover edition, 1968-1980-2023. ISBN 978-0-486-64039-6. Pag. 25 \label{BFig2-ApB} 
\bibitem{BFig3-ApB}
        Richard L. Bishop and Samuel I. Goldberg, \textit{Tensor Analysis on Manyfolds}. Garden City, New York, Dover edition, 1968-1980-2023. ISBN 978-0-486-64039-6. Pag. 36 \label{BFig3-ApB}         
\bibitem{BFig4-ApB}
        Richard L. Bishop and Samuel I. Goldberg, \textit{Tensor Analysis on Manyfolds}. Garden City, New York, Dover edition, 1968-1980-2023. ISBN 978-0-486-64039-6. Pag. 40 \label{BFig4-ApB} 
\end{thebibliography}







\addcontentsline{toc}{section}{\textbf{Bibliografía de Figuras}}

\end{document}
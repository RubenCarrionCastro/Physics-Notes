\subsection{Notación de Einstein} % Main chapter title
\label{cap1-sec1-subsec4} 

La notación de Einstein va a servir para facilitarnos la escritura, pues cada vez que tengamos un vector o una forma escrita como combinación lineal, vamos a poder redefinirlos como
\[w=\sum\limits_{i=1}^n\lambda^iv_i\equiv\lambda^iv_i\]
esto para un vector. Para una forma, tendremos
\[p=\sum\limits_{i=1}^n\mu_if^i\equiv\mu_i f^i\]
Además, para simplificar aún más la notación y dejarnos de tantas letras, vamos a identificar los escalares de $w$ como 
\[\lambda^i\equiv w^i\]
Así, los vectores como combinación lineal de otros vectores, los escribiremos como
\[w=w^iv_i\]
Y para las formas, haremos la identificación
\[\mu_i\equiv p_i\]
Así, las formas como combinación lineal de otras formas se escribirán como
\[p=p_if^i\]
\begin{example}
Un ejemplo de ello, será a la hora de identificar un vector en los términos de su base, pues suponiendo un $V$ espacio vectorial sobre el cuerpo $\mathbb{K}$ y cuya base sea $B=\curlybraces{v_1,v_2,\dots,v_n}$, tomando un $u\in V$, lo denotaremos como,
\[u=u^iv_i\]
\end{example}
\begin{example}
    Otro ejemplo será a la hora de identificar una forma en términos de la base dual, pues suponiendo un $V^*$ espacio dual de $V$, cuya base dual es $B^*=\curlybraces{f^1,f^2,\dots,f^n}$, tomando un $q\in V^*$, lo denotaremos como,
    \[q=q_if^i\]
\end{example}
\begin{note}
    En un artículo físico, se identifica directamente el escalar con el vector, es decir,
    \[w^i\equiv w\]
    pues se presupone que existe una base donde $w$ está bien definido. Así, los físicos usaremos de forma indistinguible los vectores y sus componentes respecto de una base fijada.
\end{note}
\section{Ecuación de la desviación geodésica} % Main chapter title
\label{cap5-sec3} 

Recordamos que $T^{\mu}$ es el vector tangente a la geodésica, que satisface $T^{\mu}\nabla_{\mu}T^{\nu}=0$ y que $\gamma_s(t)=\gamma_{p(s)}(t)$, es decir, la familia de geodésicas $\gamma_s(t)$ definen una superficie 2-D en la variedad $\mathscr{M}$, tal que $t$ y $s$ pueden definir coordenadas en la superficie (siempre que las geodésicas se crucen).\\ \\
Si definimos $S^{\mu}$ como el campo vectorial que define el desplazamiento infinitesimal entre geodésicas, entonces $T^{\mu}$ y $S^{\mu}$ son vectores de una base coordenada, es decir, $\alpha_sT^{\mu}=0$, de forma que
\[S^{\mu}\nabla_{\mu}T^{\nu}=T^{\mu}\nabla_{\mu}S^{\nu}\]
Por tanto, definimos la velocidad relativa de separación entre geodésicas como $v^{\mu}=T^{\nu}\nabla_{\nu}S^{\mu}$.\\ \\
Veamos ahora como escribir esto en función de la curvatura, y como $T^{\mu}$ y $S^{\mu}$ son vectores de una base coordenada, es decir, conmutan, así
\[\begin{array}{rl}
    a^{\mu} & =T^{\nu}\nabla_{\nu}(T^{\rho}\nabla_{\rho}S^{\mu})=T^{\nu}\nabla_{\nu}(S^{\rho}\nabla_{\rho}T^{\mu})=(T^{\nu}\nabla_{\nu}S^{\rho})\nabla_{\rho}T^{\mu}+T^{\nu}S^{\rho}\nabla_{\nu}\nabla_{\rho}T^{\mu}= \\
     & =\underbrace{(T^{\nu}\nabla_{\nu}S^{\rho})}_{S^{\nu}\nabla_{\nu}T^{\rho}}\nabla_{\rho}T^{\mu}+T^{\nu}S^{\rho}\nabla_{\rho}\nabla_{\nu}T^{\mu}-\mathscr{R}_{\nu\rho\sigma}^{\mu}S^{\rho}T^{\nu}T^{\sigma}= \\
     &=S^{\nu}\nabla_{\nu}\underbrace{(T^{\rho}\nabla_{\rho}T^{\mu})}_{=0\text{ (al ser geodésica)}}-\mathscr{R}_{\nu\rho\sigma}^{\mu}S^{\rho}T^{\nu}T^{\sigma}=-\mathscr{R}_{\nu\rho\sigma}^{\mu}S^{\rho}T^{\nu}T^{\sigma}=\frac{d^2S^{\mu}}{dt^2}
\end{array}\]
siendo esta la ecuación de la geodésica, donde $t$ es el parámetro afín de la geodésica.
\subsection{Efecto de la una onda gravitacional que pasa cerca del observador}
Tomemos un conjunto de objetos en reposo relativo inicialmente con cuadrivelocidad $u^{\mu}=\delta_{t}^{\mu}$. Suponemos que dos de estos objetos están separados una distancia $S^{\mu}$. La ecuación de la geodésica es,
\[\frac{d^2S^{\mu}}{dt^2}=\mathscr{R}_{\nu00}^{\mu}S^{\nu}=...=\frac{1}{2}(\partial^2_th^{\mu}_{\nu})S^{\nu}\]
donde $h^{\mu}_{\nu}$ es una perturbación que cumple el gauge radiativo (realmente da igual en qué gauge trabajar, pues la curvatura y las ecuaciones de Einstein son independientes de los gauges; usamos este gauge porque sabemos la forma de $h^{\mu}_{\nu}$).\\ \\
Supongamos que $H_x=0$ y $H_t\neq0$, entonces $S^t=0=S^z$ y
tenemos dos ecuaciones geodésicas, tales que
\[\frac{d^2S^x}{dt^2}=\frac{1}{2}S^xH_+(-\omega^2e^{ik_{\mu}x^{\mu}})\]
\[\frac{d^2S^y}{dt^2}=-\frac{1}{2}S^yH_+(-\omega^2e^{ik_{\mu}x^{\mu}})\]
Integramos estas ecuaciones y obtenemos,
\[S^x=S^x(0)\left(1+\frac{1}{2}H_+e^{ik_{\mu}x^{\mu}}\right)\]
\[S^y=S^y(0)\left(1-\frac{1}{2}H_+e^{ik_{\mu}x^{\mu}}\right)\]
El efecto de la onda gravitacional sería el siguiente,
\begin{Figura}
    \centering
    \includegraphics[width=0.8\textwidth]{onda1.png}
    \captionof{figure}{Efecto que sufren los objetos cuando pasa la onda gravitacional.}
    \label{fig5.1}
\end{Figura}
es decir, los objetos se contraen en la dirección vertical y horizontal.\\ \\
Análogamente, para la polarización $H_x\neq0$ y $H_+=0$, tenemos las soluciones,
\[S^x=S^x(0)+\frac{1}{2}H_xS^y(0)e^{ik_{\mu}x^{\mu}}\]
\[S^y=S^y(0)+\frac{1}{2}H_xS^x(0)e^{ik_{\mu}x^{\mu}}\]
Los efectos de la onda gravitacional son los siguientes,
\begin{Figura}
    \centering
    \includegraphics[width=0.8\textwidth]{onda2.png}
    \captionof{figure}{Efecto que sufren los objetos cuando pasa la onda gravitacional.}
    \label{fig5.2}
\end{Figura}
es decir, los objetos se contraen en las direcciones oblicuas.\\ \\
No es difícil ver que si realizamos una rotación de 180º, el resultado es invariante, por tanto, las ondas gravitacionales \textbf{tienen espín 2}.\\ \\
También existe la polarización circular, tal que
\[H_{+2}=\frac{1}{\sqrt{2}}(H_++iH_x)\]
\[H_{-2}=\frac{1}{\sqrt{2}}(H_+-iH_x)\]
cuyo efecto es una rotación de los objetos sin contraerlos.\\ \\
Si tenemos un interferómetro de Fabry-Perot, podemos llegar a medir estas ondas gravitacionales, pues dejamos los espejos en un péndulo, y cuando pase la onda hará que se muevan, tal que
\begin{Figura}
    \centering
    \includegraphics[width=0.8\textwidth]{f-p.png}
    \captionof{figure}{Efecto de la onda gravitacional sobre un interferómetro de Fabry-Perot.}
    \label{fig:enter-label}
\end{Figura}
\subsection{Generación de las ondas gravitacionales}
Las ecuaciones de Einstein en el gauge de Lorentz las podemos escribir como,
\[\hspace{0mm}^{(1)}8\pi GT_{\mu\nu};\hspace{4mm}-\frac{1}{2}\partial^{\sigma}\partial_{\sigma}\overline{h}_{\mu\nu}=8\pi GT_{\mu\nu};\hspace{4mm}\partial^{\mu}T_{\mu\nu}=0\Rightarrow\partial^{\mu}\overline{h}_{\mu\nu}=0\]
Esta ecuación la podemos resolver utilizando las funciones de Green, tal que
\[\overline{h}_{\mu\nu}(x)=-16\pi \mathscr{G}\int_{\mathscr{M}}G(x-y)T_{\mu\nu}(y)d^4y\]
donde $x$ es la posición de la onda gravitacional, $y$ es la posición de la fuente y $\mathscr{G}$ es una función de Green. Por tanto, podemos decir que las funciones de Green son fuentes en el espacio.\\ \\
De todas las funciones de Green usaremos la función de Green \textit{retardada}, que representa la onda gravitacional de una fuente puntual que viaja hacia el futuro, siendo
\[\mathscr{G}(x-y)=-\frac{1}{4\pi|\vec{x}-\vec{y}|}\delta\left(|\vec{x}-\vec{y}-(x^0-y^0)|\right)\cdot\Theta(x^0-y^0)\]
donde
\[\Theta(x^0-y^0)=\left\lbrace\begin{array}{rl}
    1 & \text{si}\hspace{3mm} x^0>y^0 \\
    0 &   \text{en otro caso}
\end{array}\right.\]
Así, nos garantizar poder evaluar el cono de luz pasado de la onda gravitacional.\\ \\
Notemos que $t=x^0$ es el tiempo presente.\\ \\
Si $y^0\geq x^0=t$, entonces $\mathscr{G}(x-y)=0$, es decir, solo nos interesan las contribuciones de las fuentes en tiempos $y^0<t$. Luego, integrando en $y_0$ tenemos,
\[\overline{h}_{\mu\nu}(t,\vec{x})=-4G\int\frac{1}{|\vec{x}-\vec{y}|}T_{\mu\nu}\delta\left(t-|\vec{x}-\vec{y}|,\vec{y}\right)d^3\vec{y}\]
Nos referimos a $t_r=t-\frac{|\vec{x}-\vec{y}|}{c}$ como \textit{tiempo retardado}. Así, dejando $t_r$ fijo y tomando $|\vec{x}|>>|\vec{y}|$, tenemos que $|\vec{x}-\vec{y}|\approx R$, tal que
\[\overline{h}_{\mu\nu}(t,\vec{x})=-\frac{4G}{R}\int T_{\mu\nu}(t_r,\vec{y})d^3\vec{y}\]
Como $\partial_t\overline{h}_{0\nu}=\sum\limits_{i=1}^{3}\partial_i\overline{h}_{i\nu}$, entonces $\overline{h}_{i\nu}$ determina a $\overline{h}_{0j}$ y $\overline{h}_{00}$. Por tanto, solo necesitamos estudiar la siguiente integral,
\[\int T^{ij}(t_r,\vec{y})d^3y=\int d^3y\left[\cancelto{0}{\frac{\partial}{\partial y^k}\left(T^{ik}y^j\right)}-\cancelto{\partial_tT^{0i}}{\frac{\partial T^{ik}}{\partial y^k}}y^j\right]=-\frac{\partial}{\partial t_r}\int d^3y T^{0i}y^j=...=\frac{1}{2}\frac{\partial^2}{\partial t_r^2}\int d^3y T^{00}y^iy^j\]
Además podemos escribirlo en términos del tensor de momento cuadrupolar, tal que
\[\int T^{ij}(t_r,\vec{y})d^3y=\frac{1}{6}\partial^2_{t_r}q^{ij}\]
donde 
\[q^{ij}(t_r,R)=3\int d^3y T^{00}(t_r,y)y^iy^j\]
es el tensor de momento cuadrupolar. En resumen tenemos,
\begin{equation}
    \overline{h}_{ij}(t,\vec{x})=-\frac{2G}{3R}\partial^2_{t_r}q_{ij}(t_r,R)
\end{equation}
con $t_r=t-R/c$, siendo $R$ la distancia a la fuente, $t$ el tiempo de recepción y $t_r$ el tiempo de emisión. Por tanto, si $\partial^2_{t_r}\neq0$, entonces se formarán ondas gravitacionales. No hay contribución dipolar debido a la conservación del tensor energía-impulso $T_{\mu\nu}$.
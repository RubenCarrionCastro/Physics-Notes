%SECCION 3
\section{Variedad diferenciable} % Main chapter title
\label{cap3-sec4} 
%------------------------------------------------------------------------------
Una variedad diferenciable es un conjunto de puntos de dimensión $n$ que localmente se parecen a $\mathbb{R}^n$.
\subsection{Definiciones}
\begin{definition}
    Un espacio topológico es un conjunto de puntos y sus entornos.
\end{definition}
\begin{definition}
    Un espacio topológico es de tipo Haussdorff si todo par de puntos poseen entornos abiertos disjuntos.
\end{definition}
Esto nos da una noción de 'separación' entre puntos.
\begin{definition}
    Un homeomorfismo (cartas coordenadas) es un aplicación entre dos espacios topológicos, tal que ella y su inversa son biyecciones continuas, es decir, mapean abiertos en abiertos.
\end{definition}
\begin{definition}
    Una variedad topológica es un espacio topológico de tipo Haussdorff, tal que cada punto tiene un entorno abierto hoeomorfo a un abierto de $\mathbb{R}^n$. La denotaremos por $\mathscr{M}$.
\end{definition}
\begin{definition}
    Una carta (o sistema de coordenadas) es una dupla $(U,\phi)$, que viene dada por un abierto $U\subset\mathscr{M}$ y un homeomorfismo 
    \[\phi:u\to\phi(u)\in\mathbb{R}^n\]
    que mapea un entorno abierto $u\in U$ a $\mathbb{R}^n$.
\end{definition}
\begin{definition}
    Un atlas $\mathscr{C}^{\infty}$ es un conjunto de cartas $(U_{\alpha},\phi_{\alpha})$ que cumplen:
    \begin{itemize}
        \item Cubren todo $\mathscr{M}$, es decir, $\mathscr{M}=\bigcup_{\alpha}U_{\alpha}$.
        \item Las cartas deben ser compatibles, es decir, si consideramos dos abiertos $U_{\alpha}$, $U_{\beta}$ tal que $U_{\alpha}\cap U_{\beta}\neq\emptyset$, entonces la función de transición entre ambas cartas viene dada por
        \[F=\phi_{\alpha}\circ\phi_{\beta}^{-1}=\phi_{\alpha}\left(\phi_{\beta}^{-1}\right)\]
        tal que
        \[F:\phi_{\beta}\left(U_{\alpha}\cap\phi_{\beta}\right)\to\phi_{\alpha}\left(U_{\alpha}\cup U_{\beta}\right)\]
        siendo $F$ una función $\mathscr{C}^{\infty}$ entre abiertos de $\mathbb{R}^n$.
    \end{itemize}
\end{definition}
Además, $\mathscr{M}$ tiene dimensión $n$, es decir, $p\in\mathscr{M}$ es una colección de $n$ números reales.
\begin{definition}
    Una variedad diferenciable es $\mathscr{C}^{\infty}$ (suave) si es una variedad topológica con un atlas $\mathscr{C}^{\infty}$.
\end{definition}
Veamos de forma visual cómo actúan todas estas cosas,
\begin{Figura}
    \centering
    \includegraphics[width=0.8\textwidth]{Capitulos/Capitulo3/Seccion3/IMG_0127[1].JPG}
    \captionof{figure}{Esquema visual de cómo actúan los homeomorfismos sobre variedades.}
    \label{fig3.1}
\end{Figura}
\begin{note}
    Las aplicaciones $\phi_{\alpha}$ pueden no ser $\mathscr{C}^{\infty}$ (solo continuas), pero $F=\phi_{\alpha}\circ\phi_{\beta}^{-1}$ sí debe ser $\mathscr{C}^{\infty}$.
\end{note}
Denotaremos por $x^{\mu}$ a las coordenadas de un punto $p\in\mathscr{M}$ definidas por $\phi_{\alpha}(p)$, es decir, $x^{\mu}\equiv\phi^{\mu}(p)$.
\begin{definition}
    Una variedad es orientable si y solo si admite un atlas, tal que para cada par de abiertos $U_{\alpha}$ y $U_{\beta}$ no disjuntos, el Jacobiano entre coordenadas sea positivo,
    \[\det\left(\frac{\partial x_{\alpha}}{\partial x_{\beta}}\right)>0\]
\end{definition}
Si la variedad tiene una frontera $\partial\mathscr{M}$, entonces sustituimos el mapeo a $\mathbb{R}^n$ por el mapeo a $\frac{1}{2}\mathbb{R}^n$, siendo
\[\frac{1}{2}\mathbb{R}^n=\curlybraces{x\in\mathbb{R}^n|x'\leq0}\]
de $(n-1)$-dimensión.
\subsection{Transformaciones de coordenadas}
Definimos las transformaciones de coordenadas como
\[x'=\phi'\circ\phi^{-1}(x)\]
donde $\phi$ puede ser un escalar, un vector, un campo vectorial, una uno-forma o un tensor. Veamos cada uno de ellos:
\subsubsection{Escalares}
Sea una función cualquiera $f:\mathscr{M}\to\mathbb{R}$, esta induce a $\Tilde{f}:\mathbb{R}^n\to\mathbb{R}$ mediante la regla
\[\Tilde{f}(x)=(f\circ\phi^{-1})(x)=f(\phi^{-1}(x))\]
Diremos que $f$ es suave o $\mathscr{C}^{\infty}$, si $\Tilde{f}=f\circ\phi^{-1}$ lo es.
\begin{proposition}
    Si tenemos $(U,\phi)$ y $(U',\phi')$, tal que $U$ y $U'$ son disjuntos, entonces se tiene que
    \[\Tilde{f}(x')=\Tilde{f}(x)\]
    en $p\in U\cap U'$. Esta es la regla de transformación de los escalares.
\end{proposition}
En adelante vamos a referirnos a $\Tilde{f}$ como $f$ y llamaremos $F_{\mathscr{M}}$ al conjunto de funciones $\mathscr{C}^{\infty}$ sobre $\mathscr{M}$.
\subsubsection{Vectores}
Como las variedades $\mathscr{M}$ no son espacios vectoriales, no podemos definir vectores aquí. Entonces, lo que hacemos es definir el espacio tangente $T_p$ en un punto $p\in\mathscr{M}$, definiéndose punto a punto. Consideramos para ello el conjunto de curvas $\gamma(t)$ que pasan por $p$, es decir,
\[\gamma(t):\mathbb{R}\to\mathscr{M}\]
tal que $\gamma(t=t_0)=p$.\\ \\
Consideramos una función $f$ arbitraria de $F_{\mathscr{M}}$ y calculamos,
\[\frac{d}{dt}\left.\left(f\circ\gamma(t)\right)\right|_{t=t_0}=\frac{d}{dt}\left.\left(\overbrace{\underbrace{(f\circ\phi^{-1})}_{f(x)}}^{\mathbb{R}^n\to\mathbb{R}}\circ\overbrace{(\underbrace{\phi\circ\gamma(t)}_{\gamma^{\mu}(t)})}^{\mathbb{R}\to\mathbb{R}^n}\right)\right|_{t=t_0}=\]
\[=\frac{d}{dt}\left.\left(f(x)\circ\gamma^{\mu}(t)\right)\right|_{t=t_0}=\frac{d\gamma^{\mu}}{dt}\left.\frac{\partial f}{\partial x^{\mu}}\right|_{t=t_0}\equiv \vec{v}_{\gamma}(f)|_{t=t_0}\]
donde $\vec{v}_{\gamma}(f)|_{t=t_0}$ se define como un vector que vive en el espacio tangente (espacio vectorial)0 de un punto $p\in\mathscr{M}$.\\ \\
Podemos interpretar $\dot{\gamma}_{\mu}$ como las coordenadas del vector $\vec{v}_{\gamma}$, es decir,
\[v_{\gamma}^{\mu}=\dot{\gamma}^{\mu}\]
y además, $\partial_{\mu}$ se interpreta como una base coordenada de vectores, es decir, $\partial_{\mu}\equiv\vec{e}_{\mu}$, ya que $f$ es arbitraria.
\\ \\
La notación que usaremos para los vectores, $\vec{v}:F_{\mathscr{M}}\to\mathbb{R}$, será $\vec{v}=v^{\mu}\vec{e}_{\mu}$ en $T_p$; diremos que son vectores contravariantes.\\ \\
La transformación de vectores se realizará usando dos bases coordenadas $\curlybraces{\vec{e}_{\mu}}$ y $\curlybraces{\vec{e}'_{\mu}}$ asociadas a las coordenadas $x^{\mu}$ y $x^{'\mu}$, por lo que aplicando el Jacobiano tendremos
\begin{equation}
    \vec{e}'_{\mu}=\frac{\partial x^{\mu}}{\partial x^{'\nu}}=\vec{e}_{\nu}\Longrightarrow v^{'\mu}=\frac{\partial x^{'\mu}}{\partial x^{\nu}}v^{\nu}
\end{equation}
siendo esta la transformación de vectores. Además, los vectores de $T_p$ satisfacen:
\begin{enumerate}
    \item $\vec{v}(\alpha f+\beta g)=\alpha\vec{v}(f)+\beta\vec{v}(g);\hspace{2mm}\forall f,g\in F_{\mathscr{M}},\forall\alpha,\beta\in\mathbb{R}$.
    \item La regla de Leibtniz:
    \[\vec{v}(fg)=g\vec{v}(f)+f\vec{v}(g);\hspace{2mm}\forall f,g\in F_{\mathbb{M}}\]
\end{enumerate}
\subsubsection{Campo vectorial en la variedad}
Un campo vectorial en $\mathscr{M}$ es una asignación de un vector $v\in T_p$ en cada punto $p\in\mathscr{M}$, tal que sus componentes en cualquier base coordenada sean suaves $(\mathscr{C}^{\infty})$.\\ \\
El conmutador, o corchete de Lie, se define como
\[\brackets{\vec{v},\vec{w}}=\vec{v}\circ\vec{w}-\vec{w}\circ\vec{v}=v^{\mu}(\partial_{\mu}w^{\nu})\partial_{\nu}-w^{\mu}(\partial_{\mu}v^{\nu})\partial_{\nu}\]
donde $\brackets{\vec{v},\vec{w}}=\vec{u}$, tal que $u^{\mu}=v^{\nu}\partial_{\nu}w^{\mu}-w^{\nu}\partial_{\nu}v^{\mu}$.\\ \\
Sus propiedades son:
\begin{enumerate}
    \item Antisimetría:
    \[\brackets{\vec{v},\vec{w}}=-\brackets{\vec{w},\vec{v}}\]
    \item Identidad de Jacobi:
    \[\brackets{\vec{u},\brackets{\vec{v},\vec{w}}}+\brackets{\vec{v},\brackets{\vec{w},\vec{u}}}+\brackets{\vec{w},\brackets{\vec{u},\vec{v}}}=0\]
    \item Para una base coordenada, $\brackets{\vec{e}_{\mu},\vec{e}_{\nu}}=0$.
    \item Para una base no coordenada, $\brackets{\vec{e}_{\mu},\vec{e}_{\nu}}\neq0$
\end{enumerate}
\begin{proposition}
    Sea $\curlybraces{\vec{e}_{\mu}}$ un base coordenada, entonces su transformación
    \[\vec{e}'_{\mu}=\frac{\partial x^{\nu}}{\partial x^{'\mu}}\vec{e}_{\nu}\]
    también es una base coordenada.
\end{proposition}
Denotaremos las bases coordenadas con índices latinos, es decir, usaremos $\vec{e}_{a}$, con $\vec{v}=v^a\vec{e}_a$, y nos referiremos a $v^a$ y $\vec{v}$ indistintamente.
\subsubsection{Uno-formas}
Las uno-formas son aplicaciones lineales reales en $T_p$, tal que
\[\vec{f}:T_p\to\mathbb{R}\]
donde el producto escalar $<\vec{f},\vec{v}>\in\mathbb{R}$ con $\vec{v}\in T_p$.\\ \\
Dada una base coordenada $\curlybraces{\vec{e}_a}$ en $T_p$, existe un único conjunto de uno-formas $\curlybraces{\vec{e}^a}$ que cumplen $<\vec{e}^a,\vec{e}_b>=\delta_b^a$, donde $\delta_b^a$ es la delta de Kronecker. Diremos que $\curlybraces{\vec{e}^a}$ es una base dual de $T_p^*$, que el espacio dual tangente a $p\in\mathscr{M}$, también llamado \textbf{espacio cotangente}.\\ \\
Dados $\vec{v}=v^a\vec{e}_a$ y $\vec{f}=f_a\vec{e}^a$, entonces $<\vec{f},\vec{v}>=f_av^a$, por tanto, el producto escalar de vectores con uno-formas se puede interpretar como una contracción de índices.\\ \\
Además, diremos que $f_a$ son las componentes de un vector covariante.\\ \\
Dada una función $f:\mathscr{M}\to p$, podemos construir una uno-forma de la siguiente manera:\\ 
-Dado un $\vec{v}$,
\[\vec{v}(f)=v^{\mu}\partial_{\mu}f=v^{\mu}\delta_{\mu}^{\nu}\partial_{\nu}f=v^{\mu}\partial_{\nu}f<\vec{e}^{\nu},\vec{e}_{\mu}>=<\partial_{\nu}f\vec{e}^{\nu},v^{\mu}\vec{e}_{\mu}>=d\vec{f},\vec{v}<>\]
por tanto,

\[d\vec{f}=\partial_{\mu}f\vec{e}^{\mu}=\partial_{\mu}fdx^{\mu}\]
De tal forma que $<dx^{\mu},\partial_{\nu}>=\delta_{\nu}^{\mu}$. Por tanto, las uno-formas transforman como
\begin{equation}
\vec{e}^{'\mu}=\frac{\partial x^{'\mu}}{\partial x^{\nu}}\vec{e}^{\nu}\Longrightarrow f'_{\mu}=\frac{\partial x^{\nu}}{\partial x^{'\mu}}f_{\nu}
\end{equation}
\subsubsection{Tensores}
Un tensor es una aplicación multilineal real de tipo $(r,s)$, denotada como $T_{\mu_1\mu_2\dots\mu_r}^{\nu_1\nu_2\dots\nu_s}$ que actúa como,
\[T(\vec{v}_1,\vec{v}_2,\dots,\vec{f}_1,\vec{f}_2,\dots)=T_{\mu_1\mu_2\dots}^{\nu_1\nu_2\dots}v_1^{\mu_1}v_2^{\mu_2}\dots f_{1\nu_1}f_{2\nu_2}\dots\]
Los tensores transforman bien bajo transformaciones de coordenadas.\\ \\
Dos tensores del mismo rango se pueden sumar (multiplicación por escalares).
\begin{definition}
    Definimos la parte simétrica de un tensor de rango 2 como
    \[T_{(\mu\nu)}=\frac{1}{2}(T_{\mu\nu}+T_{\nu\mu})\]
\end{definition}
\begin{definition}
    Definimos la parte antisimétrica de un tensor de rango 2 como
    \[T_{[\mu\nu]}=\frac{1}{2}(T_{\mu\nu}-T_{\nu\mu})\]
\end{definition}
Podemos cambiar el carácter tensorial, realizando una multiplicación tensorial o haciendo una contracción de índices.
\begin{example}
    Dados $R_{\mu\nu}$ y $T_{\mu\nu}$, podemos construir $R\otimes T\to R_{\mu\nu}T_{\sigma\rho}\equiv G_{\mu\nu\sigma\rho}$, siendo el producto tensorial
\end{example}
\begin{example}
    Dados $T_{\mu\nu}$ y $v^{\mu}$, podemos definir $T_{\mu\nu}v^{\nu}=h_{\mu}$, siendo una contracción de índices.
\end{example}
\textbf{Tensor métrico}\\ \\
Definimos el tensor métrico como la distancia cuadrática infinitesimal al desplazarnos en una dirección. Será por tanto un tensor de rango $(0,2)$, pues es cuadrática, y además asumimos que es un tensor simétrico.\\ \\
En una base coordenada, las componentes del tensor métrico son
\[g_{\mu\nu}=g(\vec{e}_{\mu},\vec{e}_{\nu})\]
También lo escribiremos como
\[ds^2=g_{\mu\nu}dx^{\mu}dx^{\nu}\]
donde $ds^2$ es el elemento de línea.\\ \\
Definimos la norma de un vector como
\[|\vec{v}|=\sqrt{|g_{\mu\nu}v^{\mu}v^{\nu}|}\]
Definimos el ángulo entre dos vectores $\vec{v}$ y $\vec{w}$ como
\[\cos\theta=\frac{g_{\mu\nu}v^{\mu}w^{\nu}}{\sqrt{|g_{\mu\nu}v^{\mu}v^{\nu}|}\sqrt{|g_{\mu\nu}w^{\mu}v^{\nu}|}}\]
con $|\vec{v}|\neq0$ y $|\vec{w}|\neq0$.\\ \\
Diremos que $g_{\mu\nu}$ es una métrica \textbf{degenerada} en un punto $p$ si es una matriz singular en cualquier base coordenada $\curlybraces{\vec{e}_{\mu}}$, es decir, que su determinante es igual a cero. En caso contrario, existirá un tensor de rango $(2,0)$ que cumple que $g^{\mu\nu}g_{\nu\rho}=\delta_{\rho}^{\mu}$. Es decir, $g^{\mu\nu}$ es la \textbf{inversa} de $g_{\mu\nu}$.\\ \\
La métrica (y su inversa) nos permiten 'subir' y 'bajar' índices, por tanto podremos transformar vectores en uno-formas, $v_{\mu}=g_{\mu\nu}v^{\nu}$, y viceversa, $f^{\mu}=g^{\mu\nu}f_{\nu}$.\\ \\
\textbf{Signatura}\\ \\
La signatura de la métrica se obtiene diagonalizando la matriz $g_{\mu\nu}$. Si la signatura es del tipo $(++++)$, diremos que es una métrica riemanniana; si es del tipo $(-+++)$, diremos que es lorentziana. \\ \\
Las métricas físicas son lorentzianas e introducen una estructura casual, tal que
\begin{itemize}
    \item Si $v^{\mu}v^{\nu}g_{\mu\nu}>0$, entonces $v^{\mu}$ es un vector tipo espacial.
    \item Si $v^{\mu}v^{\nu}g_{\mu\nu}<0$, entonces $v^{\mu}$ es un vector tipo temporal.
    \item Si $v^{\mu}v^{\nu}g_{\mu\nu}=0$, entonces $v^{\mu}$ es un vector tipo luz o nulo.
\end{itemize}
También existe la métrica lorentziana $(+---)$ que se usa en física de partículas relativista, donde la estructura casual se invierte, es decir, el vector espacial pasa a temporal y viceversa.
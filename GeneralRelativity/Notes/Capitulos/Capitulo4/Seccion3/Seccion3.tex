%SECCION 1
\section{Principio de Covariancia General} % Main chapter title
\label{cap4-sec3} 
%------------------------------------------------------------------------------
La construcción de las ecuaciones de Einstein se conoce como \textit{Principio de Covariancia General}, pues asumimos que no hay elementos universales más allá de la propia métrica del espacio-tiempo, es decir, que no existen regiones o direcciones privilegiadas. Es decir,
\begin{center}
    \textit{''Las leyes de la física deben ser las mismas en cualquier sistema de coordenadas y no depender de un sistema de referencia particular''.}
\end{center}
Con este principio pasamos de las derivadas parciales a la derivada covariante, y de la métrica de Minkowwski a una métrica curva, tal que
\[\partial_{\mu}\longrightarrow\nabla_{\mu};\hspace{4mm}\text{y}\hspace{4mm}\eta_{\mu\nu}\longrightarrow g_{\mu\nu}\]
La métrica $g_{\mu\nu}$  determina la estructura del espacio-tiempo, es dinámica (satisface las ecuaciones de Einstein), es lorentziana y de dimensión 4. Asumimos que es $\mathscr{C}^{\infty}$. Además, al igual que en el espacio de Minkowski, esta métrica induce una estructura causal según
\[g_{\mu\nu}V^{\mu}V^{\nu}\left\lbrace\begin{array}{lcl}
     <0 & \longrightarrow & \text{temporal} \\
    =0 & \longrightarrow & \text{nula} \\
    >0 & \longrightarrow & \text{espacial}
\end{array}\right.\]

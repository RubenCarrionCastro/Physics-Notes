%SECCION 1
\section{Condiciones de energía} % Main chapter title
\label{cap4-sec5} 

Estas condiciones se emplearán en el desarrollo de los teoremas de singularidad. Pues, al tensor energía-impulso se le deben imponer algunas propiedades para determinar si pueden formarse o no formarse singularidades en nuestro universo.
\subsection*{Condición de energía débil}
Dado el tensor energía-impulso $T_{\mu\nu}$, y cualquier vector temporal futuro $u^{\mu}$, entonces la condición de energía débil es $T_{\mu\nu}u^{\mu}u^{\nu}\geq0$, es decir, la densidad de energía es positiva, cosa que implica que la masa es positiva y la gravedad es atractiva.
\subsection*{Condición de energía fuerte}
Dado el tensor de energía-impulso $T_{\mu\nu}$, y $u^{\mu}$ cualquier vector temporal futuro, entonces la condición de energía fuerte es
\begin{equation}
    \left(T_{\mu\nu}-\frac{1}{2}g_{\mu\nu}T\right)u^{\mu}u^{\nu}\geq0
\end{equation}
\subsection*{Condición dominante de energía}
Dado el tensor de energía-impulso $T_{\mu\nu}$, y $u^{\mu}$ temporal futuro, entonces la condición dominante de energía es que el vector $(-T^{\mu\nu}u_{\nu})$ es temporal, nulo futuro o cero.
\subsection*{Condición nula de energía}
Dado el tensor de energía-impulso $T_{\mu\nu}$, y un vector $k^{\mu}$ nulo futuro cualquiera, entonces la condición de energía es que $T_{\mu\nu}k^{\mu}k^{\nu}\geq0$.\\ \\
Estas condiciones se impican entre sí, pero no de cualquier forma, sino que la cadena de implicaciones es,
\[\text{dominante}\Rightarrow\text{débil}\Rightarrow\text{nula}\Leftarrow\text{fuerte}\]
La condición nula es la que nos garantiza los Teoremas de Singularidad, por tanto, bastará con ver alguna de las condiciones de energía para ver si la condición nula se satisface, pues en ocasiones la condición nula no es fácil de calcular.
\subsection{Teoremas de Singularidad}
Definiciones preliminares:
\begin{itemize}
    \item \textbf{Geodésica inextensible incompleta:} Una trayectoria en el espacio-tiempo que termina abruptamente porque el espacio-tiempo deja de estar definido. Indica la presencia de una singularidad.
    \item \textbf{Superficie atrapada:} Una región del espacio-tiempo donde la gravedad es tan intensa que la luz y la materia no pueden escapar, típica de agujeros negros.
    \item \textbf{Condición de energía fuerte:} Supone que la energía y la presión de la materia cumplen ciertos límites, asegurando que la gravedad sea atractiva.
    \item \textbf{Espacio-tiempo globalmente hiperbólico:} Un espacio-tiempo que tiene una estructura causal bien definida, sin curvas cerradas de tipo tiempo (que implicarían paradojas causales).
\end{itemize}
A continuación, se presentan los enunciados de los teoremas de singularidad de Penrose y Hawking, traducidos de las publicaciones originales para reflejar fielmente sus formulaciones. Estos enunciados están extraídos y adaptados de los artículos citados y del libro \textit{The Large Scale Structure of Space-Time de Hawking y Ellis} (1973), que formaliza estos resultados:
\begin{theorem}{Teorema de Singularidad de Penrose (1965)}

\textit{
"Si una superficie atrapada existe en el espacio-tiempo, y si la relatividad general es válida, junto con la condición de energía fuerte, entonces hay al menos una geodésica de tipo tiempo o de tipo nulo inextensible incompleta en el espacio-tiempo."
}
\end{theorem}
Es decir, una superficie atrapada es una región donde todos los rayos de luz convergen hacia adentro debido a la gravedad extrema. Este resultado implica que, en el colapso gravitacional (por ejemplo, en la formación de un agujero negro), el espacio-tiempo necesariamente desarrolla una singularidad donde las geodésicas terminan, es decir, el espacio-tiempo deja de ser definible en esa región.

\begin{theorem}{Teorema de Singularidad de Hawking (1965)}

\textit{
"Si el espacio-tiempo es globalmente hiperbólico, satisface la condición de energía fuerte, y existe una superficie de Cauchy donde la expansión de las geodésicas de tipo tiempo es positiva en todo momento, entonces necesariamente existe una geodésica de tipo tiempo o de tipo nulo inextensible incompleta en el pasado."
}
\end{theorem}
Es decir, este teorema se aplica al contexto cosmológico. Afirma que, si el universo está en expansión (como lo está nuestro universo observable), entonces debe haber existido una singularidad en el pasado (el Big Bang) que marcó el inicio del espacio-tiempo.

\begin{theorem}{Teorema de Singularidad de Hawking-Penrose (1970)}

\textit{
"Si un espacio-tiempo contiene una trampa causal (como un horizonte de eventos), satisface la condición de energía media y es globalmente hiperbólico en el sentido de que no contiene curvas cerradas de tipo tiempo, entonces debe contener al menos una geodésica de tipo tiempo o de tipo nulo inextensible incompleta."
}
\end{theorem}
Es decir, este teorema generaliza los anteriores y establece que, bajo condiciones físicas razonables (como la validez de la relatividad general y la ausencia de curvas cerradas de tipo tiempo), las singularidades son inevitables tanto en el colapso gravitacional como en la expansión inicial del universo.


%SECCION 1
\section{Constante cosmológica} % Main chapter title
\label{cap4-sec7} 

La constante cosmológica $\Lambda$ es una consecuencia natural de las ecuaciones de Einstein, pues el tensor de Einstein $G_{\mu\nu}$ no está unívocamente definido, 
\[G_{\mu\nu}\longrightarrow G_{\mu\nu}+\Lambda g_{\mu\nu}\equiv \hspace{0mm}^{\Lambda}G_{\mu\nu}\]
Pero esta forma del tensor no es compatible con un universo estático, es decir, si $\Lambda\neq0$, tendremos un universo dinámico (en expansión o contracción, dependiendo del signo de $\Lambda$). También se cumple que
\[\nabla^{\mu}( \hspace{0mm}^{\Lambda}G_{\mu\nu})=0\]
Originalmente, la constante cosmológica fue introducida por Einstein para describir un universo estático. Hoy forma parte del modelo estándar de cosmología $\Lambda$CDM. Además, esta constante puede relacionarse con el fluido que permea el universo, \textit{''Dark Energy''}.\\ \\
La constante cosmológica se puede interpretar como un fluido perfecto, que las ecuaciones de Einstein
\[G_{\mu\nu}+\Lambda g_{\mu\nu}=\kappa T_{\mu\nu}\]
podemos reescribirlas como,
\[G_{\mu\nu}=\kappa(T_{\mu\nu}+\hspace{0mm}^{(\Lambda)}T_{\mu\nu});\hspace{5mm}\hspace{0mm}^{(\Lambda)}T_{\mu\nu}=\frac{-\Lambda}{8\pi G}g_{\mu\nu}\]
Teniendo una densidad de energía del fluido de la constante cosmológica tal que
\[\hspace{0mm}^{(\Lambda)}T_{\mu\nu}u^{\mu}u^{\nu}=\rho_{\Lambda}=\frac{\Lambda}{8\pi G}\]
y una presión 'cosmológica',
\[P_{\Lambda}=\frac{1}{3}(u^{\mu}u^{\nu}+g^{\mu\nu})\hspace{0mm}^{(\Lambda)}T_{\mu\nu}=-\frac{\Lambda}{8\pi G}\]
En resumen, tenemos la ecuación de estado,
\begin{equation}
    P_{\Lambda}=-\rho_{\Lambda}
\end{equation}
La presión negativa hace que el universo se expanda y evita que colapse. Además, si el universo está colapsando, la constante cosmológica contribuye a este colapso.\\ \\
El valor actual de la constante cosmológica es $\Lambda=10^{-52}$ m$^{-2}$.








%SECCION 4
\section{Contracción de longitudes} % Main chapter title
\label{cap2-sec4} 
%------------------------------------------------------------------------------
Tomamos dos eventos del espacio-tiempo, tal que
\[\Delta t'=\gamma\left(\Delta t-\frac{v}{c^2}\Delta x\right)\]
\[\Delta x'=\gamma\left(\Delta x-v\Delta t\right)\]
Asumimos que tomamos eventos que no están separados temporalmente, es decir, como si en $S'$ tomásemos una foto, así, $\Delta t'=0$. Por tanto, tendremos que $\Delta x'=L'$ y $\Delta x=L$. Luego, sustituyendo tenemos que
\[\Delta x'=\frac{\Delta x}{\gamma}\Longrightarrow L'=\frac{L}{\gamma}\]
Además, como $\gamma>1$, tendremos que $L>L'$, por tanto, se habla de contracción de longitudes; donde $L$ se conoce como \textbf{longitud propia}, que es la longitud del objeto respecto a un SRI en reposo respecto al objeto, es decir, el SRI centro de masas del objeto.
\begin{note}
    Las transformaciones de Lorentz dejan invariante las distancias espacio-temporales, pues dados dos eventos $(t_1,x_1)$ y $(t_2,x_2)$ en $S$, y los eventos correspondientes $(t_1',x_1')$ y $(t_2',x_2')$ en $S'$, entonces
    \[-c^2(t_2-t_1)^2+(x_2-x_1)^2=-c^2(t_2'-t_1')+(x_2'-x_1')^2\]
    por tanto, tenemos una cantidad que es invariante al SRI.
\end{note}
%SECCION 3
\section{Dilatación temporal} % Main chapter title
\label{cap2-sec3} 
%------------------------------------------------------------------------------
La dilatación temporal es una causa directa de los postulados de Einstein. Veámoslo con un esquema,
\begin{multicols}{2}
    \begin{Figura}
        \centering
        \includegraphics[width=0.8\textwidth]{Capitulos/Capitulo2/Seccion3/Lrep.png}
        \captionof{figure}{Espejos en reposo.}
        \label{fig2.1}
    \end{Figura}
    \begin{Figura}
        \centering
        \includegraphics[width=0.8\textwidth]{Capitulos/Capitulo2/Seccion3/Lmov.png}
        \captionof{figure}{Espejos en movimiento.}
        \label{fig2.2}
    \end{Figura}
\end{multicols}
Si nos fijamos en la Figura \ref{fig2.1}, al estar los espejos en reposo, el rayo de luz que sale de la linterna vuelve en un tiempo $\Delta t=\frac{2l_0}{c}$. En cambio, suponiendo que los espejos se mueven a velocidad $\vec{v}$, y que la distancia de los brazos del rayo es $D$, entonces ahora el tiempo que tarda el rayo en ir y volver es $\Delta t'=\frac{2D}{c}$. Usando el Teorema de Pitágoras podemos calcular $D$, tal que
\[D^2=l_0^2+\left(\frac{\Delta t'v}{2}\right)^2\]
Sustituyendo $D$ y $l_0$ de las ecuaciones de $\Delta t$ y $\Delta t'$, tenemos
\[\left(\frac{\Delta t'c}{2}\right)^2=\left(\frac{\delta t'v}{2}\right)^2+\left(\frac{\Delta tc}{2}\right)^2\]
Por tanto, tenemos que el tiempo se dilata de la forma,
\begin{equation}
    \Delta t'=\gamma\Delta t
\end{equation}
y como $\gamma>1$ siempre, entonces $\Delta t'>\Delta t$, por eso se dilata el tiempo.\\ \\
Vemos que en el SRI $S'$ los relojes van más lento que en el SRI $S$, pues si consideramos como reloj el rebote de los fotones en los espejos, entonces en $S$ los fotones van más rápido que los fotones en $S'$.

%SECCION 1
\section{Breve repaso histórico} % Main chapter title
\label{cap3-sec1} 
%------------------------------------------------------------------------------

La Geometría Diferenciable surge de una generalización de la geometría euclidiana que lleva a cabo Riemann, el cuál lo presentó en la Universidad de Gotinga, en su lección inaugural con el título \textit{"Sobre las hipótesis en las que se basa la geometría"}, teniendo en el tribunal al mismísimo Gauss.\\ \\
Hasta entonces, la geometría se había estudiado siempre en el espacio tridimensional euclídeo, y el concepto de 'curvatura' se concebía únicamente para curvas y superficies en el espacio, mientras que Riemann planteó poder trabajar con 'espacios coordenados' sin suponerlos contenidos en el espacio euclídeo ni en ningún otro espacio. Puso en evidencia su relación con la curvatura que Gauss había definido para superficies en el espacio euclídeo, abriendo las puertas a definir un  concepto general de curvatura que permitía afirmar, por ejemplo, que un espacio tridimensional fuera 'curvo', cosa inconcebible hasta entonces.\\ \\
Una vez asentado los conceptos y demostrado la congruencia de la exposición de Riemann, pasamos al siglo XX, cuando Albert Einstein toma el tratado de Levi-Civita para estudiar el cálculo tensorial que usaría para desarrollar la teoría general de la relatividad. Por aquel entonces, el cálculo tensorial era un mar de subíndices y superíndices que subían y bajaban de una fórmula a otra, pero no tardaron muchos matemáticos en encontrar un enfoque más conceptual que permitiera llegar a una comprensión más profunda de la teoría, desarrollando una 'geometría sin índices' o 'intrínseca', en la que los conceptos fundamentales de la geometría diferenciable son objetos algebraicos abstractos globales, y las expresiones coordenadas (con índices), son solo representaciones auxiliares locales, que en ocasiones son convenientes de usar.
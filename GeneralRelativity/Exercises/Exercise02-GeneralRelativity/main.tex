\documentclass[11pt]{article}
%\usepackage[spanish]{babel}
\RequirePackage{etex}
\usepackage[utf8]{inputenc}
\usepackage{braket}
%\usepackage[sc]{mathpazo}
% \linespread{1.5}
%\usepackage[T1]{fontenc}
%\usepackage{heuristica}
%\usepackage[erewhon,vvarbb,bigdelims]{newtxmath}
%\renewcommand*\oldstylenums[1]{\textosf{#1}}
\usepackage{enumitem}
\usepackage{array}
\usepackage{textcomp}
\usepackage{fancyhdr}
\usepackage{amsmath, amsthm}
\usepackage{slashed}
\usepackage[normalem]{ulem}
\usepackage{amsfonts}
\usepackage{amssymb}
\usepackage{mathtools}
\usepackage{float}
\usepackage{soul}
\usepackage{graphicx}
\usepackage{hyperref}
\usepackage{graphicx}
\usepackage{pstricks-add}
\usepackage{color}
\usepackage{caption}
\usepackage[margin=0.9in]{geometry}
\usepackage{marvosym}
\usepackage{mathtools}
\usepackage{framed}
\usepackage{calrsfs}
\usepackage[mathscr]{euscript}
\usepackage{tensor}
% \usepackage{autonum}
\usepackage{cancel}
\usepackage[most]{tcolorbox}
\usepackage{fancyhdr} % Headers and footers
\DeclareMathAlphabet{\mathcal}{OMS}{cmsy}{m}{n}

 % All pages have headers and footers
% Estilo de encabezado personalizado
\fancyhf{} % Limpia todos los campos de encabezado y pie de página
\renewcommand{\headrulewidth}{0pt} % Elimina la línea horizontal en el encabezado
\fancyhead[RO,LE]{\thepage} % Números de página en el encabezado, en posición derecha en páginas impares y en posición izquierda en páginas pares

% Configuración para las páginas iniciales de capítulos
\fancyhead[R]{Rubén Carrión Castro}
\fancyhead[L]{Entregable II}

\newtheorem{thm}{Teorema}[section]
\newtheorem{theorem}{Teorema}[section]
\newtheorem{proposition}[thm]{Proposición} 
\newtheorem{lemma}[thm]{Lema}
\newtheorem{corollary}[thm]{Corolario} 
\newtheorem{conv}[thm]{Convención}
\newtheorem{defi}[thm]{Definición}
\newtheorem{definition}[theorem]{Definición}
\newtheorem{notation}[thm]{Notación} 
\newtheorem{exe}[thm]{Ejemplo}
\newtheorem{conjecture}[thm]{Conjetura} 
\newtheorem{prob}[thm]{Problema}
\newtheorem{remark}[thm]{Observación}
\newtheorem{example}[thm]{Ejemplo}
\newtheorem{note}[thm]{Nota}

\newcommand{\brackets}[1]{\left[#1\right]}
\newcommand{\curlybraces}[1]{\left\{#1\right\}}
\newcommand{\qedh}{\hfill\hspace{5mm}\fbox{\phantom{\rule{.5ex}{.5ex}}}}
\newcommand{\scalar}[2]{\langle #1, #2 \rangle}
\newcommand{\ptensor}[2]{#1 \otimes #2}
\newcommand{\pcart}[2]{#1 \times #2}
\newcommand{\vvector}[4]{\begin{pmatrix}#1\\ #2\\ #3\\#4\end{pmatrix}}
\newcommand{\covector}[4]{\begin{pmatrix}#1 & #2 & #3 & #4\end{pmatrix}}
\newcommand{\abss}[1]
{\begin{vmatrix}#1\end{vmatrix}^2}
\newcommand{\upmunu}[2]{#1^{\mu}#2^{\nu}}
\newcommand{\upnumu}[2]{#1^{\nu}#2^{\mu}}
\newcommand{\downmunu}[2]{#1_{\mu}#2_{\nu}}
\newcommand{\downnumu}[2]{#1_{\nu}#2_{\mu}}
\newcommand{\prima}[1]{#1'}
\newcommand{\ricci}[2]{\mathcal{R}_{#1 \rho #2}^{\rho}=\partial_{\lambda}\Gamma_{#1 #2}^{\lambda}-\partial_{#2}\Gamma_{#1 \lambda}^{\lambda}+\Gamma_{#1 #2}^{\sigma}\Gamma_{\sigma\lambda}^{\lambda}-\Gamma_{#1 \lambda}^{\sigma}\Gamma_{\sigma #2}^{\lambda}}

\newcommand{\riccis}[4]{\partial_{#3}\Gamma_{#1 #2}^{#3}-\partial_{#2}\Gamma_{#1 #3}^{#3}+\Gamma_{#1 #2}^{#4}\Gamma_{#4 #3}^{#3}-\Gamma_{#1 #3}^{#4}\Gamma_{#4 #2}^{#3}}

\newcommand{\ricciss}[5]{\partial_{#3}\Gamma_{#1 #2}^{#3}-\partial_{#2}\Gamma_{#1 #3}^{#3}+\Gamma_{#1 #2}^{#4}\Gamma_{#4 #3}^{#3}-\Gamma_{#1 #3}^{#4}\Gamma_{#4 #2}^{#3}+\Gamma_{#1 #2}^{#5}\Gamma_{#5 #3}^{#3}-\Gamma_{#1 #3}^{#5}\Gamma_{#5 #2}^{#3}}

\newtcolorbox[auto counter, number within=section]{mytheorem}[2][]{
  enhanced,
  breakable,
  title=Teorema~\thetcbcounter: #2,
  #1,
}
\newtcolorbox[auto counter, number within=section]{propositionbox}[2][]{
  enhanced,
  breakable,
  title=Proposition~\thetcbcounter: #2,
  #1,
}

\newtcolorbox[auto counter, number within=section]{corollarybox}[2][]{
  enhanced,
  breakable,
  title=Corollary~\thetcbcounter: #2,
  #1,
}

\newtcolorbox[auto counter, number within=section]{remarkbox}[2][]{
  enhanced,
  breakable,
  title=Remark~\thetcbcounter: #2,
  #1,
}

\newtcolorbox[auto counter, number within=section]{notebox}[2][]{
  enhanced,
  breakable,
  title=Note~\thetcbcounter: #2,
  #1,
}


\newenvironment{Figura}
  {\par\medskip\noindent\minipage{\linewidth}}
  {\endminipage\par\medskip}
%\usepackage[spanish]{babel}
\title{\huge{\textbf{Entregable II. Relatividad General}}}
\author{\textbf{}\\ \\Rubén Carrión Castro\\}
% \textit{Los Chavales}
\date{Octubre 2024}
\begin{document}
\maketitle
\setcounter{page}{1}
\pagestyle{fancy}
\begin{enumerate}
\item \textbf{Considera el sistema gravedad-materia dado por la acción:}
\[\mathbf{S=\int d^4x\sqrt{-g}\left(\frac{1}{2\kappa}\mathcal{R}-\frac{1}{4}F_{\mu\nu}F^{\mu\nu}\right)}\]
\textbf{donde }$\mathbf{\kappa=8\pi G}$\textbf{, }$\mathcal{R}$\textbf{ el escalar de Ricci, }$\mathbf{F_{\mu\nu}}$\textbf{ es el tensor electromagnético, y la métrica en coordenadas cartesianas viene dada por:}
\[\mathbf{ds^2=(1+\kappa z^2)dt^2-(1-\kappa y^2)dx^2-(1-\kappa y^2)^{-1}dy^2-(1+\kappa z^2)^{-1}dz^2}\]
\begin{enumerate}
    \item \textbf{¿Cuáles son los vectores de Killing asociados a esta métrica?}
    \item \textbf{Calcula el tensor de Ricci y el escalar de curvatura.}
    \item \textbf{Resuelve las ecuaciones de Einstein y calcula las componentes no nulas del tensor electromagnético.}
\end{enumerate}
\end{enumerate}
\newpage
Tenemos la métrica
\begin{equation}
ds^2=(1+\kappa z^2)dt^2-(1-\kappa y^2)dx^2-(1-\kappa y^2)^{-1}dy^2-(1+\kappa z^2)^{-1}dz^2
\label{ec1}
\end{equation}
Por tanto, sabiendo que $ds^2=g_{\mu\nu}dx^{\mu}dx^{\nu}$, el tensor métrico vendrá dado por
\begin{equation}
    g_{\mu\nu}\equiv\begin{pmatrix}
        1+\kappa z^2 & 0 & 0 & 0 \\
        0 & -(1-\kappa y^2) & 0 & 0 \\
        0 & 0 & -(1-\kappa y^2)^{-1} & 0 \\
        0 & 0 & 0 & -(1+\kappa z^2)^{-1}
    \end{pmatrix}
\end{equation}
donde vemos que es una matriz diagonal, y por tanto, $g^{\mu\nu}=1/g_{\mu\nu}$.
\subsubsection*{Apartado (a)}

 Sabemos que la ecuación de Killing es
\begin{equation}
\nabla_\mu k^{\nu}+\nabla_\nu k^{\mu}=0
\label{ec2}
\end{equation}
que viene de la ecuación
\begin{equation}
\mathscr{L}_kg_{\mu\nu}=k^{\rho}\partial_{\rho}g_{\mu\nu}+g_{\rho\nu}\partial_{\mu}k^{\rho}+g_{\mu\rho}\partial_{\nu}k^{\rho}\equiv0
\label{ec3}
\end{equation}
donde para llegar a (\ref{ec2}) hemos asumido que la variedad está equipada con la conexión de Levi-Civita. Además, sabemos que los campos vectoriales que solucionen la ecuación (\ref{ec2}) serán los denominados vectores de Killing.\\ \\
Pero resolver esto aplicando la ecuación nos lleva a resolver un sistema de mínimo 10 EDPs, por lo que vamos a intentar buscar otra forma. Analicemos la métrica para ello,
\[ds^2=\underbrace{(1+\kappa z^2)dt^2-(1+\kappa z^2)^{-1}dz^2}_{(1)}-\underbrace{(1-\kappa y^2)dx^2-(1-\kappa y^2)^{-1}dy^2}_{(2)}\]
Haciendo en (1) el cambio de variable,
\[z=r;\hspace{5mm}\kappa=\frac{1}{R_0}\Longrightarrow(1)=\left(1+\frac{r^2}{R_0^2}\right)dt^2-\left(1+\frac{r^2}{R_0^2}\right)^{-1}dz^2\]
tenemos la métrica de anti-De Sitter dos-dimensional.\\
Para encontrar el cambio de variable en (2), debemos resolver la ecuación diferencial siguiente,
\[\alpha\equiv\int\frac{dy}{\sqrt{1-\kappa y^2}}=\frac{\arcsin(\sqrt{\kappa}y)}{\sqrt{\kappa}}+C\]
tomando $C=0$ por simplicidad, tenemos el cambio de variable,
\[y=\frac{\sin(\sqrt{\kappa}\alpha)}{\sqrt{\kappa}}\Longrightarrow(1-\kappa y^2)=(1-\sin^2(\sqrt{\kappa}\alpha))=\cos^2(\sqrt{\kappa}\alpha)\]
Por tanto,
\[(2)=-\cos^2(\sqrt{\kappa}\alpha)d\varphi^2-d\alpha^2\]
Redefinimos $\theta=\sqrt{\kappa}\alpha+\frac{\pi}{2}$, así $\cos(\sqrt{\kappa}\alpha)=\sin\theta$ y $d\theta=\sqrt{kappa}d\alpha$. Así tenemos,
\[y=\frac{\cos\theta}{\sqrt{\kappa}}\Longrightarrow 1-\kappa y^2=1-\cos^2\theta=\sin^2\theta\]
y por tanto, tomando $x=\sqrt{\kappa}\varphi$, tenemos
\[(2)=-\kappa(\sin^2\theta d\varphi^2+d\theta^2)\]
 siendo la métrica de una esfera bidimensional de radio unidad.\\ \\
 Así tenemos la métrica,
 \[ds^2=\underbrace{\left(1+\frac{r^2}{R_0^2}\right)dt^2-\left(1+\frac{r^2}{R_0^2}\right)^{-1}dz^2}_{\text{Métrica de anti-De Sitter}}-\underbrace{\kappa(\sin^2\theta d\varphi^2+d\theta^2)}_{\text{Métrica de una esfera}}\]
 Denotaremos el espacio de anti-De Sitter como $AdS^2$ y el espacio de la esfera unidad bidimensional como $\mathbb{S}^2$.\\
 Por tanto, esta métrica describe un espacio tensorial, tal que
 \[AdS^2\otimes\mathbb{S}^2\]
Teniendo así un nuevo tensor métrico,
\[\overline{g}_{\mu\nu}\equiv\begin{pmatrix}
    1+\frac{r^2}{R_0^2} & 0 & 0 & 0\\
    0 & -\left(1+\frac{r^2}{R_0^2}\right)^{-1} & 0 & 0\\
    0 & 0 & -\kappa\sin^2\theta & 0 \\
    0 & 0 & 0 & -\kappa
\end{pmatrix}\]
La construcción de los vectores de Killing en el espacio $AdS^2\otimes\mathbb{S}^2$, implica una combinación de los vectores de Killing asociados a los factores $AdS^2$ y $\mathbb{S}^2$, de manera que respeten la estructura del espacio.\\ \\
Para $AdS^2$, el grupo isométrico es $S=(2,1)$, y hay 3 vectores de Killing asociados a las isometrías del espacio anti-De Sitter bidimensional, que son
\[\begin{array}{rcl}
    \overline{\xi}^{(1)}_{AdS^2} & = & \partial_t \\ \\
    \overline{\xi}^{(2)}_{AdS^2} & = & t\partial_t+r\partial_r\\ \\
    \overline{\xi}^{(3)}_{AdS^2} & = & \left(t^2+\frac{1}{1+r^2/R_0^2}\right)\partial_t+2tr\partial_r
\end{array}\]
Para $\mathbb{S}^2$, el grupo isométrico es $SO(3)$, y hay 3 vectores de Killing asociados a las rotaciones en la esfera bidimensional, que son
\[\begin{array}{rcl}
    \overline{\xi}^{(4)}_{\mathbb{S}^2} & = & \partial_{\varphi} \\ \\
    \overline{\xi}^{(5)}_{\mathbb{S}^2} & = & -\sin(\varphi)\partial_{\theta}-\cot(\theta)\cos(\varphi)\partial_{\varphi}\\ \\
    \overline{\xi}^{(6)}_{\mathbb{S}^2} & = & \cos(\varphi)\partial_{\theta}-\cot(\theta)\sin(\varphi)\partial_{\varphi}
\end{array}\]
Teniendo así un total de 6 vectores de Killing en el espacio $AdS^2\otimes\mathbb{S}^2$. Además, como este espacio tiene las dos métricas separadas, significa que los dos factores son ortogonales entre sí, y las isometrías de $AdS^2$ no afectan las coordenadas de $\mathbb{S}^2$, y viceversa.\\ \\
Por tanto, los vectores de Killing del espacio completo se obtienen como una combinación directa de los vectores de Killing de cada factor. Formalmente, esto significa que:
\begin{itemize}
    \item Cada vector de Killing de $AdS^2$ se extiende al espacio total como un campo vectorial que actúa solo sobre las coordenadas de $AdS^2$, dejando inalteradas las coordenadas de $\mathbb{S}^2$. Así, si $\xi^{(i)}_{AdS^2}$ es un vector de Killing de $AdS^2$, entonces en el espacio total se representa como,
    \[\xi^{(i)}=(\xi^{(i)}_{AdS^2},0)\]
    donde el "0" indica que no hay componentes en las direcciones de $\mathbb{S}^2$.
    \item De manera análoga, cada vector de Killing de $\mathbb{S}^2$ se extiende al espacio total como un campo vectorial que actúa solo sobre las coordenadas de $\mathbb{S}^2$, dejando onalteradas las coordenadas de $AdS^2$, tal que
    \[\xi^{(j)}=(0,\xi^{(j)}_{\mathbb{S}^2})\]
\end{itemize}
Por tanto, el conjunto completo de vectores de Killing de $AdS^2\otimes\mathbb{S}^2$ es simplemente la unión disjunta de los vectores de Killing de $AdS^2$ y $\mathbb{S}^2$, tal que
\[\curlybraces{\xi^{(i)}_{AdS^2}\text{ extendidos a }AdS^2\otimes\mathbb{S}^2}\bigcup\curlybraces{\xi^{(j)}_{\mathbb{S}^2}\text{ extendidos a }AdS^2\otimes\mathbb{S}^2}\]
teniendo un total de 6 vectores de Killing. Así, los vectores de Killing del espacio total $AdS^2\otimes\mathbb{S}^2$ son,
\[\begin{array}{ll}
     \overline{\xi}_{(1)}=\partial_t;  & \overline{\xi}_{(4)}=\partial_{\varphi} \\ \\
    \overline{\xi}_{(2)}=t\partial_t+r\partial_r; & \overline{\xi}^{(5)}= -\sin(\varphi)\partial_{\theta}-\cot(\theta)\cos(\varphi)\partial_{\varphi}\\ \\
    \overline{\xi}^{(3)}=\left(t^2+\frac{1}{1+r^2/R_0^2}\right)\partial_t+2tr\partial_r; & \overline{\xi}^{(6)}=\cos(\varphi)\partial_{\theta}-\cot(\theta)\sin(\varphi)\partial_{\varphi}
\end{array}\]
Ahora debemos calcular los vectores de Killing de la métrica original, sabiendo que transforman como vectores bajo cambios de coordenadas, tal que
\begin{equation}
    \xi^{\mu}=\frac{\partial x^{\mu}}{\partial\overline{x}^{\nu}}\overline{\xi}^{\nu}
\end{equation}
Tenemos,
\[\begin{array}{ccccc}
    g_{\mu\nu} & \longrightarrow & \overline{g}_{\mu\nu} & \Longrightarrow & \text{Derivadas no nulas}\\ \\
    \begin{matrix}
        x^t=\partial_t & t=t
    \end{matrix} & & \begin{matrix}
        \overline{x}^t=\partial_t
    \end{matrix} & & \frac{\partial x^t}{\partial\overline{x}^t}=1\\ \\
    \begin{matrix}
        x^x=\partial_x & x=\sqrt{\kappa}\varphi
    \end{matrix} & & \begin{matrix}
        \overline{x}^r=\partial_r 
    \end{matrix} & & \frac{\partial x^x}{\partial\overline{x}^{\varphi}}=\sqrt{\kappa} \\ \\
    \begin{matrix}
        x^y=\partial_y & y=\cos(\theta)/\sqrt{\kappa}
    \end{matrix} & & \overline{x}^{\varphi}=\partial_{\varphi} & & \frac{\partial x^y}{\partial\overline{x}^{\theta}}=-\frac{\sin\theta}{\sqrt{\kappa}}\\ \\
    \begin{matrix}
        x^z=\partial_z & z=r
    \end{matrix} & & \overline{x}^{\theta}=\partial_{\theta} & & \frac{\partial x^z}{\partial\overline{x}^r}=1
\end{array}\]
Así tenemos,
\[\xi_{(1)}^{\mu}=\frac{\partial x^{\mu}}{\partial\overline{x}^{\nu}}\overline{\xi}_{(1)}^{\nu}=\frac{\partial x^{\mu}}{\partial\overline{x}^{\nu}}\delta^{\nu}_{\mu}=\frac{\partial x^{\mu}}{\partial\overline{x}^t}=\delta_t^{\mu}\equiv(1,0,0,0)\]

\[\xi_{(2)}^{\mu}=\frac{\partial x^{\mu}}{\partial\overline{x}^{\nu}}\overline{\xi}_{(2)}^{\nu}=\frac{\partial x^{\mu}}{\partial\overline{x}^{\nu}}(t\delta_t^{\nu}+r\delta_r^{\nu})=\frac{\partial x^{\mu}}{\partial\overline{x}^{t}}t+\frac{\partial x^{\mu}}{\partial\overline{x}^r}r=t\delta_t^{\mu}+z\delta_z^{\mu}\equiv(t,0,0,z)\]

\[\xi_{(3)}^{\mu}=\frac{\partial x^{\mu}}{\partial\overline{x}^{\nu}}\overline{\xi}_{(3)}^{\nu}=\frac{\partial x^{\mu}}{\partial\overline{x}^{\nu}}\brackets{\left(t^2+\frac{1}{1+r^2/R_0^2}\right)\delta^{\nu}_t+2tr\delta^{\nu}_r}=\frac{\partial x^{\mu}}{\partial\overline{x}^{t}}\left(t^2+\frac{1}{1+r^2/R_0^2}\right)+\frac{\partial x^{\mu}}{\partial\overline{x}^{r}}2tr=\]
\[=\left(t^2+\frac{1}{1+r^2/R_0^2}\right)\delta_t^{\mu}+2tr\delta_r^{\mu}=\left(t^2+\frac{1}{(1+\kappa z^2)}\right)\delta_t^{\mu}+2tz\delta_z^{\mu}\equiv\left(t^2+\frac{1}{(1+\kappa z^2)},0,0,2tz\right)\]

\[\xi_{(4)}^{\mu}=\frac{\partial x^{\mu}}{\partial\overline{x}^{\nu}}\overline{\xi}_{(4)}^{\nu}=\frac{\partial x^{\mu}}{\partial\overline{x}^{\nu}}\delta_{\varphi}^{\nu}=\frac{\partial x^{\mu}}{\partial\overline{x}^{\varphi}}=\sqrt{\kappa}\delta_{\varphi}^{\mu}=\delta_x^{\mu}\equiv(0,1,0,0)\]
donde usamos $\varphi=x/\sqrt{\kappa}$.

\[\xi_{(5)}^{\mu}=\frac{\partial x^{\mu}}{\partial\overline{x}^{\nu}}\overline{\xi}_{(5)}^{\nu}=\frac{\partial x^{\mu}}{\partial\overline{x}^{\nu}}\brackets{-\sin(\varphi)\delta_{\theta}^{\nu}-\cot(\theta)\cos(\varphi)\delta_{\varphi}^{\nu}}=\frac{\partial x^{\mu}}{\partial\overline{x}^{\theta}}(-\sin(\varphi))+\frac{\partial x^{\mu}}{\partial\overline{x}^{\varphi}}(-\cot(\theta)\cos(\varphi))=\]
\[=\frac{\sin(\theta)\sin(\varphi)}{\sqrt{\kappa}}\delta_{\theta}^{\mu}-\sqrt{\kappa}\cot(\theta)\cos(\varphi)\delta_{\varphi}^{\mu}=\frac{\sin(\arccos(\sqrt{\kappa}y))}{\sqrt{\kappa}}\sin\left(\frac{x}{\sqrt{\kappa}}\right)\delta_y^{\mu}-\sqrt{\kappa}\cot(\arccos(\sqrt{\kappa}y))\cos\left(\frac{x}{\sqrt{\kappa}}\right)\delta_x^{\mu}\equiv\]
\[\equiv\left(0,-\sqrt{\kappa}\cot(\arccos(\sqrt{\kappa}y))\cos\left(\frac{x}{\sqrt{\kappa}}\right),\frac{\sin(\arccos(\sqrt{\kappa}y))}{\sqrt{\kappa}}\sin\left(\frac{x}{\sqrt{\kappa}}\right),0\right)\]

donde usamos que $\theta=\arccos(\sqrt{\kappa}y)$ y $\varphi=x/\sqrt{\kappa}$.

\[\xi_{(6)}^{\mu}=\frac{\partial x^{\mu}}{\partial\overline{x}^{\nu}}\overline{\xi}_{(6)}^{\nu}=\frac{\partial x^{\mu}}{\partial\overline{x}^{\nu}}\brackets{\cos(\varphi)\delta_{\theta}^{\nu}-\cot(\theta)\sin(\varphi)\delta_{\varphi}^{\nu}}=\frac{\partial x^{\mu}}{\partial\overline{x}^{\theta}}\cos(\varphi)-\frac{\partial x^{\mu}}{\partial\overline{x}^{\varphi}}\cot(\theta)\sin(\varphi)=\]
\[=\frac{\sin(\theta)\sin(\varphi)}{\sqrt{\kappa}}\delta_{\theta}^{\mu}-\sqrt{\kappa}\cot(\theta)\sin(\varphi)\delta_{\varphi}^{\mu}=\frac{\sin(\arccos(\sqrt{\kappa}y))}{\sqrt{\kappa}}\cos\left(\frac{x}{\sqrt{\kappa}}\right)\delta_y^{\mu}-\sqrt{\kappa}\cot(\arccos(\sqrt{\kappa}y))\sin\left(\frac{x}{\sqrt{\kappa}}\right)\delta_x^{\mu}\equiv\]
\[\equiv\left(0,-\sqrt{\kappa}\cot(\arccos(\sqrt{\kappa}y))\sin\left(\frac{x}{\sqrt{\kappa}}\right),\frac{\sin(\arccos(\sqrt{\kappa}y))}{\sqrt{\kappa}}\cos\left(\frac{x}{\sqrt{\kappa}}\right),0\right)\]

donde usamos que $\theta=\arccos(\sqrt{\kappa}y)$ y $\varphi=x/\sqrt{\kappa}$.\\ \\
Por tanto, los 6 vectores de Killing de esta métrica son,

\[
\xi_{(1)}^{\mu}=(1,0,0,0)\]
\[
\xi_{(2)}^{\mu}=(t,0,0,z)\]
\[\xi_{(3)}^{\mu}=\left(t^2+\frac{1}{\kappa(1+\kappa z^2)},0,0,2tz\right)
\]
\[\xi_{(4)}^{\mu}=(0,1,0,0)
\]
\[\xi_{(5)}^{\mu}=\left(0,-\sqrt{\kappa}\cot(\arccos(\sqrt{\kappa}y))\cos\left(\frac{x}{\sqrt{\kappa}}\right),\frac{\sin(\arccos(\sqrt{\kappa}y))}{\sqrt{\kappa}}\sin\left(\frac{x}{\sqrt{\kappa}}\right),0\right)\]
\[\xi_{(6)}^{\mu}=\left(0,-\sqrt{\kappa}\cot(\arccos(\sqrt{\kappa}y))\sin\left(\frac{x}{\sqrt{\kappa}}\right),\frac{\sin(\arccos(\sqrt{\kappa}y))}{\sqrt{\kappa}}\cos\left(\frac{x}{\sqrt{\kappa}}\right),0\right)\]
\subsubsection*{Apartado (b)}
Sabemos que el tensor de Ricci viene dado por el tensor de Riemann, que es
\begin{equation}
    \mathcal{R}_{\mu\nu\rho}^{\lambda}=\partial_{\nu}\Gamma_{\mu\rho}^{\sigma}-\partial_{\mu}\Gamma_{\nu\rho}^{\sigma}+\Gamma_{\mu\rho}^{\lambda}\Gamma_{\lambda\nu}^{\sigma}-\Gamma_{\nu\rho}^{\lambda}\Gamma_{\lambda\mu}^{\sigma}
\end{equation}
donde los $\Gamma_{\mu\rho}^{\lambda}$ y derivados, son lo símbolos de Christoffel. El tensor de Ricci viene dado por,
\begin{equation}
\mathcal{R}_{\mu\nu}=\mathcal{R}_{\mu\rho\nu}^{\rho}=\partial_{\lambda}\Gamma_{\mu\nu}^{\lambda}-\partial_{\nu}\Gamma_{\mu\lambda}^{\lambda}+\Gamma_{\mu\nu}^{\sigma}\Gamma_{\sigma\lambda}^{\lambda}-\Gamma_{\mu\lambda}^{\sigma}\Gamma_{\sigma\nu}^{\lambda}
\end{equation}
y el escalar de curvatura o escalar de Ricci viene dadod por,
\begin{equation}
    \mathcal{R}=g^{\mu\nu}\mathcal{R}_{\mu\nu}
\end{equation}
es decir, es la traza del tensor de Ricci.\\ \\
Primero debemos calcular los símbolos de Christoffel de las componentes, de forma general vienen dados por,
\begin{equation}
    \Gamma_{\mu\nu}^{\lambda}=\frac{1}{2}g^{\lambda\sigma}\left(\partial_{\mu}g_{\nu\sigma}+\partial_{\nu}g_{\mu\sigma}-\partial_{\sigma}g_{\mu\nu}\right)
    \label{ec5}
\end{equation}
Vemos que como las componentes del tensor métrico que no están en la diagonal se anulan, y que las componentes de la diagonal solo dependen de alguna de las variables $(t,x,y,z)$, se anulan bastantes símbolos de Christoffel, pues las derivadas que permanecen no nulas son las siguientes,
\[\partial_zg_{tt} = 2\kappa z; \hspace{5mm} \partial_yg_{yy} =\frac{-2\kappa y}{(1-\kappa y^2)^2};\]
\[\partial_yg_{xx} = 2\kappa y; \hspace{5mm} \partial_zg_{zz} =\frac{2\kappa z}{(1+\kappa z^2)^2};\]
Nos damos cuenta que las derivadas no nulas son para los coeficientes del tensor métrico iguales, es decir, en la ecuación (\ref{ec5}), tenemos que $\lambda=\sigma$. Pero además, en las parciales solo están $z$ e $y$, por lo que $\mu$ y $\nu$ son $z$ ó $y$. Además, sabemos que los símbolos de Christoffel son simétricos, es decir, $\Gamma_{\mu\nu}^{\lambda}=\Gamma_{\nu\mu}^{\lambda}$. Así tenemos,
\[\Gamma_{tz}^t=\Gamma_{zt}^{t}=\frac{1}{2}g^{tt}\left(\cancelto{0}{\partial_tg_{zt}}+\partial_{z}g_{tt}-\cancelto{0}{\partial_tg_{tz}}\right)=\frac{1}{2g_{tt}}\partial_zg_{tt}=\frac{1}{2(1+\kappa z^2)}2\kappa z=\frac{\kappa z}{1+\kappa z^2}\]
\[\Gamma_{zz}^z=\frac{1}{2}g^{zz}\left(\cancel{\partial_zg_{zz}}+\partial_{z}g_{zz}-\cancel{\partial_zg_{zz}}\right)=\frac{1}{2g_{zz}}\partial_zg_{zz}=-\frac{(1+\kappa z^2)}{2}\frac{2\kappa z}{(1+\kappa z^2)^2}=\frac{-\kappa z}{(1+\kappa z^2)}\]
\[\Gamma_{xy}^x=\Gamma_{yx}^{x}=\frac{1}{2}g^{xx}\left(\cancelto{0}{\partial_xg_{yx}}+\partial_{y}g_{xx}-\cancelto{0}{\partial_xg_{xy}}\right)=\frac{1}{2g_{xx}}\partial_yg_{xx}=\frac{-1}{2(1-\kappa y^2)}2\kappa y=\frac{-\kappa y}{1-\kappa y^2}\]
\[\Gamma_{yy}^{y}=\frac{1}{2}g^{yy}\left(\cancel{\partial_yg_{yy}}+\partial_{y}g_{yy}-\cancel{\partial_tg_{yy}}\right)=\frac{1}{2g_{yy}}\partial_yg_{yy}=\frac{-1}{2}(1-ky^2)\frac{-2\kappa y}{(1-\kappa y^2)^2}=\frac{\kappa y}{1-\kappa y^2}\]
\[\Gamma_{tt}^z=\frac{1}{2}g^{zz}\left(\cancelto{0}{\partial_tg_{tz}}+\cancelto{0}{\partial_tg_{tz}}-\partial_zg_{tt}\right)=\frac{-1}{2g_{zz}}\partial_zg_{tt}=\frac{1}{2}(1+\kappa z^2)2\kappa z=\kappa z(1+\kappa z^2)\]
\[\Gamma_{xx}^y=\frac{1}{2}g^{yy}\left(\cancelto{0}{\partial_xg_{xy}}+\cancelto{0}{\partial_xg_{xy}}-\partial_yg_{xx}\right)=\frac{-1}{2g_{yy}}\partial_yg_{xx}=\frac{1}{2}(1-\kappa y^2)2\kappa y=\kappa y(1-\kappa y^2)\]
Por tanto, podemos formar el tensor de Ricci componente a componente, sabiendo que algunas de ellas son nulas, pues hay algunos símbolos de Christoffer nulos.\\ \\
Fijándonos en la ecuación (7), podemos ir calculando las componentes del tensor de Ricci, tal que,
\[
    \mathcal{R}_{tt}=\ricci{t}{t}
\]
Si nos fijamos en los términos no nulos de los símbolos de Christofel con $t$, tenemos que son distintos de cero $\Gamma_{tz}^t$ y $\Gamma_{tt}^{z}$. Por tanto, tendremos que $\lambda=t,z$ y $\sigma=t,z$. Así tenemos:
\[\begin{array}{rl}
    \mathcal{R}_{tt} & =\ricciss{t}{t}{t}{t}{z}+\\ \\
    &+\ricciss{t}{t}{z}{t}{z} = \partial_z\Gamma_{tt}^z-\Gamma_{tz}^{t}\Gamma_{tt}^{z}+\Gamma_{tt}^{z}\Gamma_{zz}^{z}=\\ \\
     & = \kappa(1+\kappa z^2)+2\kappa^2z^2-\kappa^2z^2-\kappa^2z^2=\kappa(1+\kappa z^2)\checkmark
\end{array}\]
Seguimos,
\[\mathcal{R}_{ti}=\ricci{t}{i}\]
pero como $i\neq t$, $\Gamma_{ti}^{\lambda}=0$, $\partial_t\Gamma_{\lambda i}^{\lambda}=0$ y $\partial_{\lambda}\Gamma_{ti}^{\lambda}=0$. El único término que puede que no se anule es el tercer término usando $i=z$, pero vemos que si $\sigma=z$, necesariamente para que el primer símbolo no se anule, $\lambda=t$, pero entonces el segundo Christofel se anula, pues $\Gamma_{zz}^t=0$, y si $\sigma=t$, necesariamente para no anularse el primer Christofel, $\lambda=z$, pero vemos que $\Gamma_{zz}^t=0$. Por tanto,
\[\mathcal{R}_{tx}=\mathcal{R}_{xt}=0;\hspace{4mm}\mathcal{R}_{ty}=\mathcal{R}_{yt}=0;\hspace{4mm}\mathcal{R}_{tz}=\mathcal{R}_{zt}=0\]
Seguimos,
\[\mathcal{R}_{ij}=\ricci{i}{j}\]
Si $i\neq j$, entonces $\Gamma_{ij}^{\lambda}=0$, así
\[\mathcal{R}_{xy}=\mathcal{R}_{yx}=\cancelto{0}{-\partial_y\Gamma_{\lambda x}^{\lambda}}+\Gamma_{x\lambda}^{\sigma}\Gamma_{\sigma y}^{\lambda}=\Gamma_{xx}^{x}\cancelto{0}{\Gamma_{xy}^{x}}+\cancelto{0}{\Gamma_{xy}^y}\Gamma_{yy}^{y}=0\]
Análogamente, los demás términos cruzados se anulan,
\[\mathcal{R}_{xy}=\mathcal{R}_{yx}=0;\hspace{4mm}\mathcal{R}_{xz}=\mathcal{R}_{zx}=0;\hspace{4mm}\mathcal{R}_{yz}=\mathcal{R}_{zy}=0\]
Por tanto, nos centramos en $i=j$,
\[\mathcal{R}_{ii}=\ricci{i}{i}\]
que trataremos uno por uno.
\[\begin{array}{rl}
    \mathcal{R}_{xx} & =\ricci{x}{x}=\\ \\
   &= \ricciss{x}{x}{y}{x}{y}+\\ \\
   &+\riccis{x}{x}{x}{y} = \partial_y\Gamma_{xx}^{y}-\Gamma_{xy}^{x}\Gamma_{xx}^{y}+\Gamma_{xx}^{y}\Gamma_{yy}^{y}=\\ \\
   &=\kappa(1-\kappa y^2)-2\kappa^2y^2+\kappa^2y^2+\kappa^2y^2=\kappa(1-\kappa y^2)\checkmark
\end{array}\]
\[\begin{array}{rl}
    \mathcal{R}_{yy} & =\ricci{y}{y}= \\ \\
     & =\ricciss{y}{y}{y}{x}{y}+\\ \\
     &+\ricciss{y}{y}{x}{x}{y}=\\ \\
     &=-\partial_y\Gamma_{yx}^{x}+\Gamma_{yy}^{y}\Gamma_{yx}^{x}-\Gamma_{yx}^{x}\Gamma_{xy}^{x}=\frac{\kappa(1-\kappa y^2)+2\kappa^2y^2}{(1-\kappa y^2)^2}-\frac{\kappa^2y^2}{(1-\kappa y^2)^2}-\frac{\kappa^2y^2}{(1-\kappa y^2)^2}=\frac{\kappa}{(1-\kappa y^2)}\checkmark
\end{array}\]
\[\begin{array}{rl}
    \mathcal{R}_{zz} &=\ricci{z}{z}=  \\ \\
     & =\ricciss{z}{z}{z}{z}{t}+\\ \\
     & +\ricciss{z}{z}{t}{z}{t}=\\ \\
     &=-\partial_z\Gamma_{zt}^{t}+\Gamma_{zz}^{z}\Gamma_{zt}^{t}-\Gamma_{zt}^{t}\Gamma_{tz}^{t} =-\frac{\kappa(1+\kappa z^2)-2\kappa^2z^2}{(1+\kappa^2z^2)^2}-\frac{\kappa^2z^2}{(1+\kappa z^2)^2}-\frac{\kappa^2z^2}{(1+\kappa z^2)^2}=\frac{-\kappa}{(1+\kappa z^2)}\checkmark
\end{array}\]
Por tanto, el tensor de Ricci queda,
\[\mathcal{R}_{\mu\nu}\equiv\begin{pmatrix}
    \kappa(1+\kappa z^2) & 0 & 0 & 0 \\
    0 & \kappa(1-\kappa y^2) & 0 & 0 \\
    0 & 0 & \frac{\kappa}{(1-\kappa y^2)} & 0\\
    0 & 0 & 0 & \frac{-\kappa}{(1+\kappa z^2)}
\end{pmatrix}\]
Por tanto, el escalar de Ricci queda,
\[\mathcal{R}=g^{\mu\nu}\mathcal{R}_{\mu\nu}\]
donde
\[g^{\mu\nu}\equiv\begin{pmatrix}
    (1+\kappa z^2)^{-1} & 0 & 0 & 0\\
    0 & -(1-\kappa y^2)^{-1} & 0 & 0\\
    0 & 0 & -(1-\kappa y^2) & 0 \\
    0 & 0 & 0 & -(1+\kappa z^2)
\end{pmatrix}\]
Por tanto,
\[\small
\mathcal{R}=\begin{pmatrix}
    (1+\kappa z^2)^{-1} & 0 & 0 & 0\\
    0 & -(1-\kappa y^2)^{-1} & 0 & 0\\
    0 & 0 & -(1-\kappa y^2) & 0 \\
    0 & 0 & 0 & -(1+\kappa z^2)
\end{pmatrix}\begin{pmatrix}
    \kappa(1+\kappa z^2) & 0 & 0 & 0 \\
    0 & \kappa(1-\kappa y^2) & 0 & 0 \\
    0 & 0 & \frac{\kappa}{(1-\kappa y^2)} & 0\\
    0 & 0 & 0 & \frac{-\kappa}{(1+\kappa z^2)}
\end{pmatrix}=
\]
\[=\kappa-\kappa-\kappa+\kappa=0\]
Luego, el escalar de Ricci es nulo.
\subsubsection*{Apartado (c)}
Sabemos que la ecuación de Einstein es
\begin{equation}
    \mathcal{R}_{\mu\nu}-\frac{1}{2}g_{\mu\nu}\mathcal{R}+\Lambda g_{\mu\nu}=\frac{8\pi G}{c^4}\mathcal{T}_{\mu\nu}
\end{equation}
donde $\mathcal{T}_{\mu\nu}$ es el tensor de energía-impulso, $G$ es la constante de gravitación universal, $c$ es la velocidad de la luz y $\Lambda$ es la constante cosmológica, que es distinta de cero en universos no estacionarios (en expansión). Vamos a considerar un universo estacionario y así $\Lambda=0$. Como trabajamos con unidades atómicas, consideramos $c=1$. Además hemos visto que el escalar de Ricci es nulo, por tanto, las ecuaciones de Einstein quedan como,
\begin{equation}
    \mathcal{R}_{\mu\nu}=8\pi G\mathcal{T}_{\mu\nu}
\end{equation}
Por tanto, el tensor de energía-momento es,
\begin{equation}
    \mathcal{T}_{\mu\nu}=\frac{1}{8\pi G}\mathcal{R}_{\mu\nu}\equiv\frac{1}{8\pi G}\begin{pmatrix}
    \kappa(1+\kappa z^2) & 0 & 0 & 0 \\
    0 & \kappa(1-\kappa y^2) & 0 & 0 \\
    0 & 0 & \frac{\kappa}{(1-\kappa y^2)} & 0\\
    0 & 0 & 0 & \frac{-\kappa}{(1+\kappa z^2)}
\end{pmatrix}
\end{equation}
Sabemos que existe una relación entre el tensor de energía-impulso y el tensor electromagnético, que viene dada por
\begin{equation}
    \mathcal{T}^{\mu\nu}=-F^{\mu\rho}F_{\rho}^{\nu}+\frac{1}{4}g^{\mu\nu}F_{\rho\lambda}F^{\rho\lambda}
\end{equation}
Primero vamos a calcular,
\begin{equation}
    g_{\mu\nu}T^{\mu\nu}=-g_{\mu\nu}F^{\mu\rho}F_{\rho}^{\nu}+\frac{1}{4}g_{\mu\nu}g^{\mu\nu}F_{\rho\lambda}F^{\rho\lambda}=-\frac{1}{4}F_{\rho\lambda}F^{\rho\lambda}
\end{equation}
Pero como $0=\mathcal{R}=g_{\mu\nu}R^{\mu\nu}=8\pi G g_{\mu\nu}T^{\mu\nu}$, entonces tenemos que $F_{\rho\lambda}F^{\rho\lambda}=0$. Luego, nos queda
\begin{equation}
    T^{\mu\nu}=-F^{\mu\rho}F^{\nu}_{\rho}
\end{equation}
Además, sabemos que $F^{\mu\rho}$ por definición es:
\begin{equation}
    F^{\mu\rho}=\begin{pmatrix}
        0 & -E_x & -E_y & -Ez\\
        E_x & 0 & -B_z & B_y\\
        E_y & B_z & 0 & -B_x\\
        E_z & -B_y & B_x & 0
    \end{pmatrix}
\end{equation}
y que $F_{\rho}^{\nu}=g_{\rho\mu}F^{\mu\nu}$. Por tanto,
\[F_{0}^{\nu}=g_{00}F^{0\nu}=(1+\kappa z^2)\begin{pmatrix}
    0 & -E_x & -E_y & -Ez
\end{pmatrix}\]
\[F_{1}^{\nu}=g_{11}F^{1\nu}=-(1-\kappa y^2)\begin{pmatrix}
    E_x & 0 & -B_z & B_y
\end{pmatrix}\]
\[F_{2}^{\nu}=g_{22}F^{2\nu}=-(1-\kappa y^2)^{-1}\begin{pmatrix}
    E_y & B_z & 0 & -B_x
\end{pmatrix}\]
\[F_{3}^{\nu}=g_{33}F^{3\nu}=-(1+\kappa z^2)^{-1}\begin{pmatrix}
    E_z & -B_y & B_x & 0
\end{pmatrix}\]
Luego,
\[F_{\rho}^{\nu}=\begin{pmatrix}
    0 & -E_x(1+\kappa z^2) & -E_y(1+\kappa z^2) & -E_z(1+\kappa z^2) \\
    -E_x(1-\kappa y^2) & 0 & B_z(1-\kappa y^2) & -B_y(1-\kappa y^2) \\
    -E_y(1-\kappa y^2)^{-1} & -B_z(1-\kappa y^2)^{-1} & 0 & B_x(1-\kappa y^2)^{-1} \\
    -E_z(1+\kappa z^2)^{-1} & B_y(1+\kappa z^2)^{-1} & -B_x(1+\kappa z^2)^{-1} & 0
\end{pmatrix}\]
Entonces podemos calcular $F^{\mu\rho}F_{\rho}^{\nu}\equiv \mathcal{F}^{\mu\nu}$, tal que
\[\small
\begin{array}{l}
\mathcal{F}^{\mu\nu}=\begin{pmatrix}
        0 & -E_x & -E_y & -Ez\\
        E_x & 0 & -B_z & B_y\\
        E_y & B_z & 0 & -B_x\\
        E_z & -B_y & B_x & 0
    \end{pmatrix}\begin{pmatrix}
    0 & -E_x(1+\kappa z^2) & -E_y(1+\kappa z^2) & -E_z(1+\kappa z^2) \\
    -E_x(1-\kappa y^2) & 0 & B_z(1-\kappa y^2) & -B_y(1-\kappa y^2) \\
    -E_y(1-\kappa y^2)^{-1} & -B_z(1-\kappa y^2)^{-1} & 0 & B_x(1-\kappa y^2)^{-1} \\
    -E_z(1+\kappa z^2)^{-1} & B_y(1+\kappa z^2)^{-1} & -B_x(1+\kappa z^2)^{-1} & 0
\end{pmatrix}
\end{array}\]
Nos damos cuenta que $F^{\mu\rho}$ y $F_{\rho}^{\nu}$, son antisimétricos, por tanto, $\mathcal{F}^{\mu\nu}$ es simétrico, tal que $\mathcal{F}^{\mu\nu}=\mathcal{F}^{\nu\mu}$.\\
Vamos a ir calculando componente a componente,
\[\mathcal{F}^{00}=E_x^2(1-\kappa y^2)+E_y^2(1-\kappa y^2)^{-1}+E_z^2(1+\kappa z^2)^{-1}\]
\[\mathcal{F}^{11}=-E_x^2(1+\kappa z^2)+B_z^2(1-\kappa y^2)^{-1}+B_y^2(1+\kappa z^2)^{-1}\]
\[\mathcal{F}^{22}=-E_y^2(1+\kappa z^2)+B_z^2(1-\kappa y^2)+B_x^2(1+\kappa z^2)^{-1}\]
\[\mathcal{F}^{33}=-E_z^2(1+\kappa z^2)+B_y^2(1-\kappa y^2)+B_x^2(1-\kappa y^2)^{-1}\]
\[\mathcal{F}^{01}=\mathcal{F}^{10}=E_yB_z(1-\kappa y^2)^{-1}-E_zB_y(1+\kappa z^2)^{-1}\]
\[\mathcal{F}^{02}=\mathcal{F}^{20}=-E_xB_z(1-\kappa y^2)+E_zB_x(1+\kappa z^2)^{-1}\]
\[\mathcal{F}^{03}=\mathcal{F}^{30}=E_xB_y(1-\kappa y^2)-E_yB_x(1-\kappa y^2)^{-1}\]
\[\mathcal{F}^{12}=\mathcal{F}^{21}=-E_yE_x(1+\kappa z^2)-B_yB_x(1+\kappa z^2)^{-1}\]
\[\mathcal{F}^{13}=\mathcal{F}^{31}=-E_xE_z(1+\kappa z^2)-B_zB_x(1-\kappa y^2)^{-1}\]
\[\mathcal{F}^{23}=\mathcal{F}^{32}=-E_yE_z(1+\kappa z^2)-B_zB_y(1-\kappa y^2)\]
Como $T_{\mu\nu}$ es diagonal, tendremos que $T^{\mu\nu}=1/T_{\mu\nu}$, es decir,
\[T^{\mu\nu}=8\pi G\begin{pmatrix}
    \frac{1}{\kappa(1+\kappa z^2)} & 0 & 0 & 0 \\
    0 & \frac{1}{\kappa(1-\kappa y^2)} & 0 & 0 \\
    0 & 0 & \frac{(1-\kappa y^2)}{\kappa} & 0\\
    0 & 0 & 0 & \frac{-(1+\kappa z^2)}{\kappa}
\end{pmatrix}=-\mathcal{F}^{\mu\nu}\]
Por tanto, igualando componente a componente tenemos el siguiente sistema de 10 ecuaciones con 6 incógnitas (sabiendo que $\kappa=8\pi G)$:
\[\frac{-1}{(1+\kappa z^2)}=E_x^2(1-\kappa y^2)+E_y^2(1-\kappa y^2)^{-1}+E_z^2(1+\kappa z^2)^{-1}
\tag{i}\label{(i)}\]
\[\frac{-1}{(1-\kappa y^2)}=-E_x^2(1+\kappa z^2)+B_z^2(1-\kappa y^2)^{-1}+B_y^2(1+\kappa z^2)^{-1}\tag{ii}
\label{(ii)}\]
\[-(1-\kappa y^2)=-E_y^2(1+\kappa z^2)+B_z^2(1-\kappa y^2)+B_x^2(1+\kappa z^2)^{-1}\tag{iii}
\label{(iii)}\]
\[(1+\kappa z^2)=-E_z^2(1+\kappa z^2)+B_y^2(1-\kappa y^2)+B_x^2(1-\kappa y^2)^{-1}\tag{iv}
\label{(iv)}\]
\[0=E_yB_z(1-\kappa y^2)^{-1}-E_zB_y(1+\kappa z^2)^{-1}\tag{v}
\label{(v)}\]
\[0=-E_xB_z(1-\kappa y^2)+E_zB_x(1+\kappa z^2)^{-1}\tag{vi}
\label{(vi)}\]
\[0=E_xB_y(1-\kappa y^2)-E_yB_x(1-\kappa y^2)^{-1}\tag{vii}
\label{(vii)}\]
\[0=-E_yE_x(1+\kappa z^2)-B_yB_x(1+\kappa z^2)^{-1}\tag{viii}
\label{(viii)}\]
\[0=-E_xE_z(1+\kappa z^2)-B_zB_x(1-\kappa y^2)^{-1}\tag{ix}
\label{(ix)}\]
\[0=-E_yE_z(1+\kappa z^2)-B_zB_y(1-\kappa y^2)\tag{x}
\label{(x)}\]
Resolver este sistema de ecuaciones es un suplicio, pues no es lineal y los métodos numéricos que he usado han fallado.\\ \\
Realizando una búsqueda por internet, he encontrado que la métrica transformada en la del espacio $AdS^2\otimes\mathbb{S}^2$, es la métrica de Bertotti-Robinson, cuyo tensor electromagnético solo tiene componentes no nulas en las direcciones $r$ y $t$, tal que
\[F_{tr}=-F_{rt}=\frac{Q}{R_0^2}=E_z\]
donde $Q$ es una constante que representa la carga eléctrica. Es decir, tenemos un campo eléctrico radial.\\ \\
Debemos deshacer el cambio de coordenadas $r=z$ y $\kappa=1/R_0^2$, usando la transformación de coordenadas de tensores, tal que
\[F_{\mu\nu}=\frac{\partial x^{\alpha}}{\partial\overline{x}^{\mu}}\frac{\partial x^{\beta}}{\partial\overline{x}^{\nu}}\overline{F}_{\alpha\beta}\]
En nuestro caso, $\mu=t$ y $\nu=r$, tal que
\[\frac{\partial x^{\alpha}}{\partial\overline{x}^t}=\left\lbrace\begin{array}{lll}
    1 & \text{si} & \alpha=t \\
    0 & \text{si} & \alpha\neq t
\end{array}\right.;\hspace{8mm}\frac{\partial x^{\beta}}{\partial\overline{x}^{r}}=\left\lbrace\begin{array}{lll}
    1 & \text{si} & \beta=z \\
    0 & \text{si} & \beta\neq z
\end{array}\right.\]
Por tanto,
\[F_{tz}=\frac{\partial x^{t}}{\partial\overline{x}^{t}}\frac{\partial x^{z}}{\partial\overline{x}^{r}}\overline{F}_{tr}=F_{tr}=Q\kappa\]
Por tanto, las únicas componentes no nulas del tensor electromagnético son $F_{tz}=E_z$ y $F_{zt}=-E_z$. Así, el tensor electromagnético queda,
\[F_{\mu\nu}=\begin{pmatrix}
    0 & 0 & 0 & -Q\kappa\\
    0 & 0 & 0 & 0\\
    0 & 0 & 0 & 0\\
    Q\kappa & 0 & 0 & 0
\end{pmatrix}\]

\end{document}

\documentclass[11pt]{article}
%\usepackage[spanish]{babel}
\RequirePackage{etex}
\usepackage[utf8]{inputenc}
\usepackage{braket}
%\usepackage[sc]{mathpazo}
% \linespread{1.5}
%\usepackage[T1]{fontenc}
%\usepackage{heuristica}
%\usepackage[erewhon,vvarbb,bigdelims]{newtxmath}
%\renewcommand*\oldstylenums[1]{\textosf{#1}}
\usepackage{enumitem}
\usepackage{array}
\usepackage{textcomp}
\usepackage{pdfpages}
\usepackage{feynmp-auto}
\usepackage{fancyhdr}
\usepackage{amsmath, amsthm}
\usepackage{slashed}
\usepackage[normalem]{ulem}
\usepackage{amsfonts}
\usepackage{amssymb}
\usepackage{mathtools}
\usepackage{float}
\usepackage{soul}
\usepackage{graphicx}
\usepackage{hyperref}
\usepackage{graphicx}
\usepackage{pstricks-add}
\usepackage{color}
\usepackage{caption}
\usepackage[margin=0.9in]{geometry}
\usepackage{marvosym}
\usepackage{mathtools}
\usepackage{framed}
\usepackage{calrsfs}
\usepackage[mathscr]{euscript}
\usepackage{tensor}
\usepackage{autonum}
\usepackage{cancel}
\usepackage[most]{tcolorbox}

\newtheorem{thm}{Teorema}[section]
\newtheorem{theorem}{Teorema}[section]
\newtheorem{proposition}[thm]{Proposición} 
\newtheorem{lemma}[thm]{Lema}
\newtheorem{corollary}[thm]{Corolario} 
\newtheorem{conv}[thm]{Convención}
\newtheorem{defi}[thm]{Definición}
\newtheorem{definition}[theorem]{Definición}
\newtheorem{notation}[thm]{Notación} 
\newtheorem{exe}[thm]{Ejemplo}
\newtheorem{conjecture}[thm]{Conjetura} 
\newtheorem{prob}[thm]{Problema}
\newtheorem{remark}[thm]{Observación}
\newtheorem{example}[thm]{Ejemplo}
\newtheorem{note}[thm]{Nota}

\newcommand{\brackets}[1]{\left[#1\right]}
\newcommand{\curlybraces}[1]{\left\{#1\right\}}
\newcommand{\qedh}{\hfill\hspace{5mm}\fbox{\phantom{\rule{.5ex}{.5ex}}}}
\newcommand{\scalar}[2]{\langle #1, #2 \rangle}
\newcommand{\ptensor}[2]{#1 \otimes #2}
\newcommand{\pcart}[2]{#1 \times #2}
\newcommand{\voverrightarrowtor}[3]{\begin{pmatrix}#1\\ #2\\ #3\end{pmatrix}}
\newcommand{\cooverrightarrowtor}[3]{\begin{pmatrix}#1 & #2 & #3\end{pmatrix}}
\newcommand{\abss}[1]{\begin{vmatrix}#1\end{vmatrix}^2}

\newtcolorbox[auto counter, number within=section]{mytheorem}[2][]{
  enhanced,
  breakable,
  title=Teorema~\thetcbcounter: #2,
  #1,
}
\newtcolorbox[auto counter, number within=section]{propositionbox}[2][]{
  enhanced,
  breakable,
  title=Proposition~\thetcbcounter: #2,
  #1,
}

\newtcolorbox[auto counter, number within=section]{corollarybox}[2][]{
  enhanced,
  breakable,
  title=Corollary~\thetcbcounter: #2,
  #1,
}

\newtcolorbox[auto counter, number within=section]{remarkbox}[2][]{
  enhanced,
  breakable,
  title=Remark~\thetcbcounter: #2,
  #1,
}

\newtcolorbox[auto counter, number within=section]{notebox}[2][]{
  enhanced,
  breakable,
  title=Note~\thetcbcounter: #2,
  #1,
}


\newenvironment{Figura}
  {\par\medskip\noindent\minipage{\linewidth}}
  {\endminipage\par\medskip}
%\usepackage[spanish]{babel}
\title{\huge{\textbf{Minitest II. Sistemas atómicos de muchos electrones}}}
\author{Rubén Carrión Castro\\}
% \textit{Los Chavales}
\date{Noviembre 2025}
\begin{document}
\maketitle
\begin{enumerate}
\item \textbf{Considere la aproximación de campo medio para el átomo de litio con carga nuclear apantallada.}\\ \\
La aproximación de campo medio, también conocida como aproximación de campo central, es la base para el cálculo en átomos polielectrónicos. Esta aproximación consiste en tomar un potencial $V(r)$ con simetría esférica y centrado en el núcleo, sometido a cada electrón del átomo. \\ \\
El Hamiltoniano será del tipo,
\begin{equation}
    H=-\frac{1}{2}\sum_{i=1}^N\nabla_i^2-\sum_{i=1}^N\frac{Z}{r_i}+\sum_{i<j=1}^N\frac{1}{r_{ij}}
\end{equation}
Pero en este caso, al tener muchos electrones, el apantallamiento que provocan es tan grande, que el término $1/r_{ij}$ siempre es del orden de $-Z/r_i$, por lo que no podremos aplicar teoría de perturbaciones. Para solucionar esto, se define un potencial de campo medio del tipo,
\begin{equation}
    V(r)=-\frac{Z}{r}+S(r)
\end{equation}
donde $S(r)$ es un potencial promedio, siendo el apantallamiento de la carga nuclear, a una distancia $r$ del núcleo, debido a la densidad de carga electrónica contenida en el volumen esférico de radio $r$.\\ \\
Se imponen ciertas condiciones al potencial, que son
\begin{itemize}
    \item A muy cortas distancias, el potencial debe tender al potencial culombiano, por lo que no hay apantallamiento,
    \begin{equation}
        \lim_{r\to0}V(r)=-\frac{Z}{r}\Longrightarrow S(r\to0)\to0
    \end{equation}
    \item A muy largas distancias, el potencial debe tender al culombiano generado por una carga $Z-(N-1)$,
    \begin{equation}
        \lim_{r\to\infty}V(r)=-\frac{Z-(N-1)}{r}\Longrightarrow S(r\to\infty)\to\frac{N-1}{r}
    \end{equation}
\end{itemize}
Así, el Hamiltoniano queda como,
\begin{equation}
    H=\underbrace{\sum_{i=1}^n\left(-\frac{\nabla_i^2}{2}-\frac{Z}{r_i}+S(r_i)\right)}_{H_C}+\underbrace{\sum_{i<j=1}^N\frac{1}{r_{ij}}-\sum_{i=1}^NS(r_i)}_{H_1}
\end{equation}
donde al contener $S(r_i)$ parte de la repulsión, el efecto de $H_1$ sobre los niveles será menor que el de la repulsión electrostática, por lo que $H_1\lll H_C$ y podemos aplicar teoría de perturbaciones.
\begin{enumerate}
\item \textbf{Usando el valor experimental de la energía de ionización dada en NIST Ionization Energies, determine la constante de apantallamiento.}\\ \\
Nos piden calcular la constante de apantallamiento, $S$, del Litio a partir de la energía de ionización experimental, que es
\begin{equation}
    E_{exp}=-0.1981418715\text{ Hartree}
\end{equation}
siendo la energía del nivel fundamental. Como el Litio tiene tres electrones, la configuración del nivel fundamental es
\begin{equation}
    \mathfrak{C}(Li)=1s^22s
\end{equation}
cuya degeneración será
\begin{equation}
    deg(1s^22s)=\begin{pmatrix}
        2\\
        2
    \end{pmatrix}\begin{pmatrix}
        2\\
        1
    \end{pmatrix}=1\cdot2=2
\end{equation}
Como nos piden determinar la CONSTANTE de apantallamiento, tomaremos el potencial apantallado como sigue,
\begin{equation}
    V(r)=-\frac{Z_{eff}}{r}=-\frac{Z-S}{r}
\end{equation}
donde $S$ es la denominada \textit{constante de apantallamiento}. Esto es la aproximación más simple del desarrollo que hicimos anteriormente.\\ \\
Como tenemos una capa llena y la otra semillena, podemos aproximar el Litio como un átomo hidrogenoide, con electrón en $n=2$, cuya energía será
\begin{equation}
    E_{n}=-\frac{Z_{eff}^2}{2n^2}=-\frac{(Z-S)^2}{2n^2}
\end{equation}
Por tanto, la energía del estado fundamental será,
\begin{equation}
    E_{2}=-\frac{(Z-S)^2}{2\cdot4}=-\frac{(Z-S)^2}{8}
\end{equation}
Entonces, la constante de apantallamiento queda
\begin{equation}
    S=Z-\sqrt{-8E_{11}}
\end{equation}
Luego, la constante de apantallamiento para el Litio $(Z=3)$ queda
\begin{equation}
    S=3-\sqrt{8\cdot0.1981418715}\approx 1.740978566
\end{equation}
Luego, la carga efectiva será
\begin{equation}
    Z_{eff}=Z-S\approx1.259021434
\end{equation}
\item \textbf{Calcule el tamaño medio del átomo.}\\ \\
Para calcular el tamaño medio del Litio, debemos introducir las funciones de onda hidrogenoides, pues estamos en esta aproximación. Estas funciones de onda son separables, del tipo
\begin{equation}
    \Psi(q)=\phi(\vec{r})\chi(\sigma)
\end{equation}
donde $\chi(\sigma)$ es la función de onda de la parte de espín y $\phi(\vec{r})$ es la función de onda espacial, que tomando simetría esférica, podemos separar la parte radial de la angular, tal que
\begin{equation}
    \phi_{nlm}(\vec{r})=\mathscr{R}_{nl}(r)Y_{lm}(\theta,\varphi)
\end{equation}
Luego, el tamaño medio del átomo será $<r>_{nlm}$, siendo un valor esperado que podemos calcular usando
\begin{equation}
    <r>_{nlm}=\int_{-\infty}^{+\infty} d^3\vec{r}\phi_{nlm}^*(\vec{r})r\phi_{nlm}(\vec{r})=\int_{0}^{\infty} r^3\mathscr{R}_{nl}^*(r)\mathscr{R}_{nl}(r)dr\int_{\Omega'}Y_{lm}^*(\Omega)Y_{lm}(\Omega)d\Omega
\end{equation}
Como los armónicos esféricos siempre están normalizados, tendremos que
\begin{equation}
    \int_{\Omega'}|Y_{lm}(\Omega)|^2d\Omega=1
\end{equation}
Luego, solo tendremos que calcular la integral radial. Para ello, usamos la forma explícita de los $\mathscr{R}_{nl}(r)$, que es
\begin{equation}
    \mathscr{R}_{nl}(r)=N_{nl}\rho^le^{-\rho/2}\mathscr{L}_{n+l}^{2l+1}(\rho)
\end{equation}
con 
\begin{equation}
    \rho=\frac{2\mu c Z\alpha}{\hbar n}r
\end{equation}
donde podemos hacer un agrupamiento de constantes que represente un 'radio de Bohr', pero en lugar de usar la masa del electrón, usamos la masa reducida del átomo, tal que
\begin{equation}
    a=\frac{4\pi\epsilon_0\hbar^2}{\mu e^2}\Longrightarrow\rho=\frac{2Zr}{na}
\end{equation}
Luego, la expresión se reduce a
\begin{equation}
    R_{nl}(r)=N_{nl}\left(\frac{2Z}{na}r\right)^le^{-Zr/na}\mathscr{L}_{n+l}^{2l+1}\left(\frac{2Z}{na}r\right)
\end{equation}
con
\begin{equation}
    N_{nl}=\left(\frac{2Z}{na}\right)^{3/2}\sqrt{\frac{(n-l-1)!}{2n(n+l)!}}
\end{equation}
donde $\mathscr{L}_{n+l}^{2l+1}(\rho)$ son los polinomios asociados de Laguerre, que cumplen la propiedad
\begin{equation}
    \int_0^{\infty}\rho^{2l+3}e^{-\rho}\left[\mathscr{L}_{n+l}^{2l+1}(\rho)\right]^2d\rho=\frac{2n(n+l)!}{(n-l-1)!}\brackets{3n^2-l(l+1)}
\end{equation}
Resolvemos la integral, usando el cambio de variable de $\rho$, por lo que
\begin{equation}
    r=\frac{na}{2Z}\rho;\hspace{3mm}dr=\frac{na}{2Z}d\rho
\end{equation}
Luego queda,
\begin{align}
       \int\limits_0^{\infty}r^3|\mathscr{R}_{nl}(r)|^2dr= & N_{nl}^2\int\limits_0^{\infty}r^3\left(\frac{2Z}{na}r\right)^{2l}e^{-2Zr/na}\left[\mathscr{L}_{n+l}^{2l+1}\left(\frac{2Z}{na}r\right)\right]^2dr= \\  \\
         = & N_{nl}^2\left(\frac{na}{2Z}\right)^3\frac{na}{2Z}\int_0^{\infty}\rho^3\rho^{2l}e^{-\rho}\left[\mathscr{L}_{n+l}^{2l+1}(\rho)\right]^2d\rho=\\ \\
         =&\cancel{\left(\frac{2Z}{na}\right)^3}\frac{(n-l-1)!}{2n(n+l)!}\left(\frac{na}{2Z}\right)^{\cancel{4}}\int_0^{\infty}\rho^{2l+3}e^{-\rho}\left[\mathscr{L}_{n+l}^{2l+1}(\rho)\right]^2d\rho=\\ \\
         =&\left(\frac{na}{2Z}\right)\cancel{\frac{(n-l-1)!}{2n(n+l)!}}\cancel{\frac{2n(n+1)!}{(n-l-1)!}}\left[3n^2-l(l+1)\right]=\\ \\
         =&n^2\frac{a}{Z}\left\lbrace1+\frac{1}{2}\left[1-\frac{l(l+1)}{n^2}\right]\right\rbrace=<r>_{nlm}
\end{align}

Entonces, como nuestro electrón de valencia está en la capa $2s$, tendremos que el tamaño medio del átomo de litio es $<r>_{200}$, y para una mejor aproximación, usaremos $Z_{eff}$ como $Z$. Así queda,
\begin{equation}
    <r>_{200}=4\frac{a}{Z_{eff}}\frac{3}{2}\approx 4.7656\cdot a
\end{equation}
Calculamos $a$ para obtener el tamaño medio en Armstrong, usando la masa reducida del litio,
\begin{equation}
    a=\frac{m_e}{\mu}a_0
\end{equation}
donde $a_0=0.529 \mathring{A}$ es el radio de Bohr. Luego, usando que la masa atómica del Litio es $M_{Li}\approx6.94$ uma$=1.15241\times10^{-26}$ kg.
\begin{equation}
    \mu=\frac{m_e\cdot M_{Li}}{M_{Li}+m_e}=\frac{9.10938\times10^{-31}\cdot1.15241\times10^{-26}}{9.10938\times10^{-31}+1.15241\times10^{-26}}\approx 9.108659994\times10^{-31}\text{ kg}
\end{equation}
Entonces,
\begin{equation}
    a=\frac{9.10938\times10^{-31}}{9.108659994\times10^{-31}}0.529=0.5290418155\mathring{A}
\end{equation}
Luego, el tamaño medio del átomo de litio será
\begin{equation}
    <r>_{200}=4.7656\cdot0.5290418155\approx 2.5212\mathring{A}
\end{equation}
\item \textbf{Con ese campo medio, prediga los valores de las energías de ionización de los primeros niveles excitados y compare con los resultados experimentales, NIST atomic levels.}\\ \\
Los primeros niveles excitados del átomo de Litio tienen las configuraciones siguientes,
\begin{equation}
    \mathfrak{C}(Li^*)=1s^22p;\hspace{3mm}\mathfrak{C}(Li^{2*})=1s^23s;\hspace{3mm}\mathfrak{C}(Li^{3*})=1s^23p;\hspace{3mm}\dots
\end{equation}
Usaremos la carga efectiva calculada antes, que es $Z_{eff}=1.259021434$, y la energía será
\begin{equation}
    E_n=-\frac{Z_{eff}^2}{2n^2}
\end{equation}
Entonces, esta energía no diferencia entre estados excitados de la misma capa, pues para las configuraciones $1s^22s$ y $1s^22p$, tendremos que $n=2$, obteniendo la misma energía que la fundamental para el primer estado excitado, que es $E_2=-0.1981418715$ Hartree. Para el segundo y tercer estado excitado también tendremos la misma energía, pues $n=3$, siendo $E_3=-0.08806305396$ Hartree.\\ \\
Las energías de los estados excitados del Litio según el NIST son las siguientes,
\begin{Figura}
    \centering
    \includegraphics[width=0.8\textwidth]{energias.png}
    \captionof{table}{Tabla del NIST de la energía de los primeros estados excitados del Litio.}
    \label{fig1}
\end{Figura}
Esta tabla toma como referencia la energía del estado fundamental, por lo que para comparar la energía de forma correcta con las que hemos obtenido, debemos sumarle a la energía de los estados excitados la energía del estado fundamental, obteniendo
\begin{center}
    \begin{tabular}{|c|c|c|}
\hline
       Configuración & Nivel (Hartree) & Valor obtenido (Hartree) \\ \hline
        $1s^22s$ & $-0.198121871$ & $-0.198121871$ \\ 
        & & \\
        $1s^22p$ & $\begin{matrix}
            -0.130215771\\
            -0.130214271
        \end{matrix}$ & $-0.1981418715$\\ 
        & & \\
        $1s^23s$ & $-0.074161671$ & $-0.08806305396$\\ 
        & & \\
        $1s^23p$ & $\begin{matrix}
            -0.057215471\\
            -0.057215471
        \end{matrix}$ & $-0.08806305396$\\ \hline
    \end{tabular}
\end{center}
Vemos que los valores obtenidos difieren bastante con los valores experimentales y además no tienen en cuenta la $l$.
\item \textbf{Usando los valores experimentales, refine su modelo de campo medio considerando un apantallamiento diferente en cada subcapa \textit{nl}, }$\mathbf{Z-S_{nl}.}$\textbf{ Determine el valor de cada constante de apantallamiento y discuta sus resultados.}\\ \\
Para hacer este apartado, solo vamos a considerar las energías experimentales sin desdoblamiento en $J$ (las que pone 'term'), pues para ello debemos hacer otro tratamiento más fino.\\ \\
Usando el mismo razonamiento que el usado en el apartado (a), pero tomando $E_{nl}$, tendremos las siguientes constantes de apantallamiento para cada configuración,
\begin{center}
    \begin{tabular}{|c|c|c|c|}
    \hline
        Configuración & Energía Exp. (Hartree) & $S_{nl}$ & $Z_{eff}^{(nl)}$ \\ \hline
        $1s^22s$ & $-0.198121871$ & 1.740978566 & 1.259021434 \\ 
        $1s^22p$ & $-0.130214771$ & 1.979354044 & 1.020645956 \\ 
        $1s^23s$ & $-0.074161671$ & 1.844616913 & 1.553830387 \\ 
        $1s^23p$ & $-0.057215471$ & 1.985170715 & 1.014829285 \\ \hline
    \end{tabular}
\end{center}
donde hemos usado que
\begin{equation}
    S_{nl}=Z-n\sqrt{-2E_{nl}}
\end{equation}
\item \textbf{Calcule el tamaño del átomo.}
Usando el tratamiento anterior, tenemos que el tamaño del átomo es
\begin{equation}
   <r>_{nlm}=n^2\frac{a}{Z_{eff}^{(nl)}}\left\lbrace1+\frac{1}{2}\left[1-\frac{l(l+1)}{n^2}\right]\right\rbrace
\end{equation}
con $a=0.52904418155\mathring{A}$.\\ \\
Luego, el tamaño medio del átomo para cada estado excitado será,

\begin{center}
    \begin{tabular}{|c|c|c|}
    \hline
        Configuración &  $Z_{eff}^{(nl)}$ & $<r>_{nlm}$ $(\mathring{A})$\\ \hline
        $1s^22s$ &  1.259021434 & 2.5212 \\ 
        $1s^22p$ &  1.020645956 & 2.5917 \\ 
        $1s^23s$ &  1.155383087 & 6.1816 \\ 
        $1s^23p$ &  1.014829285 & 6.5164 \\ \hline
    \end{tabular}
\end{center}
\end{enumerate}
\end{enumerate}

















\end{document}

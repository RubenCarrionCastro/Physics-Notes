\documentclass[11pt]{article}
%\usepackage[spanish]{babel}
\RequirePackage{etex}
\usepackage[utf8]{inputenc}
\usepackage{braket}
%\usepackage[sc]{mathpazo}
% \linespread{1.5}
%\usepackage[T1]{fontenc}
%\usepackage{heuristica}
%\usepackage[erewhon,vvarbb,bigdelims]{newtxmath}
%\renewcommand*\oldstylenums[1]{\textosf{#1}}
\usepackage{enumitem}
\usepackage{array}
\usepackage{textcomp}
\usepackage{multicol}
\usepackage{fancyhdr}
\usepackage{xcolor}
\usepackage{amsmath, amsthm}
\usepackage{slashed}
\usepackage[normalem]{ulem}
\usepackage{amsfonts}
\usepackage{amssymb}
\usepackage{mathtools}
\usepackage{float}
\usepackage{soul}
\usepackage{graphicx}
\usepackage{hyperref}
\usepackage{graphicx}
\usepackage{pstricks-add}
\usepackage{color}
\usepackage{caption}
\usepackage[margin=0.9in]{geometry}
\usepackage{marvosym}
\usepackage{mathtools}
\usepackage{framed}
\usepackage{calrsfs}
\usepackage[mathscr]{euscript}
\usepackage{tensor}
\usepackage{autonum}
\usepackage{cancel}
\usepackage[most]{tcolorbox}

\newtheorem{thm}{Teorema}[section]
\newtheorem{theorem}{Teorema}[section]
\newtheorem{proposition}[thm]{Proposición} 
\newtheorem{lemma}[thm]{Lema}
\newtheorem{corollary}[thm]{Corolario} 
\newtheorem{conv}[thm]{Convención}
\newtheorem{defi}[thm]{Definición}
\newtheorem{definition}[theorem]{Definición}
\newtheorem{notation}[thm]{Notación} 
\newtheorem{exe}[thm]{Ejemplo}
\newtheorem{conjecture}[thm]{Conjetura} 
\newtheorem{prob}[thm]{Problema}
\newtheorem{remark}[thm]{Observación}
\newtheorem{example}[thm]{Ejemplo}
\newtheorem{note}[thm]{Nota}

\newcommand{\brackets}[1]{\left[#1\right]}
\newcommand{\curlybraces}[1]{\left\{#1\right\}}
\newcommand{\qedh}{\hfill\hspace{5mm}\fbox{\phantom{\rule{.5ex}{.5ex}}}}
\newcommand{\scalar}[2]{\langle #1, #2 \rangle}
\newcommand{\ptensor}[2]{#1 \otimes #2}
\newcommand{\pcart}[2]{#1 \times #2}
\newcommand{\voverrightarrowtor}[3]{\begin{pmatrix}#1\\ #2\\ #3\end{pmatrix}}
\newcommand{\cooverrightarrowtor}[3]{\begin{pmatrix}#1 & #2 & #3\end{pmatrix}}
\newcommand{\abss}[1]{\begin{vmatrix}#1\end{vmatrix}^2}

\newtcolorbox[auto counter, number within=section]{mytheorem}[2][]{
  enhanced,
  breakable,
  title=Teorema~\thetcbcounter: #2,
  #1,
}
\newtcolorbox[auto counter, number within=section]{propositionbox}[2][]{
  enhanced,
  breakable,
  title=Proposition~\thetcbcounter: #2,
  #1,
}

\newtcolorbox[auto counter, number within=section]{corollarybox}[2][]{
  enhanced,
  breakable,
  title=Corollary~\thetcbcounter: #2,
  #1,
}

\newtcolorbox[auto counter, number within=section]{remarkbox}[2][]{
  enhanced,
  breakable,
  title=Remark~\thetcbcounter: #2,
  #1,
}

\newtcolorbox[auto counter, number within=section]{notebox}[2][]{
  enhanced,
  breakable,
  title=Note~\thetcbcounter: #2,
  #1,
}


\newenvironment{Figura}
  {\par\medskip\noindent\minipage{\linewidth}}
  {\endminipage\par\medskip}
%\usepackage[spanish]{babel}
\title{\huge{\textbf{Evaluación III. Física Estadística}}}
\author{\textbf{}\\ \\Rubén Carrión Castro\\}
% \textit{Los Chavales}
\date{Abril 2025}
\begin{document}
\maketitle
\begin{enumerate}
\item \textbf{Considere un sistema formado por }$\mathbf{N}$\textbf{ partículas \textit{idénticas} y \textit{distinguibles} que no interactúan entre sí. Cada una de estas partículas puede situarse en dos niveles de energía }$\mathbf{\curlybraces{0,\epsilon}}$\textbf{ con degeneración 2 y 4 respectivamente. Obtenga la expresión de la energía interna de este sistema en función de la temperatura y del número de partículas, }$\mathbf{U(T,N)}$\textbf{ y compruebe que coincide en el límite termodinámico, empleando:}
\begin{enumerate}
    \item \textbf{la colectividad canónica. Use esta colectividad también para calcular la fracción de partículas en el nivel fundamental y en el nivel de energía }$\mathbf{+\epsilon}$\textbf{, estudiando los límites de baja y alta temperatura.}\\ \\
Tenemos $N$ partículas distinguibles e independientes, es decir, no interactúan, con dos niveles de energía, $\curlybraces{0,\epsilon}$. Sabemos que al ser independientes, la función de partición total $Z$ del sistema puede dividirse en función de la función de partición $Z_1$ de una partícula elevada al número de partículas, tal que
\begin{equation}
    Z=(Z_1)^N;\hspace{4mm}Z_1=\sum_{\curlybraces{E}}e^{-\beta E}g_E
\end{equation}
donde $g_E$ es la degeneración de cada nivel. Luego, al tener dos niveles con degeneración, tenemos
\begin{equation}
    Z_1=1\cdot2+e^{-\beta\epsilon}\cdot4=2+4e^{-\beta\epsilon}
\end{equation}
Luego,
\begin{equation}
    Z=\left(2+4e^{-\beta\epsilon}\right)^N
\end{equation}
Sabemos que la energía interna viene dada por menos la parcial del logaritmo de $Z$ respecto $\beta$, luego,
\begin{equation}
    U=-\left(\frac{\partial\ln Z}{\partial\beta}\right)_{E,V}=4N\epsilon\frac{e^{-\beta\epsilon}}{2\left(1+2e^{-\beta\epsilon}\right)}=2N\epsilon\frac{e^{-\beta\epsilon}}{1+2e^{-\beta\epsilon}}
\end{equation}
Luego, la energía interna $U(T,N)$ viene dada por,
\begin{equation}
    U(T,N)=2N\epsilon\frac{e^{-\epsilon/kT}}{1+2e^{-\epsilon/kT}}
\end{equation}
Observamos que la energía interna es proporcional a $N$, es decir, $U(T,N)/N$ es finita y constante para cada temperatura fija. Esto confirma que el resultado obtenido es consistente en el límite termodinámico.
\\ \\
Calculamos ahora la fracción de partículas en el nivel fundamental, $n_0$, y en el nivel de energía $\epsilon$, $n_{\epsilon}$. Para ello, usamos la probabilidad de que una partícula esté en el nivel que queremos estudiar, pues al ser independientes, podemos usar esto. Luego,
\begin{equation}
    \mathscr{P}_1(E=0)=\frac{g_0}{Z_1}e^0=\frac{2}{2+4e^{-\beta\epsilon}}=\frac{n_0}{N}
\end{equation}
\begin{equation}
    \mathscr{P}_1(E=\epsilon)=\frac{g_{\epsilon}}{Z_1}e^{-\beta\epsilon}=\frac{4e^{-\beta\epsilon}}{2+4e^{-\beta\epsilon}}=\frac{n_{\epsilon}}{N}
\end{equation}
Luego,
\begin{equation}
    n_0=\frac{2N}{2+4e^{-\beta\epsilon}}
\end{equation}
\begin{equation}
    n_{\epsilon}=\frac{4Ne^{-\beta\epsilon}}{2+4e^{-\beta\epsilon}}
\end{equation}
Estudiamos ahora los límites de alta y baja temperaturas, tal que
\begin{equation}
    \lim_{T\to\infty}n_0=\lim_{\beta\to0}n_0=\lim_{\beta\to0}\frac{2N}{2+4e^{-\beta\epsilon}}=\frac{2N}{2+4}=\frac{N}{3}
\end{equation}
\begin{equation}
    \lim_{T\to\infty}n_{\epsilon}=\lim_{\beta\to0}n_{\epsilon}=\lim_{\beta\to0}\frac{4Ne^{-\beta\epsilon}}{2+4e^{-\beta\epsilon}}=\frac{4N}{2+4}=\frac{2N}{3}
\end{equation}
A altas temperaturas, la ocupación tiende a distribuirse según la degeneración: $1/3$ en el fundamental ($g_0=2$) y $2/3$ en el excitado ($g_{\epsilon}=4$), debido al principio de equiprobabilidad canónica.
\begin{equation}
    \lim_{T\to-\infty}n_0=\lim_{\beta\to\infty}n_0=\lim_{\beta\to\infty}\frac{2N}{2+4e^{-\beta\epsilon}}=\frac{2N}{2+4e^{-\infty}}=\frac{2N}{2}=N
\end{equation}
\begin{equation}
    \lim_{T\to-\infty}n_{\epsilon}=\lim_{\beta\to\infty}n_{\epsilon}=\lim_{\beta\to\infty}\frac{4Ne^{-\beta\epsilon}}{2+4e^{-\beta\epsilon}}=\frac{4Ne^{-\infty}}{2+4e^{-\infty}}=0
\end{equation}
Vemos que a temperaturas bajas, todas las partículas se encuentran en el nivel fundamental.
\item \textbf{La colectividad microcanónica.} \\ \\
En el conjunto microcanónico, el sistema tiene energía total fija $E$ y número de partículas $N$. Cada partícula puede ocupar uno de dos niveles de energía, $\{0, \epsilon\}$, con degeneraciones $g_0 = 2$ y $g_\epsilon = 4$, respectivamente. Denotamos por $n_\epsilon$ el número de partículas situadas en el nivel de energía $\epsilon$, y por $n_0 = N - n_\epsilon$ las que están en el nivel fundamental.

Entonces, la energía total del sistema es:

\begin{equation}
    E = n_0\cdot0+n_\epsilon \cdot \epsilon=n_{\epsilon}\cdot\epsilon
\end{equation}

y el número de microestados compatibles con esa energía es:

\begin{equation}
    \Omega(N,E) = \binom{N}{n_\epsilon} \cdot g_0^{n_0} \cdot g_\epsilon^{n_\epsilon} = \binom{N}{n_\epsilon} \cdot 2^{N - n_\epsilon} \cdot 4^{n_\epsilon}
\end{equation}

La entropía microcanónica se define como:

\begin{equation}
    S(E,N) = k \ln \Omega(N,E)
\end{equation}
Vemos el $ln\Omega$,
\begin{equation}
    \ln\Omega=\ln\binom{N}{n_{\epsilon}}+(N-n_{\epsilon})\ln2+n_{\epsilon}\ln4
\end{equation}

Para grandes $N$, usamos la aproximación de Stirling,
\begin{equation}
    \ln\binom{N}{n_{\epsilon}}\approx N\ln N-n_{\epsilon}\ln n_{\epsilon}-(N-n_{\epsilon})\ln(N-n_{\epsilon})
\end{equation}
Así,
\begin{equation}
    \ln\Omega(E,N)\approx N\ln N-n_{\epsilon}\ln n_{\epsilon}-(N-n_{\epsilon})\ln(N-n_{\epsilon})+(N-n_{\epsilon})\ln2+n_{\epsilon}\ln4
\end{equation}

y definimos,

\begin{equation}
    x = \frac{n_\epsilon}{N} = \frac{E}{N\epsilon}
\end{equation}
Luego,
\begin{equation}
\begin{array}{rl}
     
    \ln\Omega(N,x)&=N\ln N-Nx\ln(Nx)-(N-Nx)\ln(N-Nx)+(N-Nx)\ln2+Nx\ln4 \\ \\
     & =N\brackets{\ln N-x\ln[(N)(x)]-(1-x)\ln[N(1-x)]+(1-x)\ln2+x\ln4}=\\ \\
     &=N\brackets{\ln N-x(\ln N+\ln x)-(1-x)(\ln N+\ln(1-x))+(1-x)\ln2+x\ln4}=\\ \\
     &=N\brackets{\cancel{\ln N}-\cancel{x\ln N}-x\ln x-\cancel{(1-x)\ln N}-(1-x)\ln(1-x)}+N(1-x)\ln2+xN\ln4=\\ \\
     &=-N\brackets{x\ln x+(1-x)\ln(1-x)}+N(1-x)\ln2+xN\ln4
\end{array}
\end{equation}

Entonces, la entropía puede escribirse como:

\begin{equation}
    S = k\ln\Omega=-Nk \left[ x \ln x + (1 - x) \ln (1 - x) \right] + kN(1 - x) \ln 2 + kNx \ln 4
\end{equation}

La temperatura se define como:

\begin{equation}
    \frac{1}{T} = \left( \frac{\partial S}{\partial E} \right)_N = \left( \frac{\partial S}{\partial x} \cdot \frac{\partial x}{\partial E} \right)_N=\frac{1}{N\epsilon}\left(\frac{\partial S}{\partial x}\right)_N
\end{equation}

Calculando, obtenemos:

\begin{equation}
    \frac{1}{T} = \frac{\cancel{N}k}{\cancel{N}\epsilon}\brackets{-\ln x-1+\ln(1-x)+1-\ln2+\ln4}=\frac{k}{\epsilon}\ln\left(\frac{4(1-x)}{2x}\right)=\frac{k}{\epsilon}\ln\left(\frac{2(1-x)}{x}\right)
\end{equation}

Despejando $x$ en función de $T$, llegamos a:

\begin{equation}
    \frac{\epsilon}{kT}=\ln\left(\frac{2-2x}{x}\right)\Rightarrow e^{1/kT}=\frac{2-2x}{x}\Rightarrow xe^{1/kT}+2x=2\Rightarrow  x = \frac{2}{2 + e^{\epsilon/kT}}
\end{equation}

Finalmente, la energía interna es:

\begin{equation}
    U(T,N) = E = n_\epsilon \cdot \epsilon = N\epsilon \cdot x = N\epsilon \cdot \frac{2}{2 + e^{\epsilon/kT}} = N\epsilon \cdot \frac{2}{2 + e^{\epsilon/kT}}
\end{equation}

Esta expresión es equivalente a la obtenida en la colectividad canónica, pues

\begin{equation}
    \frac{2}{2 + e^{\beta\epsilon}}\frac{2e^{-\beta\epsilon}}{2e^{-\beta\epsilon}} =\frac{4e^{-\beta\epsilon}}{4e^{-\beta\epsilon}+2}= \frac{4e^{-\epsilon/kT}}{2 + 4e^{-\epsilon/kT}}
\end{equation}

lo que demuestra la equivalencia de resultados en el límite termodinámico.

\item \textbf{La colectividad macrocanónica. También determine la probabilidad de encontrar }$\mathbf{N}$\textbf{ partículas en el sistema }$\mathbf{P(N)}$\textbf{.} \\ \\

En el conjunto macrocanónico (o gran canónico), el sistema puede intercambiar energía y partículas con un reservorio. Las variables naturales son: temperatura $T$, volumen $V$ (no relevante aquí) y potencial químico $\mu$. El número de partículas puede variar, y el sistema se describe mediante la función de partición macrocanónica:

\begin{equation}
    \mathcal{Z} = \sum_{N=0}^{\infty} z^N Z_N(T)
\end{equation}

donde $z = e^{\beta\mu}$ es la fugacidad, $\beta = 1/kT$ y $Z_N$ es la función de partición canónica con $N$ partículas. Como ya vimos anteriormente:

\begin{equation}
    Z_N = \left( 2 + 4e^{-\beta\epsilon} \right)^N
\end{equation}

Luego:

\begin{equation}
    \mathcal{Z} = \sum_{N=0}^{\infty} \left[z \cdot (2 + 4e^{-\beta\epsilon})\right]^N = \sum_{N=0}^{\infty} (zZ_1)^N
\end{equation}

Esta es una serie geométrica de razón $zZ_1$, que converge si $zZ_1 < 1$. Entonces:

\begin{equation}
    \mathcal{Z} = \frac{1}{1 - zZ_1} = \frac{1}{1 - z(2 + 4e^{-\beta\epsilon})}
\end{equation}

A partir de aquí, podemos obtener el número medio de partículas:

\begin{equation}
    \langle N \rangle = \frac{1}{\beta} \frac{\partial \ln \mathcal{Z}}{\partial \mu} = zZ_1 \cdot \frac{1}{1 - zZ_1}
\end{equation}

De donde despejamos la fugacidad:

\begin{equation}
    \langle N \rangle = \frac{zZ_1}{1 - zZ_1} \Rightarrow zZ_1 = \frac{\langle N \rangle}{\langle N \rangle + 1}
\end{equation}

La energía media en el conjunto macrocanónico se obtiene como:

\begin{equation}
    \langle E \rangle = -\frac{\partial \ln \mathcal{Z}}{\partial \beta}
\end{equation}

Sabemos que:

\begin{equation}
    \ln \mathcal{Z} = -\ln(1 - zZ_1) \Rightarrow \frac{\partial \ln \mathcal{Z}}{\partial \beta} = \frac{z}{1 - zZ_1} \cdot \frac{\partial Z_1}{\partial \beta}
\end{equation}

Recordando que:

\begin{equation}
    Z_1 = 2 + 4e^{-\beta\epsilon} \Rightarrow \frac{\partial Z_1}{\partial \beta} = -4\epsilon e^{-\beta\epsilon}
\end{equation}

Entonces:

\begin{equation}
    \langle E \rangle = 4\epsilon \cdot \frac{z e^{-\beta\epsilon}}{1 - zZ_1} = \epsilon \cdot \frac{4z e^{-\beta\epsilon}}{1 - z(2 + 4e^{-\beta\epsilon})}
\end{equation}

Sustituyendo el valor de $zZ_1 = \langle N \rangle / (\langle N \rangle + 1)$ y reorganizando, se recupera:

\begin{equation}
    \langle E \rangle = \langle N \rangle \cdot \epsilon \cdot \frac{4e^{-\beta\epsilon}}{2 + 4e^{-\beta\epsilon}}
\end{equation}

que coincide con el resultado obtenido en los conjuntos canónico y microcanónico.

Finalmente, la probabilidad de que el sistema tenga exactamente $N$ partículas está dada por:

\begin{equation}
    P(N) = \frac{z^N Z_N}{\mathcal{Z}} = (1 - zZ_1) \cdot (zZ_1)^N
\end{equation}

Esta es una distribución geométrica, como se espera en el conjunto macrocanónico para partículas distinguibles.
    
\end{enumerate}






\item \textbf{\textit{GAS IDEAL ULTRA-RELATIVISTA EN 1 DIMENSIÓN}. Considere un \textit{gas ideal clásico} formado por }$\mathbf{N}$\textbf{ átomos idénticos atrapado en una dimensión de longitud }$\mathbf{L}$\textbf{. Cada átomo posee una posición }$\mathbf{x_i(0\leq x_i\leq L)}$\textbf{, el \textit{módulo} de su momento }$\mathbf{|p_i|(-\infty<p_i<\infty)}$\textbf{. La energía de cada átomo es }$\mathbf{\epsilon=|p|c}$\textbf{, donde }$\mathbf{c}$\textbf{ es la velocidad de la luz. La energía total del sistema es:}
\[\mathbf{E=\sum_{i=1}^N\epsilon_i=c\sum_{i=1}^N|p_i|}\]
\begin{enumerate}
    \item \textbf{Usando la \textit{colectividad canónica} calcule la entropía del sistema }$\mathbf{S(T,L,N)}$\textbf{ y compruebe que es extensiva.}\\ \\
    Debemos calcular primero la función de partición canónica, que vendrá dada por,
    \begin{equation}
        Z=\frac{1}{h^NN!}Z_1^N
    \end{equation}
    donde $Z_1$ es la función de partición de una partícula, que calculamos como,
    \begin{equation}
        Z_1=\int_0^Ldx\int_{-\infty}^{+\infty}dpe^{-\beta c|p|}=L\left(\int_0^{\infty}dpe^{-\beta cp}+\int_{-\infty}^{0}dpe^{\beta cp}\right)=L\left(\frac{1}{\beta c}+\frac{1}{\beta c}\right)=\frac{2L}{\beta c}
    \end{equation}
    Luego, la función de partición canónica queda,
    \begin{equation}
        Z=\frac{1}{h^NN!}\left(\frac{2L}{\beta c}\right)^N
    \end{equation}
    La entropía viene dada por,
    \begin{equation}
        S=-\left(\frac{\partial A}{\partial T}\right)_{V,N}=kT\left(\frac{\partial \ln Z}{\partial T}\right)_{V,N}+k\ln Z
    \end{equation}
    donde
    \begin{equation}
        \ln Z=N\brackets{\ln(2L/c)+\ln(kT)-\ln(h)}-\ln(N!)
    \end{equation}
    Luego,
    \begin{equation}
        \frac{\partial\ln Z}{\partial T}=\frac{N}{T}
    \end{equation}
    Por tanto,
    \begin{equation}
        S=kT\frac{N}{T}+k\curlybraces{N\brackets{\ln(2L/c)+\ln(kT)-\ln(h)}-\ln(N!)}
    \end{equation}
    Usamos la aproximación de Stirling,
    \begin{equation}
        S\approx kN+kN\curlybraces{\ln(2L/c)+\ln(kT)-\ln(h)}-kN\ln(N)+kN
    \end{equation}
    Reordenando,
    \begin{equation}
        S=kN\brackets{\ln\left(\frac{2LkT}{Nhc}\right)+2}
    \end{equation}
    Comprobamos que es extensiva la entropía, es decir, que depende linealmente de las variables extensivas, o en otras palabras,
    \begin{equation}
        S(T,\lambda N,\lambda L)=\lambda S(T,N,L)
    \end{equation}
    con $\lambda\in\mathbb{R}$. Veamos,
    \begin{equation}
        S(T,\lambda N,\lambda L)=k\lambda N\brackets{\ln\left(\frac{2\cancel{\lambda} LkT}{\cancel{\lambda} Nhc}\right)+2}=\lambda\curlybraces{kN\brackets{\ln\left(\frac{2LkT}{Nhc}\right)+2}}=\lambda S(T,N,L)
    \end{equation}
    Luego, la entropía es extensiva.
    \item \textbf{Determine la probabilidad de que, si el sistema se encuentra a temperatura }$\mathbf{T}$\textbf{, una partícula tenga un momento }$\mathbf{p}$\textbf{ mayor que }$\mathbf{kT/c}$\textbf{ y de que vaya dirigida en el sentido positivo del eje X.}\\ \\
    La energía de cada partícula es $\epsilon = c|p|$, por lo que la distribución de momento para una partícula en el conjunto canónico es proporcional a:

\begin{equation}
    f(p) \propto e^{-\beta \epsilon} = e^{-\beta c |p|}
\end{equation}

Normalizando esta distribución sobre todo el eje real $p \in (-\infty, \infty)$:

\begin{equation}
    Z_p = \int_{-\infty}^{\infty} e^{-\beta c |p|} dp = \int_{-\infty}^{0} e^{\beta c p} dp + \int_{0}^{\infty} e^{-\beta c p} dp = \frac{1}{\beta c} + \frac{1}{\beta c} = \frac{2}{\beta c}
\end{equation}

Así, la función de densidad de probabilidad del momento es:

\begin{equation}
    f(p) = \frac{\beta c}{2} e^{-\beta c |p|}
\end{equation}

Queremos calcular la probabilidad de que una partícula tenga momento mayor que $kT/c$ y se mueva en el sentido positivo del eje $X$, es decir, que su momento sea positivo y supere dicho umbral:

\begin{equation}
    P\left(p > \frac{kT}{c},p>0\right) = \int_{kT/c}^{\infty} f(p)\,dp = \int_{kT/c}^{\infty} \frac{\beta c}{2} e^{-\beta c p}\,dp
\end{equation}

Realizando la integral:

\begin{equation}
    P = \left. -\frac{1}{2} e^{-\beta c p} \right|_{kT/c}^{\infty} = \frac{1}{2} e^{-\beta c \cdot \frac{kT}{c}} = \frac{1}{2} e^{-1}
\end{equation}

\begin{equation}
    \boxed{P = \frac{1}{2e} \approx 0.184}
\end{equation}

Este resultado representa la probabilidad de que una partícula del gas tenga un momento superior a $kT/c$ y se desplace hacia el sentido positivo del eje $X$, siendo el $18,4\%$ de las partículas.
\end{enumerate}
\item \textbf{\textit{GAS IDEAL CLÁSICO ATRAPADO EN UNA CORTEZA TOROIDAL}. Considere un \textit{gas ideal clásico} en equilibrio formado por }$\mathbf{N}$\textbf{ partículas \textit{idénticas} de masa }$\mathbf{m}$\textbf{ que se encuentra a temperatura }$\mathbf{T}$\textbf{. El gas está confinado en una corteza bidimensional con forma de toroide, de radio menor }$\mathbf{a}$\textbf{ y radio mayor }$\mathbf{b}$\textbf{. La función Hamiltoniana de una partícula atrapada en un toroide viene dada por (véase abajo la definición de las coordenadas generalizadas)}
\[\mathbf{H_1(\alpha,\beta,p_{\alpha},p_{\beta})=\frac{1}{2ma^2}p_{\alpha}^2+\frac{1}{2m(b+a\cos\alpha)^2}p_{\beta}^2\hspace{6mm}\alpha,\beta\in[0,2\pi)}\]
\textbf{Además, cada partícula posee \textit{grados internos de libertad}, de modo que puede estar en dos niveles cuánticos de energía }$\curlybraces{0,\epsilon}$\textbf{ con degeneración 1 y 2 respectivamente.}
\begin{enumerate}
    \item \textbf{Obtenga la energía libre de Helmholtz. Compruebe que es extensiva y que no depende de la forma de la corteza, sólo de su superficie (el área de un toroide es }$\mathbf{S=4\pi^2ab}$\textbf{).}\\ \\
Al tratarse de un gas ideal clásico, la función de partición viene dada por,
\begin{equation}
    Z=\frac{1}{N!}Z_1^N
\end{equation}
donde $Z_1$ es la función de partición de una partícula y viene dada por,
\begin{equation}
    Z_1=\frac{1}{h^2}\int_0^{2\pi}\int_0^{2\pi}d\alpha d\beta\int_{-\infty}^{\infty}dp_{\alpha}\int_{-\infty}^{\infty}dp_{\beta}e^{-\beta H_1}
\end{equation}
Separando las integrales,
\begin{equation}
    Z_1=\frac{1}{h^2}\int_0^{2\pi}d\alpha\int_0^{2\pi}d\beta\left(\int_{-\infty}^{\infty}dp_{\alpha}e^{-\beta\frac{p_{\alpha}^2}{2ma^2}}\right)\left(\int_{-\infty}^{\infty}dp_{\beta}e^{-\beta\frac{p_{\beta}^2}{2m(b+a\cos{\alpha})^2}}\right)
\end{equation}
Las integrales sobre momentos son gaussianas del tipo,
\begin{equation}
    \int_{-\infty}^{\infty}e^{-Ap^2}dp=\sqrt{\frac{\pi}{A}}
\end{equation}
    Luego,
\begin{equation}
    Z_1=\frac{1}{h^2}\int_0^{2\pi}d\beta\int_0^{2\pi}d\alpha\sqrt{\frac{2\pi ma^2}{\beta}}\sqrt{\frac{2\pi m(b+a\cos\alpha)^2}{\beta}}
\end{equation}
Calculamos la integral en beta (que no es la beta de 1/kT!!) y tenemos,
\begin{equation}
    Z_1=\frac{1}{h^2}2\pi\sqrt{\frac{(2\pi m)^2a^2}{\beta^2}}\int_0^{2\pi}d\alpha(b+a\cos\alpha)=\frac{1}{h^2}\frac{(2\pi)^2ma}{\beta}2\pi b=\frac{8\pi^3mabkT}{h^2}=\frac{2\pi SkT}{h^2}
\end{equation}
donde hemos usado que el área del toro es $S=4\pi^2 ab$. Así, la función de partición total es
\begin{equation}
    Z=\frac{1}{N!}\left(\frac{2\pi SkT}{h^2}\right)^N
\end{equation}
La energía libre de Helmholtz viene dada por $A=-kT\ln Z$, luego
\begin{equation}
\begin{array}{rl}
    A&=-kT\brackets{N\ln\left(\frac{2\pi SkT}{h^2}\right)-\ln(N!)}\approx kT\brackets{N\ln\left(\frac{2\pi SkT}{h^2}\right)-N\ln N+N}=\\ \\
    &=NkT\brackets{\ln\left(\frac{2\pi SkT}{Nh^2}\right)+1}
    \end{array}
\end{equation}
Comprobamos que sea extensiva, es decir, $A(T,\lambda S,\lambda N)=\lambda A(T,S,N)$ con $\lambda\in\mathbb{R}$,
\begin{equation}
    A(T,\lambda S,\lambda N)=\lambda N kT\brackets{\ln\left(\frac{2\pi \cancel{\lambda} SkT}{\cancel{\lambda}Nh^2}\right)+1}=\lambda\curlybraces{NkT\brackets{\ln\left(\frac{2\pi SkT}{Nh^2}\right)+1}}=\lambda A(T,S,N)
\end{equation}
    Luego es extensiva. Además, como la energía libre de Helmholtz solo depende de la superficie del toro, es decir, no depende de $a$ ni de $b$ explícitamente, vemos que no depende de la forma de la corteza.
    \item \textbf{Calcule la probabilidad de encontrar una partícula que posea un ángulo }$\mathbf{\alpha}$\textbf{ comprendido en el intervalo }$\mathbf{[0,\pi/2]}$\textbf{ y que, a la vez, esté en el nivel cuántico de energía }$\mathbf{\epsilon.}$\\ \\
    Esta probabilidad vendrá dada por
    \begin{equation}
    \begin{array}{rl}
        P\left(\alpha\in\brackets{0,\pi/2};\epsilon\right)&=\frac{1}{Z}g_{\epsilon}e^{-\beta\epsilon}\int_0^{\pi/2}\int_0^{2\pi} d\alpha d\beta (b+a\cos\alpha)=\\ \\
        &=\frac{2e^{-\beta\epsilon}}{Z}2\pi\int_0^{\pi/2}(b+a\cos\alpha)d\alpha=\frac{4\pi e^{-\beta\epsilon}}{Z}\brackets{b\frac{\pi}{2}+a}
        \end{array}
    \end{equation}
    donde $Z=Z_{esp}Z_{int}$, con
    \begin{equation}
        Z_{esp}=\int_0^{2\pi}d\beta\int_0^{2\pi}d\alpha(b+a\cos\alpha)=4\pi^2b;\hspace{6mm}Z_{int}=1+2e^{-\beta\epsilon}
    \end{equation}
    Luego, la probabilidad queda,
    \begin{equation}
        P\left(\alpha\in\brackets{0,\pi/2};\epsilon\right)=\frac{4\pi e^{-\beta\epsilon}}{4\pi^2b\left(1+2e^{-\beta\epsilon}\right)}\brackets{b\frac{\pi}{2}+a}=\frac{e^{-\beta\epsilon}}{1+2e^{-\beta\epsilon}}\brackets{\frac{1}{2}+\frac{a}{\pi b}}
    \end{equation}
    \item \textbf{Calcule el valor medio }$\mathbf{<\cos^2\alpha>.}$\\ \\
    Aplicando la definición de valor medio, tenemos que
    \begin{equation}
        <\cos^2\alpha>=\frac{\int_0^{2\pi}\cos^2\alpha(b+a\cos\alpha)d\alpha}{\int_0^{2\pi}(b+a\cos\alpha)d\alpha}
    \end{equation}
    El denominador queda,
    \begin{equation}
        \int_0^{2\pi}(b+a\cos\alpha)d\alpha=2\pi b+\cancelto{0}{a\int_0^{2\pi}\cos\alpha d\alpha}=2\pi b
    \end{equation}
    El numerador queda,
    \begin{equation}
        \int_0^{2\pi}\cos^2\alpha(b+a\cos\alpha)d\alpha=b\cancelto{\pi}{\int_0^{2\pi}\cos^2\alpha d\alpha}+a\cancelto{0}{\int_0^{2\pi}\cos^3\alpha d\alpha}=b\pi
    \end{equation}
    Luego tenemos,
    \begin{equation}
        <\cos^2\alpha>=\frac{b\pi}{2\pi b}=\frac{1}{2}
    \end{equation}
\end{enumerate}
\item \textbf{\textit{HUECOS DE ABSORCIÓN}. Un sistema está formado por una superficie con un total de }$\mathbf{N_0}$\textbf{ huecos de absorción, de manera que dentro de cada uno de estos huecos pueden entrar a lo sumo un máximo de 3 partículas (\textit{fermiones}). Cuando un fermión se absorbe en un hueco, puede ocupar dos diferentes estados cuánticos de energía 0 y }$\mathbf{\epsilon}$\textbf{, con degeneración 1 y 2, respectivamente. El sistema está en equilibrio a temperatura }$\mathbf{T.}$
\begin{Figura}
    \centering
    \includegraphics[width=0.6\textwidth]{huecos.png}     
\end{Figura}
    \begin{enumerate}
        \item \textbf{Centrándonos en un hueco en particular, calcule la \textit{función de partición canónica} para los casos en los que entren en el hueco 1, 2 y 3 fermiones (}$\mathbf{Z_1(T)}$\textbf{ y }$\mathbf{Z_2(T)}$\textbf{ y }$\mathbf{Z_3(T)}$\textbf{ respectivamente).}\\ \\
        En este caso, tenemos dos fases en equilibrio, el gas y los huecos.\\ \\
        Como tratamos con fermiones, no podemos repetir estados cuánticos, pues por el Principio de Exclusión de Pauli solo podremos ocupar un máximo de una partícula por estado.\\ \\
        Para el caso de una partícula, tenemos tres niveles posibles de ocupación, una posibilidad para energía $E=0$ y dos posibilidades para energía $E=\epsilon$. Entonces, la función de partición canónica vendrá dada por,
        \begin{equation}
            Z_1(T)=g_0e^{-\beta E_0}+g_{\epsilon}e^{-\beta\epsilon}=1\cdot e^{-\beta0}+2\cdot e^{-\beta\epsilon}=1+2e^{-\beta\epsilon}
        \end{equation}
        Para el caso de dos fermiones tendremos $\left(\begin{matrix}
            3\\
            2
        \end{matrix}\right)=3$ combinaciones posibles y para cada combinación sumamos las energías de los niveles ocupados. Así,
        \begin{enumerate}
            \item \textbf{nivel 0 y nivel }$\epsilon_1$ $\to$ $E_1=0+\epsilon=\epsilon$.
            \item \textbf{nivel 0 y nivel }$\epsilon_2$ $\to$ $E_2=0+\epsilon=\epsilon$.
            \item \textbf{nivel} $\epsilon_1$ \textbf{y nivel }$\epsilon_2$ $\to$ $E_3=\epsilon+\epsilon=2\epsilon$.
        \end{enumerate}
        Luego, la función de partición canónica queda,
        \begin{equation}
            Z_2(T)=\sum_{i=1}^3e^{-\beta E_i}=2e^{-\beta\epsilon}+e^{-2\beta\epsilon}
        \end{equation}
        Para el caso de tres fermiones solo tenemos una posibilidad, pues solo tenemos 3 niveles. Luego $E=0+\epsilon+\epsilon=2\epsilon$, teniendo así una función de partición canónica, tal que
        \begin{equation}
            Z_3(T)=e^{-2\beta\epsilon}
        \end{equation}
        \item \textbf{Determine el \textit{potencial macrocanónico} del sistema de }$\mathbf{N_0}$\textbf{ huecos, }$\mathbf{J}$\textbf{. Emplee para ello la colectividad que considere más adecuada.}\\ \\
Estamos trabajando con un número variable de partículas y un número fijo de $N_0$ huecos, pero cada hueco puede contener 0, 1, 2 o 3 partículas. Por tanto, la colectividad más adecuada es la macrocanónica, en la que la temperatura $T$, el volumen $V$ y el potencial químico $\mu$ son fijos y el número de partículas y la energía pueden fluctuar.\\ \\
Comenzamos calculando la función de partición macrocanónica de un solo hueco, tal que
\begin{equation}
    \Theta_1=\sum_{n=0}^3\lambda^nZ_n(T)
\end{equation}
consideramos $Z_0=1$ pues es el vacío y los demás $Z_i$ los calculamos antes, así tenemos que
\begin{equation}
    \Theta_1=1+\lambda (1+2e^{-\beta\epsilon})+\lambda^2(2e^{-\beta\epsilon}+e^{-2\beta\epsilon})+\lambda^3e^{-2\beta\epsilon}
\end{equation}
donde $\lambda=e^{\beta\mu}$ es la fugacidad.\\ \\
Ahora calculamos la función de partición macrocanónica del sistema de $N_0$ huecos, que viene dada por,
\begin{equation}
    \Theta_{huecos}=(\Theta_1)^{N_0}=\brackets{1+\lambda (1+2e^{-\beta\epsilon})+\lambda^2(2e^{-\beta\epsilon}+e^{-2\beta\epsilon})+\lambda^3e^{-2\beta\epsilon}}^{N_0}
\end{equation}
El potencial macrocanónico se define como $J=-kT\ln\Theta$, luego
\begin{equation}
    J=-kTN_0\ln\Theta_{1}=-kTN_0\ln\brackets{1+\lambda (1+2e^{-\beta\epsilon})+\lambda^2(2e^{-\beta\epsilon}+e^{-2\beta\epsilon})+\lambda^3e^{-2\beta\epsilon}}
\end{equation}
con $\lambda=e^{\beta\mu}$.        
        \item \textbf{Calcule el \textit{número medio de fermiones} absorbidos en la superficie en función de }$\mathbf{T}$\textbf{ y }$\mathbf{\mu,<N>}$\\ \\
El número medio de fermiones absorbidos en la superficie se define como
\begin{equation}
    <N>=\frac{1}{\beta}\left(\frac{\partial\ln\Theta_{huecos}}{\partial\beta}\right)_{T,V}=\lambda\left(\frac{\partial\ln\Theta_{huecos}}{\partial\lambda}\right)_{T,V}
\end{equation}
Luego, derivamos el logaritmo de la gran función de partición de los huecos, tal que
\begin{equation}
\begin{array}{c}
    <N>=\lambda N_0\frac{1+2e^{-\beta\epsilon}+2\lambda(2e^{-\beta\epsilon}+e^{-2\beta\epsilon})+3\lambda^2e^{-2\beta\epsilon}}{1+\lambda(1+2e^{-\beta\epsilon})+\lambda^2(2e^{-\beta\epsilon})+\lambda^3e^{-2\beta\epsilon}}=N_0\frac{\lambda(1+2e^{-\beta\epsilon})+2\lambda^2(2e^{-\beta\epsilon}+e^{-2\beta\epsilon})+3\lambda^3e^{-2\beta\epsilon}}{1+\lambda(1+2e^{-\beta\epsilon})+\lambda^2(2e^{-\beta\epsilon})+\lambda^3e^{-2\beta\epsilon}}
\end{array}
\end{equation}
o bien, poniéndolo en función de las funciones de partición canónicas, tenemos
\begin{equation}
    <N>=N_0\frac{Z_1\lambda+2Z_2\lambda^2+3Z_3\lambda^3}{Z_0+Z_1\lambda+Z_2\lambda^2+Z_3\lambda^3}
\end{equation}
        
        \item \textbf{Sabiendo que los huecos se encuentran en equilibrio con un vapor, y suponiendo que dicho vapor se comporta como un gas ideal clásico, estudie el comportamiento del número medio de partículas absorbidas en los huecos en función de la presión del vapor.}\\ \\
Para relacionar $<N>$ con la presión del gas, considerando que es un gas ideal clásico, podemos usar la expresión de $\lambda_g$ del gas y sustituirla en la expresión de $<N>$ por $\lambda$ de los huecos, pues al estar ambos sistemas en equilibrio, $\lambda=\lambda_g$. Calculamos esta fugacidad, sabiendo que la función de partición canónica de un gas ideal clásico de una partícula es
\begin{equation}
    Z_{1,g}=V\frac{(2\pi mkT)^{3/2}}{h^3}
\end{equation}
Luego, la gran función de partición del gas viene dada por,
\begin{equation}
    \Theta_{gas}=e^{\lambda_gZ_{1,g}}
\end{equation}
Calculamos el potencial macrocanónico del gas,
\begin{equation}
    J=-kT\ln\Theta_{gas}=-kT\lambda_g Z_{1,g}
\end{equation}
al ser un gas ideal clásico, este potencial se relaciona con la presión y el volumen, como $J=-PV$, luego,
\begin{equation}
    J=-P\cancel{V}=-kT\lambda_g\cancel{V}\frac{(2\pi mkT)^{3/2}}{h^3}
\end{equation}
Luego,
\begin{equation}
    \lambda_g=\frac{P}{kT}\frac{h^3}{(2\pi mkT)^{3/2}}
\end{equation}
donde $P$ es la presión del gas. Luego, sabiendo que el gas y los huecos están en equilibrio, $\lambda=\lambda_g$. Por tanto,
\begin{equation}
    <N>=N_0\frac{Z_1\lambda_g+2Z_2\lambda_g^2+3Z_3\lambda_g^3}{Z_0+Z_1\lambda_g+Z_2\lambda_g^2+Z_3\lambda_g^3}
\end{equation}
donde $\lambda_g$ y $Z_i$, $i=0,1,2,3$ vienen dados por,
\[\begin{matrix}
    \lambda_g=\frac{P}{kT}\frac{h^3}{(2\pi mkT)^{3/2}}; & Z_0=1; & Z_1=1+2e^{-\beta\epsilon} & Z_2=2e^{-\beta\epsilon}+e^{-2\beta\epsilon} & Z_3=e^{-2\beta\epsilon}
\end{matrix}\]
relacionando así $<N>$ con la presión del gas $P$.
    \end{enumerate}

\end{enumerate}

\end{document}
